\documentclass[a4paper,
fontsize=11pt,
%headings=small,
oneside,
numbers=noperiodatend,
parskip=half-,
bibliography=totoc,
final
]{scrartcl}

\usepackage{synttree}
\usepackage{graphicx}
\setkeys{Gin}{width=.4\textwidth} %default pics size

\graphicspath{{./plots/}}
\usepackage[ngerman]{babel}
\usepackage[T1]{fontenc}
%\usepackage{amsmath}
\usepackage[utf8x]{inputenc}
\usepackage [hyphens]{url}
\usepackage{booktabs} 
\usepackage[left=2.4cm,right=2.4cm,top=2.3cm,bottom=2cm,includeheadfoot]{geometry}
\usepackage{eurosym}
\usepackage{multirow}
\usepackage[ngerman]{varioref}
\setcapindent{1em}
\renewcommand{\labelitemi}{--}
\usepackage{paralist}
\usepackage{pdfpages}
\usepackage{lscape}
\usepackage{float}
\usepackage{acronym}
\usepackage{eurosym}
\usepackage[babel]{csquotes}
\usepackage{longtable,lscape}
\usepackage{mathpazo}
\usepackage[normalem]{ulem} %emphasize weiterhin kursiv
\usepackage[flushmargin,ragged]{footmisc} % left align footnote

\usepackage{listings}

\urlstyle{same}  % don't use monospace font for urls

\usepackage[fleqn]{amsmath}

%adjust fontsize for part

\usepackage{sectsty}
\partfont{\large}

%Das BibTeX-Zeichen mit \BibTeX setzen:
\def\symbol#1{\char #1\relax}
\def\bsl{{\tt\symbol{'134}}}
\def\BibTeX{{\rm B\kern-.05em{\sc i\kern-.025em b}\kern-.08em
    T\kern-.1667em\lower.7ex\hbox{E}\kern-.125emX}}

\usepackage{fancyhdr}
\fancyhf{}
\pagestyle{fancyplain}
\fancyhead[R]{\thepage}

%meta
%meta

\fancyhead[L]{Th. Ernst \\ %author
LIBREAS. Library Ideas, 30 (2016). % journal, issue, volume.
\href{http://nbn-resolving.de/
}{}} % urn
\fancyhead[R]{\thepage} %page number
\fancyfoot[L] {\textit{Creative Commons BY 3.0}} %licence
\fancyfoot[R] {\textit{ISSN: 1860-7950}}

\title{\LARGE{Eine Kritik der Kritik des Open Access. \\ Zu den Debatten über das Zweitveröffentlichungsrecht und über die Wertigkeit von Print- vs. Digitalpublikationen in den Geisteswissenschaften}} %title %title
\author{Thomas Ernst} %author

\setcounter{page}{1}

\usepackage[colorlinks, linkcolor=black,citecolor=black, urlcolor=blue,
breaklinks= true]{hyperref}

\date{}
\begin{document}

\maketitle
\thispagestyle{fancyplain} 

%abstracts

%body
\emph{Zur Sache des Buches}, \emph{Der Preis des Buches und sein Wert},
\emph{Warum Bücher?}, \emph{Die Befreiung von den Büchern},
\emph{Bücherdämmerung} und \emph{Weiße Magie. Die Epoche des Papiers} --
gedruckte Bücher über gedruckte Bücher haben derzeit
Hochkonjunktur.\footnote{Vgl. Michael Hagner: Zur Sache des Buches.
  Göttingen: Wallstein 2015; Roland Reuß: FORS. Der Preis des Buches und
  sein Wert. Frankfurt am Main/Basel: Stroemfeld 2013; Michael
  Schikowski: Warum Bücher? Buchkultur in Zeiten der Digitalkultur.
  Frankfurt am Main: Bramann 2013; Günter Karl Bose: Das Ende einer
  Last. Die Befreiung von den Büchern. Göttingen: Wallstein 2013; Detlef
  Bluhm (Hg.): Bücherdämmerung. Über die Zukunft der Buchkultur.
  Darmstadt: Wissenschaftliche Buchgesellschaft 2014; Lothar Müller:
  Weiße Magie. Die Epoche des Papiers. München: Hanser 2012.} Das
aktuelle Nachdenken über den Wert gedruckter Bücher ist eine Reaktion
auf den digitalen Wandel, der neue Möglichkeiten des digitalen
Veröffentlichens mit sich bringt. Diese Möglichkeiten verändern die
gewachsenen Informationssysteme nachhaltig. Schon in der
\enquote{Gutenberg-Galaxis} haben sich verschiedene Öffentlichkeiten mit
eigenen Regelsystemen und unterschiedlichen Produktions-, Distributions-
und Rezeptionsverfahren ausgeprägt. Die jeweiligen gesellschaftlichen
Teilsysteme -- wie der Literaturbetrieb, der Journalismus und die
Wissenschaft -- erkennen im digitalen Medienwandel verschiedene
Potenziale und Probleme.

Dieser Beitrag fokussiert die Frage, wie die Geisteswissenschaften sich
zu den Potenzialen der neuen digitalen Veröffentlichungen positionieren.
Er geht dieser Frage nach, indem er konträre Diskurspositionen sowohl
allgemein zum Wert gedruckter versus digitaler Publikationen als auch
konkret zur Frage des Open Access nachzeichnet. In einem zweiten Schritt
reflektiert er vor diesem Hintergrund das seit Anfang 2014 geltende
Zweitveröffentlichungsrecht. Als zentrales diskursives Ereignis gilt der
\emph{Heidelberger Appell},\footnote{Institut für Textkritik: Für
  Publikationsfreiheit und die Wahrung der Urheberrechte, online seit
  dem 22.3.2009 unter
  \href{http://www.textkritik.de/urheberrecht/index.htm}{\emph{http://www.textkritik.de/urheberrecht/index.htm}}
  {[}Stand: 11.11.2016{]}.} der sich 2009 gegen eine stärkere
Verpflichtung von WissenschaftlerInnen auf Open
Access-Veröffentlichungen richtete und von über 2.500
SchriftstellerInnen und GeisteswissenschaftlerInnen unterzeichnet wurde.
Neben der Beschreibung der konträren Diskurspositionen fragt ein kleines
empirisches Kapitel danach, wie heute ganz praktisch zwischen
WissenschaftlerInnen und Verlagen Fragen des Open Access und des
Zweitveröffentlichungsrechts verhandelt werden, denn in den eher
grundsätzlich geführten öffentlichen Debatten geht oft unter, welche
konkreten Erfahrungen WissenschaftlerInnen mit Verlagen machen.

\section*{Digitale wissenschaftliche Erkenntnisprozesse: Von
Appellen für Open Science zu einer Standardisierung der offenen
Veröffentlichungspraxis}\label{digitale-wissenschaftliche-erkenntnisprozesse-von-appellen-fuxfcr-open-science-zu-einer-standardisierung-der-offenen-veruxf6ffentlichungspraxis}

Einen guten Eindruck, wie stark die Digitalisierung der geistigen Arbeit
die Wissenschaftskommunikation in den letzten Dekaden verändert hat,
gibt ein kleines Video aus dem Arbeitsalltag von Max
Horkheimer.\footnote{\enquote{jackogreene}: Prof Horkheimer going
  through his mail (german language), seit dem 4.2.2009 online unter
  \href{https://www.youtube.com/watch?v=NsLWJP3ZpTc}{\emph{https://www.youtube.com/watch?v=NsLWJP3ZpTc}}
  {[}Stand: 11.11.2016{]}, vgl. vor allem 00:00--01:37.} Es trägt den
Titel \emph{Prof Horkheimer going through his mail} und zeigt ihn in
seinem Büro mit seiner Sekretärin. An den Wänden stehen Bücherregale,
auf dem Tisch liegen zahlreiche Papierstapel. Vor der Sekretärin steht
eine Schreibmaschine. Zunächst berichtet Horkheimer erfreut, dass er das
geplante Telefonat aus Frankfurt erhalten habe, \enquote{und sogar noch
das mit Nürnberg ist auch gekommen, jawohl},\footnote{Ebd.,
  00:18--00:22.} um dann eine Mappe mit Briefentwürfen durchzugehen und
der Sekretärin einen Brief zu diktieren, den diese stenografisch
notiert. Sie wird ihn später mit der Schreibmaschine auf ein anderes
Papier übertragen.

In Abgrenzung von diesen Praxen ermöglicht die elektronische und
codierte Produktion, Distribution und Rezeption von Informationen eine
weitaus höhere Intensität, Schnelligkeit und Stabilität der -- somit:
digitalen -- Wissenschaftskommunikation. Elektronische Briefe (sog.
\enquote{E-Mails}), mobilisierte und visualisierte Telefonate (mit
Smartphones oder Diensten wie Skype) oder Kurznachrichten (SMS, Twitter)
verdichten und beschleunigen die geisteswissenschaftliche
Alltagskommunikation in bisher ungekannter Weise. Doch nicht nur die
Kommunikation selbst verändert sich, auch die Formate und
Distributionsformen des wissenschaftlichen Erkenntnisprozesses, also die
mediale Form von Monographien und Aufsätzen.

Eine nachhaltige Erkenntnisproduktion basiert in
geisteswissenschaftlichen Veröffentlichungen auf einer klaren
Fragestellung, einer Ausgangshypothese, einem gut ausgewählten
exemplarischen Analysegegenstand, einer angemessen ausgewählten und
angewandten Methode und einem möglichst klaren und differenzierten
Ergebnis. Der wissenschaftliche Progress innerhalb einer Disziplin oder
einer Arbeitseinheit wird dann gefördert, wenn diese Elemente der
Erkenntnisproduktion offen zugängig und nachprüfbar sind, um entweder
kritisiert und modifiziert oder verifiziert und weiter distribuiert zu
werden. In einem skeptischen Essay zur digitalen Wissenschaftssprache
bemisst Valentin Groebner folglich Wissenschaftlichkeit an der
Möglichkeit, \enquote{dass ältere Informationen wiederauffindbar
gespeichert werden, um mit neuen Daten kontrolliert und ergänzt werden
zu können. In der Praxis bedeutet das den Umgang mit ziemlich großen
Mengen an Information, mit denen der jeweils neue (oder angeblich neue)
Fund abgeglichen und eingeordnet werden kann.} Die Wissenschaft verfüge
somit über eine spezifische \enquote{Geschichte der
Informationsbewirtschaftung},\footnote{Valentin Groebner:
  Wissenschaftssprache digital. Die Zukunft von gestern: Konstanz:
  Konstanz UP 2014, S. 70.} deren schon immer zentrales Problem, welche
Informationen wie und wem zur Verfügung stehen, sich durch die
Möglichkeiten der Digitalisierung von Informationen nur anders
darstellt.

Seit etwa 2001 haben sich verschiedene Konzeptionen einer \emph{Open
Science} und einer \emph{Open Scholarship} entwickelt, die in
unterschiedlicher Weise den Zugriff auf wissenschaftliche Informationen
und die Partizipation an wissenschaftlichen Erkenntnisprozessen
vereinfachen möchten, um die Potenziale der Digitalisierung für eine
bessere wissenschaftliche Praxis zu nutzen. Eine zentrale Rolle spielt
dabei der \emph{Open Access}, also der freie Zugang zu
wissenschaftlichen Veröffentlichungen und Informationen. Im Anschluss an
die \emph{Budapest Open Access Initiative} (2002)\footnote{Budapest Open
  Access Initiative (14.2.2002), online unter
  \href{http://www.budapestopenaccessinitiative.org/read}{\emph{http://www.budapestopenaccessinitiative.org/read}}
  {[}Stand: 11.11.2016{]}.} forderten in der \emph{Berliner Erklärung
über offenen Zugang zu wissenschaftlichem Wissen} (2003) beziehungsweise
in der \emph{Göttinger Erklärung zum Urheberrecht für Bildung und
Wissenschaft} (2004) einige der wichtigsten deutschen
Forschungsorganisationen (unter anderem Wissenschaftsrat,
Hochschulrektorenkonferenz, DFG, Max-Planck-Gesellschaft) und neue
Initiativgruppen einen solchen freien Zugriff auf wissenschaftliche
Informationen. In der \emph{Göttinger Erklärung} heißt es entsprechend
offensiv und allgemein: \enquote{In einer digitalisierten und vernetzten
Informationsgesellschaft muss der Zugang zur weltweiten Information für
jedermann zu jeder Zeit von jedem Ort für Zwecke der Bildung und
Wissenschaft sichergestellt werden!}\footnote{Aktionsbündnis
  \enquote{Urheberrecht für Bildung und Wissenschaft}: Göttinger
  Erklärung zum Urheberrecht für Bildung und Wissenschaft vom 5. Juli
  2004, S. 1, online unter
  \href{http://www.urheberrechtsbuendnis.de/GE-Urheberrecht-BuW-Mitgl.pdf}{\emph{http://www.urheberrechtsbuendnis.de/GE-Urheberrecht-BuW-Mitgl.pdf}}
  {[}Stand: 11.11.2016{]}. Vgl. auch die Berliner Erklärung über den
  offenen Zugang zu wissenschaftlichem Wissen, online unter
  \href{https://openaccess.mpg.de/68053/Berliner_Erklaerung_dt_Version_07-2006.pdf}{\emph{https://openaccess.mpg.de/68053/Berliner\_Erklaerung\_dt\_Version\_07-2006.pdf}}
  {[}Stand: 11.11.2016{]}.}

Seit diesen frühen und allgemein gehaltenen Manifesten haben sich in den
Geisteswissenschaften unterschiedliche Verfahren des digitalen
Veröffentlichens etabliert. Die offiziellen Verlautbarungen zu Fragen
einer Open Science, der Open Scholarship und des Open Access lesen sich
heute, zumindest im Bereich der Digital Humanities, weniger als
politisches Postulat denn als obligatorisches Fundament einer bereits
bestehenden wissenschaftlichen Praxis. Das \emph{Working Paper der
Arbeitsgruppe Digitales Publizieren im Verband Digital Humanities im
deutschsprachigen Raum} sieht 2016, dem Verbandsziel gemäß,\footnote{Der
  Verband Digital Humanities im deutschsprachigen Raum formuliert als
  eine seiner fünf Zielsetzungen, \enquote{den freien Zugang und die
  freie Nutzung von Wissensbeständen und Verfahren (Open Access, Open
  Source) zu fördern.} Vgl. Digital Humanities im deutschsprachigen
  Raum: DHd-Satzung, online unter
  \href{https://dig-hum.de/dhd-satzung}{\emph{https://dig-hum.de/dhd-satzung}}
  {[}Stand: 11.11.2016{]}.} das \enquote{Open-Access-Publizieren} als
Standardfall. Es gibt zudem den Forschenden Empfehlungen zu den
verschiedenen Open-Access-Strategien (goldener/grüner Weg),
Finanzierungs- bzw. Geschäftsmodellen (Author-Pays-Modell,
Publikationsfonds, Hybrid-Open-Access, Freemium-Modell) und zu
rechtlichen Rahmenbedingungen und Lizensierungsformen
(Open-Content-Lizenzen, Creative Commons).\footnote{DHd-Arbeitsgruppe
  \enquote{Digitales Publizieren}: Workingpaper \enquote{Digitales
  Publizieren}, seit dem 1.3.2016 online unter
  \href{http://dhd-wp.hab.de/?q=ag-text\#abschnitt5}{\emph{http://dhd-wp.hab.de/?q=ag-text\#abschnitt5}}
  {[}Stand: 11.11.2016{]}. Hinweis: Der Autor dieses Beitrags hat in der
  Arbeitsgruppe mitgearbeitet, war allerdings vor allem am allgemeinen
  Diskussionsprozess der Arbeitsgruppe sowie an der Formulierung Passage
  zur digitalen wissenschaftlichen Autorschaft (mit Anne Baillot)
  beteiligt.} Schließlich fordert es im Sinne einer besseren
wissenschaftlichen Praxis, dass auch die \enquote{Forschungsdaten (Open
Research Data), Zusatzmaterialien (Open Extra Material),
Softwareanwendungen (Open Source) sowie {[}\ldots{}{]} in der Lehre
eingesetzte Bildungsressourcen (Open Educational Resources)}\footnote{Ebd.
  (Hervorh. bereinigt, T.E.).} offen verfügbar sein sollten.

\emph{The Vienna Principles der Arbeitsgruppe Open Access and Scholarly
Communication des Open Access Network Austria} kritisieren ebenfalls
2016 zunächst eine traditionelle Form der Wissenschaftskommunikation,
die durch \enquote{{[}r{]}estricted access and collaboration},
\enquote{{[}i{]}nefficient processes}, \enquote{{[}l{]}ack of
reproducibility and transparency}, \enquote{{[}t{]}echnical and legal
barriers} und verschiedene \enquote{{[}i{]}n\-cent\-ives in need of
improvement} (zersplitterte Veröffentlichungspraxen,
Begutachtungsverfahren) gekennzeichnet sei.\footnote{Open Access Network
  Austria: The Vienna Principles: A Vision for Scholarly Communication
  in the 21\textsuperscript{st} Century, online unter
  \href{https://zenodo.org/record/55597/files/ViennaPrinciples_v1_2016.pdf}{\emph{https://zenodo.org/record/55597/files/ViennaPrinciples\_v1\_2016.pdf}},
  S. 4f. {[}Stand: 11.11.2016{]}.} Im Gegensatz dazu stellt die
Arbeitsgruppe zwölf Prinzipien auf, die neben den bekannten Größen wie
\enquote{Accessibility} oder \enquote{Reproducibility} spezifische
digitale Arbeitsweisen einfordern. Dazu zählt, \enquote{{[}to{]} foster
collaboration and participation between researchers and their
stakeholders}, \enquote{{[}to{]} embrace the possibilities of new
technology} und \enquote{{[}to{]} provide transparent and competent
review}.\footnote{Ebd., S. 6--10.}

Diese jüngeren Manifeste bzw. Empfehlungen gehen von etablierten
digitalen Arbeitspraxen in den (Geistes-)Wissenschaften aus, deren
Erkenntnisprozesse sich stärker kollaborativ, offen und prozessual
vollziehen und die bereits auf Erfahrungen mit den juridischen,
ökonomischen und institutionellen Voraussetzungen des Open Acess
basieren. Gegen diese verschiedenen Forderungen einer offenen
geisteswissenschaftlichen Veröffentlichungspraxis, die sowohl mit dem
Selbstverständnis vieler ForscherInnen als auch mit den traditionellen
Geschäfts- und Arbeitsmodellen der Wissenschaftsverlage brechen, erhebt
sich jedoch Widerstand.

\section*{\texorpdfstring{Erkenntnisprozesse wie gedruckt: Das Buch
als \enquote{magischer Körper} und die Kritik des Open
Access}{Erkenntnisprozesse wie gedruckt: Das Buch als magischer Körper und die Kritik des Open Access}}\label{erkenntnisprozesse-wie-gedruckt-das-buch-als-magischer-kuxf6rper-und-die-kritik-des-open-access}

Die Kritik an der zunehmenden politischen und institutionellen Förderung
von Open-Access-Modellen kulminiert in den Geisteswissenschaften im
\emph{Heidelberger Appell} vom 22. März 2009, dessen Programm unter dem
Titel \emph{Für Publikationsfreiheit und die Wahrung der Urheberrechte}
steht und der 2636 UnterzeichnerInnen findet, darunter AutorInnen wie
Günter Grass und Herta Müller, VertreterInnen der wichtigsten deutschen
Verlage und zahlreiche renommierte (Geistes-)WissenschaftlerInnen, unter
ihnen unter anderem der Editionsphilologe und Appell-Initiator Roland
Reuß sowie der Wissenschaftshistoriker Michael Hagner, deren Beiträge
zur Debatte im Folgenden genauer analysiert werden.

Der \emph{Heidelberger Appell} verbindet zwei unterschiedliche Probleme,
nämlich einerseits die auf internationalen Social-Media-Plattformen wie
Google Books und Youtube begangenen Urheberrechtsverletzungen, die
\enquote{in ungeahntem Umfang und ohne strafrechtliche
Konsequenzen}\footnote{Institut für Textkritik: Für Publikationsfreiheit
  und die Wahrung der Urheberrechte (Anm. 3).} stattfänden, und
andererseits den nationalen Einsatz der Allianz der deutschen
Wissenschaftsorganisationen für die Umsetzung von
Open-Access-Strategien.\footnote{Ich habe mich an anderer Stelle bereits
  zum Heidelberger Appell positioniert, vgl. Thoms Ernst: Das Internet
  und die digitale Kopie als Chance und Problem für die Literatur und
  die Wissenschaft. Über die Verabschiedung des geistigen Eigentums, die
  Transformation der Buchkultur und zum Stand einer fehlgeleiteten
  Debatte. In: kultuRRevolution. zeitschrift für angewandte
  diskurstheorie. Heft 57, Oktober 2009, S. 29-37; online verfügbar
  unter
  \href{http://www.thomasernst.net/ThomasErnst-InternetLiteratur(kRR57).pdf}{\emph{http://www.thomasernst.net/ThomasErnst-InternetLiteratur(kRR57).pdf}}
  {[}Stand: 11.11.2016{]}.} Ohne den Begriff des Open Access zu nennen
oder explizit auf die konkreten Inhalte der \emph{Schwerpunktinitiative
\enquote{Digitale Information} der
Allianz-Partnerorganisationen}\footnote{Vgl. Allianz der
  Wissenschaftsorganisationen: Schwerpunktinitiative \enquote{Digitale
  Information} der Allianz-Partnerorganisationen, online seit dem
  11.06.2008 unter
  \href{http://www.wissenschaftsrat.de/download/archiv/Allianz-digitale\%20Info.pdf}{\emph{http://www.wissenschaftsrat.de/download/archiv/Allianz-digitale\%20Info.pdf}}
  {[}Stand: 11.11.2016{]}. Zur Allianz der Wissenschaftsorganisationen
  zählen: Alexander von Humboldt-Stiftung, Deutscher Akademischer
  Austauschdienst, Deutsche Forschungsgemeinschaft,
  Fraunhofer-Gesellschaft, Helmholtz-Gemeinschaft Deutscher
  Forschungszentren, Hochschulrektorenkonferenz, Leibniz-Gemeinschaft,
  Max-Planck-Gesellschaft und der Wissenschaftsrat.} zu verweisen,
konstatieren die UnterzeichnerInnen, dass die internationalen
\enquote{Raubkopien} und die nationalen Open-Access-Initiativen schwere
Eingriffe in die Freiheit der Wissenschaft und die Selbstbestimmung der
UrheberInnen vornähmen: \enquote{Jeder Zwang, jede Nötigung zur
Publikation in einer bestimmten Form ist ebenso inakzeptabel wie die
politische Toleranz gegenüber Raubkopien, wie sie Google derzeit
massenhaft herstellt.}\footnote{Ebd.}

Zwar veröffentlicht die Allianz der deutschen
Wissenschaftsorganisationen schon drei Tage später eine gemeinsame
Erklärung und beharrt darauf, dass Open Access \enquote{{[}k{]}ein
Eingriff in die Publikationsfreiheit} sei.\footnote{Vgl. Allianz der
  Wissenschaftsorganisationen: Gemeinsame Erklärung der
  Wissenschaftsorganisationen. Open Access und Urheberrecht: Kein
  Eingriff in die Publikationsfreiheit, online seit dem 25.03.2009 unter
  \href{http://www.wissenschaftsrat.de/download/archiv/allianz_erklaerung_25-03-09.pdf}{\emph{http://www.wissenschaftsrat.de/download/archiv/allianz\_erklaerung\_25-03-09.pdf}}
  {[}Stand: 11.11.2016{]}.} Dennoch ist der \emph{Heidelberger Appell}
als diskursives Ereignis ein wichtiger Referenzpunkt für die weiteren
Debatten.\footnote{Diese Debatte wurde 2009 bereits in Heft 15 von
  LIBREAS geführt, vgl. dazu die folgenden Beiträge: - Uwe Jochum: Der
  Souverän: In: LIBREAS. Library Ideas, Heft 15 (2009),
  \href{http://libreas.eu/ausgabe15/texte/006.htm}{\emph{http://libreas.eu/ausgabe15/texte/006.htm}}.
  - Joachim Eberhardt: Wiederholung erzeugt keine Wahrheit. Jochum
  schreibt immer noch gegen Open Access. In: LIBREAS. Library Ideas,
  Heft 15 (2009),
  \href{http://libreas.eu/ausgabe15/texte/007.htm}{\emph{http://libreas.eu/ausgabe15/texte/007.htm}}.
  - Joachim Losehand: Moskenstraumen. In: LIBREAS. Library Ideas, Heft
  15 (2009),
  \href{http://libreas.eu/ausgabe15/texte/008.htm}{\emph{http://libreas.eu/ausgabe15/texte/008.htm)}})
  {[}alle Texte: Stand: 11.11.2016{]}.} Daher werden wir uns erstens
näher ansehen, welche buch- und medientheoretischen Annahmen hinter
diesem Appell stehen, sowie zweitens, welche differenzierteren Argumente
gegen Open Access die zentralen Vertreter des \emph{Heidelberger
Appells} an anderer Stelle vorbringen. Dazu sollen mit dem Aufsatz
\emph{Autorverantwortung und Text} (2009) sowie den Büchern \emph{Ende
der Hypnose. Vom Netz und zum Buch} (2012) sowie \emph{FORS. Der Preis
des Buches und sein Wert} (2013) zentrale Schriften von Roland Reuß, dem
Initiator des \emph{Heidelberger Appells}, kursorisch betrachtet werden,
in denen er sein Engagement gegen digitale Veröffentlichungen und Open
Access begründet. Daneben rückt die Monographie \emph{Zur Sache des
Buches} (2015) von Michael Hagner in den Blick, in der Hagner, ein
Unterzeichner des \emph{Heidelberger Appells}, sich mit der Eignung
sowohl gedruckter als auch digitaler Medien für wissenschaftliche
Erkenntnisprozesse beschäftigt und sich intensiv und kritisch mit Open
Access auseinandersetzt.

Hagner wiederum bezieht sich positiv auf Lothar Müllers \emph{Weiße
Magie. Die Epoche des Papiers} (2012) und Müller, Hagner und Reuß teilen
eine besondere Präferenz für das gedruckte und wohlgestaltete Buch, das
sie als ein ganz besonderes Medium der Erkenntnis bewerten. Müllers
Ausführungen über die Geschichte des Papiers, die schließlich im
gedruckten Buch gipfelt, stehen in einem deutlichen Kontrast zur
Ausrufung einer digitalen Gesellschaft. Selbst digitale
Textverarbeitungsprogramme, konstatiert Müller, griffen noch immer auf
papierne Symbole zurück: \enquote{So raschelt es im elektronischen
Papierkorb, wenn wir eine Datei löschen, und Scherensymbole bieten das
Ausschneiden markierten Textes an.} Solche Beispiele überhöht Müller zur
starken These, dass trotz aller digitalen Mediennutzung die
\enquote{Digitale Gesellschaft} noch nicht existiere: \enquote{Wir
leben, bis auf weiteres, immer noch in der Epoche des
Papiers.}\footnote{Müller, Weiße Magie (Anm. 2), S. 352.} In dieser
Epoche des Papiers schwinde nicht die gesellschaftliche Bedeutung des
Buches, vielmehr werde die Relevanz des einzelnen Druckwerks in einer
zunehmend digitalen Medienumgebung gesteigert. Gedruckte Bücher, so
Müller, seien zwar keine wirklichen Unikate, da sie Reproduktionen einer
Vorlage seien, aber \enquote{es könnte sein, daß {[}dem gedruckten{]} im
Kontrast zum elektronischen Buch die Aura zuwächst, das
Original\emph{format} zu sein.}\footnote{Ebd., S. 351.} Dieses an Walter
Benjamin angelehnte und historisch-technisch inverse Argument greift
wiederum Hagner auf, indem er -- ebenfalls in Abgrenzung von digitalen
Texten -- die \enquote{physiognomische Individualität},
\enquote{Stabilität} und \enquote{rechte Anordnung} von Büchern preist,
die \enquote{\emph{für sich selbst} {[}existieren{]}, weil sie nicht von
einer Sekunde auf die andere gelöscht oder manipuliert werden können,
und das macht ihre Eigentümlichkeit aus.}\footnote{Hagner, Zur Sache des
  Buches (Anm. 2), S. 242, 244.} Hagner überträgt hier den Begriff der
Eigentümlichkeit, der um 1800 dem genialischen Dichten zugeschrieben und
somit der Legitimation des geistigen Eigentums Pate stand,\footnote{Vgl.
  Wolfgang Fleischhauer: Eigentümlichkeit. Ein Beitrag zur
  Wortgeschichte. In: Gerald Gillespie/Edgar Lohner (Hg.): Herkommen und
  Erneuerung. Essays für Oskar Seidlin. Tübingen: Niemeyer 1976, S.
  56-63, vor allem S. 60f.} auf das Buch selbst: Das Buch materialisiert
heute, was einst der Gedanke des Autors war, und bewahrt somit die
Möglichkeit einer schöpferischen Subjektivität.

Bei Roland Reuß erhält das Lob des Buches eine religiöse wie auch
widerständige Dimension, die an zwei Beispielen verdeutlicht werden
kann. Erstens nutzt Reuß in \emph{Ende der Hypnose} ein längeres Zitat
des protestantischen Theologen Dietrich Bonhoeffer, der für seinen
Widerstand gegen den Nationalsozialismus 1945 im Konzentrationslager
Flossenbürg hingerichtet wurde. Bonhoeffer empfiehlt darin den Weg zur
Besinnung durch die Rückwendung zum Buch.\footnote{Vgl. Roland Reuß,
  Ende der Hypnose. Vom Netz und zum Buch. Frankfurt am Main/Basel:
  Stroemfeld 2013 (OA 2012), 3. Aufl., S. 81-83.} Zweitens rahmt Reuß
sein Buch \emph{FORS. Der Preis des Buches und sein Wert}, indem er zwei
vorbildliche Buchhandlungen beschreibt, zunächst die Buchhandlung Rieck
aus Aulendorf (und am Ende des Buchs eine aus Hamburg). Reuß nähert sich
der Buchhandlung Rieck über Zitate von otl aicher an, der sich an die
Buchhandlung Rieck als einen Ort des Widerstands im Nationalsozialismus
erinnert, denn \enquote{{[}\ldots{}{]} schon ein, zwei bücher können das
holz sein, auf dem man im meer überlebt.}\footnote{Reuß, FORS (Anm. 2),
  S. 10.} Reuß recherchiert über die Buchhandlung Rieck und zitiert eine
Quelle, \enquote{daß sie nach wie vor eine der wichtigsten theologischen
Buchhandlungen in der Welt sei}, abschließend bewertet er die
Buchhandlung Rieck als eine \enquote{kleine Provinzbuchhandlung}, die
allerdings \enquote{als dezentraler Ort des Widerstands} gelten
könne.\footnote{Ebd., S. 11.}

Während der Buchhändler Rieck in Jörg Schröders Darstellung schon 1972
rückblickend als zwar sympathischer, aber vor allem weltfremder und
esoterischer Buchhändler gilt,\footnote{Bei Schröder heißt es u.a.:
  \enquote{Riecks Vorstellung war wohl die: Ich hole mir einen guten
  Mann, der wird schon gute Sachen machen. Es sollten aber gute Sachen
  sein, die exakt sein altes System und dessen Inhalte immer weiter
  dröselten. Das ging nicht. Und sehr langsam, in Wochen, dämmerte es
  mir auf: Rieck wollte überhaupt keinen Mann, der irgend etwas Neues
  machte: er wollte einen Klosterbruder haben, einen Menschen, den er in
  die Einsamkeit seines von ihm erbauten Ersatzklosters holen und ihn
  dabehalten konnte. Und er, Josef Rieck, wollte der Abt sein.} Jörg
  Schröder erzählt Ernst Herhaus: Siegfried. Erftstadt: area 2004 (OA
  1972), S. 93; siehe auch S. 88-98.} stellt Reuß seinen Kampf für das
Medium Buch mit Bonhoeffer und der Buchhandlung Rieck in eine Linie des
religiös fundierten Widerstands gegen den Nationalsozialismus. Reuß'
hyperbolische Rhetorik, die er auch in seinen zahlreichen
feuilletonistischen Beiträgen zur Open-Access-Debatte nutzt,\footnote{Um
  nur ein weiteres Beispiel zu nennen: In einem polemischen
  Debattenbeitrag für die \emph{FAZ} bewertet Reuß ein Interview, das
  Bildungsministerin Johanna Wanka der Zeitung \emph{Die Welt} gab, als
  \enquote{ein schwer goutierbares Ragout aus krud neoliberalen
  Vorstellungen von Wissenschaftsmärkten (\enquote{Monitoring} darf,
  natürlich, nicht fehlen), virtueller DDR 5.0 (mit Enteignung der
  geistigen Produktion) und Staatsautoritarismus wilhelminischer
  Anmutung.} Roland Reuß: Staatsautoritarismus, groß geschrieben. In:
  Frankfurter Allgemeine Zeitung, 28.9.2016, S. N4. Vgl. zu Reuß'
  Argumentationen und Rhetorik auch seine weiteren Debattenbeiträge,
  u.a.: - Roland Reuß: Open Access. Eine heimliche technokratische
  Machtergreifung. In: FAZ, 5.5.2009, online unter
  \href{http://www.faz.net/aktuell/feuilleton/debatten/open-access-eine-heimliche-technokratische-machtergreifung-1775488.html}{\emph{http://www.faz.net/aktuell/feuilleton/debatten/open-access-eine-heimliche-technokratische-machtergreifung-1775488.html}}
  {[}Stand: 11.11.2016{]}; - Roland Reuß: Autoren- und Urheberrechte.
  Eine Kriegserklärung an das Buch. In: FAZ, 13.10.2015, online unter
  \href{http://www.faz.net/aktuell/feuilleton/debatten/roland-reuss-ueber-autoren-und-urheberrechte-13852733.html}{\emph{http://www.faz.net/aktuell/feuilleton/debatten/roland-reuss-ueber-autoren-und-urheberrechte-13852733.html}}
  {[}Stand: 11.11.2016{]}; - Roland Reuß: Open Access. Der Geist gehört
  dem Staat. In: FAZ, 30.12.2015, online unter
  \href{http://www.faz.net/aktuell/feuilleton/forschung-und-lehre/baden-wuerttemberg-entrechtet-seine-wissenschaftlichen-autoren-13988149.html}{\emph{http://www.faz.net/aktuell/feuilleton/forschung-und-lehre/baden-wuerttemberg-entrechtet-seine-wissenschaftlichen-autoren-13988149.html}}
  {[}Stand: 11.11.2016{]}; - Roland Reuß: Reform des Urheberrechts. Was
  freie Autoren brauchen. In: FAZ, 13.4.2016, online unter
  \href{http://www.faz.net/aktuell/feuilleton/forschung-und-lehre/urheberrecht-freiheit-fuer-die-wissenschaft-14173320.html}{\emph{http://www.faz.net/aktuell/feuilleton/forschung-und-lehre/urheberrecht-freiheit-fuer-die-wissenschaft-14173320.html}}
  {[}Stand: 11.11.2016{]}.

  Vgl. u.a. Hagner, Zur Sache des Buches (Anm. 2), S. 92.} könnte man --
angesichts seines scharfen Plädoyers für das Erkenntnisprimat von
Druck\-er\-zeug\-nissen -- auch als performativen Selbstwiderspruch bewerten.
Reuß' Vergleiche legen also nahe, dass die digitalen Kräfte, die das
Medium Buch relativieren, eine faschistische, barbarische Gefahr seien,
gegen die nur die religiöse Kraft des Mediums Buch als Antidot wirke,
schließlich sei das Buch ein \enquote{\emph{magischer} Quader innerer
Sammlung und nicht einfach ein gleichgültiger, vorübergehender
Intermediär eines Transfers.}\footnote{Reuß, Ende der Hypnose (Anm. 23),
  S. 104f.} Diese \enquote{magische} Qualität des Buches sei gebunden an
seine materielle Einmaligkeit, seine ausgewählte Typographie und seinen
gestalteten Umschlag: \enquote{Das Buch bildet \emph{seinen eigenen
Kontext} aus, und je mehr Parameter der äußeren Gestalt bei seiner
Herstellung reflektiert wurden, desto adäquater ist in und an ihm
\emph{sein} individueller Inhalt präsent und wahrnehmbar.}\footnote{Ebd.,
  S. 88.}

Im Gegensatz zu dieser reflektierten und stabilen Gestaltung eines
Buches erscheinen für Reuß die digitalen Texte des Computers und vor
allem des Internets als instabil, kontextlos und profan: Während das
Buch \enquote{eine scharf umrissene Grenze gegenüber \enquote{außen}}
ziehe, seien die Online-Leser \enquote{immer schon mittendrin in der
Welt der Geschäfte, der \enquote{Kommunikation}, der Versteigerungen,
der \enquote{Freundeskreise}, der Nachrichtenkanäle, der Musik- und
Video-\enquote{Plattformen}.}\footnote{Ebd., S. 84.} Die mediale
\enquote{Natur} digitaler Medien und vor allem des Internets
verunmögliche folglich Muße und Reflexion, daher setze \enquote{geistige
Erfahrung} voraus, \enquote{daß der Stecker gezogen, die Verstrickung
ins \enquote{Netz} gelöst wird und man für die notwendige Phase des
Atemholens sich aus der Reiz-/Reaktionskette der Vernetzten befreit
{[}\ldots{}{]}.}\footnote{Ebd., S. 96.}

In etwas anderer Weise hierarchisiert Michael Hagner den Nutzen
digitaler versus gedruckter Veröffentlichungen für die
Geisteswissenschaften. Zunächst erklärt er -- in Abgrenzung von den
Naturwissenschaften mit ihren Experimenten, Tabellen und Statistiken --
die Geisteswissenschaften pauschal zu rein sprachbasierten
Wissenschaften (wodurch beispielsweise die Digital Humanities und ihre
quantitativen Methoden ausgegrenzt werden). In einem Rückgriff auf
Walter Benjamin erklärt er dann die Sprache zum \enquote{Skelett des
Gedankens}, der sich letztlich nur im Buch angemessen formulieren ließe,
da nur das Buch Schrift, Stil und Gedanken adäquat verschränke. Es sei
daher nicht erstrebenswert, \enquote{für das Netz zu schreiben}, sondern
\enquote{{[}e{]}s geht darum, das Netz zu nutzen, um bessere Texte für
das Papier zu schreiben.}\footnote{Hagner, Zur Sache des Buches (Anm.
  2), S. 247.}

Die zentralen Gegner des Open Access wie Michael Hagner und Roland Reuß
wollen also viel mehr als nur eine Kritik einer Open Science und der
Open Scholarship. Sie kämpfen schon seit knapp zwei Dekaden\footnote{Richard
  Kämmerlings verweist schon 1998 in einem Konferenzbericht darauf, dass
  Roland Reuß auf \enquote{die praktischen Probleme einer
  Online-Edition} hingewiesen habe und auf das grundsätzliche Problem,
  dass die \enquote{im Netz verwendeten Sprachen {[}\ldots{}{]} keine
  \enquote{standgenaue Übertragung} von Dokumenten {[}ermöglichen{]}, da
  je nach Einstellung des Browsers Texte unterschiedlich dargestellt
  werden. Als Medium für wissenschaftliche Editionen sei das Buch
  unentbehrlich}. Vgl. Richard Kämmerlings: Lesesaal, Gedächtnisort,
  Datenraum. Der Standort der Bücher: Auf dem Weg zur hybriden
  Bibliothek. In: Frankfurter Allgemeine Zeitung, 9.10.1998, Nr. 234, S.
  46.} für das medientheoretische Apriori, dass das Buch mehr als ein
Übertragungsmedium sei, und zwar ein \enquote{Denkkörper} (Hagner)
beziehungsweise ein \enquote{magischer Gegenstand} (Reuß), während die
Online-Medien die Konzentration und Reflexion eher störten denn
beförderten. Eine solche scharfe Differenzierung von \enquote{sakralem
Buch} einerseits und \enquote{profanen Digitalmedien} andererseits wird
in den neueren Arbeiten der Buchwissenschaft auf nüchterne Weise
eingeebnet. Svenja Hagenhoff beschreibt beide Seiten neutral als
\enquote{\emph{Lesemedien}} mit unterschiedlichen Eigenschaften, die
jedoch in keine Hierarchie zu bringen seien. Vielmehr verweist Hagenhoff
darauf, dass ein Lesemedium \enquote{nur ein Element in einem komplexen
Kommunikationssystem ist}, zu dem auch \enquote{die Beschaffenheit
technischer, organisatorischer und logistischer Infrastrukturen} gehört,
weshalb die Wertbestimmung eines Lesemediums für sich alleine
\enquote{eher unnütz} sei.\footnote{Svenja Hagenhoff:
  Buch/Buchsachgruppen. In: Jan Krone/Tassilo Pellegrini (Hg.): Handbuch
  Medienökonomie. Wiesbaden: Springer Fachmedien 2016 , hier S. 7, 10.
  DOI: 10.1007/978-3-658-09632-8\_30-1.}

Die problematische Priorisierung und Sakralisierung des
\enquote{magischen Erkenntniskörpers Buch} durch die
Open-Access-Kritiker muss nun aber nicht heißen, dass deren Argumente
nur vernachlässigbare Scheingefechte in einem Medienkrieg seien.
Vielmehr kann es sich lohnen, die wichtigsten Argumente gegen Open
Access zu betrachten, um auch die Verfahren des Open Access kritisch zu
prüfen. Zwar beteiligt sich Roland Reuß regelmäßig mit Zeitungsartikeln
in der FAZ und anderen Medien an dieser Debatte, seine Beiträge fallen
in der Regel jedoch besonders polemisch aus, weshalb im Folgende die
wichtigsten Argumenten von Michael Hagner gegen Open Access aufgeführt
werden sollen.

Im \emph{Heidelberger Appell} und auch von Hagner und Reuß wird vor
allem argumentiert, dass die zunehmende Verpflichtung der Forscher auf
Open-Access-Veröffentlichungen die grundgesetzlich geschützte Freiheit
der Forschung beschränke, zu der auch die freie Wahl der
Veröffentlichungsform gehört. Hagner verweist auf Beispiele aus dem
anglo-amerikanischen Raum und der Schweiz, in denen WissenschaftlerInnen
sanktioniert werden, wenn sie ihre Texte nicht offen verfügbar
machen.\footnote{Vgl. u.a. Hagner, Zur Sache des Buches (Anm. 2), S. 92.}
Eine solche staatliche Regulierung des Veröffentlichens habe es zuletzt
in der höfischen Gesellschaft gegeben.\footnote{Ebd., S. 96.} Vor allem
stört Hagner, dass die erweiterten Veröffentlichungsmöglichkeiten ihren
Reiz verlören, \enquote{wenn Politiker, Lehr- und
Forschungsinstitutionen oder forschungsfinanzierende Organisationen
Zwangsmaßnahmen ergreifen}.\footnote{Ebd.} Reuß gewichtet das Verhältnis
der AutorInnen zu ihren Verlagen und dieses Betreuungsverhältnis als
besonders wichtig für die Produktion und Rezeption eines Werks:
\enquote{Jeder Versuch, auf dieses vertrauensvolle und schützenswerte
Verhältnis zwischen Verlag und Autor von außen einen dirigistischen
Einfluß zu nehmen, zeichnet deshalb auch immer Brutalität gegenüber dem
Faktum der innigen Beziehung zwischen Urheber und Produkt einer
geistigen Schöpfung aus.}\footnote{Roland Reuß: Autorverantwortung und
  Text. In: Roland Reuß/Volker Rieble (Hg.): Autorherrschaft als
  Werkherrschaft in digitaler Zeit. Symposium Frankfurt 15. Juli 2009.
  Frankfurt am Main: Klostermann 2009, S. 9-20, hier S. 13.} Im
Online-Veröffentlichen wittert Reuß den \enquote{Tod der
wissenschaftlichen Verlagsbranche}, aus dem wiederum der Tod
\enquote{des pluralistischen Publizierens}\footnote{Ebd., S. 14.}
resultieren werde. Diesen Argumenten kann man jedoch entgegnen, dass
noch vor wenigen Dekaden ein ähnlicher (Karriere-)Druck auf
WissenschaftlerInnen lastete, auf jeden Fall ihre Texte von
Wissenschaftsverlagen drucken und veröffentlichen zu lassen und dabei
(teilweise hohe und eigenfinanzierte) Druckkostenzuschüsse zu bezahlen,
und dass dieses Verfahren kaum als massiver Eingriff in die
Forschungsfreiheit gewertet wurde. Zudem beanspruchen auch digitale
Wissenschaftsveröffentlichungen Dienstleistungen von Verlagen (Lektorat,
Satz, Werbung, Hosting), weshalb die Verlagsbranche nicht komplett
verschwinden, ihre Funktionen sich allerdings modifizieren.

Relevanter erscheint Hagners Kritik an der Qualitätssicherungs- und
Archivierungspraxis von Open-Access-Plattformen, denn tatsächlich wird
es digital wesentlich einfacher, eine undifferenzierte Masse an
wissenschaftlichen Veröffentlichungen verfügbar zu machen, weshalb
Hagner prophezeit, \enquote{daß unzählige wissenschaftliche Artikel im
Nirwana des Netzes nachhaltiger verschwinden als ihre auf Papier
gedruckten Vorgänger in den Bibliotheksregalen.}\footnote{Hagner, Zur
  Sache des Buches (Anm. 2), S. 72.} Auch die Qualität von
Open-Access-Zeitschriften sei meist enttäuschend, weshalb renommierte
WissenschaftlerInnen in solchen Zeitschriften nicht
veröffentlichten.\footnote{Vgl. ebd., S. 101f., 104.} Schließlich ist
Reuß natürlich zuzustimmen, dass AutorInnen \enquote{aus ethischen
Gründen {[}\ldots{}{]} etwas dagegen haben} können, ihre Werke
\enquote{auf einer durch Werbeeinnahmen finanzierten Plattform
wiederzufinden (weil das schon an sich etwas Herabziehendes
hat).}\footnote{Reuß, Autorverantwortung und Text (Anm. 39), S. 15.}
Reuß bezieht sich hier auf die Angebote von \emph{GoogleBooks}, aber man
kann genereller konstatieren, dass viele digitale Veröffentlichungen von
WissenschaftlerInnen auf kommerziellen Plattformen veröffentlicht
werden, die eine problematische Gestaltung nutzen, die Teil ihres
jeweiligen Geschäftsmodells ist. Diese Postulat nach einer
wissenschaftlichen Qualitätssicherung im Bereich der
Open-Access-Plattformen und einer forschungs- und lesefreundlichen
Webumgebung wird von den jüngeren Pro-Open-Access-Manifesten
aufgenommen; zudem gilt es natürlich auch für gedruckte
Veröffentlichungen.

Reuß beklagt zudem massiv, dass ökonomische Motive zur einer
\enquote{Missachtung} der Autorenrechte geführt hätten und Open Access
der \enquote{verzweifelte Befreiungsschlag einer unterfinanzierten
Bibliothekenszene} sei. Hinter den Geschäftsmodellen des Open Access, so
Reuß, verbürgen sich \enquote{Geschäftsmodelljodler}, die \enquote{ein
unstillbar großes Interesse daran haben, mit der Arbeit anderer Geld zu
verdienen.}\footnote{Ebd., S. 17f.} Tatsächlich ist es für eine
gesellschaftlich und wissenschaftlich verantwortungsvolle Praxis des
Open Access wichtig, dass die Veröffentlichungskosten fair und
ausgeglichen bleiben. Dies ist momentan, insbesondere bei
internationalen Verlagen mit großen Repositorien, nur unzureichend
gewährleistet.

Schließlich ruft Hagner noch praktische Probleme von (offenen) digitalen
wissenschaftlichen Veröffentlichungen auf. Dazu zählt seine Invektive
gegen Open Data, denn durch die Bereitstellung \enquote{von Big Data}
ergäben sich \enquote{gravierende ethische und rechtliche
Probleme}.\footnote{Ebd., S. 112.} Zudem widersprächen sich
international die rechtlichen Vorgaben für Open-Access-Publikationen und
die Vertragsmodelle der Verlage, weshalb konkrete Veröffentlichungen zu
absurden Widersprüchen führen könnten.\footnote{Vgl. ebd., S. 86f.}
Schließlich verhielten sich die verschiedenen Gruppen, die am
wissenschaftlichen Buchmarkt partizipieren, also WissenschaftlerInnen,
Verlage und BibliothekarInnen, höchst widersprüchlich: \enquote{Eine
kontinuierlich wachsende Mehrheit aller Wissenschaftler befürwortet OA,
ist bei der Umsetzung allerdings eher träge. Nur eine kleine Gruppe von
OA-Aktivisten unter den Wissenschaftlern dringt auf die zwangsweise
Durchsetzung eines nicht-kommerziellen akademischen Publikationssystems.
Die Verlage vertreten den goldenen Weg und akzeptieren den grünen Weg
nur so lange, wie er nicht zu Abbestellungen ihrer Zeitschriften führt.
Bibliothekare dagegen favorisieren den grünen Weg beziehungsweise ein
Mischmodell aus grün und golden, weil sie ihr neu definiertes
Aufgabenspektrum {[}\ldots{}{]} nur dann realisieren können, wenn das
Modell des Universitätsservers sich durchsetzt}.\footnote{Ebd., S. 121.}
Tatsächlich können solche widersprüchlichen und praktischen Hürden zu
Problemen bei der Realisierung von Open-Access-Veröffentlichungen
führen. Daher ist es wichtig, im Bewusstsein um diese Probleme mit den
praktischen Hürden des Open Access umzugehen.

\section*{Offene Wissenschaft in der Praxis: Das
Zweitveröffentlichungsrecht im
Selbstversuch}\label{offene-wissenschaft-in-der-praxis-das-zweitveruxf6ffentlichungsrecht-im-selbstversuch}

Was aber müssen WissenschaftlerInnen heute beim Veröffentlichen von
wissenschaftlichen Publikationen im Sinne des Open Access und bei der
Nutzung des Zweitveröffentlichungsrechts konkret beachten? Wir müssen
zunächst die beiden Formen einer Open-Access-Veröffentlichung
unterscheiden, den goldenen und den grünen Weg.\footnote{Teilweise wird
  auch noch ein dritter, der \emph{graue Weg} genannt. Dieser Begriff
  soll die Veröffentlichung sog. \enquote{grauer Literatur} im Sinne des
  Open Access bezeichnen, er ist jedoch für unseren Erkenntnisgang nicht
  relevant.} Der \emph{goldene Weg} beschreibt die Erstveröffentlichung
eines Beitrags in einem Medium oder einer Form, die von Anfang an eine
offene Verfügbarkeit garantiert, zum Beispiel in einem
Open-Access-Journal. Es kann sein, dass für den Prozess der
Veröffentlichung, Begutachtung und Einrichtung einer solchen
Veröffentlichung eine zusätzliche Zahlung an einen Verlag oder eine
Institution, die die entsprechende Plattform pflegt, notwendig wird.
Eine solche \enquote{goldene} Open Access-Veröffentlichung
\enquote{erscheint aus urheberrechtlicher Sicht vollständig
konfliktfrei}, so der Rechtswissenschaftler Sebastian Krujatz in seiner
Arbeit über Open Access.\footnote{Sebastian Krujatz: Open Access. Der
  offene Zugang zu wissenschaftlichen Informationen und die ökonomische
  Bedeutung urheberrechtlicher Ausschlussmacht für die wissenschaftliche
  Informationsversorgung. Tübingen: Mohr Siebeck 2012, S. 329.}

Anders stellt dies bei den hybriden Veröffentlichungen des \emph{grünen
Wegs} dar,\footnote{Ich danke Eric Steinhauer für seine kritischen
  Hinweise, die mir geholfen haben, die folgenden sieben Absätze -- im
  Vergleich zur Erstveröffentlichung -- zu modifizieren und zu
  präzisieren.} also bei Veröffentlichungen, die einerseits als
\emph{Erstveröffentlichung} bei Verlagen oder in Zeitschriften gedruckt
veröffentlicht worden sind und andererseits zusätzlich als digitale
\emph{Zweitveröffentlichung} im Sinne der Selbstarchivierung auf der
eigenen Webseite oder einem institutionellen Dokumentenserver frei
verfügbar gemacht werden. Hier sind grundsätzlich zwei Situationen
denkbar: Entweder wird bei der Erstveröffentlichung kein Vertrag
zwischen AutorIn und Verlag geschlossen oder eben doch. Wenn kein
separater Vertrag geschlossen wird, garantiert der § 38, die Absätze 1
und 2, die digitale Zweitveröffentlichung nach einem Jahr:

\begin{quote}
Gestattet der Urheber die Aufnahme des Werkes in eine {[}\ldots{}{]}
Sammlung, so erwirbt der Verleger oder Herausgeber im Zweifel ein
ausschließliches Nutzungsrecht zur Vervielfältigung, Verbreitung und
öffentlichen Zugänglichmachung. Jedoch darf der Urheber das Werk nach
Ablauf eines Jahres seit Erscheinen anderweit vervielfältigen,
verbreiten und öffentlich zugänglich machen, wenn nichts anderes
vereinbart ist.\footnote{Bundesministerium der Justiz und für
  Verbraucherschutz: Gesetz über Urheberrecht und verwandte Schutzrechte
  (Urheberrechtsgesetz): § 38 Beiträge zu Sammlungen, online unter
  \href{https://www.gesetze-im-internet.de/urhg/__38.html}{\emph{https://www.gesetze-im-internet.de/urhg/\_\_38.html}}
  {[}Stand: 11.11.2016{]}. Absatz 1 formuliert das
  Zweitveröffentlichungsrecht für Werke in periodisch erscheinenden
  Sammlungen, Absatz 2 für Werke in nicht periodisch erscheinenden
  Sammlungen.}
\end{quote}

Dieses grundsätzliche Zweitveröffentlichungsrecht gilt seit der
Einführung des Urheberrechts am 1. Januar 1966 und geht zurück auf den §
42 des vorherigen Verlagsgesetzes. Mit der Novelle des
Urheberrechtsgesetzes von 2013 (\enquote{Dritter Korb}) wurde allerdings
die \enquote{öffentliche Zugänglichmachung}, also die Möglichkeit zur
digitalen Zweitveröffentlichung, explizit ergänzt, außerdem wurde ein
zusätzliches unabdingbares Zweitveröffentlichungsrecht für
Veröffentlichungen von in Drittmittelprojekten angestellten
WissenschaftlerInnen in Periodika eingeführt, das in jedem Fall alle
verlagsvertraglichen Vereinbarung bricht (§ 38, Absatz 4).

Im Regelfall haben die Wissenschaftsverlage schon in der Vergangenheit
versucht, sich durch Verlagsverträge mit ihren AutorInnen das
ausschließliche Nutzungsrecht einräumen zu lassen, das trotz des neu
eingeführten Zweitveröffentlichungsrecht für die oben genannten
spezifischen Fälle noch immer für die Mehrheit aller wissenschaftlichen
Aufsätze gelten würde. Für diese Fälle haben der Börsenverein des
deutschen Buchhandels und der Deutsche Hochschulverband im Jahre 2000
gemeinsam Vertragsnormen für wissenschaftliche Werke vorgelegt, die
einseitig zu Gunsten der Verlage geklärt werden, wie Krujatz 2012
betont: Diese Vertragsnormen \enquote{sehen in der Regel eine
ausschließliche Nutzungsrechtseinräumung vor, welche eine
Zweitveröffentlichung des wissenschaftlichen Beitrags durch den Urheber
oder einen Dritten verbietet.} In der Praxis wird diese
Ausschließlichkeit jedoch nicht immer konsequent durchgesetzt, wie
Krujatz kontastiert: \enquote{Jedoch hat sich in der relevanten
internationalen Verlagslandschaft eine vielfältige, von den genannten
Vertragsnormen abweichende Vertragspraxis herausgebildet.}\footnote{Ebd.}

Vor diesen Hintergründen versuchte die interdisziplinäre \emph{AG
Potenziale digitaler Medien in der Wissenschaft} in der Global Young
Faculty III in einem Praxisprojekt die Frage zu klären, ob und, wenn ja,
wie ihre geistes- und gesellschaftswissenschaftlichen
Veröffentlichungen, die bei deutschen Verlagen veröffentlicht wurden und
durch einen Verlagsvertrag geschützt sind, im Sinne des
Zweitveröffentlichungsrechts verfügbar zu machen wären.\footnote{Die AG
  wurde bei der Projektarbeit in Rechtsfragen unterstützt von John H.
  Weitzmann und Matthias Spielkamp vom \emph{iRights.Lab}, finanziell
  und logistisch vom Mercator Research Center Ruhr und in der
  praktischen Umsetzung von Kristina Petzold.} Das erste Ergebnis des
Projekts ist die Broschüre \emph{Zweitveröffentlichungsrecht für
Wissenschaftler: Geltende Rechtslage und Handlungsempfehlungen} von
Matthias Spielkamp von iRights.Lab,\footnote{Vgl. Matthias Spielkamp:
  Zweitveröffentlichungsrecht für Wissenschaftler: Geltende Rechtslage
  und Handlungsempfehlungen. Berlin: iRights.Lab 2015, online seit dem
  27.4.2015 unter
  \href{http://irights-lab.de/assets/Uploads/Documents/Publications/zweitveroeffentlichungsrecht-20150425.pdf}{\emph{http://irights-lab.de/assets/Uploads/Documents/Publications/zweitveroeffentlichungsrecht-20150425.pdf}}
  {[}Stand: 11.11.2016{]}.} in der Spielkamp für \emph{noch
unveröffentlichte Werke} empfiehlt, in etwaigen Verlagsverträgen
Streichungen und/oder Ergänzungen vorzunehmen, um sich das
Zweitveröffentlichungsrecht an der eigenen Publikation gegebenenfalls
schon parallel zur Erstveröffentlichung zu sichern -- wenn man nicht
ohnehin das in § 38, Absatz 1 und 2, des Urheberrechts garantierte
Zweitveröffentlichungsrecht in Anspruch nehmen will. Für etwaige
Vertragsverhandlungen mit Verlagen empfiehlt iRights.lab als Idealfall
erstens, in Verlagsverträgen die komplette Abgabe aller Rechte
durchzustreichen sowie zweitens den folgenden, 2006 von Reto Mantz
formulierten, Zusatz zu ergänzen:

\begin{quote}
Der Urheber erteilt dem Verlag für die elektronische Publikation nur ein
einfaches Nutzungsrecht. Er behält sich vor, das Werk unter eine Open
Access-Lizenz, z. B. die \enquote{Digital Peer Publishing License} zu
stellen, die die elektronische Verbreitung gestattet.\footnote{Reto
  Mantz: Open Access-Lizenzen und Rechtsübertragung bei Open
  Access-Werken. In: Gerald Spindler (Hg.): Rechtliche Rahmenbedingungen
  von Open Access-Publikationen, Göttinger Schriften zur
  Internetforschung. Band 2. Göttingen: Universitätsverlag 2006, S.
  55--103, hier S. 103.}
\end{quote}

Neben diesen Empfehlungen für künftige Veröffentlichungen definiert
Spielkamp auch Wege, das Zweitveröffentlichungsrecht für \emph{bereits
veröffentlichte Werke} zu erhalten, die -- je nach Veröffentlichungsort,
-zeit und Vertragslage -- mit oder ohne Rücksprache mit dem Verlag
möglich sind.\footnote{Vgl. Spielkamp, Zweitveröffentlichungsrecht für
  Wissenschaftler (Anm. 48), S. 7--9.} Wie aber gestalten sich solche
Verhandlungen mit Verlagen, denen die ausschließlichen Nutzungstexte an
einer Publikation übertragen wurden? Zur Klärung dieser Frage, die nicht
den Status einer repräsentativen empirischen Untersuchung beanspruchen
kann, sondern eher als Stichprobe aus dem Bereich der Geistes- und
Gesellschaftswissenschaften zu bewerten ist, beteiligten sich drei
Mitglieder der \emph{AG Potenziale Digitaler Medien in der Wissenschaft}
in der Global Young Faculty III an einem Experiment. Insgesamt wurden
sechzehn kleine, mittlere und große Verlage mit einer für die jeweilige
Konstellation modifizierten Email kontaktiert, um das
Zweitveröffentlichungsrecht für bereits veröffentlichte geistes-
beziehungsweise gesellschaftswissenschaftliche Aufsätze zu erhalten, bei
denen das ausschließliche Nutzungsrecht bei den Verlagen lag.\footnote{Zum
  Gegenstand dieses Projekts wurden Veröffentlichungen von Mitgliedern
  der Global Young Faculty III, bei denen die AutorInnen entweder in
  einem Verlagsvertrag dem Verlag das ausschließliche Nutzungsrecht
  übertragen hatten oder aber nicht mehr genau wussten, ob ein solcher
  Verlagsvertrag vorliegt.

  Der Grundtext der Emails an die Verlage wurde mit dem iRights.Lab
  abgestimmt und liest sich wie folgt: Betreff: Zweitveröffentlichung
  meines Aufsatzes // Sehr geehrte Damen und Herren, // im Jahr {[}X{]}
  erschien bei Ihnen mein Aufsatz \enquote{{[}X{]}} in dem Sammelband
  \enquote{{[}X{]}} (hrsg. von {[}X{]}). / Wie Sie sicherlich wissen,
  ist es für Nachwuchswissenschaftlerinnen und -wissenschaftler
  inzwischen zumindest faktisch (und teils sogar rechtlich)
  obligatorisch, Neuveröffentlichungen für die übrige Wissenschaftswelt
  möglichst frei zugänglich zu machen. Der deutsche Gesetzgeber hat 2014
  daher ein unabdingbares Zweitveröffentlichungsrecht für
  wissenschaftliche Publikationen ein Jahr nach ihrem Erscheinen
  eingeführt. Ich würde meinen Aufsatz daher gern im Wege des Open
  Access gemäß Berliner Erklärung von 2003 frei zugänglich machen. / Mir
  liegt momentan die Vereinbarung mit Ihnen nicht schriftlich vor, aber
  ich gehe davon aus, dass es mir ganz im Geiste der neuen gesetzlichen
  Regelung gestattet ist, die Redigatsfassung meines Artikels ab {[}X{]}
  {[}auf der Website unseres Instituts \textbar{} im eigenen Namen unter
  der f-DPPL der jeweils neuesten Version \textbar{} im eigenen Namen
  unter der Creative-Commons-Lizenz BY-SA der jeweils neuesten Version
  \textbar{} im eigenen Namen unter der Creative-Commons-Lizenz BY der
  jeweils neuesten Version{]} online zu stellen. / Ich möchte Sie
  außerdem darauf hinweisen, dass ich diese Anfrage im Rahmen einer
  Studie der Global Young Faculty
  {[}\href{http://global-young-faculty.de/}{\emph{http://global-young-faculty.de/}}{]}
  stelle, die sich einen Überblick über die aktuellen Leitlinien und
  Vorgehensweisen von Verlagen in Bezug auf das
  Zweitveröffentlichungsrecht verschaffen will. Wir würden uns daher
  vorbehalten, Ihre Antwort (wenn Sie es entsprechend vermerken auch
  anonymisiert) im Rahmen der Studie zu verwenden und öffentlich zu
  machen. Gerne können Sie uns daher auch ganz allgemein den Standpunkt
  und die Strategie Ihres Verlages zur (späteren) freien Zugänglichkeit
  wissenschaftlicher Texte kurz erläutern. / Bei Rückfragen wenden Sie
  sich gern an {[}X{]}. // Mit freundlichen Grüßen //
  {[}Unterschrift{]}.}

Die Ergebnisse dieser Stichprobe waren teilweise erwartbar, teilweise
überraschend:\footnote{Das Datenmaterial der Untersuchung kann beim
  Verfasser dieses Aufsatzes angefragt werden, es ist teilweise
  anonymisiert, weil zwei Verlage dies erwünscht haben.} Es war
erwartbar, dass nicht alle Verlage sofort antworten würden, dies trifft
für fünf der sechzehn Verlage zu, darunter sind mit dem Reclam Verlag
und dem VS Verlag (Springer) zwei größere Verlage. Eher unerwartet ist,
dass die elf Antworten durchweg dem Wunsch entsprechen, die jeweilige
Publikation zweitveröffentlichen zu dürfen, allerdings in
unterschiedlicher Weise: Sechs Verlage antworten ganz kurz, dies sei
überhaupt kein Problem und bedingungslos möglich. Allerdings wird
zumeist der Verweis auf die Erstveröffentlichung zur Bedingung gemacht.
Zu diesen sechs Verlagen zählen Wissenschaftsverlage unterschiedlicher
Größe: der Aisthesis Verlag (Antwort von Detlev Kopp), die Pabst Science
Publishers (Wolfgang Pabst), der Peter Lang Verlag (Annette Reese; das
Zweitveröffentlichungsrecht wird allerdings nur für die
Pre-Publication-Version zugestanden), der Schneider Verlag Hohengehren
(Ulrich Schneider) und der Verlag Westfälisches Dampfboot (Günter
Thien), ein Verlag bat um Anonymisierung seiner Antwort.\footnote{Hier
  ist die passende Stelle, um auch dem Wilhelm Fink Verlag (Paderborn)
  und seinen MitarbeiterInnen, insbesondere Andreas Knop und Mechthild
  Vogt, für die unkomplizierte Erlaubnis, diese überarbeitete Version
  meines Textes digital verfügbar zu machen, zu danken.} Vier eher
kleine Wissenschaftsverlage entsprachen einer Bitte der Anfrage und
erörterten zunächst, weshalb das Zweitveröffentlichung grundsätzlich
problematisch und für die konkrete Verlagsarbeit bedrohlich sei, um dann
-- als Ausnahme -- doch das Zweitveröffentlichungsrecht zuzugestehen. In
einem elften Fall wurde das Zweitveröffentlichungsrecht ebenfalls
zugestanden, allerdings war die Antwort komplexer, denn sie enthielt --
unbeabsichtigt -- gleich sechs E-Mails, die in dem mittelgroßen Verlag
zwischen verschiedenen Abteilungen hin und her ging, um die
Zuständigkeit für die Anfrage zu klären und das Verfahren zu definieren.

\section*{Link fixed: Was von der Kritik des Open Access
bleibt}\label{link-fixed-was-von-der-kritik-des-open-access-bleibt}

Dieser Beitrag interessiert sich für den digitalen Medienwandel und
seine Folgen für die Veröffentlichungspraxis in den
Geisteswissenschaften. In den letzten Dekaden sorgte die Digitalisierung
für eine enorme Beschleunigung und Intensivierung der
Wissenschaftskommunikation. Zudem haben sich in den letzten anderthalb
Dekaden verschiedene digitale Veröffentlichungsformen in der
Wissenschaft differenziert und etabliert. Während noch im Jahr 2002
Manifeste für die Open Science eher wissenschaftspolitische Forderungen
aufstellten, firmieren Open Scholarship und Open Access inzwischen als
Standards für viele Forschungspraxen.

Die Kritik an Open Access ist jedoch weiterhin intensiv und kulminierte
2009 im \emph{Heidelberger Appell}. Dieser Beitrag skizzierte
Diskurspositionen der Kritiker am Beispiel von Michael Hagner, Lothar
Müller und Roland Reuß und zeigte, dass sie von einer
medientheoretischen Überhöhung, einer Sakralisierung des
\enquote{magischen Erkenntniskörpers Buch} ausgehen, während sich
digitale Medien entweder noch nicht final durchgesetzt hätten (Müller)
oder aber für Erkenntnis- und Reflexionsprozesse ungeeignet seien
(Hagner, Reuß).

Neben diesen bibliophilen Intentionen werden andere Kritikpunkte
sichtbar, die auch von den VerfechterInnen des digitalen
Open-Access-Publizierens diskutiert werden sollten. Es ist eine wichtige
Aufgabe für GeisteswissenschaftlerInnen, die sich für Open Scholarship
und Open Access einsetzen, sich dieser Kritik in ihrer
Veröffentlichungspraxis zu stellen. Dazu gehören die Absicherung von
Qualitätsstandards, die zielgerichtete Nutzung hybrider
Publikationsmodelle, der Aufbau nachhaltiger öffentlicher
Infrastrukturen und Repositorien, die Inanspruchnahme spezifischer
Verlagsdienstleistungen sowie die Entwicklung einer verantwortungsvollen
Preisstruktur für Open Access-Publikationen und internationaler
Standardisierungen und Vereinheitlichungen.

In der gegenwärtigen medialen Transformationsphase sucht das
Zweitveröffentlichungsrecht einen Ausgleich zwischen den (vermeintlich)
gegenläufigen Interessen der WissenschaftlerInnen und der Verlage.
Allerdings erfordert die rechtliche Situation im Regelfall noch immer
Verhandlungen zwischen den WissenschaftlerInnen und ihrem jeweiligen
Verlag. Eine Stichprobe zeigte, dass geistes- und
gesellschaftswissenschaftliche Verlage überraschend permissiv auf die
Bitte reagieren, WissenschaftlerInnen das Zweitveröffentlichungsrecht an
ihren Aufsätzen zuzugestehen, selbst wenn die Verlage vertraglich das
ausschließliche Nutzungsrecht besitzen.

Vilém Flusser sah bereits 1987 die digitalen Codes in einem intensiven
Kampf gegen die Buchstaben und prophezeite einen schnellen Sieg der
Digitalisierung: \enquote{Den Texten ist erst nach dreitausendjährigem
Kampf, erst im 18. Jahrhundert der Aufklärung gelungen, die Bilder und
ihre magischen Mythen in Winkel wie Museen und das Unterbewußtsein zu
drängen. Der gegenwärtige Kampf wird nicht so lange währen. Das digitale
Denken wird weit schneller siegen.} Allerdings weist er darauf hin, dass
das 20. Jahrhundert auch von einem Aufstand der Bilder gegen die Schrift
geprägt sei. Daher stellt er die Frage: \enquote{Dürfen wir in
unvorhersehbarer Zukunft mit einem reaktionären Aufstand der verdrängten
Texte gegen die Computerprogramme rechnen?}\footnote{Vilém Flusser: Die
  Schrift. Hat Schreiben Zukunft? Göttingen: European Photography 2002
  (OA: 1987), 5., durchges. Aufl., S. 141.} Wie diese Kämpfe aussehen
könnten, deutet sich in den Debatten um Open Access und das
Zweitveröffentlichungsrecht bereits an. Noch befinden wir uns in einer
Phase medialer Transformationen, wovon auch dieser Text zeugt,
\sout{indem
er digitale Potenziale in einer medialen Form reflektiert, die diese
Potenziale nicht fruchtbar zu machen weiß} indem er als digitale
Zweitveröffentlichung in verschiedenen Varianten (html,\footnote{Vgl.
  \href{https://github.com/libreas}{\emph{https://github.com/libreas}}.}
als PDF in einem Repositorium) erscheint. \sout{Broken} Link fixed:
\url{https://www.merkur-zeitschrift.de/2016/10/24/siggenthesen/}.

\emph{Eine überarbeitete Digitalversion von: Thomas Ernst: Wie offen
sollten die Geisteswissenschaften sein? Print- vs.~Digitalpublikationen
und die Debatten um Open Access und das Zweitveröffentlichungsrecht. In:
Thomas Ernst/Georg Mein (Hg.): Literatur als Interdiskurs. Realismus und
Normalismus, Interkulturalität und Intermedialität von der Moderne bis
zur Gegenwart. Festschrift zum 60. Geburtstag von Rolf Parr. München:
Fink 2016, S. 653-667.}\footnote{Ich danke Ben Kaden für sein Interesse,
  diesen in einer gedruckten Festschrift erstveröffentlichten Beitrag
  digital und im bibliothekswissenschaftlichen Kontext verfügbar zu
  machen, und für seine hilfreichen Hinweise für die Überarbeitung sowie
  auf passende Zeitungsfundstücke. Zudem danke ich den AutorInnen der
  \#Siggenthesen für ihre wichtigen Anregungen.Weitere Informationen
  dazu:
  \href{https://libreas.wordpress.com/2016/10/24/siggener-thesen-wissenschaftskommunikation/}{\emph{https://libreas.wordpress.com/2016/10/24/siggener-thesen-wissenschaftskommunikation/}}}

%autor
\begin{center}\rule{0.5\linewidth}{\linethickness}\end{center}

\textbf{Dr.~Thomas Ernst} studierte Philosophie und Germanistik in
Duisburg, Berlin, Bochum und Leuven/Belgien und promovierte 2008 an der
Universität Trier zum Thema Pop, Minoritäten, Untergrund. Subversive
Konzepte in der deutschsprachigen Gegenwartsprosa. Seit November 2010
ist er wissenschaftlicher Mitarbeiter an der Universität Duisburg-Essen
im Studiengang ``Literatur und Medienpraxis'', dort schreibt er derzeit
an seinem Habilitationsprojekt zur Geschichte des geistigen Eigentums.

\end{document}
