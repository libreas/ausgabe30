\documentclass[a4paper,
fontsize=11pt,
%headings=small,
oneside,
numbers=noperiodatend,
parskip=half-,
bibliography=totoc,
final
]{scrartcl}

\usepackage{synttree}
\usepackage{graphicx}
\setkeys{Gin}{width=.4\textwidth} %default pics size

\graphicspath{{./plots/}}
\usepackage[ngerman]{babel}
\usepackage[T1]{fontenc}
%\usepackage{amsmath}
\usepackage[utf8x]{inputenc}
\usepackage [hyphens]{url}
\usepackage{booktabs} 
\usepackage[left=2.4cm,right=2.4cm,top=2.3cm,bottom=2cm,includeheadfoot]{geometry}
\usepackage{eurosym}
\usepackage{multirow}
\usepackage[ngerman]{varioref}
\setcapindent{1em}
\renewcommand{\labelitemi}{--}
\usepackage{paralist}
\usepackage{pdfpages}
\usepackage{lscape}
\usepackage{float}
\usepackage{acronym}
\usepackage{eurosym}
\usepackage[babel]{csquotes}
\usepackage{longtable,lscape}
\usepackage{mathpazo}
\usepackage[normalem]{ulem} %emphasize weiterhin kursiv
\usepackage[flushmargin,ragged]{footmisc} % left align footnote

\usepackage{listings}

\urlstyle{same}  % don't use monospace font for urls

\usepackage[fleqn]{amsmath}

%adjust fontsize for part

\usepackage{sectsty}
\partfont{\large}

%Das BibTeX-Zeichen mit \BibTeX setzen:
\def\symbol#1{\char #1\relax}
\def\bsl{{\tt\symbol{'134}}}
\def\BibTeX{{\rm B\kern-.05em{\sc i\kern-.025em b}\kern-.08em
    T\kern-.1667em\lower.7ex\hbox{E}\kern-.125emX}}

\usepackage{fancyhdr}
\fancyhf{}
\pagestyle{fancyplain}
\fancyhead[R]{\thepage}

%meta

\fancyhead[L]{D. Kühnlein \& M. Reichardt \\ %author
LIBREAS. Library Ideas, 30 (2016). % journal, issue, volume.
\href{http://nbn-resolving.de/urn:nbn:de:kobv:11-100244196}{urn:nbn:de:kobv:11-100244196}} % urn
\fancyhead[R]{\thepage} %page number
\fancyfoot[L] {\textit{Creative Commons BY 3.0}} %licence
\fancyfoot[R] {\textit{ISSN: 1860-7950}}

\title{\LARGE{Rahmenbedingungen für das Publizieren eines Open-Access-Journals. Ein Szenario}} 
\author{Dorothea Kühnlein \& Mareen Reichardt \\ \small{unter Mitarbeit von Vivien Damm, Sebastian Grau und Katharina Weile}} %author

\setcounter{page}{136}

\usepackage[colorlinks, linkcolor=black,citecolor=black, urlcolor=blue,
breaklinks= true]{hyperref}

\date{}
\begin{document}

\maketitle
\thispagestyle{fancyplain} 

%abstracts
\begin{abstract}
unter Mitarbeit von Vivien Damm, Sebastian Grau und Katharina Weile

Inspiriert von den Erfahrungen aus dem DFG-Projekt `Future Publications
in den Humanities' werden auf Grundlage eines gegebenen fiktiven
Szenarios für das Publizieren eines geisteswissenschaftlichen
Open-Access-Journals die verschiedenen Strategien und Möglichkeiten des
Open Access erörtert. Als mögliche Option wird zum einen der Grüne Weg
mit der Weiterführung der Printausgabe über den alten Verlag und dem
Open-Access-Publizieren der Artikel auf einem Repositorium
identifiziert. Weiterhin kann der hybride Goldene Weg herausgestellt
werden, mit der Möglichkeit über einen neuen Verlag die Printausgabe um
eine elektronische Parallelausgabe zu erweitern und einzelne Artikel im
Open-Access-Verfahren bereitzustellen. Auch der Goldene Weg im
Selbstverlag wird als Alternative erläutert. Es werden jeweils die Vor-
und Nachteile aller Optionen abgewogen.
\end{abstract}

%body
\section*{Einleitung}\label{einleitung}

Im Sommersemester 2015 wurde im Rahmen des DFG-Projektes \enquote{Future
Publications in den Humanities} am Institut für Bibliotheks- und
Informationswissenschaft der Humboldt-Universität zu Berlin ein
studentisches Projektseminar realisiert. Die Studierenden erarbeiteten
in Projektgruppen bestimmte Themenfelder. Das Ergebnis einer der
Projektgruppen soll in diesem Beitrag vorgestellt werden. Die
Zielsetzung bestand darin, die verschiedenen Varianten des Open Access,
also den freien Zugang zu wissenschaftlichen Volltexten, in Abwägung
ihrer jeweiligen Vor- und Nachteile darzustellen. Den Ausgangspunkt
bildete ein fiktives Publikationsszenario aus den Geisteswissenschaften.
In dem Szenario geht es um einen Herausgeber eines Printjournals,
welcher an der Humboldt Universität zu Berlin arbeitet. Er ist in der
Wissenschaftsgemeinschaft sehr aktiv und hat eine erhebliche Reputation.
Seit längerem wünscht er sich mehr Offenheit gegenüber dem
Open-Access-Verfahren und eine effizientere Qualitätssicherung. Letztere
sollte idealerweise als Peer-Review, statt wie bisher mittels Editorial
Review, durchgeführt werden. Einen größeren finanziellen und
organisatorischen Aufwand möchte der Herausgeber wenn möglich vermeiden.
Sein bisheriger Verlag befindet sich in einer wirtschaftlichen Krise und
möchte diese mit einer Neustrukturierung überwinden. Aus diesem Grund
ist die Perspektive der Zeitschrift bei diesem Verlag ungewiss. Es
besteht jedoch die Möglichkeit, zu einem größeren Verlag zu wechseln und
gleichzeitig auch eine elektronische Parallelausgabe zu veröffentlichen.
Beide Verlage setzen eher auf das bestehende Publikationssystem. Jedoch
ist eine gewisse Verhandlungsbereitschaft zur Realisierung des
Open-Access-Ansatzes vorhanden. Der größere Verlag würde einer
Open-Access-Publikation über den hybriden Goldenen Weg mit sehr hohen
Publikationsgebühren, sogenannten Article Processing Charges (APC),
zustimmen. Der bisherige Verlag ist aufgrund fehlender Ressourcen nicht
in der Lage, eine elektronische Ausgabe bereitzustellen, zeigt sich
jedoch offen gegenüber dem Grünen Weg des Open Access. Darüber hinaus
verlangt der neue Verlag eine weitreichendere Rechteabtretung seitens
der Autoren für Nicht-Open-Access-Artikel. Dem Wunsch nach einem
Peer-Review-Ver\-fahr\-en will der neue Verlag nicht mit eigenen Mitteln
unterstützen, fordert es aber de facto vom Herausgeber ein. Der
bisherige Verlag zeigt sich diesbezüglich offener. Beide Verlage
bestehen zudem auf die Weiterführung der Printausgabe.

Aus dem Szenario ergaben sich folgende Leitfragen:

\begin{enumerate}
\def\labelenumi{\arabic{enumi}.}
\tightlist
\item
  Soll der Herausgeber bei seinem alten Verlag bleiben und weiter eine
  Printausgabe produzieren?
\item
  Soll der Herausgeber zu einem anderen Verlag wechseln und Kompromisse
  eingehen?
\item
  Gibt es andere Möglichkeiten einer Open-Access-Publikation? Wenn ja,
  mit welchen Partnern, welchem Aufwand und unter welchen Bedingungen?
\item
  Braucht die Publikation ein Peer Review und falls ja, wie lässt sich
  ein solches Verfahren organisieren?
\end{enumerate}

Anhand dieser Fragen soll nun erläutert werden, welche
Handlungsmöglichkeiten der Herausgeber für dieses Szenario hat.

\section*{Methodische Beschreibung}\label{methodische-beschreibung}

Zu Beginn der Projektarbeit wurde sich mit den einzelnen Akteuren,
Problemen und Themen des Szenarios auseinandergesetzt. Im Anschluss
wurden die von der Projektleitung vorgegebenen Leitfragen erörtert und
für die weitere Bearbeitung ein Flussdiagramm erstellt, um die möglichen
Handlungsoptionen zu entwickeln und darzustellen. Die konkreten
Handlungsvorschläge wurden unter Zuhilfenahme einschlägiger
Fachliteratur ausgearbeitet und im weiteren Verlauf gegeneinander
abgewogen.

\section*{Analyse}\label{analyse}

Das Publizieren im Open-Access-Verfahren bringt viele Vorzüge mit sich,
ist aber auch einigen Vorbehalten ausgesetzt. Der wohl entscheidendste
Grund für Open Access ist der schnelle und kostenfreie Zugang zu
wissenschaftlichen Informationen und ein damit verbundener,
uneingeschränkter globaler Zugriff auf Forschungsergebnisse, die auf
Basis von öffentlich finanzierter Forschung zustande gekommen sind.
Informationen, die uneingeschränkt und für den Nutzer kostenfrei
zugänglich sind, werden stärker wahrgenommen als Informationen deren
Zugang beschränkt ist. Im Internet sind diese Informationen von
Suchmaschinen und anderen Nachweisdiensten leichter indexierbar.
Suchmaschinen, wie die Bielefeld Academic Search Engine (BASE) und auch
das Directory of Open Access Journals (DOAJ) sind spezialisiert auf das
Indexieren von Open-Access-Dokumenten. Durch die zunehmende Verbreitung
und Sichtbarmachung von Open-Access-Informationen können diese vermehrt
genutzt und weiterverwendet werden. Die freie Verfügbarkeit von
Informationen fördert auch die ländergrenzen- und fachübergreifende
Zusammenarbeit. Auch Staaten, in denen weniger Geld in die Forschung
investiert wird oder werden kann, erhalten so Zugriffsmöglichkeiten auf
relevante Informationen und bereits vorhandenes Wissen. Schließlich kann
auch der Forschungsprozess beschleunigt werden, weil Ergebnisse nicht
nur schneller publiziert, sondern Dokumente zum selben Zeitpunkt von
mehreren Personen bearbeitet oder eingesehen werden können. Zweifel am
Open Access entstehen häufig durch eine Unsicherheit im Umgang mit dem
Urheberrecht sowie durch Vorbehalte gegenüber der Qualität und der
Langzeitarchivierung der Dokumente. Nur mit der richtigen Archivierung
von Informationen im Internet auf entsprechenden Archivservern kann eine
langfristige Erhaltung und Verfügbarkeit der Informationen gewährleistet
werden (Georg-August-Universität Göttingen. Niedersächsische Staats- und
Universitätsbibliothek 2015).

Es gibt verschiedene Ausprägungen des Open Access. Im Folgenden werden
diese Möglichkeiten mit ihren Vor- und Nachteilen erörtert. Dabei wird
konkret auf die bereits vorgestellten Leitfragen eingegangen.

\subsection*{Soll der Herausgeber bei seinem alten Verlag bleiben und
weiter eine Printausgabe
produzieren?}\label{soll-der-herausgeber-bei-seinem-alten-verlag-bleiben-und-weiter-eine-printausgabe-produzieren}

Obgleich sich der bisherige Verlag derzeit in wirtschaftlichen
Schwierigkeiten befindet und eine Neustrukturierung seines Geschäftes
vornehmen muss, stellt der Verbleib der Zeitschrift bei diesem Verlag
dennoch eine attraktive Option dar. Zwar ist die wirtschaftliche
Perspektive des Verlages ungewiss, jedoch veranlasst die derzeitige
Krise den alten Verlag auch dazu, sich neuen Möglichkeiten zu öffnen, um
die renommierten Zeitschrift auch zukünftig an sich zu binden. Für den
Herausgeber ergibt sich die Chance die bisherige Zusammenarbeit in einer
neuen Form fortzusetzen und auch seinem Wunsch nach Open Access
nachzugehen. Da der Verlag aufgrund seiner derzeitigen Schwierigkeiten
allerdings weder personelle noch technische Ressourcen für die
Weiterentwicklung der Zeitschrift in Richtung Open Access zur Verfügung
stellen kann, würde er aber die Öffnung der Zeitschrift für den Grünen
Weg des Open Access unterstützen. Da es sich aus der Perspektive des
Verlages hier nur um eine vertragsrechtliche Änderung handelt, wäre
diese Variante besonders kostengünstig, da lediglich eine entsprechende
Klausel in den Autorenverträgen angepasst werden müsste.

Konkret bedeutet dies, dass in der Zeitschrift veröffentlichte Artikel
als elektronische Dokumente archiviert und so kostenfrei öffentlich
verfügbar gemacht werden können. Prinzipiell sind hier drei Varianten
möglich: Das Individual-, das Institutional- oder das
Central-Self-Archiving. Beim Individual-Self-Archiving erfolgt die
Archivierung und Bereitstellung des Artikels durch den Autor, zum
Beispiel auf einer privaten Homepage. Das Institutional-Self-Archiving
meint die Archivierung auf einem institutionellen Repositorium, das
Central-Self-Archiving die Archivierung auf einem disziplinären
Repositorium (Müller \& Schirmbacher 2007).

Zwar erfolgt die Archivierung beim Grünen Weg in der Regel durch den
Autor selbst, allerdings lässt sich dieser Prozess auch durch den
Herausgeber organisieren, was im vorliegenden Fall auch empfehlenswert
wäre. Dafür sind die Einholung eines einfachen Nutzungsrechts beim Autor
sowie die Zustimmung zu einer Deposit-Licence durch den Autor nötig.
Dieser Prozess ließe sich mit geringem Aufwand durch den Herausgeber
organisieren. Beispielsweise könnte eine entsprechende
Zustimmungspflicht der Autoren zur Veröffentlichung ihrer Artikel auf
einem Repositorium bereits mit dem \emph{Call for Papers} oder auch in
einem Autorenvertrag, kommuniziert werden. Durch dieses Verfahren wäre
sichergestellt, dass jeder in der Printausgabe veröffentlichte Artikel
auch tatsächlich als archiviertes elektronisches Dokument zur Verfügung
stünde. Außerdem wäre zu klären, ob die elektronische Veröffentlichung
der Artikel als Preprint oder als Postprint erfolgen soll. Der
Herausgeber muss zudem - bestenfalls in Absprache mit den Autoren und
dem Verlag - entscheiden, auf welcher Art Repositorium die Artikel
veröffentlicht werden sollen. Mit den Verantwortlichen des
institutionellen Repositoriums der Humboldt-Universität zu Berlin, dem
edoc-Server, wäre zu klären, ob eine Veröffentlichung von Artikeln auch
dann möglich ist, wenn die Autorenschaft oder Co-Autorschaft nicht bei
einem Angehörigen der Humboldt-Universität liegt (vgl. AG Elektronisches
Publizieren 2014). Sollte dem nicht so sein, bliebe als Alternative die
Zugänglichmachung über ein institutionelles Repositorium. Für sehr viele
Fachgebiete existiert in Deutschland zumindest ein spezifisches
Repositorium, das beispielsweise mithilfe des Registry of Open Access
Repositories (ROAR) oder dem Directory of Open Access Repositories
(DOAR) ermittelt werden kann.

Besteht die Möglichkeit zwischen mehreren Angeboten zu wählen, sollten
bestimmte Kriterien die Entscheidung lenken. Erstens sollte die
Langzeitverfügbarkeit der Dokumente auf dem Repositorium gesichert sein.
Hier spielt auch die Reputation des Repositorien-Betreibers eine Rolle.
Zweitens sollten die Betreiber des Repositoriums Hilfe bei der
Erstellung, dem Upload und der Verwaltung der elektronischen Dokumente
anbieten. Im Idealfall sollte es zudem über das {[}DINI-Zertifikat
verfügen, das neben den oben genannten Punkten weitere wichtige
Kriterien zur Beurteilung eines Repositoriums enthält.

Auch der finanzielle und organisatorische Aufwand hielte sich bei dieser
Variante relativ in Grenzen. Für den alten Verlag entstände kein
zusätzlicher Aufwand, was sich positiv auf die wirtschaftliche Erholung
auswirken könnte. Der Herausgeber beziehungsweise Verlag müsste bei
diesem Modell einerseits grundlegende Vereinbarungen mit den beteiligten
Autoren treffen (Autorenvertrag, Einfaches Nutzungsrecht, Deposit
Licence), ein geeignetes Repositorium auswählen und für die Erstellung
sowie die Zugänglichmachung der elektronischen Dokumente Sorge tragen,
wobei ein Großteil dieses Prozesses gegebenenfalls durch die
Repositorien-Betreiber abgedeckt werden könnte. Schließlich würden die
Sichtbarkeit und der Verbreitungsgrad der in der Zeitschrift
veröffentlichten Artikel durch das Verfügbarmachen der elektronischen
Dokumente erhöht. Die Printausgabe der Zeitschrift würde in dieser
Variante allerdings nicht überflüssig, da es sich bei der
Veröffentlichung im Rahmen des Grünen Weges nicht um eine elektronische
Parallelausgabe handelt. Hierin liegt allerdings auch ein Nachteil, da
die einzelnen Artikel auf dem Repositorium auch nur als solche
wahrnehmbar wären. Eine elektronische Entsprechung der gesamten
Zeitschrift existierte nicht, zumal außerdem nicht garantiert wäre, dass
tatsächlich alle Artikel verfügbar gemacht werden können. Weitere
Bestandteile der Printausgabe, wie zum Beispiel Kurzmeldungen,
Veranstaltungshinweise et cetera fänden sich ebenfalls nicht auf dem
Repositorium wieder. Abzuwägen wäre ebenfalls die Gefahr, dass
Abonnenten die Printausgaben abbestellen, wenn ein Großteil der Artikel
elektronisch verfügbar gemacht wird. Die nicht garantierte
Vollständigkeit lässt dies allerdings als weniger wahrscheinlich
erscheinen. Außerdem ist damit zu rechnen, dass viele Printabonnenten
aus Gewohnheit, Verlagsaffinität oder anderen Gründen auch weiterhin die
Printzeitschrift bevorzugen.\footnote{Vgl.
  \url{https://info.ub.hu-berlin.de/annualreports/2014/fupush/}.}

Insgesamt betrachtet, ließe sich die Variante der
Open-Access-Veröffentlichungen auf dem Grünen Weg ohne größere Hürden
umsetzen, da zwischen Herausgeber und altem Verlag keinerlei
Interessenkonflikte bestehen.

\subsection*{Soll der Herausgeber zu einem anderen Verlag wechseln und
Kompromisse
eingehen?}\label{soll-der-herausgeber-zu-einem-anderen-verlag-wechseln-und-kompromisse-eingehen}

Bei einem Wechsel zum neuen Verlag wird dem Herausgeber für seinen
Wunsch nach mehr Offenheit gegenüber dem Open Access der hybride Goldene
Weg angeboten. Der einfache Goldene Weg bezeichnet die
Erstveröffentlichung von wissenschaftlichen Beiträgen, die den
Bedingungen des Open Access folgen, was bedeutet, dass die Beiträge
sofort frei verfügbar sind. Die Veröffentlichung erfolgt je nach
Zeitschrift entweder kostenlos oder durch Zahlung einer
Veröffentlichungsgebühr in Form von APC, welche in der Regel von
Forschungsorganisationen und Forschungsförderern übernommen wird,
mitunter aber auch direkt von den Autorinnen und Autoren,
beziehungsweise aus deren Forschungsbudgets.~Hier können die Autoren
sich auch über institutionelle Publikationsfonds, diese werden zum
Beispiel gefördert von der Deutschen Forschungsgemeinschaft (DFG), oder
durch institutionelle Mitgliedschaften von wissenschaftlichen
Einrichtungen finanzielle Unterstützung einholen.

Im Gegensatz dazu ist der hybride Goldene Weg eine gesonderte Variante,
bei denen vor allem den Autoren grundsätzlich zwei Möglichkeiten offen
liegen. Bezahlen sie eine Publikationsgebühr, wird ihr Artikel sofort
als Open Access veröffentlicht. Andernfalls wird der Artikel als Closed
Access nur den Abonnenten der Zeitschrift zur Verfügung gestellt. APC
beinhalten in der Regel die Publikationskosten sowie die Kosten für das
Gutachterverfahren, das Online-Hosting und die Verbreitung
beziehungsweise die Bewerbung des Artikels. Darüber hinaus kann bei
dieser Finanzierung von einer hohen Gewinnspanne seitens des Verlages
ausgegangen werden. Hohe Publikationsgebühren können daher dazu führen,
dass aus Kostengründen nur einzelne Artikel von den Autorinnen und
Autoren für Open Access freigeschaltet werden können, bzw. in reinen
Gold-Open-Access-Journals nur Autorinnen und Autoren publizieren können,
die über entsprechende Mittel verfügen.

Der Einsatz von APC birgt mitunter auch die Gefahr des sogenannten
\enquote{double dippings}. Dies bedeutet, dass einerseits durch die
Abonnenten Subskriptionsgebühren und andererseits durch die Autoren
Publikationsgebühren für denselben Artikel an den Verlag bezahlt werden.
Somit verdient der Verlag zusätzlich zu den Einnahmen aus den
Abonnements ein zweites Mal durch die Open-Access-Gebühren.

Im Szenario wäre das Begutachtungsverfahren der eingereichten Artikel
eigentlich durch die APC im Prinzip finanziell abgedeckt, jedoch stellt
der neue Verlag die dafür notwendigen Ressourcen nicht zur Verfügung.
Außerdem besteht der Verlag auf eine weitreichende Rechteabtretung
seitens der Autorinnen und Autoren für die Nicht-Open-Access-Artikel. So
ist beispielsweise eine Zweitverwertung nicht mehr ohne weiteres
möglich. Wissenschaftliche Einrichtungen müssen dann die Artikel
entweder einzeln oder über Lizenzverträge und Subskriptionen
zurückkaufen, sonst können sie nicht mehr auf die gesamte Zeitschrift
zugreifen. In Anbetracht der Tatsache, dass
Open-Access-Veröffentlichungen in den Geisteswissenschaften bislang im
Vergleich zu den Naturwissenschaften weniger etabliert sind, bleiben die
Akzeptanz sowie der Finanzierungswille seitens der Autoren fraglich
(vgl.Kleineberg 2016: 16).

Dieses Szenario bietet aber auch eine Reihe an Vorteilen. Wenn der
Herausgeber diesen Weg mit dem neuen Verlag einschlagen möchte, gelten
erst einmal die gleichen Bedingungen wie für den einfachen Goldenen
Open-Access-Weg, nur dass die Publikation von einem Verlag
veröffentlicht wird und dieser auch die Richtlinien für alle Beteiligten
vorgibt. Die Betreuung wäre vom Verlag gegeben, womit der Herausgeber
wenig bis gar keinen organisatorischen Aufwand hat. Die Reputation
seiner Zeitschrift wäre weiterhin gewährleistet. Auch das bisher bereits
stattfindende Gutachterverfahren zur Qualitätskontrolle könnte wie
bisher durchgeführt werden. Ein weiterer Vorteil für den Herausgeber
könnte, sofern vom Verlag auch gewährleistet, eine Langzeitarchivierung
und -verfügbarkeit seiner Zeitschrift über die entsprechend angebotenen
Strukturen des neuen Verlages sein.

\subsection*{Gibt es andere Möglichkeiten einer Open-Access-Publikation?
Wenn ja, mit welchen Partnern, welchem Aufwand und unter welchen
Bedingungen?}\label{gibt-es-andere-muxf6glichkeiten-einer-open-access-publikation-wenn-ja-mit-welchen-partnern-welchem-aufwand-und-unter-welchen-bedingungen}

Bei einer Publikation ohne Verlagsbeteiligung als dritte Alternative
würde die bisherige Printzeitschrift in eine rein elektronische
Zeitschrift umgewandelt und nach dem Goldenen Weg des Open
Access-Verfahrens publiziert. Dieses Verfahren ist auf zwei denkbaren
Wegen realisierbar. Erstens könnte der Herausgeber eine Partnerschaft
mit einem sogenannten Open-Access-Verlag eingehen. Dieser bietet neben
der benötigten Software auch die Absicherung der Mindestanforderungen an
das Qualitätssicherungsverfahren. Darüber hinaus besitzt er ein
Mitspracherecht bei der Verteilung der Inhalte und er legt das
finanzielle Betriebsmodell fest. Letztlich übernimmt er alle Aufgaben
und Pflichten eines klassischen Verlages, sichert aber die frei
zugängliche Veröffentlichung der Zeitschrift zu (Reckling 2013: 5--8).
Zweitens könnte die Zeitschrift wie beim bereits genannten Grünen Weg
auch ganz ohne Verlagsbeteiligung, dafür in kompletter Eigenregie des
Herausgebers erscheinen. Der Zugang zu den von der Universität
bereitgestellten Ressourcen zur Veröffentlichung nach Open Access würde
sinnvollerweise über eine Zusammenarbeit mit der Arbeitsgruppe
\enquote{Elektronisches Publizieren} erfolgen, welche sich aus der
Universitätsbibliothek und dem Computer- und Medienservice
zusammensetzt. Während die Universitätsbibliothek die Zeitschrift mit
Bestandsaufnahme, Metadaten und Erschließung unterstützt, sichert der
Computer- und Medienservice die Bereitstellung auf dem institutionellen
Repositorium, dem edoc-Server der HU Berlin ab. Diese Art der
Bereitstellung gewährleistet gleichzeitig die Langzeitarchivierung der
Zeitschrift (Dobratz 2007: 28--30; Schirmbacher 2005: 6).

Bei der Veröffentlichung unter Verzicht auf Verlagsstrukturen fallen
alle organisatorischen und finanziellen Aufgaben dem Herausgeber zu.
Insbesondere letztere sind nur zu bewältigen, wenn tragfähige
Finanzierungswege gefunden und umgesetzt werden. Grundsätzlich fallen
bei Open-Access-Publikationen geringere Kosten an als bei klassischen
Printzeitschriften und als bei Closed-Access-Zeitschriften, denn sowohl
die Druckkosten als auch die Kosten für die Abonnentenverwaltung,
worunter auch die Identifikation von autorisierten und
nicht-autorisierten Nutzern sowie das Mahnverfahren fallen, entfallen
(Gradmann 2007: 42). Die Produktionskosten inklusive eventuell
anfallender Personalkosten sowie der zeitliche Aufwand können zudem
durch die Nutzung von Open-Source-Software für elektronische
Zeitschriften, beispielsweise Open Journals Systems (OJS), auf ein
Minimum reduziert werden. Personelle Ressourcen sind bei diesem Weg die
hauptsächlichen Kostenfaktoren. Denkbar wäre hier wie beim Goldenen Weg
des Open Access eine Finanzierung über die Erhebung von APC, wobei diese
aus genannten Gründen wiederum eine Barriere für die Autorinnen und
Autoren bei der Wahl ihres Veröffentlichungskanals darstellen könnten
(Gradmann 2007: 43--45; Schmidt 2006: 48--52). Sollen die Kosten nicht
von den Autoren getragen werden, wäre je nach Renommee der Zeitschrift
auch das Akquirieren von Sponsoren und Werbekunden eine
Finanzierungsoption, insbesondere wenn diese eine enge Verbindung zur
fachlichen Ausrichtung der Zeitschrift aufweisen (Schmidt 2006: 34).
Auch die Nutzung von öffentlichen Mitteln, beispielsweise aus dem
Förderfonds \enquote{Elektronisches Publizieren} der DFG oder aus
eventuellen Publikationsfonds der Universität sind denkbar (Deutsche
Forschungsgemeinschaft 2014; Schmidt 2006: 33).

Der letzte und vielleicht langfristig sicherste Finanzierungsweg ist die
Gründung eines Vereins, in dessen Satzung das Herausbringen der
Zeitschrift festgelegt wird. Mithilfe der erhobenen Vereinsbeiträge ist
die Finanzierung der Zeitschrift abgedeckt, weiterer Personal- und
Zeitaufwand für die Akquirierung von Einnahmen entfallen und
insbesondere das Renommee des Herausgebers und der Zeitschrift sollte
Mitglieder der Wissenschaftsgemeinschaft zu diesem Beitrag zur Sicherung
des Fortbestandes der Zeitschrift bewegen (vgl. LIBREAS e.V. 2013).
Hierüber könnten auch notwendige Tätigkeiten wie etwa die Kontaktarbeit,
sowohl mit potentiellen als auch mit bisherigen Autoren, sowie das
allgemeine Networking innerhalb der Fachcommunity, das Marketing für die
Zeitschrift und typische Verlagsarbeiten, wie etwa die Layout-Gestaltung
abgedeckt werden.

Eine Weiterführung der Zeitschrift ohne die Beteiligung eines Verlages
ist demnach theoretisch auf jeden Fall möglich, stößt in der Praxis aber
auf einige Hindernisse: Das wohl größte ist der Wegfall der
Printausgabe, was nicht im Sinne des Herausgebers ist. In erster Linie
birgt die rein elektronische Erscheinungsweise die Gefahr, dass die
abonnierenden Bibliotheken und ihre Benutzer den gewohnten Zugangsweg
verlieren (Gersmann 2007: 78). Deshalb müssen Strategien zur Information
über die neue Publikationsform entwickelt werden. Dennoch wird es
wahrscheinlich \enquote{traditionell denkende} Geisteswissenschaftler
geben, die gewohnheitsmäßig und ausschließlich die Printausgabe
rezipieren wollen und demzufolge dauerhaft als Leser wegfallen. Der
zweite Nachteil dieses Publikationsweges ist der deutlich steigende
Organisationsaufwand, der nun allein beim Herausgeber läge. Der Wegfall
der Verlagsstrukturen sorgt zwar für viel selbst zu leistende Arbeit,
ist aber gleichzeitig ein Vorteil. Anstatt von Verlagen und ihren
Vorgaben abhängig zu sein, nähert sich die Wissenschaft dem Ideal,
selbstständig und unbeeinflusst ihre Ergebnisse allen bereit zu stellen,
an (Schirmbacher 2005: 4). Mithilfe der richtigen Informations- und
Marketingstrategie kann die Zeitschrift von ihrer höheren Verbreitung,
besseren Zugänglichkeit, gesteigerten Rezeption und Nachnutzbarkeit für
alle Wissenschaftler und Interessenten profitieren (Schmidt 2006: 8).

\subsection*{Braucht die Publikation ein Peer Review und falls ja, wie
lässt sich ein solches Verfahren
organisieren?}\label{braucht-die-publikation-ein-peer-review-und-falls-ja-wie-luxe4sst-sich-ein-solches-verfahren-organisieren}

Der Herausgeber hat die Qualitätssicherung der Zeitschrift bislang im
Editorial Review-Ver\-fahr\-en gewährleistet und somit eigenständig die beim
Verlag eingereichten Artikel und Beiträge von Wissenschaftlern danach
überprüft, ob sie den gegebenen Anforderungen der Zeitschrift und denen
einer wissenschaftlichen Publikation gerecht werden. Im besten Fall
folgte er dabei einer Editorial Policy, welche bestimmte Richtlinien
aufstellt, nach denen die Bewertung vorgenommen wird. Üblich beim
Editorial Review sind proaktive Einladungen zu Artikeln seitens des
Herausgebers, vor allem in geistes- und sozialwissenschaftlichen
Journalen und Sammelwerken (Herb 2016). Neben dieser etablierten und
eher klassischen Variante der Qualitätssicherung können auch
verschiedene Peer Review-Verfahren durchgeführt werden, deren Vorteile
der Herausgeber im Zuge einer möglichen Open-Access-Version der
Zeitschrift künftig gern nutzen möchte. Damit ginge die Einrichtung
eines Herausgebergremiums (Editorial Board) einher. Beim Peer Review
werden eingereichte Artikel durch Wissenschaftler desselben Fachgebietes
begutachtet. Damit soll vor allem die Verständlichkeit der Beiträge
ebenso wie die sachliche Korrektheit, die Validität und die Objektivität
der eingereichten Arbeiten -- sprich: die wissenschaftliche Qualität --
gewährleistet beziehungsweise erhöht werden. Ein Peer-Review-Verfahren
kann sehr unterschiedlich organisiert werden. Beim sogenannten
Single-Blind-Verfahren ist der Name des Autors dem jeweiligen Gutachter
bekannt, wohingegen beim Double-Blind-Verfahren beide, sowohl der Autor
als auch der Gutachter, anonym bleiben.~Im Folgenden werden die
jeweiligen Möglichkeiten für die bereits vorgestellten
Open-Access-Szenarien vorgestellt.

Für den Grünen Weg mit Fortsetzung der Veröffentlichung der Printausgabe
über den bisherigen Verlag und einer davon unabhängigen
Open-Access-Ausgabe der einzelnen Artikel ergeben sich mehrere Wege der
Qualitätssicherung. Hierbei sind noch einmal die wirtschaftliche Lage
des Verlages und die damit einhergehende Umstrukturierung zu betonen.
Neuen Review-Verfahren gegenüber zeigt sich der Verlag zwar prinzipiell
aufgeschlossen, jedoch ist er nicht in der Lage die nötigen Ressourcen
und Kapazitäten zur Verfügung zu stellen. Entsprechend könnte der
Herausgeber weiterhin das bewährte Editorial Review praktizieren,
welches angesichts der guten Reputation der Zeitschrift durchaus
angemessen erscheint. Die Qualitätssicherung der Artikel würde dann wie
gehabt über die weiter erscheinende Printausgabe erfolgen.~Weiterhin hat
er die Option mit dem bisherigen Verlag in Verhandlungen zu gehen, um
eine für beide Seiten umsetzbare Peer-Review-Strategie zu erarbeiten.
Wichtig für die Open-Access-Version der Zeitschrift ist in jedem Fall
die Veröffentlichung der Postprints, da nur diese der Printausgabe
entsprechen. Preprints, also die Manuskripte der Autoren, sind zum einen
zumindest nicht mit Verweis auf die Zeitschrift zitierbar und zum
anderen sind sie meist nicht identisch mit den veröffentlichten Artikeln
in der Printausgabe.

In der Variante einer hybriden Open-Access-Veröffentlichung der
Zeitschrift bei einem neuen Verlag ist die Qualitätssicherung über Peer
Review ein schwieriges Vorhaben. Der neue Verlag signalisiert, dass ein
Verfahren dieser Art zur Qualitätssicherung eingesetzt werden soll,
verweigert aber die entsprechenden Ressourcen. Hier helfen nur
Verhandlungen mit dem Verlag und eine solide Argumentationsgrundlage.
Kann der Verlag nicht dazu bewegt werden von seiner Haltung abzuweichen
und dem Herausgeber finanziell und personell entgegenzukommen, bleibt
dem Herausgeber letztlich nur, weiterhin das Editorial Review für die
Printausgabe fortzuführen. Für die einzelnen Open-Access-Artikel kann
zusätzlich ein Post-Publication-(Open)-Review beziehungsweise
Transparent-Peer-Review eingesetzt werden, bei welchem die rezipierende
Wissenschaftsgemeinschaft über eine bereitgestellte Kommentarfunktion im
Nachhinein die Qualität des Beitrages bewerten kann.

Auch beim Goldenen Weg im Selbstverlag ist ein Verfahren zur
Qualitätssicherung zwingend notwendig. Der Herausgeber hat hier neben
dem Editorial Review-Verfahren alle Freiheiten, die der jeweilige
finanzielle und personelle Rahmen hergibt. Theoretisch ist also jedes
Verfahren denkbar, die praktische Umsetzung muss individuell bewertet
werden. Zu bedenken ist, dass für die Einrichtung eines
Herausgebergremiums beziehungsweise einer verlässlichen Peer Group
sowohl das Single-Blind- als auch das Double-Blind-Peer-Review den
Aufbau und die Pflege eines aktiven Wissenschaftlerkreises
beziehungsweise die Vernetzung zu einer bereits bestehenden aktiven
Fachcommunity erfordern. Dies wird möglicherweise durch das
hervorragende Renommee der Zeitschrift und des Herausgebers erleichtert.

\section*{Zusammenfassung}\label{zusammenfassung}

Alle vorgestellten Vorgehensweisen sind möglich und denkbar, jeder Weg
hat für sich individuelle Vorzüge, aber auch Nachteile. Beim Goldenen
Weg des Open Access im Selbstverlag fiele die Printausgabe komplett weg,
es würden nur noch die einzelnen Artikel publiziert. Eine elektronische
Parallelausgabe zur Printausgabe der Zeitschrift gäbe es lediglich bei
Publikation über die Vereinsbildung oder dem Wechsel zum neuen Verlag.
Die Printausgabe bliebe bei letzterem wie auch mit dem alten Verlag
weiterhin erhalten. Wie schwerwiegend die wirtschaftliche Krise des
bisherigen Verlages ist, kann an dieser Stelle nicht endgültig beurteilt
werden, ebenso wenig wie der Ausgang der geplanten Neustrukturierung.
Dies bedeutet, dass somit auch die Zukunft der Zeitschrift -- zumindest
in der Printversion -- weiterhin ungewiss bliebe, würde sich der
Herausgeber für eine Fortführung der Zusammenarbeit mit dem alten Verlag
entscheiden. Der Herausgeber hat jedoch die Möglichkeit in jedem Fall
gemeinsam mit dem bisherigen Verlag den Grünen Weg des Open Access zu
gehen. In diesem Fall wird für die Publikation der Artikel im Open
Access keine zusätzliche Finanzierung benötigt, da bereits vorhandene
Infrastrukturen, wie etwa ein geeignetes Fachrepositorium, genutzt
werden können. Man könnte hier aber im Hinblick auf Wirtschaftlichkeit
und Krisenbewältigung über Möglichkeiten zusätzlicher Finanzeinnahmen
nachdenken. Kann die Umstrukturierung dem Verlag nicht aus seiner Krise
helfen, so hat der Herausgeber zu einem späteren Zeitpunkt immer noch
die Option, die Zeitschrift ohne Printausgabe über den Goldenen Weg im
Selbstverlag zu publizieren oder sich auf einen neuen Verlag
einzulassen. Sollte sich die wirtschaftliche Situation des Verlages
wieder stabilisieren, würde sich diese Option im Nachhinein als
besonders lukrativ erweisen. Denn die Printversion bliebe beim
bisherigen Verlag, für die Rezipienten gäbe es dahingehend keine
Änderung. Da sich der Verlag prinzipiell offen zeigt für
Open-Access-Verfahren, hätte der Herausgeber außerdem freie Hand bei der
Organisation.

Durch einen Verlagswechsel bliebe die Reputation der Zeitschrift ohne
Risiko erhalten beziehungsweise könnte sich je nach Renommee des
entsprechenden Verlages durchaus sogar noch steigern. Negativ sind die
weitreichende Rechteabtretung für die Autoren und die hohen APC, wenn
sie ihre Artikel im Open Access publizieren. Des Weiteren befände sich
der Herausgeber in einer Abhängigkeit mit wenig Spielraum für die
Umsetzung eigener Interessen.

Bei der Wahl des Goldenen Weges im Selbstverlag müsste zunächst die
Finanzierung gesichert und die gesamte Infrastruktur für das Open Access
geschaffen werden. Positiv betrachtet bietet dieser Weg somit das
höchste Maß an Unabhängigkeit und organisatorischer Freiheit. Jedoch
bestünde durch die Abwendung von Verlegern die Gefahr, dass das
Fortbestehen der Zeitschrift als Open-Access-Version von den Rezipienten
nicht bemerkt beziehungsweise auch nicht angenommen werden könnte. Im
schlimmsten Fall stünde die Reputation der Zeitschrift auf dem Spiel.
Aus diesem Grund ist hier intensive Marketingarbeit und eine sinnvolle
Strategie erforderlich, um das Fortbestehen der Zeitschrift zu sichern.
Eine Organisation und Verteilung dieser zahlreichen Aufgaben sowie eine
Absicherung in Hinblick auf Urheber-, Autoren- und Verlagsrechte wären
über die Gründung eines Vereins möglich, der neben der Finanzierung auch
festgelegte und damit dauerhafte Strukturen sicherte.

Die endgültige Entscheidung für einen bestimmten Weg muss letztlich
durch den Herausgeber getroffen werden. Nach Abwägung des Für und Wider
erscheint jedoch der Verbleib beim bisherigen Verlag mit dem Grünen Weg
des Open Access und der Option für eine spätere Erweiterung des
Review-Verfahrens als sinnvollste Variante. Sollte der Verlag scheitern,
kann der Herausgeber immer noch eine der anderen Varianten in Betracht
ziehen.

\section*{Literatur}\label{literatur}

Arbeitsgruppe Elektronisches Publizieren (2014): Hinweise für
Open-Access-Autoren. {[}abrufbar unter:
\url{http://edoc.hu-berlin.de/e_autoren/index-oa.php}{]}.

Deutsche Forschungsgemeinschaft (2014): Merkblatt Open Access
Publizieren. Bonn. {[}abrufbar unter:
\url{http://www.dfg.de/formulare/12_20/12_20_de.pdf}{]}.

Dobratz, Susanne (2007): Open-Access-Repositories am Beispiel des
edoc-Servers der Humboldt-Universität zu Berlin. In: Malina, Barbara
(Hrsg.): Open Access. Chancen und Herausforderungen. Bonn: Deutsche
UNESCO-Kommission e.V. S. 28--32.

Georg-August-Universität Göttingen. Niedersächsische Staats- und
Universitätsbibliothek (2015): Informationsplattform Open Access.
Göttingen.\\
{[}abrufbar unter: \url{https://www.open-access.net/}{]}.

Gersmann, Gudrun (2007): Open Access in den Geisteswissenschaften. In:
Malina, Barbara (Hrsg.): Open Access. Chancen und Herausforderungen.
Bonn: Deutsche UNESCO-Kommission e.V. S. 78--79.

Gradmann, Stefan (2007): Finanzierung von Open-Access-Modellen. In:
Malina, Barbara (Hrsg.): Open Access. Chancen und Herausforderungen.
Bonn: Deutsche UNESCO-Kommission e.V. S. 42--45.

Gradmann, Stefan (2009): Publizieren im Open-Access-Modell. Allgemeine
Einführung und Grundaussagen. In: cms-journal (32). Berlin.\\
{[}abrufbar unter:
\url{http://edoc.hu-berlin.de/cmsj/32/gradmann-stefan-20/PDF/gradmann.pdf}{]}.

Herb, Ulrich (2016.): Wissenschaftliches Publizieren. Qualitätssicherung
und -messung, in: scinoptica {[}abrufbar unter:
\url{http://www.scinoptica.com/pages/de/materialien/wissenschaftliches-publizieren/wissenschaftliches-publizieren-qualitaetssicherung-und--messung.php}{]}.

Kleineberg, Michael (2016): Open Humanities? Expertenmeinungen über Open
Access in den Geisteswissenschaften. In: Berliner Beiträge zu Digital
Humanities, hrsg. vom Einstein-Zirkel, Berlin. Preprint. {[}abrufbar
unter:
\url{https://zenodo.org/record/50598/files/Open_Humanities_Kleineberg.pdf}{]}.

LIBREAS e.V. (2013): LIBREAS. Vereinssatzung. Berlin.\\
{[}abrufbar unter: \url{http://www.libreas-verein.eu/satzung}{]}.

Mittermaier, Bernhard (2015): Double Dipping beim Hybrid Open Access --
Chimäre oder Realität? In: Informationspraxis (1). Heidelberg.\\
{[}abrufbar unter: \url{http://dx.doi.org/10.11588/ip.2015.1.18274}{]}.

Müller, Uwe; Schirmbacher, Peter (2007): Der \enquote{Grüne Weg zu Open
Access} in Deutschland. ZfBB 54 (4-5). S. 183--193.\\
{[}abrufbar unter:
\url{http://edoc.hu-berlin.de/oa/articles/retHrv7eeUFo2/PDF/23tfNyzkDjYo.pdf}{]}.

Reckling, Falk (2013): Open Access. Aktuelle internationale und
nationale Entwicklungen. Wien.\\
{[}abrufbar unter:
\url{http://www.fwf.ac.at/fileadmin/files/Dokumente/News_Presse/News/FWF_OA-2013.pdf}{]}.

Schirmbacher, Peter (2005): Open Access -- die Zukunft des
wissenschaftlichen Publizierens. In: cms-journal (27). Berlin.\\
{[}abrufbar unter:
\url{http://edoc.hu-berlin.de/docviews/abstract.php?lang=ger\&id=25486}{]}.

Schmidt, Birgit (2006): Open Access. Freier Zugang zu wissenschaftlichen
Informationen - das Paradigma der Zukunft? In: Berliner Handreichungen
zur Bibliotheks- und Informationswissenschaft (144). Berlin. Institut
für Bibliotheks- und Informationswissenschaft der Humboldt-Universität
zu Berlin.

Sherpa (2006): Definition and Terms.\\
{[}abrufbar unter: \url{http://www.sherpa.ac.uk/romeoinfo.html}{]}.

Alle Links wurden letztmalig am 4.12.2016 geprüft.

%autor

\end{document}
