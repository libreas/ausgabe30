\documentclass[a4paper,
fontsize=11pt,
%headings=small,
oneside,
numbers=noperiodatend,
parskip=half-,
bibliography=totoc,
final
]{scrartcl}

\usepackage{synttree}
\usepackage{graphicx}
\setkeys{Gin}{width=.4\textwidth} %default pics size

\graphicspath{{./plots/}}
\usepackage[ngerman]{babel}
\usepackage[T1]{fontenc}
%\usepackage{amsmath}
\usepackage[utf8x]{inputenc}
\usepackage [hyphens]{url}
\usepackage{booktabs} 
\usepackage[left=2.4cm,right=2.4cm,top=2.3cm,bottom=2cm,includeheadfoot]{geometry}
\usepackage{eurosym}
\usepackage{multirow}
\usepackage[ngerman]{varioref}
\setcapindent{1em}
\renewcommand{\labelitemi}{--}
\usepackage{paralist}
\usepackage{pdfpages}
\usepackage{lscape}
\usepackage{float}
\usepackage{acronym}
\usepackage{eurosym}
\usepackage[babel]{csquotes}
\usepackage{longtable,lscape}
\usepackage{mathpazo}
\usepackage[normalem]{ulem} %emphasize weiterhin kursiv
\usepackage[flushmargin,ragged]{footmisc} % left align footnote

\usepackage{listings}

\urlstyle{same}  % don't use monospace font for urls

\usepackage[fleqn]{amsmath}

%adjust fontsize for part

\usepackage{sectsty}
\partfont{\large}

%Das BibTeX-Zeichen mit \BibTeX setzen:
\def\symbol#1{\char #1\relax}
\def\bsl{{\tt\symbol{'134}}}
\def\BibTeX{{\rm B\kern-.05em{\sc i\kern-.025em b}\kern-.08em
    T\kern-.1667em\lower.7ex\hbox{E}\kern-.125emX}}

\usepackage{fancyhdr}
\fancyhf{}
\pagestyle{fancyplain}
\fancyhead[R]{\thepage}

%meta
%meta

\fancyhead[L]{T. Roeder \\ %author
LIBREAS. Library Ideas, 30 (2016). % journal, issue, volume.
\href{http://nbn-resolving.de/urn:nbn:de:kobv:11-100244161}{urn:nbn:de:kobv:11-100244161}} % urn
\fancyhead[R]{\thepage} %page number
\fancyfoot[L] {\textit{Creative Commons BY 3.0}} %licence
\fancyfoot[R] {\textit{ISSN: 1860-7950}}

\title{\LARGE{Daten auf Wanderschaft} \\ Migration von Daten als Herausforderung der Digital Humanities am Beispiel von MUSICI und MusMig} %title %title
\author{Torsten Roeder} %author

\setcounter{page}{104}

\usepackage[colorlinks, linkcolor=black,citecolor=black, urlcolor=blue,
breaklinks= true]{hyperref}

\date{}
\begin{document}

\maketitle
\thispagestyle{fancyplain} 

%abstracts

%body
\section*{Einleitung}\label{einleitung}

Migration bringt für den Wandernden -- oder auch: \emph{das} Wandernde
-- stets eine Veränderung des Umfelds mit sich. Kontakte und Reibungen
mit der neuen Umgebung entstehen und Assimilationsprozesse werden in
Gang gesetzt. Das Ergebnis solcher Prozesse ist kaum vorhersehbar:
Harmonie und Ausgewogenheit sind dabei ebenso gut möglich wie Konflikte
und Differenzen.

Sowohl in den Geisteswissenschaften als auch in der Informatik ist
\enquote{Migration} -- jeweils mit einer ganz eigenen Ausprägung des
Begriffs -- zu einem relevanten Thema avanciert. Der Schwerpunkt der
folgenden Untersuchung liegt auf einem Überschneidungsbereich der beiden
Disziplinen, in dem sich derzeit das Feld der Digital Humanities
etabliert.

Diese spezifische Verortung der Digital Humanities, in denen
geisteswissenschaftliche und informatische Fragestellungen
zusammenfließen, soll in den folgenden Ausführungen veranschaulicht
werden. Exemplarisch werden dafür zwei Projekte skizziert, in denen das
Thema Migration gleich in zweifacher Hinsicht, auf beiden der oben
genannten Ebenen, zum Tragen kommt.

\section*{Begriffliche Grundlagen}\label{begriffliche-grundlagen}

\subsection*{Der Begriff der Migration in den
Geisteswissenschaften}\label{der-begriff-der-migration-in-den-geisteswissenschaften}

In den Geisteswissenschaften versteht man unter \enquote{Migration} ganz
allgemein die Bewegung von Individuen oder ethnologischen Gruppen im
geographischen Raum; im weiteren Sinne auch die Wanderung von
Gegenständen oder Ideen; und im weitesten Sinne auch \enquote{innere}
Bewegungen im menschlichen Denken und Schaffen. Am Ende eines
Migrationsweges steht das Subjekt dabei stets in einer neuen Umgebung
und erzeugt mit dieser Wechselwirkungen. Die Geisteswissenschaften
analysieren Ursachen und Wirkungen dieses Phänomens sowohl in einzelnen
Fallstudien als auch zusammenfassend im Hinblick auf größere
Kulturströmungen. So hat sich Migration zu einem wichtigen
sozialhistorischen Forschungsthema entwickelt, welches insbesondere am
Beispiel der Musik -- als stets unmittelbarer und sprachunabhängiger
Ausdruck von Identität -- seine besondere Ergiebigkeit darin bewies, an
ihm Kulturströmungen aufzeigen und charakteristisch nachzeichnen zu
können.\footnote{Für eine breite musikhistorische Übersicht siehe z.B.
  Ehrmann-Herfort/Leopold 2013; zu politisch verursachten Migrationen
  des 20.~Jh. siehe z.B. Brinkmann/Wolff 1999 und Davis et al. 2011
  (hier Part III, Traveling Sound: Music and Migration); zu
  zeitgenössischen globalen Strömungen siehe z.~B. MAiA 2011/2. Diese
  Auswahl ist exemplarisch und nicht repräsentativ.}

\subsection*{Der Begriff der Migration in der
Informatik}\label{der-begriff-der-migration-in-der-informatik}

In den Informationswissenschaften hingegen steht \enquote{Migration} vor
allem für die Übertragung von Daten aus einem System in ein anderes
System. Auch hier wird der Begriff \enquote{Umgebung} verwendet, der
sich in diesem Kontext vor allem auf die informationstechnischen
Vorgaben (etwa einer bestimmten Plattform) bezieht. Meist projekt- oder
auch geschäftsorientiert ist Migration in der IT ein fast durchwegs
praxisbezogenes Betätigungsfeld, in dem Aspekte der Theoriebildung eher
im Hintergrund stehen. Erst bedingt durch über mehrere Generationen von
Betriebssystemen und Anwendungsprogrammen hinweg erfahrenen Konsequenzen
einer unzulänglichen Datenübertragbarkeit konnte die Migration von
elektronisch gespeicherten Daten auch die erhöhte Aufmerksamkeit der
theoretischen Informatik gewinnen. Ein bereits intensiv bestelltes Feld
ist die Migration von Software,\footnote{Zur Problematik von
  \emph{Legacies} (Altsystemen) vgl. z.~B. Wagner 2014.} welches an
dieser Stelle jedoch nicht näher einbezogen werden soll -- hier geht es
ausschließlich um die Migration von Daten.

Levine, 2009, beschrieb das Datenmigrationsproblem wie folgt:

\begin{quote}
\emph{\enquote{Data migration is the process of making a copy of data
and moving it from one device or system to another, preferably without
disrupting or disabling active business processing.}}
\end{quote}

Sieht man von dem Fokus auf Geschäftsvorgänge ab -- an dieser Stelle
könnte schließlich ebenso gut eine wissenschaftliche Datenverarbeitung
stehen -- wird hier ein Kernprinzip der Migration erkennbar: Objekte aus
einer Ursprungsumgebung müssen zukünftig in einer neuen Umgebung
funktionieren. Das Zitat impliziert, dass dieser Prozess nicht zwingend
reibungslos verläuft (\enquote{preferably without disrupting}). Es muss
stets mit Inkonsistenzen und Anpassungsbedarf gerechnet werden. Die zu
migrierenden Daten müssen daher nicht nur strukturell transformiert,
sondern gegebenenfalls auch inhaltlich angepasst werden.

\subsection*{Datenmigration als Aufgabenfeld der Digital
Humanities}\label{datenmigration-als-aufgabenfeld-der-digital-humanities}

Die Verbindung der beiden Migrationsverständnisse hat in zwei
Forschungsprojekten konkret Gestalt angenommen, die sich inhaltlich mit
der Migration von Musikern im Europa des 17. und 18. Jahrhunderts
befassen. Das erste Projekt, MUSICI\footnote{\emph{Musicisti europei a
  Venezia, Roma e Napoli (1650-1750): musica, identità delle nazioni e
  scambi culturali}, gefördert durch die Agence Nationale de la
  Recherche und die Deutsche Forschungsgemeinschaft von 2010--2012,
  \url{http://www.musici.eu/} (24.05.2016).} (2009--2013), widmete sich
Musiker-Migrationen nach Italien; das nachfolgende Projekt
MusMig\footnote{\emph{Music Migrations in the Early Modern Age: the
  Meeting of the European East, West and South}, gefördert durch
  Humanities in the European Research Area von 2013--2016,
  \url{http://musmig.eu/} (24.05.2016).} (2013--2016) konzentriert sich
hingegen auf Migrationsbewegungen zwischen Süd-, Mittel- und Osteuropa.
Die MUSICI-Daten\-bank ging mit dem Projektabschluss im Jahr 2013 in
Betrieb.\footnote{Musicisti europei a Venezia, Roma e Napoli
  (1650-1750), hrsg. von Michela Berti, Gesa zur Nieden und Torsten
  Roeder, Berlin/Rom 2013, \url{http://www.musici.eu/database}
  (24.05.2016).} Die Datenbank des Nachfolge-Projektes MusMig basiert
technisch auf dem gleichen System wie die MUSICI-Daten\-bank und soll
zukünftig die Daten des MUSICI-Projekts als eine von mehreren
Teildatenmengen inkludieren. Dafür müssen die Daten des MUSICI-Projekts
in das System der MusMig-Datenbank migriert werden; während die zugrunde
liegenden Datenstrukturen auf der technischen Ebene miteinander
kompatibel sind, unterscheiden sie sich jedoch inhaltlich in vielerlei
Hinsicht. Das gemeinsame Thema der beiden Projekte gab den Anstoß, sich
in Analogie zu natürlichen Migrationsstrukturen den inhaltlichen
Konsequenzen der Datenübertragung anzunähern. Aus diesen und
vergleichbaren Vorhaben ergeben sich die für die Digital Humanities
spezifischen Problemstellungen.

Vernachlässigt man den rein praktischen Aspekt der Datenmigration und
setzt voraus, dass zwei Umgebungen zumindest von der technischen Seite
her miteinander kompatibel sind, öffnet sich die Perspektive auf die
zahlreichen inhaltlichen Herausforderungen: Wie verändern sich
Informationen\footnote{Zur begrifflichen Abgrenzung: \enquote{Daten}
  wird hier unspezifisch aufgefasst und bezeichnet eine undefinierte
  Menge von Datenobjekten; \enquote{Informationen} hingegen meint hier
  konkretisierbare, einzelne Datenobjekte mit Aussagegehalt.} in neuen
Umgebungen? Wie wirkt sich zum Beispiel abweichendes Vokabular der neuen
Umgebungen auf die migrierten Informationen aus? Sind in der neuen
Umgebung bereits Daten vorhanden? Werden einheitliche Referenzsysteme
genutzt?

Diese Fragen können bei jedem Datenmigrationsprozess erneut auftreten.
Durch die aktuelle Forschungslandschaft, die sich zum einen durch eher
kurzfristige Projekte und zum anderen durch eine relativ offene
Verfügbarkeit von Daten (etwa im Rahmen von Open Access) definiert,
bleiben Daten stets der Möglichkeit einer Nachnutzung durch
Folgeprojekte oder Dritte ausgesetzt. Dies ist wissenschaftlich gesehen
wünschenswert, bringt aber gleichzeitig Aufgaben mit sich.

Eine weitere Form von Datenmigration thematisiert Bernhard Thalheim im
Zusammenhang mit der \enquote{Evolution von Systemen}, das heißt in
diesem Falle: mit dem Anstieg der Funktionalität.\footnote{Vgl.
  Thalheim/Wang 2011, 2013; dazu auch Seacord et al. 2003; Klettke 2011.}
Grundannahme ist, dass die Ansprüche an die Funktionalität eines Systems
sich mit der Zeit steigern und sich Systeme entsprechend
weiterentwickeln müssen. Dies kann etwa durch eine Folge von
Ausbaustufen geschehen. Einen deutlich höheren Anstieg der
Funktionalität, als es die allmähliche Erweiterung eines bestehenden
Systems bewirken könne, bringe der Wechsel eines Systems mit sich. Der
Wechsel aber erfordere zwangsläufig die Migration der bereits
bestehenden Daten. Als zentrale und zu berücksichtigende Problempunkte
werden unter anderem genannt: Die divergente Interpretierbarkeit der
Daten in den verschiedenen Systemen, die damit verbundenen
unterschiedlichen Methodiken und Mittel, mit denen die Daten erfasst
werden und schließlich die Änderung der Spezifikationen von System zu
System. All dies ist unabhängig davon, ob die Datenmigration zwischen
zwei voneinander gänzlich unabhängigen Systemen erfolgt oder ob sie im
Rahmen eines größeren Versionssprungs geschieht.

Das abstrakte Konzept der Evolution von Systemen ist allerdings nicht
einseitig linear zu verstehen. Ein Anstieg von Funktionalität kann
ebenso auch Verluste in der Leistungsfähigkeit eines Systems mit sich
bringen. Ein neues System wird -- und soll -- nur sehr selten in der
Lage sein, das bisherige System hundertprozentig zu inkludieren:
Systemwechsel sind häufig auch konzeptuell motiviert und dienen nicht
nur der Leistungsoptimierung. Analog dazu steht auch in den
Geisteswissenschaften einer reinen Anhäufung und Vermehrung von Wissen
eine immer wieder neue Kontextualisierung und Reinterpretation der
Wissenselemente entgegen. In den Digital Humanities verbindet sich
beides unter anderem auf der Ebene der Datenmodellierung: Hier fließen
informationstechnische und geisteswissenschaftliche Konzepte zusammen.

Migrationen von Forschungsdaten sind daher nicht als rein technische
Aufgabe zu verstehen, sondern sie müssen von einem wissenschaftlichen
Prozess begleitet werden, welcher die Konsequenzen der konzeptionellen
Veränderungen in Datenstrukturen transparent werden lässt.

\section*{Das Zusammenspiel von Daten, Kontext und
Narration}\label{das-zusammenspiel-von-daten-kontext-und-narration}

\subsection*{\texorpdfstring{Datenbanken und ihre \enquote{Weltsichten} --
ein
Beispiel}{Datenbanken und ihre Weltsichten -- ein Beispiel}}\label{datenbanken-und-ihre-weltsichten-ein-beispiel}

Durch eine Gegenüberstellung mehrerer musikwissenschaftlicher digitaler
Ressourcen soll im Folgenden die konzeptuelle Divergenz von
Datenbanksystemen exemplifiziert werden. Als Beispiel dienen drei
inhaltlich und strukturell sehr unterschiedliche Kurzbiographien des
Komponisten Carl Ditters von Dittersdorf. Der Fokus soll, in Anlehnung
an das MUSICI-Projekt, auf die Beschreibung seines Verhältnisses zu
Italien gerichtet werden. Herangezogen werden das \emph{Österreichische
Musik-Lexikon} (ÖML), das \emph{Bayerische Musiker-Lexikon Online}
(BMLO) und \emph{Musica Migrans}.

Das ÖML bringt in dem Artikel keinen einzigen Hinweis auf eine Beziehung
Dittersdorfs zu Italien.\footnote{\emph{Österreichisches Musik-Lexikon.
  Online-Ausgabe}, Wien: Österreichische Akademie der Wissenschaften,
  2002--2013, \url{http://www.musiklexikon.ac.at/} = Oesterreichisches
  Musiklexikon, Wien: Verlag der Österreichischen Akademie der
  Wissenschaften, 2002--2006. Dort: \emph{Ditters von Dittersdorf,
  Johann Carl}, \url{http://musiklexikon.ac.at/0xc1aa500d_0x0001cbcb}
  (24.05.2016).} Das BMLO hingegen führt unter dem Schlagwort
\enquote{Wirkungsorte} die Städte Bologna, Mantua, Parma, Triest und
Venedig.\footnote{\emph{Bayerisches Musiker-Lexikon Online}, hrsg. von
  Josef Focht, München: Ludwig-Maximilians-Universität, 2004--2014,
  {[}\url{http://bmlo.de/}. Dort: \emph{Dittersdorf, Karl Ditters von},
  \url{http://bmlo.de/d0473} (24.05.2016).} Musica Migrans nennt dagegen
eine Konzertreise nach Italien im Jahr 1763 mit Christoph Willibald
Gluck.\footnote{\emph{Musica Migrans}, Universität Leipzig,
  \url{http://www.musicamigrans.de/}. Dort: \emph{Ditters von
  Dittersdorf, Carl},
  \url{http://www.musicamigrans.de/pages/individual/person.php?id=topic-233658}
  (24.05.2016).} Diese sehr unterschiedliche Auswahl der Informationen
hängt sowohl mit der Struktur als auch mit der inhaltlichen Ausrichtung
der Datenbanken zusammen.

Das ÖML ist nicht ausschließlich biographisch ausgerichtet und als
ursprüngliches Printprodukt vom Umfang her begrenzt, weshalb die
Informationen dort auf das für Österreich Relevante konzentriert sind.
BMLO und Musica Migrans können hingegen als native Datenbankprojekte
mehr Informationen aufnehmen, und so findet sich hier trotz der
jeweiligen Interessengebiete (Bayern beziehungsweise Ost- und
Mitteleuropa) in beiden Fällen ein Hinweis auf Dittersdorfs
Italienaufenthalte. Dabei steht im BMLO eine schlagwortartige Struktur
im Vordergrund, welche eine Vielzahl von Wirkungsorten auflistet, jedoch
ohne Gewichtung und Kontext. Der Eintrag in Musica Migrans hingegen --
eine aus Einzelinformationen zusammengestellte Liste mit semantischem
Markup -- erklärt die genaueren Umstände eines Italienaufenthalts,
allerdings ohne die konkreten Orte zu nennen.

Als Gesamteindruck bleibt, dass in jeder Datenbank eine andere
Sichtweise auf Dittersdorfs Italienaufenthalte repräsentiert ist. Aus
der Zusammenschau der Datenbanken wird nicht deutlich, ob die in Musica
Migrans erwähnte Reise tatsächlich mit den im BMLO genannten Stationen
(oder einigen davon) in Zusammenhang steht. Es ist ohne weitere
Recherche nicht möglich, die Informationen miteinander in Verbindung zu
bringen.\footnote{Die etwas ausführlichere Deutsche Biographie (ADB/NDB)
  klärt, dass es sich um mehrere Aufenthalte handelte. Vgl. Günter
  Birkner: \emph{Ditters von Dittersdorf, Carl}. In: Neue Deutsche
  Biographie 4 (1959), S. 1--2,
  \url{http://www.deutsche-biographie.de/ppn118679856.html}
  (24.05.2016); J. Maehly: \emph{Dittersdorf, Karl von}. In: Allgemeine
  Deutsche Biographie 5 (1877), S. 262--266, ebd.} Strukturen und
Inhalte der Datenbanken sind so divergent, dass sie als miteinander
nicht kompatibel erscheinen.

\subsection*{Die Kontextualisierung von
Daten}\label{die-kontextualisierung-von-daten}

An dem Beispiel wurde ersichtlich, dass eine Information aus einer
Datenbank im Kontext einer anderen Datenbank sowohl die eigene
Aussagekraft als auch die der anderen Daten relativiert. Diese
Relativität mag problematisch erscheinen, und doch ist sie wichtig: Zwar
könnte eine Information prinzipiell auch ohne ihren Kontext erschlossen
werden, denn bereits in rein textueller Form liefert sie eine Aussage.
Allerding ließe sich daraus ihre Relevanz nicht bewerten: Diese wird
erst durch eine Einordnung in Zusammenhänge erkennbar, wenn also
ersichtlich wird, wie sich die Information im Zusammenspiel mit anderen
Informationen verhält. Die Möglichkeiten der Kontextualisierung sind
fast unbegrenzt variierbar.

So taucht etwa die Information, dass Dittersdorf mit Gluck auf
Konzertreise in Bologna war, zwar häufig in den Biographien Dittersdorfs
auf, viel seltener aber in denen über Gluck, so dass die Relevanz einer
Information offensichtlich mit ihrem jeweiligen Kontext variiert.
Angenommen, sie würde in der Biographie Glucks auftauchen, könnte dies
sowohl die Wahrnehmung Glucks als auch von Ditterdorf verändern.

Der Kontext entsteht bereits während der Auswahl und Zusammenstellung
der Daten, die stets durch eine bestimmte Perspektive und ein vorher
festgelegtes Ziel determiniert sind. Wissenschaftlich und methodisch ist
dies von großem Interesse, da der Erkenntnisgewinn mehr durch die
Modellierung und weniger durch beliebiges Sammeln von Daten befördert
wird.\footnote{In diesem Zusammenhang kann eine Betrachtung von
  \enquote{Big Data} lohnen, allerdings ist dies für die vorliegende
  Untersuchung nachrangig. Während \enquote{Big Data} aufgrund der
  übermäßigen Menge den Inhalt oder die Struktur der Daten nicht im
  Einzelnen berücksichtigt und auf alternative Methoden der Auswertung
  zielt, überwiegt in geisteswissenschaftlich ausgerichteten Datenbanken
  die Bedeutung der konkreten konzeptionellen Anlage. Christof Schöch
  unterschied in diesem Sinne zwischen \enquote{Big Data} und
  \enquote{Smart Data} und plädiert für eine Symbiose beider Qualitäten:
  \enquote{only smart big data enables intelligent quantitative methods}
  (Schöch 2013).} In den oben erwähnten Beispielen BMLO und ÖML bestünde
dieses in der Fokussierung auf die Bedeutung des Komponisten Dittersdorf
für Bayern respektive Österreich.

Umgekehrt gesehen bewirkt schon die Zuordnung einer Datenmenge zu einem
bestimmten Vorhaben eine Kontextualisierung. So ist aus der Tatsache,
dass Dittersdorf im BMLO und im ÖML vertreten ist, ableitbar, dass der
Lebensweg des Komponisten offenbar in einer Verbindung zu Bayern und
Österreich stand; ebenso ist damit jede einzelne Information aus der
Biographie vor diesem Hintergrund interpretierbar. Auf diese Weise
bleiben Daten dem Kontext ihrer Entstehung verbunden.

Insofern ist das Erfassen von Daten im Idealfall ein von
wissenschaftlicher Erkenntnisabsicht geleiteter Prozess, der stets die
spezifische Ausrichtung eines Vorhabens transportiert. Die
Zielperspektive der Datenerfassung wohnt damit den Daten inne und
bestimmt die Wertigkeit der einzelnen Informationen.

Bei einer Migration von Daten entsteht somit zum einen unweigerlich eine
Vermischung ursprünglich differierender Kontexte, zum anderen verschiebt
sich die Relevanz der Informationen untereinander und die Integrität des
Datenkorpus verwässert. Daher ist bei einer Migration der ursprüngliche
Kontext der Daten stets zu berücksichtigen.

\subsection*{Narratives Potenzial und spezifische
Terminologien}\label{narratives-potenzial-und-spezifische-terminologien}

Der theoretische Ansatz des Personendaten-Repositoriums,\footnote{Personendaten-Repositorium,
  \href{http://pdr.bbaw.de/}{\emph{http://pdr.bbaw.de}} (24.05.2016).}
einer digitalen Infrastruktur für Personendaten (auf der sowohl MUSICI
als auch MusMig basieren), geht davon aus, dass beim biographischen
Arbeiten durch Prozesse der Informationsselektion spezifische
\enquote{Narrationen} kreiert werden.\footnote{\enquote{Biographisches
  Arbeiten ist also ein konstruierender Prozess, der aus der
  Mannigfaltigkeit an vorhandenen Informationen und Daten schöpft}
  (Walkowski 2011).} Durch Veränderung der Sichtweise auf die Daten sind
wiederum neue Narrationen erzeugbar: So könnten prosopographische Daten
etwa nach chronologischen, geographischen oder sachbezogenen Aspekten
geordnet werden. Auch die Auswahl müsste sich nicht zwangsläufig auf die
Person als zentrales Narrationsobjekt beschränken, sondern könnte ebenso
auch von einer bestimmten Berufsgruppe ausgehen und alle dazu
verfügbaren Informationen zu einer neuen Narration formen.

Beispielsweise könnten aus einer Musiker-Datenbank alle Informationen
zur Person \enquote{Dittersdorf} extrahiert und zu einer chronologischen
Biographie des Komponisten angeordnet werden. Dieses wäre die klassische
biographische Narration. Eine tiefe semantische Erschließung der Daten
vorausgesetzt, könnten der Datenbank auch alle Informationen zum Ort
\enquote{Bologna} entnommen und zu einer lokalen Musikgeschichte
zusammengestellt werden (selbstverständlich unter der Berücksichtigung
des allgemeinen Erfassungskontextes der Daten). Die Auswahl der Daten
kann anhand der Datenbank-Systematik parametrisiert werden,
beispielsweise auf Klavierbauer in Wien\footnote{Beispielrecherche im
  BMLO:
  \url{http://bmlo.de/Q/Musikalische_Tätigkeit=Klavierbauer/Wirkungsort=Wien}
  (24.05.2016).} oder auf Aktivitäten französischer Sänger in Rom
zwischen 1650 und 1750.\footnote{Beispielrecherche in der
  MUSICI-Daten\-bank:
  \url{http://www.musici.eu/?id=90\&content=franc*\&contentTag=provenienza\&place=Roma\&placeTag=Permanenza}
  (24.05.2016).} Diesem Prinzip folgend ist anzunehmen, dass jede
Datenbank ein fast unerschöpfliches Potenzial von Narrationen birgt,
welches lediglich durch die Tiefe der semantischen Erschließung und
durch die Zugriffsmöglichkeiten limitiert ist. Hierin liegt das
\enquote{narrative Potenzial} einer Datenbank.

Hinter einer semantischen Erschließung verbergen sich in der Regel
hochspezialisierte Fachterminologien. So ist die Suche nach
\enquote{Lautenmachern} oder \enquote{Kastraten} am ehesten in
musikwissenschaftlichen Datenbanken möglich, während sich Termini wie
\enquote{Zellularpathologie} oder \enquote{Verein sozialistischer Ärzte}
vorrangig in medizinhistorischen Datenbanken anwenden lassen. Diese
meist streng kontrollierten Vokabulare spiegeln die
\enquote{Weltordnung} einer Datenbank und spielen gleichzeitig eine
bedeutende Rolle für deren narratives Potenzial. Bei einer Vermischung
von Daten infolge einer Migration kommt es zum einen zu einer
Veränderung des narrativen Potenzials (es wird größer, aber auch
beliebiger), zum anderen kommt es gegebenenfalls zu einer Vermengung der
Terminologien, sofern diese nicht miteinander kompatibel sind oder
angepasst werden. Diesem Umstand ist bei einer Datenmigration Rechnung
zu tragen.

\subsection*{Konsequenzen für die
Datenmigration}\label{konsequenzen-fuxfcr-die-datenmigration}

In Datenbanken spielen die Parameter Kontext, Datenkorpus und
Terminologie bedeutende Rollen und müssen bei einer Datenmigration wie
folgt berücksichtigt werden:

\begin{itemize}
\item
  Der abstrakte Kern der Information -- also basale Parameter wie Zeit,
  Raum und Identitäten -- dürfen nicht verändert werden.
\item
  Der allgemeine Kontext einer Datenbank bestimmt die Erfassungsvorgänge
  und damit auch die spätere Einordnung der Daten. Es muss daher
  transparent bleiben, in welchem Kontext die Daten ursprünglich erfasst
  wurden.
\item
  Das Datenkorpus als Ganzes bestimmt die Relevanz einzelner
  Informationen. Bei einer Datenmigration treffen die Parameter der
  Ausgangs- und Zieldatenbank aufeinander und beeinflussen sich in einem
  wechselseitigen Prozess: Nicht nur die Daten der Ausgangsdatenbank
  verändern sich in ihrer neuen Umgebung, sondern auch die Zieldatenbank
  verändert sich durch eine Integration neuer Daten. Daher ist dafür zu
  sorgen, dass die Integrität der jeweiligen Korpora erhalten bleibt.
\item
  Die spezifischen Terminologien von Datenbanken repräsentieren deren
  jeweilige \enquote{Weltordnung} und beeinflusst die Nutzbarkeit auf
  semantischer Ebene. Sie müssen bei einer Migration den neuen
  Strukturen so angepasst werden, dass sie in mit den neuen Daten
  zusammen interagieren können und dass gleichzeitig die ursprünglichen
  Strukturen nachvollziehbar und nutzbar bleiben.
\end{itemize}

Der letzte Punkt -- die Umstellung von Terminologien -- bildet das
komplexeste Feld und soll daher eine kurze Ausführung erfahren. Hier ist
ein Mapping zu entwickeln, welches die alte Terminologie (T1) auf die
neue Terminologie (T2) abbildet. Sofern nicht in beiden Datenbanken ein
ontologisches Referenzsystem wie zum Beispiel CIDOC-CRM\footnote{CIDOC
  Conceptual Reference Model, \url{http://cidoc-crm.org/} (24.05.2016).}
verwendet wird, welches einen automatischen Abgleich ermöglichen würde,
muss ein neues Mapping entworfen werden. Dabei können zwischen den
einzelnen Begriffen der Terminologien folgende Differenzen vorliegen:

\begin{itemize}
\item
  Mehrere Termini in T1 entsprechen einem Terminus in T2
  (Generalisierung). Die betroffenen Daten müssen zusammengelegt werden.
\item
  Ein Terminus in T1 entspricht mehreren Termini in T2
  (Spezialisierung). Die betroffenen Daten müssen inhaltlich analysiert
  werden, um eine Aufteilung vorzunehmen.
\item
  Ein Terminus in T1 entspricht nur teilweise einem Terminus in T2
  (Inkongruenz). Die betroffenen Daten müssen inhaltlich analysiert
  werden, um Aufteilungen und Zusammenlegungen vorzunehmen.
\item
  Ein Terminus in T1 entspricht keinem Terminus in T2 (Exklusion). Es
  muss entschieden werden, ob und in welcher Form die betroffenen Daten
  erhalten bleiben sollen.
\end{itemize}

Insgesamt können Datenmigrationen eine Vielzahl von inhaltlichen
Auswirkungen auf verschiedensten Ebenen bewirken. Um den möglichen
Konsequenzen eines Migrationsvorgangs eine höhere Transparenz zu
verleihen, empfiehlt sich eine Dokumentation der Datenherkunft in der
Zieldatenbank, die auch dem Nutzer kommuniziert wird. Dabei sollten
Kontext, Korpus und Terminologie der Ausgangsdatenbank erkennbar
bleiben. Eine undokumentierte Integration von Daten würde eine
Verzerrung der Informationen bedeuten, die möglicherweise die
Datenkorpora wissenschaftlich wertlos und unbrauchbar macht. Daher ist
vor einer Datenübernahme zu untersuchen, welche Folgen diese haben
könnte und welche Aspekte dokumentiert werden müssen, um die Nutzbarkeit
der Daten nicht zu gefährden. Eine Datenmigration wird kaum ohne
Veränderungen der Ursprungsdaten möglich sein. Im besten Falle würden
die Daten sowohl in dem alten als auch in dem neuen Kontext
gleichermaßen gut operieren und innerhalb des neuen Kontextes als
Untermenge abgrenzbar bleiben.

Die genannten Punkte berühren mehrere Kernaspekte des
Migrationsbegriffes: Die Einpassung in die neue Umgebung und das Ablegen
alter Strukturen, das Einbringen eigener Strukturen in die neue Umgebung
und die Bewahrung von Integrität der Daten. Migrierende Daten sind dabei
nicht selbständig wie Individuen, jedoch begeben auch sie sich
gewissermaßen auf Wanderschaft und gelangen in einen im übertragenen
Sinne vergleichbaren, wechselseitigen Integrationsprozess.

\section*{Von MUSICI nach MusMig}\label{von-musici-nach-musmig}

Anhand des Fallbeispiels der Datenmigration von MUSICI nach MusMig
sollen die theoretischen Ansätze einem konkreten Anwendungsfall
gegenübergestellt werden. Die beiden Projekte liegen grundsätzlich
aufgrund ihrer Thematik, ihrer zeitlichen und räumlichen Abdeckung und
ihrer Methodik nah beieinander. Der Fokus liegt beiderseits auf
Kulturprozessen, die durch Wanderungsbewegungen von Musikern angetrieben
werden. Die Datenbanken enthalten vorrangig biographische Informationen
zu Musikern und dokumentieren die beruflichen Aktivitäten und
Beziehungen der Musiker an verschiedenen Orten. Kontrollierte Vokabulare
und Referenzsysteme helfen bei der systematischen Erfassung von Orten,
Institutionen, Berufen, Beziehungen und Sachbegriffen.

\subsection*{Datenmodell}\label{datenmodell}

Beide Datenbanken basieren auf der Architektur des
Personendaten-Repositoriums der BBAW, deren Datenstruktur den folgenden
strukturellen Regeln gehorcht:\footnote{Personendaten-Repositorium:
  Datenmodellierung,
  \url{http://pdr.bbaw.de/projekt/vorstellung/datenmodellierung}
  (24.05.2016).}

\begin{itemize}
\item
  \emph{Aspekte und Personen:} Die basale Dateneinheit besteht in diesem
  Modell aus einer einzelnen Information zu einer Person, hier Aspekt
  genannt. Es kann sich dabei sowohl um Lebensereignisse (wie Geburt,
  Heirat, Ausbildungsabschluss) als auch um Personeneigenschaften
  handeln (wie Namen, Zugehörigkeiten). Jeder Aspekt ist mindestens
  einer Person zuzuordnen.
\item
  \emph{Markup:} Die Information jedes einzelnen Aspektes kann mithilfe
  eines Markup-Vokabulars semantisch angereichert werden. Möglich sind
  Verweise auf Orte, Personen, Körperschaften, Sachbegriffe und
  kalendarische Daten. Das Vokabular kann, abgesehen von wenigen
  Basisbegriffen, frei definiert werden. Es ist hierarchisch in vier
  Ebenen gegliedert, wobei die unteren, detaillierteren Ebenen die
  oberen, allgemeineren inkludieren.
\item
  \emph{Kategorien:} Jeder Aspekt muss mindestens einer biographischen
  Kategorie (zum Beispiel Werdegang, Familienereignisse, Reisen)
  zugeordnet werden. Die Kategorien können, abgesehen von wenigen
  Basiskategorien, durch ein Projekt frei definiert werden.
\item
  \emph{Quellen:} Jeder Aspekt muss mit einer bibliographischen Angabe
  belegt werden.
\item
  \emph{Projekte:} Eine Instanz des Repositoriums kann beliebig viele
  Projekte enthalten. Aspekte, Personen und Quellen gehören jeweils zu
  genau einem Projekt.
\end{itemize}

Aufgrund der identischen Systeme erscheint eine Migration der Daten von
MUSICI nach MusMig technisch gesehen relativ leicht.

\subsection*{Betroffene Bereiche}\label{betroffene-bereiche}

Schwerwiegender sind die inhaltlichen Konsequenzen der Datenmigration,
denn trotz sehr ähnlicher Ausgangsbedingungen bestehen zwischen den
Projekten Unterschiede, welche eine Übernahme der Daten nicht ohne
Modifikationen ermöglichen. Die Unterschiede schlagen sich in diesen
Bereichen nieder:

\begin{itemize}
\item
  Projektkontext
\item
  Erfassungssprache
\item
  Biographische Kategorien
\item
  Kontrollierte Vokabulare
\item
  Referenzsyteme
\end{itemize}

In den folgenden Abschnitten wird exemplarisch untersucht, welche
Maßnahmen konkret durchgeführt werden müssten, um ein harmonisches
Zusammenspiel beider Datenmengen zu gewährleisten. Abschließend wird
behandelt, welche Konsequenzen aus diesen Erkenntnissen für nachfolgende
Datenbankprojekte zu ziehen sind, für die eine Integration der
MusMig-Daten infrage käme.

\subsubsection*{Projektkontext}\label{projektkontext}

Bereits die Unterschiede in der geographischen und chronologischen
Ausrichtung der Projekte beeinflussen Wahrnehmung und Bewertung der
Informationen. Während MUSICI auf ausländische Musiker in Venedig, Rom
und Neapel zwischen 1650 und 1750 einging,\footnote{Eine Besprechung der
  MUSICI-Daten\-bank findet sich bei Berti/Roeder 2015.} erforscht MusMig
ein breiteres geographisches Spektrum im Zeitraum des 17. und 18.
Jahrhunderts mit Schwerpunkt auf Osteuropa.\footnote{Die
  unterschiedlichen Ausrichtungen von MUSICI und MusMig wurden
  detaillierter bei Over/Roeder 2015 erörtert.} Könnte eine in MUSICI
vertretene Person den Kontext einer auf Italien zielenden Migration
hinüber in die MusMig-Datenbank transportieren, wäre dies ein Gewinn an
Information; gleichzeitig wäre sie die damit von dem
Osteuropaschwerpunkt abgrenzbar.

Die Umsetzung einer solchen \enquote{Kontext-Migration} könnte
unterschiedlichste Gestalten annehmen. Ohne weiteres vorstellbar wäre
eine zusätzliche Quellenangabe oder auch ein zusätzliches Aspekt-Objekt;
zusätzlich würde eine grafische Hervorhebung die Transparenz der
Datenherkunft befördern.

Gleichzeitig sollte der Nutzer die Möglichkeit bekommen, die
Datenkorpora getrennt voneinander zu nutzen. Dies ist im
Personendaten-Repositorium realisierbar, da Daten nach Projekten
trennbar sind, aber dennoch auf einer gemeinsamen Oberfläche erscheinen
können. Die Zusammenfassung von Personen, die in mehreren Projekten
auftaucht, geschieht mittels Normdateien oder projektinternen
Identifikatoren.

\subsubsection*{Erfassungssprache}\label{erfassungssprache}

Da bei MUSICI alle Daten aus italienischen Quellen stammen, wurden die
Daten direkt in italienischer Sprache erfasst. Zudem war Italienisch die
\emph{lingua franca} des Projekts und eignete sich für die Kommunikation
besser als Englisch. Bei MusMig hingegen ist die Quellenlage disparater,
weshalb dort die Informationen in Kurzform auf Englisch erfasst werden
und die Originaltexte zusätzlich als Referenz geführt werden. Es ist in
diesem Falle davon auszugehen, dass eine Übersetzung der Inhalte nicht
geleistet werden kann. Insofern wird der Sprachunterschied erhalten
bleiben und auf mehreren Ebenen erheblichen Einfluss ausüben: Zum einen
werden zentrale Funktionen der Datenbank wie zum Beispiel Volltextsuche
schwerer handhabbar. Der Nutzer muss darauf vorbereitet werden, dass er
mit zwei Sprachen operieren oder ansonsten auf sprachunabhängige
Suchfunktionen ausweichen muss, wenn er gleichzeitig in beiden
Datenmengen suchen möchte. Ein interessanter Nebeneffekt besteht darin,
dass die italienische Sprache der MUSICI-Daten in der MusMig-Datenbank
der Hervorhebung dessen dienlich sein wird, dass es sich um Daten aus
einem anderen Kontext handelt. Als einfache technische Überbrückung
empfiehlt sich eine Kollationsliste für die wichtigsten und häufigsten
Termini. Zum anderen müssen die biographischen Kategorien und das
Markup-Vokabular sprachlich angepasst werden. Bei den biographischen
Kategorien kann dies bewerkstelligt werden, ohne die ursprünglichen
Kategorien zu entfernen, da es erlaubt ist, mehrere Kategorien aus
unterschiedlichen Projekten parallel zu führen. Bei Markup-Vokabular
jedoch funktioniert dies nicht. Sollen die Daten gemeinsam miteinander
funktionieren, muss hier Anpassungsarbeit geleistet werden.

Informationstechnisch gesehen erscheinen mehrsprachige Inhalte
suboptimal. Allerdings repräsentieren die multilingualen Daten die
kulturelle Breite des Forschungsgegenstandes viel unmittelbarer als
einheitssprachliche Daten. Hier kommt ein interkulturelles Potenzial zum
Vorschein, welches bislang wenig untersucht wurde. Im Hinblick auf sich
immer enger und globaler vernetzende Datenkorpora erscheint eine
Vielfalt von Sprachen auf längere Sicht unumgänglich und bedeutet
informationstechnisch eine große Herausforderung. Gleichzeitig steht sie
aber in einem globalgesellschaftlich und wissenschaftlich völlig
akzeptablen Licht.

\subsubsection*{Biographische Kategorien}\label{biographische-kategorien}

Die folgende Übersicht zeigt die biographischen Kategorien beider
Projekte im direkten Vergleich (die Reihenfolge entspricht der aktuellen
systematischen Auflistung der Projekte):

\begin{longtable}[]{@{}ll@{}}
\toprule
\textbf{MUSICI} & \textbf{MusMig}\tabularnewline
\midrule
\endhead
01 Nome di norma & 16 Normalized name\tabularnewline
02 Nomi (altri) & 17 Name variants\tabularnewline
03 Descrizione principale & 18 Biographical data\tabularnewline
04 Dati biografici & 19 Relatives\tabularnewline
05 Provenienza & 20 Affiliations\tabularnewline
06 Permanenza & 21 Profession/function\tabularnewline
07 Viaggi & 22 Training/schooling\tabularnewline
08 Manoscritti & 23 Employments\tabularnewline
09 Attività musicali & 24 Residences/sojourns/travels\tabularnewline
10 Altre attività & 25 Professional/social contacts\tabularnewline
11 Contatti con colleghi & 26 Works/products\tabularnewline
12 Contatti con mecenati & 27 Reception\tabularnewline
13 Contatti con maestri di musica & 28 Miscellaneous\tabularnewline
14 Altri contatti & 29 Commentary (public)\tabularnewline
15 Informazioni famigliari & 30 Commentary (private)\tabularnewline
\bottomrule
\end{longtable}

In der Gegenüberstellung der Terminologien treten die konzeptionellen
Unterschiede und die Interessenschwerpunkte beider Projekte deutlich
hervor:

\begin{itemize}
\item
  Die MUSICI-Kategorie \enquote{Herkunft} (05) geht bei MusMig in
  \enquote{Zugehörigkeiten} (20) auf. MusMig ordnet dieser Kategorie
  jedoch nicht nur ethnische Herkunft, sondern zum Beispiel auch die
  Konfession zu.
\item
  Die MUSICI-Kategorien \enquote{Aufenthalt} (06) und \enquote{Reisen}
  (07) finden am ehesten in der MusMig-Kategorie
  \enquote{Wohnsitz/Aufenthalte/Reisen} (24) ihre Entsprechungen, wobei
  das italienische \enquote{permanenza} (dauerhafter Aufenthalt) nicht
  dem englischen \enquote{sojourn} (kurzer Aufenthalt) entspricht.
\item
  MUSICI führt die Kategorie \enquote{Manuskripte} (08), welche MusMig
  überhaupt nicht verwendet.
\item
  MUSICI differenziert bei beruflichen Aktivitäten, die in
  \enquote{musikalische Aktivitäten} (09) und \enquote{sonstige
  Aktivitäten} (10) unterteilt sind, sowie bei Kontakten, die in vier
  Untergruppen \enquote{Kollegen} (11), \enquote{Mäzene} (12),
  \enquote{Lehrer} (13) und \enquote{sonstige} (14) differenziert
  werden. MusMig generalisiert in beiden Fällen zu
  \enquote{Dienstverhältnissen} (23) beziehungsweise
  \enquote{beruflichen und sozialen Kontakten} (25). Die
  \enquote{Lehrerkontakte} (13) fielen bei MusMig allerdings unter
  \enquote{Ausbildung} (22).
\item
  MusMig führt eine Kategorie für \enquote{Werke} (26),
  \enquote{Rezeption} (27), \enquote{Verschiedenes} (28) und zwei
  Kategorien für \enquote{interne} und \enquote{öffentliche Kommentare}
  (29, 30).
\end{itemize}

Offenbar sind die MUSICI-Kategorien tendenziell feiner ausdifferenziert
als die MusMig-Ka\-te\-go\-rien. Dies entspricht auch der jeweiligen
inhaltlichen Tendenz der Projekte: MUSICI kann sich aufgrund der
Konzentration auf Rom, Venedig und Neapel mit feineren Strukturen wie
zum Beispiel dem Mäzenatentum befassen, während MusMig durch den
regional weit übergreifenden Kontext allgemeiner arbeiten muss.

Da das Personendaten-Repositorium im Falle der biographischen Kategorien
erlaubt, mehrere Kategoriensysteme parallel zu führen, können die
MUSICI-Kategorien beibehalten und weiter genutzt werden, unabhängig
davon, welche MusMig-Kategorie ihnen zugewiesen ist. Dies ist für die
Nachnutzbarkeit der Daten von sehr großem Vorteil, da auf diese Weise
der ursprüngliche Erfassungskontext mittransportiert werden kann und
transparent bleibt.

\subsubsection*{Kontrollierte Vokabulare}\label{kontrollierte-vokabulare}

Auch dieser Bereich fällt in die Kategorie der terminologischen
Umstellungen. Anders als bei den biographischen Kategorien verhält es
sich bei den kontrollierten Vokabularen, die im Datenmodell des
Personendaten-Repositoriums auf der Markup-Ebene eingesetzt werden.
Jeder markierten Textstelle kann genau ein Begriff (oder Unterbegriff)
eines Vokabulars zugeordnet werden.\footnote{Theoretisch ermöglicht es
  das Datenschema, die Markup-Attribute zu tokenisieren und außerdem
  Namespaces zu vergeben, so dass prinzipiell auch mehrere Vokabulare
  parallel verwendet werden könnten (z.~B.
  @subtype=\enquote{musmig:kingdom~musici:regno}). Dies ist jedoch im
  gedanklichen Ansatz des Modells nie formuliert worden, wurde nie
  getestet und wird von den Software-Komponenten in keiner Weise
  unterstützt. Im Hinblick auf die praktische Nutzung ist dies also kein
  gangbarer Weg; für zukünftige Datenmodelle könnte dies jedoch eine
  Alternative darstellen.} Damit ist es technisch nicht möglich, die
Markup-Vokabulare von MUSICI und MusMig parallel zu führen.\footnote{Vgl.
  \emph{Personendaten-Repositorium, Schema documentation for aodl.xsd},
  \url{https://pdrprod.bbaw.de/wiki/lib/exe/fetch.php?media=de:schema:aodl.pdf}
  (24.05.2016), S. 18.} Entsprechend muss das MUSICI-Vokabular entweder
erhalten bleiben oder an das MusMig-Vokabular angepasst werden. Eine
dritte Möglichkeit besteht in der Erstellung eines gemischten
Vokabulars, welches auf den oberen hierarchischen Ebenen die Begriffe
von MusMig adaptiert, auf den unteren, differenzierteren Ebenen hingegen
die Begriffe von MUSICI beibehält.

Zwei weitere Schwachpunkte des PDR-Datenmodells erschweren die
reibungslose Migration der Daten. So kann erstens nicht festgehalten
werden, nach welchem Vokabular die Auszeichnung erfolgt. Man kann
lediglich mit der Annahme operieren, dass sich alle Daten mit einer
bestimmten Projektnummer an ein vorher vereinbartes Vokabular halten.
Zweitens erfolgt bei Änderungen des Vokabulars weder eine automatische
Angleichung der Daten, noch werden die Daten gegen das Vokabular
validiert, wodurch im Verlaufe eines Projektes markante Inkonsistenzen
in den Daten entstehen können. Infolgedessen gibt es keine Garantie,
dass die Daten einem bestimmten Vokabular gehorchen. Ein Test der
MUSICI-Daten ergab, dass ein drittes, undokumentiertes, lediglich
implizit vorhandenes Vokabular besteht, welches aus den Daten heraus
rekonstruierbar ist. Vor dem Transfer nach MusMig wäre dieses mit dem
offiziellen Vokabular abzugleichen.

Vom Umfang her sind die Terminologien von MUSICI und MusMig mit jeweils
300-350 Begriffen vergleichbar. Für eine detaillierte Gegenüberstellung
ist in dieser Untersuchung nicht der Platz, jedoch seien exemplarisch
folgende charakteristische Unterschiede erwähnt:

\begin{itemize}
\item
  MUSICI führt eine fein ausdifferenzierte Terminologie zur Beschreibung
  von Organisationen, die insgesamt 80 Begriffe und Eigennamen umfasst,
  während MusMig hier mit 20 Begriffen in einer flachen Hierarchie
  auskommt.
\item
  MusMig differenziert sehr stark bei der Erfassung der Sprachvarianten
  von Personennamen, während bei MUSICI die Unterscheidung zwischen
  Geburtsname und italianisierter Form genügte.
\item
  MusMig stellt ein deutlich komplexeres Vokabular zur Beschreibung von
  Personenbeziehungen bereit; hier sind zum Beispiel auch Freundschafts-
  oder Handelsbeziehungen erfassbar, während bei MUSICI Familien- und
  Förderbeziehungen im Vordergrund standen.
\item
  MUSICI führt Vokabulare sowohl für Instrumente als auch für
  Instrumentalistenberufe, während MusMig lediglich die Berufe
  berücksichtigt.
\end{itemize}

Diese Aufzählung könnte man noch weiter fortsetzen. Unter dem Strich
steht, dass MUSICI in vielen Punkten differenzierter arbeitet, während
MusMig häufig eine vereinfachte Terminologie bevorzugt, die zudem
weniger hierarchisch operiert.

\subsubsection*{Referenzsysteme}\label{referenzsysteme}

Eine beschwerliche Angelegenheit ist die Übertragung des geographischen
Referenzsystems, welches bei MUSICI völlig anders als bei MusMig
gehandhabt wird. Während MUSICI das Markup-Vokabular für eine Normierung
der Namensformen nutzt,\footnote{Bei MUSICI liegt eine inkonsistente
  Verwendung des Datenmodells vor, da innerhalb des Hierarchiebaumes von
  beschreibenden Begriffen zu Eigennamen gewechselt wird. Dies ist bei
  einer überschaubaren Menge von Namen noch handhabbar, jedoch hätte
  dieses Vorgehen bei MusMig den Rahmen des Vokabulars gesprengt. Aus
  diesem Grund entschied man sich bei MusMig für eine Verwendung eines
  allgemeinen Georeferenzierungssystems.} verwendet MusMig die
Identifikatoren von Geonames\footnote{GeoNames
  \url{http://www.geonames.org/} (24.05.2016).} als Referenzsystem. Das
Land \enquote{Portugal} wird bei MUSICI also mit dem italienischen Namen
\enquote{Portogallo} in den Markup-Attributen identifiziert, während
MusMig in einem Referenzattribut auf die GeoNames-ID 2264397\footnote{Siehe
  \url{http://www.geonames.org/2264397/portuguese-republic.html}
  (24.05.2016).} referenziert. Bei der Übertragung nach MusMig müssen
die in MUSICI verwendeten Ortsnamen also manuell mit Geoidentifikatoren
angereichert werden. Auch Familien- und Organisationsnamen sind in dem
MUSICI-Vokabular enthalten und sollten gegebenenfalls auf eine
GND-Referenzierung umgestellt werden.\footnote{Auch die Verwendung des
  Getty Thesaurus of Geographic Names
  \url{http://www.getty.edu/research/tools/vocabularies/tgn/}
  (24.05.2016) wurde erwogen, da hier auch eine Referenzierung auf
  historische Gebiete möglich ist. Allerdings ist die Abdeckung für die
  untersuchten Regionen nicht ausreichend, weshalb die Entscheidung
  zugunsten von GeoNames ausfiel.}

\section*{Zusammenfassung}\label{zusammenfassung}

Durch die Untersuchung wurde deutlich, dass der Vorgang einer
Datenmigration weniger auf der technischen Ebene zu betrachten ist,
sondern hierbei vielmehr die inhaltlichen Konsequenzen im Mittelpunkt
stehen müssen. Dabei spielen sowohl Kontexte, Sprachen und Terminologien
also auch Referenzsysteme eine Rolle. Am problematischsten erscheint die
Umstellung von Terminologien, da diese am stärksten die
Forschungsinhalte einer Datenbank semantisch strukturieren. Das Beispiel
von MUSICI und MusMig zeigt, dass hier zahlreiche Umstellungen notwendig
wären, um die MUSICI-Daten innerhalb der MusMig-Datenbank nutzbar zu
machen.

Die Vorstellung von einer einseitigen Datenanpassung erscheint
angesichts der vielen Faktoren als irrtümlich. Stattdessen sollte die
Aufgabe darin gesehen werden, die Erfassungskontexte von Daten deutlich
transparenter zu gestalten und die Herkunft von Daten genauer zu
dokumentieren. Es gilt, die Übertragungsprozesse sehr bewusst
durchzuführen, vor allem in den Digital Humanities, aber ebenso
allgemein in den Informationswissenschaften. Bei der Umsetzung erwiesen
sich insbesondere parallel führbare Terminologien und allgemeine
Referenzsysteme als hilfreich und erscheinen auch im Hinblick auf
weitere Datenmigrationen als sinnvoll.

Auch wenn die Reise eines Musikers im 17. Jahrhundert nach heutigen
Maßstäben beschwerlich gewesen sein mag, so lag auch damals in der
Überwindung einer räumlichen Distanz nicht die eigentliche
Herausforderung der Migration. Diese ist stattdessen vorrangig in der
Auseinandersetzung der hergebrachten Strukturen mit den Strukturen der
neuen Umgebung zu sehen. Dabei bleibt einiges erhalten oder verändert
sich nur geringfügig, während sich anderes sehr grundsätzlichen
Wandlungen unterzogen wird. Dieser Prozess vollzieht sich an jeder
Station von neuem; der Wandernde bleibt jedoch in seiner Integrität
unverändert.

\section*{Bibliographie}\label{bibliographie}

\textless{}Berti/Roeder 2015\textgreater{} Michela Berti; Torsten
Roeder: \emph{The »Musici« Database. An Interdisciplinary Cooperation}.
In: Europäische Musiker in Venedig, Rom und Neapel. Les musiciens
européens à Venise, Rome et Naples. Musicisti europei a Venezia, Roma e
Napoli. Deutsch-italienische Round-Table-Gespräche (= Analecta
musicologica 52), hrsg. von Anne-Madeleine Goulet und Gesa zur Nieden,
Kassel u.~a.: Bärenreiter, 2015, S. 633--645.

\textless{}Brinkmann/Wolff 1999\textgreater{} Reinhold Brinkmann;
Christoph Wolff: \emph{Driven into Paradise. The musical migration from
Nazi Germany to the United States}, Berkeley/London: University of
California Press, 1999.

\textless{}Ehrmann-Herfort/Leopold 2010\textgreater{} Sabine
Ehrmann-Herfort; Silke Leopold (Hrsg.): \emph{Migration und Identität.
Wanderbewegungen und Kulturkontakte in der Musikgeschichte} (= Analecta
musicologica 49), Kassel u.~a.: Bärenreiter, 2013.

\textless{}MAiA 2011/2\textgreater{} \emph{Special Issue: Music and
Migration} = Music and Arts in Action, Volume 3, Issue 3 (2011/2),
\url{http://musicandartsinaction.net/index.php/maia/issue/view/Vol\%203,\%20No\%203}
(24.05.2016).

\textless{}Klettke/Thalheim 2011\textgreater{} Meike Klettke; Bernhard
Thalheim: \emph{Evolution and Migration of Information Systems}. In:
Handbook of Conceptual Modeling, hrsg. von David W. Embley und Bernhard
Thalheim, Heidelberg: Springer, 2011, S. 381--420.

\textless{}Levine 2009\textgreater{} Robert Levine: \emph{Data migration
strategies}. In: Wikibon, 2009,
\url{http://wikibon.org/wiki/v/Data_migration_strategies} (24.05.2016).

\textless{}Over/Roeder 2015\textgreater{} Berthold Over; Torsten Roeder:
\emph{MUSICI und MusMig. Kontinuitäten und Diskontinuitäten}. In:
Zeitschrift für Digital Humanities, Sonderband 1,
\url{http://dx.doi.org/10.17175/sb001_017} (24.05.2016).

\textless{}Schöch 2013\textgreater{} Christoph Schöch: \emph{Big? Smart?
Clean? Messy? Data in the Humanities}. In: Journal of Digital
Humanities, Vol.~2, No.~3 (Summer 2013),
\url{http://journalofdigitalhumanities.org/2-3/big-smart-clean-messy-data-in-the-humanities/}
(24.05.2016).

\textless{}Seacord et al. 2003\textgreater{} Robert C. Seacord; Daniel
Plakosh; Grace A. Lewis: \emph{Modernizing Legacy Systems: Software
Technologies, Engineering. Process and Business Practices}. Boston:
Addison-Wesley Longman, 2003.

\textless{}Thalheim/Wang 2011\textgreater{} Bernhard Thalheim; Qing
Wang: \emph{Towards a Theory of Refinement for Data Migration}. In:
Lecture Notes in Computer Science 6998 (= Conceptual Modeling -- ER
2011), Heidelberg: Springer, 2011, S. 318--331.

\textless{}Thalheim/Wang 2013\textgreater{} Bernhard Thalheim; Qing
Wang: \emph{Data migration: A theoretical perspective}. In: Data \&
Knowledge Engineering 87 (Sept.~2013), S. 260--278.

\textless{}Wagner 2014\textgreater{} Christian Wagner (Hrsg.):
\emph{Model-Driven Software Migration: A Methodology. Reengineering,
Recovery and Modernization of Legacy Systems}, Wiesbaden: Springer,
2014.

\textless{}Walkowski 2009\textgreater{} Niels-Oliver Walkowski:
\emph{Das Konzept einer polysemischen Datenbank und seine
Konkretisierung im Personendaten-Repositorium der BBAW}. In: Jahrbuch
für Computerphilologie online, 2011
\url{http://computerphilologie.digital-humanities.de/jg09/walkowski.html}
(24.05.2016).

%autor
\begin{center}\rule{0.5\linewidth}{\linethickness}\end{center}

\textbf{Torsten Roeder} studierte Musikwissenschaft und Italienisch in
Hamburg, Rom und Berlin. Im Anschluss betreute er eine Reihe von digital
orientierten Projekten an der Berlin-Brandenburgischen Akademie der
Wissenschaften sowie in freiberuflicher Tätigkeit. Seit 2014 arbeitet er
als wissenschaftlicher Mitarbeiter im Projekt „Richard Wagner
Schriften`` am Institut für Musikforschung der Universität Würzburg.

\end{document}
