\documentclass[a4paper,
fontsize=11pt,
%headings=small,
oneside,
numbers=noperiodatend,
parskip=half-,
bibliography=totoc,
final
]{scrartcl}

\usepackage{synttree}
\usepackage{graphicx}
\setkeys{Gin}{width=.4\textwidth} %default pics size

\graphicspath{{./plots/}}
\usepackage[ngerman]{babel}
\usepackage[T1]{fontenc}
%\usepackage{amsmath}
\usepackage[utf8x]{inputenc}
\usepackage [hyphens]{url}
\usepackage{booktabs} 
\usepackage[left=2.4cm,right=2.4cm,top=2.3cm,bottom=2cm,includeheadfoot]{geometry}
\usepackage{eurosym}
\usepackage{multirow}
\usepackage[ngerman]{varioref}
\setcapindent{1em}
\renewcommand{\labelitemi}{--}
\usepackage{paralist}
\usepackage{pdfpages}
\usepackage{lscape}
\usepackage{float}
\usepackage{acronym}
\usepackage{eurosym}
\usepackage[babel]{csquotes}
\usepackage{longtable,lscape}
\usepackage{mathpazo}
\usepackage[normalem]{ulem} %emphasize weiterhin kursiv
\usepackage[flushmargin,ragged]{footmisc} % left align footnote

\usepackage{listings}

\urlstyle{same}  % don't use monospace font for urls

\usepackage[fleqn]{amsmath}

%adjust fontsize for part

\usepackage{sectsty}
\partfont{\large}

%Das BibTeX-Zeichen mit \BibTeX setzen:
\def\symbol#1{\char #1\relax}
\def\bsl{{\tt\symbol{'134}}}
\def\BibTeX{{\rm B\kern-.05em{\sc i\kern-.025em b}\kern-.08em
    T\kern-.1667em\lower.7ex\hbox{E}\kern-.125emX}}

\usepackage{fancyhdr}
\fancyhf{}
\pagestyle{fancyplain}
\fancyhead[R]{\thepage}

%meta
%meta

\fancyhead[L]{A. Hartsch \\ %author
LIBREAS. Library Ideas, 30 (2016). % journal, issue, volume.
\href{http://nbn-resolving.de/
}{}} % urn
\fancyhead[R]{\thepage} %page number
\fancyfoot[L] {\textit{Creative Commons BY 3.0}} %licence
\fancyfoot[R] {\textit{ISSN: 1860-7950}}

\title{\LARGE{Das Hybride als Normalität. Digitaler Wandel und analoge Herkunftswelten}} %title %title
\author{Andreas Hartsch} %author

\setcounter{page}{1}

\usepackage[colorlinks, linkcolor=black,citecolor=black, urlcolor=blue,
breaklinks= true]{hyperref}

\date{}
\begin{document}

\maketitle
\thispagestyle{fancyplain} 

%abstracts
\begin{abstract}
Diskursanalytisch untersucht werden konzeptionelle Verunsicherungen bei
der Medienausrichtung von Bibliotheken, die von einer unsteten, auf
Durchsetzungserfolge bauenden Orientierung an unbewiesenem
Gleichzeitigen oder Zukünftigen herzurühren scheinen. Dabei gerät
Bewiesenes unter Begründungszwänge und Verdrängungsdruck. Paradigmatisch
dafür wird der Satz ausgemacht: `Das Neue ist das Gute', von dem eine
Steigerungsform, nun unter gänzlichem Verzicht auf Qualitätsurteile
existiert: `Das Neue ist das Selbstverständliche'. Das wird hier
hinterfragt mit transdisziplinärem Blick auf Diskussionsbeiträge
hauptsächlich aus 2014 bis 2016. Herausgearbeitet wird ein
informationswissenschaftliches Plädoyer für Mediensymbiosen aller Art,
auf die das hybride Bibliotheksverständnis - in den
Geisteswissenschaften zumal - selbstredend seit Jahrzehnten zu Recht
aufsetzt und dessen Verstetigung dringend empfohlen wird.
\end{abstract}

%body
\begin{quote}
\enquote{In the roiling cauldron of change now being felt by academic
libraries, it would be foolhardy to hazard definitive responses to most
of these questions, though some futures seem more attractive than others
and worth some effort to bring into being.} (2012)\footnote{Fischer,
  Karen et al. (2012) Give 'Em What They Want. In: College and Research
  Libraries. 73(2012)5, S. 492.}
\end{quote}

Allenthalben herrscht prognostisches Unbehagen, permanente
Rekontextualisierung oder diffuse Vorwegnahme des vermeintlich
Kommenden: \enquote{Perpetual beta} als problematisches
Entwicklungsprinzip nach dem Motto: \enquote{Nur was sich ändert,
bleibt}.\footnote{Motto des Bibliothekartags 1998, Frankfurt am Main.}
\enquote{Change Management} scheint zum Dauertrend\footnote{Zu dessen
  Problematisierung vgl. Fuhrmans, Marc (2016) Change Management --
  Mainstream oder unverzichtbarer Werkzeugkasten? In: Perspektive
  Bibliothek, 5.1(2016), S. 3-24.} auch in Bibliotheken geworden zu
sein. Das \enquote{Handbuch Bibliothek} von Umlauf/Gradmann (2012)
systematisiert das Nachdenken über die Zukunft der Bibliothek in
Bibliotheksutopien, Bibliotheksideale, Bibliotheksprognosen und
Bibliotheksszenarien.\footnote{Umlauf, Konrad (Hg. et al. 2012) Handbuch
  Bibliothek, hier S. 387-390.}

Dieser \enquote{roiling cauldron of change} verbindet sich mit einem
\enquote{Akzelerationismus}, wie ihn nicht nur der US-amerikanische
Dichter Kenneth Goldsmith (2015) wahrnimmt: \enquote{Wir vollführen
atemlos {[}\ldots{}{]} unsere digitalen Rituale
{[}\ldots{}{]}.}\footnote{Goldsmith, Kenneth (2015) Der digitale
  Todestrieb. In: Kulturaustausch 65(2015)2, S. 55.}

Beide Faktoren, permanenter Wechsel in der Form eines sich
verstetigenden Veränderungsmanagements und hohes Tempo,\footnote{Vgl.
  Kotter, John P. (2014) Accelerate. Building strategic agility for a
  faster-moving world. Boston : Harvard, 2014. {[}Dt. Übers. München :
  Vahlen, 2015{]}, S. 3: \enquote{Today any company that isn't
  rethinking its direction at least every few years (as well as
  constantly adjusting to shifting contexts) and then quickly making
  necessary operational changes is putting itself at risk.}} werden
schon seit Jahrzehnten als systemischer Kontext, auch für das
Bibliothekswesen, konstatiert: \enquote{Die Rapidität der Entwicklung
ist zum Signum des technischen Zeitalters geworden, das die ständige
Neuorientierung einschließt.}\footnote{Raabe, Paul (1986) Die Bibliothek
  als humane Anstalt betrachtet, S. 13.}

Verstetigter Wandel und steigendes Tempo als Konstanten der
Modernisierung verbinden sich zudem mit einer Vorwegnahme des
vermeintlich Kommenden, wie sie Uwe Jochum (2007) in seiner
\enquote{Kleinen Bibliotheksgeschichte}, zu recht kritisch,
vermerkt\footnote{Vgl. Jochum, Uwe (2007), Kleine Bibliotheksgeschichte.
  3. Aufl., S. 222.}. Sie spuke schon seit den 70er Jahren als
implizites Telos\footnote{Für die geschichtsphilosophische
  Unerläßlichkeit der Zukunftsoffenheit siehe Hölscher, Lucian (2009)
  Semantik der Leere. Grenzfragen der Geschichtswissenschaft.} durch die
Bibliotheksgeschichte.

Gegenstand und Vision dieses Telos ist jene vorauseilende Vorwegnahme
der ultimativ vernetzten, vollständig auf digitaler Technik basierenden,
gänzlich virtuell gewordenen globalen Datenbank aller Wissensobjekte,
mithin die Volldigitalisierung von Weltkultur. Dies in der Regel gerne
kombiniert mit dem felsenfesten Glauben an das definitive Verschwinden
des Buches gleich morgen oder übermorgen: das Ende von
\enquote{Paperage}\footnote{Sehenswert der Kurzfilm, Gewinner des
  Webvideopreises 2013: Paperage. Online-Ressource: siehe Bibliographie.}.

In solchen Phasen kultureller Unübersichtlichkeit und konzeptioneller
technikinduzierter Desorientierung, so Michael Hagner (2011), würden oft
grundsätzliche Zweifel gesät und es käme zu medialen Heilserwartungen,
die sich vornehmlich auf neue Technologien stützten.\footnote{Vgl.
  Hagner, Michael (2011) Ein Buch ist ein Buch ist ein Buch; keine
  Datei. In: Zintzen, Clemens (Hg. 2011) Die Zukunft des Buches.
  Stuttgart, 2011. S. 49-51.}

So wird \enquote{gemenetekelt}, dass auch dem Berufsstand, der für die
Verwaltung des Wissenscontainers Buch in der Vergangenheit
verantwortlich zeichnete, der Untergang\footnote{\enquote{dead-end-job}
  librarian vgl. Bonte, Achim (2015) Was ist eine Bibliothek? In: ABI
  Technik 35(2015)2, S. 95.} drohe: \enquote{Zutritt zur Cloud hat nur,
wer einen Beitrag zur eigenen Überflüssigkeit leistet.}\footnote{Gödert,
  Winfried (2015), Hashtag Erschließung. Online-Ressource:
  \url{http://eprints.rclis.org/24643/} (27.02.2015). Hier S. 1.}

Es ist unmittelbar einsichtig, dass in einem solchen Kontext das
Nachdenken über ein Phänomen, das zu einer Art von unauffälligem
Residuat im steten Wandel ohne Aufmerksamkeitspotential geworden ist,
das Hybridkonzept, kein leichtes Unterfangen, aber vielleicht ein
durchaus notwendiges darstellt. Dabei hat ein weitgreifendes Verständnis
des Hybridbegriffs durchaus das Potential, umfassend alle Formen und
Begegnungsräume analoger Herkunftswelten mit Digitalisierung abzudecken
und die Normalität ihrer symbiotischen Beziehung aufzuzeigen.

Dies mit Blick auf die Geisteswissenschaften vor dem Hintergrund des
aktuellen Diskussionsstandes zum digitalen Wandel herauszuarbeiten und
argumentativ zu untermauern, ist der Fokus und gleichzeitig die
Beschränkung dieses Versuches einer Diskursanalyse.

\section*{Vorüberlegungen}\label{voruxfcberlegungen}

Ausgangspunkt der Erkundungen ist der Typ der wissenschaftlichen
Bibliothek im Bereich der Geistes- und Kulturwissenschaften. Der Fokus
der Überlegungen liegt auf diesem Bibliothekstyp. Allgemeine Aussagen
zur Bibliothek als solcher, als kultureller Institution werden nur
vereinzelt getroffen. Eine mangelnde Zielansprache des Bibliothekstyps
widerfährt oft in Fachbeiträgen, sei es, dass gar nicht präzisiert wird,
von welchem Bibliothekstyp gesprochen wird, sei es, dass je nach
Fortschreiten der Argumentation mit einer gewissen Beliebigkeit der
Fokus wechselt. Das soll hier vermieden werden. Fokus ist stets der Typ
der geisteswissenschaftlichen Spezialbibliothek und ihr Fachpublikum.

Der digitale Wandel, die digitale \enquote{Revolution} relativiert
diesen Fokus. Der \enquote{Digitale Wandel} muss als das übergeordnete
Phänomen begriffen werden, in dem Forschung und Wissenschaft, mithin die
Gesellschaft als Ganzes im Prozess einer systemischen Transformation
sich befinden.

Bei einem solchen theoriegeleiteten Versuch geht es letztlich um die
Klärung aus bibliothekarischer Sicht, inwieweit der Hybridansatz als
Dauerkonzept unerlässlich sein könnte für die
\enquote{Werkstatt},\footnote{Adolf von Harnack (1905):
  \enquote{Bibliotheken {[}\ldots{}{]} sind Speicher und Werkstätte und
  Instrument der Wissenschaft zugleich.} Zitiert nach Fabian, Bernhard
  (1983) Buch, Bibliothek und geisteswissenschaftliche Forschung, S. 23.
  Jüngst wieder aufgegriffen von Wolfram Horstmann in seiner Rede zur
  Amtseinführung als Direktor der Niedersächsischen Staats- und
  Universitätsbibliothek Göttingen am 24. Juli 2014. Siehe: Bibliothek,
  Forschung und Praxis, 38(2014)3, S. 503-505.} das
\enquote{Laboratorium}\footnote{Vgl. Fabian, Bernhard (1983) Buch,
  Bibliothek und geisteswissenschaftliche Forschung, S. 28f.} der
Geisteswissenschaften. Wäre sogar von einer Notwendigkeit für
textorientierte Buchwissenschaften zu sprechen? Ist es dauerhafter
Ausdruck eines Komplementaritätsprinzips von analogen Herkunftswelten
und digitalen Simulacra? Ist die mediale Andersartigkeit zwischen
Analogem und Digitalem für die Geisteswissenschaften unüberbrückbar,
geht sie hybrid-synergetisch zusammen oder kann sogar von
\enquote{Verschmelzung} gesprochen werden? Kann Dematerialisierung für
textbasierte Wissenschaften die Zukunft sein? Und nicht zuletzt: Müssen
Geisteswissenschaften im medientechnologischen Sinne stärker ihre
Eigenart gegenüber STM-Fächern\footnote{\enquote{Naturwissenschaftler
  haben offensichtlich kein Problem damit, dass ihre Veröffentlichungen
  nur noch digital existieren. Aber warum sollten Geisteswissenschaftler
  sich diesen Habitus zu eigen machen?} Wilfried Sühl-Strohmenger (2016)
  in seiner Besprechung des Buches von Michael Hagner: Zur Sache des
  Buches. In: Bub 68(2016)4, S. 209.} betonen?

Hier werden unter den Kultur- und Geisteswissenschaften ganz
traditionell die Wissenschaften verstanden, deren Leitmedium oder
\enquote{Königsformat} für Forschungsinhalte und wissenschaftliches
Arbeiten das Printbuch beziehungsweise die Monografie ist. Diese
sogenannten \enquote{Buchwissenschaften} oder textorientierte
Wissenschaften werden als unterschieden verstanden in ihrem
hermeneutischen Bemühen und ihren Methoden von den Naturwissenschaften
und quantifizierenden Sozialwissenschaften, trotz aller
Transdisziplinarität. Dabei kann die Plausibilität dieser
Basisunterscheidung im Rahmen der hier vorliegenden Ausarbeitungen nicht
umfassend erneut begründet werden, sondern muss als allgemein
akzeptierbar vorausgesetzt werden.\footnote{Zur Problematisierung dieser
  Unterscheidung siehe Oexle, Otto Gerhard (Hg. 1998) Naturwissenschaft,
  Geisteswissenschaft, Kulturwissenschaft. Einheit -- Gegensatz --
  Komplementarität? Göttingen, 1998.}

Es wird dennoch ein ausreichend weiter Rahmen aufzuspannen sein, in den
die Teilthemen dann eingehängt werden können. Das wird mit sich bringen,
dass das engere Thema, die mediale Hybridität, häufiger überschritten
wird. Dies ist unvermeidlich, weil nur so Querverbindungen und die
Multidisziplinarität des analog-digitalen Begegnungsraumes wenigstens
ansatzweise deutlich werden können.

Der Ort der Bibliothek gehört von seinen Möglichkeiten her, wie die
Profession dies gerade trendsicher wiederentdeckt, zu den realen Räumen
von Wissensgenerierung auch im digitalen Zeitalter. Wer von
Wissensgenerierung spricht, wird selbstredend von der medialen
Vermittlung von Informationen und Wissen zu sprechen haben, also
Anleihen bei der Medienwissenschaft machen müssen, die Hybridität als
nach wie vor sperrigen Begegnungsraum analoger und digitaler Medien
mitzudenken haben, Kognitionswissenschaften, Aufmerksamkeitsökonomie,
anthropologische Grundkonstanten im Mensch-Technik-Verhältnis ebenso
mitdenken müssen wie systemische Einwirkungen auf das Subsystem
Bibliothek durch digitalen Wandel und Phänomene der computergetriebenen
Beschleunigung des modernen Zeitregimes; Konvergenz im technischen
Sinne, Emergenz\footnote{Begriff für die komplexe Logik von
  Aufwärtskausalität, d.h. die Entstehung komplexer, ganz neuartiger
  Eigenschaften und Begriffe aus der Interaktion einfacherer Elemente.
  Vgl. Draguhn, Andreas: Angriff auf das Menschenbild?, S. 268. In:
  Hilgert, Markus (Hg. et al. 2012) Menschen-Bilder.} neuer
Wissensinhalte, Latenz\footnote{Vgl. Gumbrecht, Hans Ulrich (Hg. et al.
  2011) Latenz. Blinde Passagiere in den Geisteswissenschaften.
  Göttingen : Vandenhoek \& Ruprecht, 2011. Ellrich, Lutz (Hg. et al.
  2009) Die Unsichtbarkeit des Politischen. Theorie und Geschichte
  medialer Latenz. Bielefeld : Transcript, 2009. Diekmann, Stefanie (Hg.
  et al. 2007) Latenz. 40 Annäherungen an einen Begriff. Berlin :
  Kadmos, 2007.} potentieller Inhalte im Analogen und Digitalen.

\begin{quote}
\enquote{Der Koppelung von Medientechnik, Denkfiguren und
Wissensformationen ist nicht zu entkommen, weil keine geistige Tätigkeit
im immateriellen Raum stattfindet, sondern auf die Materialität ihrer
Mittel angewiesen ist.}\footnote{Karpenstein-Eßbach, Christa (2004)
  Einführung in die Kulturwissenschaft der Medien, S. 99.}
\end{quote}

Nicht thematisiert werden soll, trotz aller folgenden Reflexionen zum
Medium Buch, zur Medienform Text und zum Lesen als den hier ausgewählten
Kernelementen von Wissensgenerierung, die Buchkultur als solche.

Auch der Zeitschriftenmarkt und damit die Produktion und Distribution
von wissenschaftlichen Aufsätzen und Forschungsartikeln kann nicht in
den Blick genommen werden, weil er von sehr komplexen Mechanismen
bestimmt wird, unter denen Open-Access eine zentrale Strategie ist, wie
sie Michael Hagner (2015) im Kontext von
\enquote{Informationskapitalismus}\footnote{Hagner, Michael (2015) Zur
  Sache des Buches. Siehe insbesondere das Kapitel \enquote{Alles
  umsonst? Open Access}, S. 63-130. Einführend auch Münch, Richard
  (2011) Akademischer Kapitalismus. Zur politischen Ökonomie der
  Hochschulreform. Berlin : Suhrkamp, 2011.} analysiert, der die hier
abgehandelte mediale Hybridität nicht im Kern betrifft, sondern ihr
systemisches Umfeld. Die entscheidenden Aussagen richten sich am Format
der Monografie, der Narrative in Buchform aus.

Was ebenfalls keine Erwähnung findet, sind die Schreibprozesse, die in
den \enquote{Werkstätten} der Geisteswissenschaften, mithin den
Lesesälen, natürlich auch stattfinden.

\begin{quote}
\enquote{Nun ist es allerdings so, dass im Lesesaal keineswegs nur Lesen
{[}\ldots{}{]} stattfindet, sondern vor allem das Schreiben.
{[}\ldots{}{]} Mithin kann man den Lesesaal auch Schreibwerkstatt
nennen. {[}\ldots{}{]} Bleibt festzustellen, dass ein derartig prekärer,
sozial wie kognitiv und kommunikativ konstituierter Ort der
Wissensgesellschaft bis heute nicht begriffen ist.}\footnote{Schneider,
  Ulrich Johannes (2015) Wozu Lesesäle? In: FAZ Nr. 186, Donnerstag, den
  13.08.2015, S. 12.}
\end{quote}

Nicht der intellektuelle Output, und damit insbesondere auch nicht
kollaborative Schreibprozesse,\footnote{Vgl. dazu Krameritsch, Jakob
  (2007) Geschichte(n) im Netzwerk. Hypertext und dessen Potenziale für
  die Produktion, Repräsentation und Rezeption der historischen
  Erzählung. Münster : Waxmann, 2007. S. 18: \enquote{Das Medium
  Internet -- speziell die \enquote{Potenzialität} Hypertext -- kommt
  wie kein anderes Medium vernetzten kollektiven Schreibprozessen
  {[}\ldots{}{]} entgegen.}} wohl aber die Wissensaufnahme und
Wissensgenerierung qua Lesen soll in den Blick kommen in ihrer starken
Abhängigkeit von medialen und räumlichen Bedingungen.

\section*{Systemische Rahmenbedingungen hybrider
Begegnungsräume}\label{systemische-rahmenbedingungen-hybrider-begegnungsruxe4ume}

Hartmut Rosa (2005) lieferte in seiner wichtigen
Habilitationsschrift\footnote{Rosa, Hartmut (2005) Beschleunigung.
  Frankfurt, Main : Suhrkamp, 2005. Auf diese Arbeit wird häufig Bezug
  genommen in jüngeren Zeitdiagnosen, so besonders bei Aleida Assmann
  (2013), auch jüngst Andreas Rödder (2015), vgl.Kapitel I: Welt 3.0, S.
  18-39. Siehe Bibliographie.} eine sehr umfassende zeitsoziologische
Untersuchung des oben erwähnten \enquote{Akzelerationismus}\footnote{Vgl.
  weiter oben die Einleitung S. 1, Fußnote 3.}, für den auch er die
Digitalisierung als gravierenden Beschleunigungsimpuls seit den 90er
Jahren ausmacht:

\begin{quote}
\enquote{Meine heuristisch leitende Hypothese ist {[}\ldots{}{]} die
Vermutung, \emph{dass die in der Moderne konstitutiv angelegte soziale
Beschleunigung in der \enquote{Spätmoderne} einen kritischen Punkt
übersteigt, jenseits dessen sich der Anspruch auf gesellschaftliche
Synchronisation und soziale Integration nicht mehr aufrechterhalten
lässt.} {[}\ldots{}{]} An diesem Umschlagpunkt ändert sich
{[}\ldots{}{]} die Qualität der {[}\ldots{}{]} Zeit selbst: Individuelle
wie kollektive Zeitmuster und -perspektiven werden situativ und
kontextabhängig mit dem Fluss der Zeit immer wieder neu bestimmt
(\enquote{verzeitlicht}), was zu historisch neuartigen Formen
\enquote{situativer Identität} und \enquote{situativer Politik}
führt.}\footnote{Rosa, Hartmut (2005) Beschleunigung, S. 48ff.}
\end{quote}

Als Indikator für Beschleunigung in der geisteswissenschaftlichen
Textproduktion kann man die Renaissance der Kurzformen ansehen, der
Miszelle, des wissenschaftlichen Essays, eine Aufwertung von
Rezensionstext, Blogbeitrag, des Snippets und des Tweet. Dem Ende der
\enquote{großen Erzählungen}\footnote{Douglas Rushkoff (2014) spricht in
  apokalyptischem Ton vom \enquote{narrativen Kollaps}. Vgl. Rushkopff,
  Douglas (2014, zuerst engl. 2013) Present Shock, S. 19-76.} scheint
die Verkürzung der Texte und der \enquote{Book sprint}\footnote{Methode
  zum gemeinschaftlichen Verfassen von Büchern. Vgl. Artikel
  \enquote{Book sprint} in Wikipedia, Version vom 6. März 2016, 17:23.}
zu folgen. Geschlossene Textformen sollen sich öffnen, auflösen, in
einen \enquote{flow} übergehen, nicht mehr vom Geist eines Autors,
sondern der Intelligenz des Schwarms\footnote{Zu kollektiver Intelligenz
  insbesondere in ihrer medialen Abhängigkeit siehe: Ghanbari, Nacim
  (2013 et al.) Was sind Medien kollektiver Intelligenz? Eine
  Diskussion. In: Zeitschrift für Medienwissenschaft, 2013, H. 8, S.
  145-155.} sich nähren: \enquote{Versionierungen mit dauerhafter
Fortschreibungsmöglichkeit} (Klaus Ceynowa 2014). Es entstehe, so
Ceynowa weiter, ein \enquote{kontinuierlich fortschreibbares Ökosystem
digitaler Objekte} für \enquote{nicht-narrative Inhalte}. In diesen
\enquote{vernetzten Wissensräumen} seien die neuen Wissensarbeiter als
\enquote{agil bewegende Entdecker} unterwegs. Es gehe um
\enquote{Immersivität}. Diesen Gedanken des Generaldirektors der
Bayerischen Staatsbibliothek in seinem Beitrag \enquote{Der Text ist
tot. Es lebe das Wissen!}\footnote{Ceynowa, Klaus (2014) Der Text ist
  tot. Es lebe das Wissen! Kultur ohne Text. In: Hohe Luft 1(2014), S.
  52-57.} sind Anregungen für die hier vorliegenden Erkundungen
entnommen.

Über eben diesen Tod des Textes, zumindest seine Agonie, wird schon
länger gemutmaßt, so auch vom Direktor der Library of Congress Daniel J.
Boorstin:

\begin{quote}
\enquote{We now have of course, elaborated communication with unimagined
new devices {[}\ldots{}{]}. We have complicated the machinery of sending
messages in fantastic new ways in order to make it possible for everyone
to receive messages effortlessly {[}\ldots{}{]}. Our faith in progress
leads us to assume that the bad is always, if gradually, being displaced
by the good, and the good is being displaced by the better.
{[}\ldots{}{]}. Now, the displacive fallacy is the belief, that a new
technology necessarily displaces the old. {[}\ldots{}{]} Every great
innovation in technology creates a new environment for all earlier
technologies, and so gives surprising new roles to earlier techniques.
{[}\ldots{}{]} In our collaborative age {[}\ldots{}{]} the book
{[}\ldots{}{]} remains an island of individualism, the utopia of the
non-collaborator. {[}\ldots{}{]} Anyone alert to the problems of
communication in our country today {[}\ldots{}{]} will have no
difficulty in writing his own prescription for the ideal communication
device. {[}\ldots{}{]} There is no better example of the technological
amnesia that afflicts the most highly developed civilizations -- our
tendency to forget simple ways of doing things in our desperate
preoccupation with complex ways of doing them -- than our need to be
reminded that we already possess precisely this device. The name for it
(a wonderful four-letter word) is {[}\ldots{}{]}.}\footnote{Boorstin,
  Daniel J. (1974): A design for an anytime, do-it-yourself, energy-free
  communication device. In: Harpers Magazine, Jan. 1, 1974 (248), S.
  83-86. Die einzelnen Gedanken und Argumente von Boorstin wurden von
  mir neu angeordnet, um auf die kleine Pointe hinzuwirken; der Sinn von
  Boorstins Aussagen wird dadurch in keiner Weise entstellt.}
\end{quote}

Es ist nicht das \enquote{iPod}, sondern das \enquote{book}. Der Artikel
erschien vor über 40 Jahren und stellt eine der frühen Reflexionen dar
auf den \enquote{Tod des Textes} und das von Uwe Jochum konstatierte
Telos vom zwangsläufigen Verschwinden des Buches, dem durch
vorauseilende Vorwegnahme der vermeintlich kommenden Allverfügbarkeit
elektronischer Medien im allesverbindenden Netz entgegengearbeitet
werden könne oder müsse.

Das Ende der \enquote{Gutenberg-Galaxis} dräut ja bereits seit dem
Erscheinen des so betitelten Buches von Herbert Marshall McLuhan
1962.\footnote{McLuhan, Marshall Herbert: Die Gutenberg-Galaxis. Zuerst
  engl. Toronto, 1962. Dt. Ausg. Hamburg, 2011.} In seiner Nachfolge
formuliert insbesondere Norbert Bolz (1993) seine medientheoretischen
Überlegungen zum \enquote{Ende der Gutenberg-Galaxis}.\footnote{Bolz,
  Norbert (1993) Am Ende der Gutenberg-Galaxis. München, 1993.} Und Uwe
Jochum (2011) datiert die \enquote{Furie des Verschwindens} (Hegel), und
den \enquote{Beginn der Selbstabschaffung der Bibliotheken} auf das Jahr
1965 zurück, in dem J.C.R. Lickliders Programmschrift \enquote{Libraries
of the future} erschien.\footnote{Vgl. Jochum, Uwe (2011) Die
  Selbstabschaffung der Bibliotheken. In: Jochum, Uwe ; Schlechter,
  Armin (Hg.) Das Ende der Bibliothek? Vom Wert des Analogen (2011), S.
  11-25. Hier S. 11.}

Markus Buschhaus (2008) erkennt in diesen zahllosen gleichgearteten
Medienanalysen des Verdrängens und Verschwindens eine \enquote{Rhetorik
der Verabschiedung}, die sich verstetigt habe:

\begin{quote}
\enquote{Als Denkfigur, welche zwischen Revolution und Tradition,
zwischen Historie und Historiographie, zwischen Technik und Kultur
vermittelt, stellt die Gutenberg-Galaxis in der Tat so etwas wie eine
medienwissenschaftliche Urszene dar. Sie erlaubt es schließlich, die
Revolution auf Dauer zu stellen, den Ausnahmezustand als Regelfall
einzuführen, die Alarmbereitschaft aufrecht zu erhalten und
aufmerksamkeitsökonomische Ansprüche geltend zu machen. Das liest sich
dann wie folgt: \enquote{Die Gutenberg-Galaxis hört nicht auf zu enden.}
Oder auch so: \enquote{Die Geschichte vom Ende des Buches ist eine
unendliche Geschichte.}}\footnote{Vgl. Buschhaus, Markus (2008) Am einen
  \& am anderen Ende der Gutenberg-Galaxis. In: Grampp, Sven (Hg. et
  al., 2008) Revolutionsmedien -- Medienrevolutionen, S. 205-228. Hier
  S. 212 f.}
\end{quote}

Tatsächlich verführt heute Digitalisierung durch den Prozess der
Konversion, im Ergebnis eine Dematerialisierung, immer erneut zum
Nachdenken über das Verschwinden der Originale. Aus
kulturwissenschaftlicher Sicht analysierte Aleida Assmann (2013) im
Kontext von Zeit- und Wandlungsbeschleunigung das dahinter stehende
Evolutionsprinzip:

\begin{quote}
\enquote{Die Dialektik von Innovation und \enquote{Antiquation}
reguliert {[}\ldots{}{]} die Ersetzungsprozesse der technischen
Evolution {[}\ldots{}{]}. Auf dem Markt zieht die Produktion des Neuen
Aufmerksamkeit an und setzt Begehren frei, während sie zugleich das
Bestehende und Bekannte als unattraktiv und obsolet erscheinen lässt.
Obsoleszenz ist eine schleichende und unscheinbare Form des Vergessens
durch Entwertung und Aufmerksamkeitsentzug.}\footnote{Assmann, Aleida
  (2013) Ist die Zeit aus den Fugen?, S. 203. Buchvorstellung am
  Deutschen Historischen Institut Paris am 13. November 2014.}
\end{quote}

Diesem Charakter von technischer Evolution scheint nicht nur der Kosmos
aller Wissensobjekte in Bibliotheken, Verlagen, Archiven und Museen seit
über 40 Jahren und mehr unterworfen. Markus Buschhaus (2008):
\enquote{Das \enquote{Ende des Buchzeitalters}, das \enquote{Ende des
fotografischen Zeitalters} und das \enquote{Ende des musealen
Zeitalters} haben {[}\ldots{}{]} die Gemeinsamkeit, dass Buch,
Fotografie und Museum ihrer letztlich stets digitalen Herausforderung
{[}\ldots{}{]} zum Opfer fallen.}\footnote{Buschhaus, Markus (2008) Am
  einen \& am anderen Ende der Gutenberg-Galaxis, S. 216.}

Als Telos hat diese Sicht technikgetriebener Evolution als Dialektik
zwischen Innovation und Obsoleszenz auch den bibliothekarischen
Berufsstand erfasst und untergraben. So kann ein Insider der
bibliothekarischen und informationswissenschaftlichen Ausbildung 2015
der Meinung sein, \enquote{dass die Community der Bibliotheksbeflissenen
derzeit stark durch Aktivitäten auf sich aufmerksam macht, die eigene
Überflüssigkeit unter Beweis zu stellen.}\footnote{Gödert, Winfried
  (2015), Hashtag Erschließung. Online-Ressource:
  \url{http://eprints.rclis.org/24643/} (27.02.2015). Hier S. 1.} Eine
Neigung zur selbstgenügsamen Agonie sei unverkennbar.\footnote{Gödert,
  Winfried (2015) in einem Kommentar zur Verleihung der
  Karl-Preusker-Medaille an Konrad Umlauf und dessen in seiner
  Dankesrede vorgestellten sieben Thesen zur Zukunft der Bibliotheken
  und bibliothekarischen Berufe. Vgl. die Mailing-Liste inetbib.de, Mail
  vom 24.11.2015, 12:36. Dort: 3. These.}

Was macht diese Bedenken auslösende Wirkkraft des digitalen Wandels aus?
Warum wirken neben dem Digitalen immer mehr analoge Techniken und die
sie bedienenden Menschen gestrig oder \enquote{kraftlos}, wie Simon
Strauss (2015) sich in der FAZ ausdrückte?\footnote{\enquote{Die
  digitale Revolution ist so ein Sauger {[}Vampir{]}. Sie entkräftet
  ihre Opfer nach und nach, bis sie, außen noch einigermaßen intakt,
  innen jedoch blutleer, saft- und kraftlos in sich zusammensacken.}
  Strauss, Simon (2015), Und wo sind hier die Bücher. Bibliothek der
  Zukunft. In: FAZ Nr. 229 vom 2. Okt. 2015, S. 20.} Gilt das
Verdrängungsprinzip nicht schon immer seit den frühesten Tagen des
menschlichen Werkzeuggebrauchs? Oder ist dieses Verdrängungsprinzip ein
Trugschluss, wie Broostin meinte: \enquote{\ldots{} the displacive
fallacy is the belief, that a new technology necessarily displaces the
old}\footnote{Boorstin, Daniel J. (1974): A design for an anytime,
  do-it-yourself, energy-free communication device. In: Harpers
  Magazine, Jan. 1, 1974 (248), S. 83-86.}, und der sich damit dem
Prinzip nach medienhistorisch als Anhänger der Riepl'schen Annahme der
Medienkomplementarität von 1913\footnote{Ausführlich dazu siehe weiter
  unten S. 21ff.} zeigt.

Aleida Assmann (2013), in der Folge von Hartmut Rosa (2005) und
anderen,\footnote{Eine Aneinanderreihung von Tempophänomenen durch die
  Jahrhunderte bei Borscheid, Peter (2004): Das Tempo-Virus. Eine
  Kulturgeschichte der Beschleunigung. Frankfurt, Main : Campus, 2004.}
macht das Zeitregime der Moderne aus als Motor eines verstetigten
Wandels:

\begin{quote}
\enquote{Da die Vergangenheit in den Augen der
Modernisierungstheoretiker dazu tendiert, sich in Form einer
\enquote{einmal eingelebten Einstellung} zu verfestigen, ist mit einem
einmaligen Bruch nichts getan, vielmehr ist ein unentwegtes Brechen mit
ihr angesagt. In solchen Akten des Brechens müssen permanent Bestände
aus der Gegenwart aussortiert, verworfen und für ungültig erklärt
werden. Dieser andauernde Abwurf von Ballast geschieht performativ durch
ein Zur-Vergangenheit-Erklären dessen, was bisher noch Anspruch auf
Gegenwart und damit zugleich auch Geltung besessen hatte. Eine besonders
markante rhetorische Form, im Kontinuum der Zeit solche
Hiatus-Erfahrungen zu produzieren, ist die fortgesetzte Verkündung des
\enquote{Todes}{[}sic{]} aller möglicher kultureller Institutionen und
Werte.}\footnote{Assmann, Aleida (2013), Ist die Zeit aus den Fugen?, S.
  142.}
\end{quote}

Das Digitale hat in ganz besonderem Maße den Nimbus des täglich Neuen
mit quasi-religiöser Aureole, wie es der Historiker Valentin Groebner
ausdrückt: \enquote{Die Digitalisierung, scheint es, ist unaufhörlicher
Anfang und sich ständig erneuerndes Versprechen, Hoffnung,
Neuland.}\footnote{Groebner, Valentin (2014), Wissenschaftssprache
  digital, S. 10.} \enquote{{[}\ldots{}{]} das ist durchaus theologisch,
eine elektronische Immer-Neu-Ewigkeit.}\footnote{Ebd., S. 121.}

So ist das Digitale in der Wahrnehmung der einen tägliche Morgenröte,
Zukunftsversprechen und Wunderland, während es sich für andere invasiv
und dystopisch zeigt als Horrorszenario und Herrschaft der
Algorithmen.\footnote{Vgl. die Dystopie von Miriam Meckel (2011), Next.
  Erinnerungen eines ersten humanoiden Algorithmus.} Von Vertretern der
Digital Humanities wird der Beginn dieser jungen Disziplin häufig
zurückdatiert auf 1949 und den Index Thomisticus von Roberto Busa SJ
(1913-2011).\footnote{Vgl. König, Mareike (2016) Was sind Digital
  Humanities? Definitionsfragen und Praxisbeispiele aus der
  Geschichtswissenschaft. \url{https://dhdhi.hypotheses.org/2642}.}
Trotzdem konstatiert der Historiker Wolfgang Schmale im Sinne der o.g.
\enquote{Immer-Neu-Ewigkeit} 2015 noch immer: \enquote{Die Digital
Humanities stehen am Anfang.}\footnote{Schmale, Wolfgang (2015),
  Einleitung Digital Humanities, S. 13.}

Im Zeitalter globalisierter Informationsströme und einer
\enquote{beschleunigten} Gegenwart ist das Digitale, so wurde
herausgearbeitet, in einer Weise in unsere Lebenswirklichkeit
hineingestellt, dass permanente Positionsbestimmung notwendig wird. Hier
darf man einen der Hauptgründe für die Allgegenwart der Klagen von
Informationsflut, dem Overload der Informationskanäle, der
\enquote{Konfrontation mit beständiger Überkomplexität}\footnote{Gumbrecht,
  Hans-Ulrich (2014), FAZ vom 11.03.2014, Nr. 59, S. 14.} vermuten.

Tim Cole (2015, Internetexperte) bilanziert die aktuelle Situation:

\begin{quote}
\enquote{Wenn sich {[}\ldots{}{]} das, was wir \enquote{Wirklichkeit}
nennen, so fundamental wandelt, dass wir unsere Lebensverhältnisse neu
daran anpassen müssen, dann müssen wir mit unserem Denken wahrscheinlich
auch unseren gesamten Lebensplan neu ausrichten. Anders als die
Vordenker der klassischen Aufklärung können sich deren digitale
Nachfolger keine geruhsame Reflexion mehr leisten. Die digitale
Aufklärung muss sich immer wieder der Herausforderung einer permanenten
Beschleunigung dessen stellen, was wir Wirklichkeit nennen
{[}\ldots{}{]}.}\footnote{Cole, Tim (2015) Kein Grund zur Panik. In:
  Kulturaustausch 65(2015)4, S. 18.}
\end{quote}

Diese komplexe Gemengelage schreckte Hans-Ulrich Gumbrecht,
Literaturwissenschaftler, aus \enquote{geruhsamer Reflexion} auf. In der
Frankfurter Allgemeinen Zeitung startete er im März 2014 eine Kolumne
mit der Formel: \enquote{Das Denken muss nun auch den Daten folgen.} Er
sah die \enquote{Grundlagen der menschlichen Existenz} durch das
Digitale angefasst, erkannte nichts Geringeres als die \enquote{für das
Überleben der Menschheit möglicherweise entscheidende und bis vor kurzem
kaum geahnte Herausforderung} durch die elektronischen Technologien und
sah die Zeit gekommen für nichts weniger als eine \enquote{Epistemologie
der elektronischen Zeit}.\footnote{Gumbrecht, Hans-Ulrich (2014), FAZ
  vom 11.03.2014, Nr. 59, S. 14.}

Das sieht auch Ramón Reichert (2014, Kultur- und Medientheoretiker) so.
Umfassende Neuorientierung tue Not, um die durch Daten aller Art
ausgelösten tektonischen Verschiebungen der Gegenwartsgesellschaft in
allen Bereichen des Alltags angemessen reflektieren zu können.\footnote{Vgl.
  Reichert, Ramón (2014) Big Data, Einleitung, S. 28.}

Das Editorial im \enquote{Züricher Jahrbuch für Wissensgesellschaft
2013} konstatiert: \enquote{\emph{Big data} ist in den
Geisteswissenschaften angekommen. Nachdem diese in den letzten
Jahrzehnten mit einigen \emph{turns}\footnote{Eine Aufzählung der
  verschiedenen \enquote{turns} bei: Paravicini, Werner (2010) Die
  Wahrheit der Historiker. S. 6f. Eine Textstelle drückt die Distanz des
  langjährigen Direktors des Deutschen Historischen Instituts Paris
  gegenüber diesen \enquote{\emph{turns}} aus: " {[}\ldots{}{]} wie
  Thomas Thiel feststellte: \enquote{\emph{Nach dem Turn ist vor dem
  Turn}}, denn die Ursache dieser Hatz ist nicht Erkenntnisfortschritt,
  sondern Karrierekonkurrenz." Aus dem Blickwinkel der hier auch
  interessierenden Technikgeschichte sichtet Stefan Krebs (2015) nach
  dem \emph{linguistic}, dem \emph{pictorial} oder \emph{iconic}, dem
  \emph{aural} oder \emph{sonic turns} jetzt den \emph{sensorial turn}:
  \enquote{Die Sinnlichkeit der Technik betont die Körperlichkeit im
  Umgang mit Technik {[}\ldots{}{]}.} In: Technikgeschichte, 82(2015),
  H. 1, S. 3-9. Eine Auflistung von 17 \enquote{turns} in den Geistes-,
  Kultur- und Sozialwissenschaften auch bei Theo Hug (2012) Kritische
  Erwägungen zur Medialisierung des Wissens im digitalen Zeitalter, S.
  25. Den \emph{archival turn} für die Kulturwissenschaften dokumentiert
  das Handbuch Archiv, hg. von Marcel Lepper et al. 2016, S. 21f. Ein
  \emph{library}- oder \emph{bibliological turn} oder ähnliches konnte
  nicht gesichtet werden.} konfrontiert waren, haben wir es nun mit dem
\emph{digital turn} zu tun, auch wenn es noch reichlich unklar ist, was
man sich darunter vorstellen soll.}\footnote{Hagner, Michael ; Hirschi,
  Caspar (2013) Editorial. In: Nach Feierabend 2013 (9), S. 7. Der
  \emph{digital turn} scheint jedoch bereits abgelöst durch \enquote{The
  \emph{computational turn} conference}, Swansea University, 09.03.2010,
  organisiert von David M. Berry, Senior Lecturer in Digital Media,
  Swansea University, UK.}

Dieses immer erneute Wachwerden, teils Aufschrecken dem Digitalen
gegenüber zeitigt eine gewisse Hilflosigkeit, die Konzeptbildungen per
se nicht förderlich sein kann und welche unter Umständen in permanenten
Transformationsgesellschaften auch gar nicht mehr nötig oder gewünscht
sind. Bemerkenswert ist dabei, dass bereits früh und immer wieder in
geradezu heideggerscher Radikalität über die Bedeutung und Rolle von
Technologie als konstitutivem Element der gesamten Seinsverfassung und
Daseinsbewältigung des Menschen nachgedacht wurde. Karl-Heinz Ott (2014)
weist im Themenheft \enquote{Digital} der Zeitschrift \enquote{Die
Politische Meinung} darauf hin:

\begin{quote}
\enquote{Während bis heute der Glaube vorherrscht, dass die Technik
einzig und allein ein Hilfsmittel ist, mit dem sich unser Leben
erleichtern lässt, versucht Heidegger nachzuweisen, dass sie unser
gesamtes Selbst- und Weltverhältnis prägt. {[}\ldots{}{]} Laut Heidegger
begegnet uns das Technische nämlich weit mehr in den Formen jenes
logischen, rechnerischen, instrumentellen Denkens, {[}\ldots{}{]} als
bloß in solchen sichtbaren Dingen wie Maschinen, Apparaten und
Automaten.}\footnote{Ott, Karl-Heinz (2014), Gewichtige Werke oder
  digitales Gewurstel. In: Die Politische Meinung 2014 (59) 526, S. 85f.}
\end{quote}

Hier lässt sich auch das visionäre Buch von Marshall McLuhan von 1962
verorten, der, fasziniert von \enquote{Elektrobiologischem},\footnote{McLuhan,
  Herbert Marshall (2011) Die Gutenberg-Galaxis, S. 60.} von
Nichtlinearität und Mosaikstrukturen,\footnote{Ebd., S. 343:
  \enquote{Das vorliegende Buch ist {[}\ldots{}{]} einem mosaikartigen
  Wahrnehmungs- und Beobachtungsmuster gefolgt.}} versuchte, die Fesseln
und Beschränkungen des \enquote{typographischen} Menschen aufzuzeigen in
seiner Befangenheit in der linearen, sequenziellen Gutenberg-Galaxis.
Ihn beschäftigte die \enquote{{[}\ldots{}{]} Fragmentierung der
menschlichen Psyche durch die Buchdruckkultur {[}\ldots{}{]}.}\footnote{Ebd.,
  S. 43.}

Ähnlich ganzheitlich äußert sich auch der Mitbegründer und ehemalige
Direktor des Media Lab am Massachusetts Institute of Technology Nicholas
Negroponte 1995: \enquote{Der Umgang mit dem Computer hat nichts mehr
mit Rechnen und Berechnen zu tun -- er ist ein Lebensstil
geworden.}\footnote{Nekroponte, Nicholas (1995): Being digital. Dt.
  Total digital (1995), S. 13. Vgl. zu dieser \enquote{digitalen
  Seinsform} in postmodernen Zeiten auch: Wirth, Sabine (2014):
  Computer/Internet, S. 84. In: Metzler Lexikon moderner Mythen (2014).}

Um für die Untersuchung zur notwendigen Komplementarität analoger und
digitaler Medien im Rahmen eines umfassenden Hybridansatzes gegenüber
der Flut der Literatur ein Selektionskriterium zu finden, bietet sich
die Lebenswirklichkeit und Empirie der Autoren selbst gegenüber der
elektronischen Vernetzungstechnologie als valides Kriterium an. Diese
Empirie ist vor Mitte der 90er Jahre eine andere und mit Rücksicht auf
die Innovationsdynamik der Computertechnologie 2015/2016 nochmals
dramatisch anders.

Es kann hier der Distanznahme Valentin Groebners (2014) gegenüber
Vordenkern der digitalen Revolution gefolgt werden:

\begin{quote}
\enquote{Die Phänomene der neuen digitalen Kommunikationskanäle
{[}\ldots{}{]} könnten nur mit Hilfe älterer Theoretiker überhaupt
korrekt eingeordnet und verstanden werden: McLuhan, Foucault, Deleuze
und Luhmann sind dabei besonders beliebte Kandidaten. Das geschieht
ungeachtet der Tatsache, dass diese Autoren in ihrem eigenen Berufsleben
keine ähnlichen technischen Installationen gesehen oder benutzt haben.
{[}\ldots{}{]} Wer im 21. Jahrhundert mit Theorien aus den 1950er,
1970er und 1980er Jahren über die Phänomene der digitalen
Kommunikationskanäle schreibt, glaubt entweder an richtig starke
Rückkopplungsphänomene, also an Konstellationen, in der neue technische
Phänomene sehr viel ältere Argumente nachträglich bewahrheiten.
{[}\ldots{}{]} Oder er glaubt, dass man in den Begriffen der großen
Denker gar nichts anderes sagen könne als etwas, was schon irgendwie
stimmen werde, im abstrakten Sinn.}\footnote{Groebner, Valentin (2014),
  Wissenschaftssprache digital, S. 22f.}
\end{quote}

Es ist demnach wenig sinnvoll, weiter als bis zu der breitenwirksamen
Phase des Internets Mitte der 90er Jahre zurückzuschauen.

\section*{\texorpdfstring{Dauerkonzept
\enquote{Hybridbibliothek}?}{Dauerkonzept Hybridbibliothek?}}\label{dauerkonzept-hybridbibliothek}

Evoziert wurde ein kontinuierlicher Druck zur Positionierung gegenüber
innovativer Technik und insbesondere für das Digitale, dem Nimbus des
ständig Neuen und dem Neuen inhärent dessen Verdrängungspotenzial als
ein Prinzip technischen Fortschritts. Es scheint, dass seit Jahrzehnten
ein digitaler \enquote{Bald-Anders} durch das Land wandert, dem der Sinn
nach stetem Wandel steht, nach \enquote{perpetual beta}.\footnote{Vgl.
  die Ausführungen von Aike Schaefer-Rolffs (2013) in ihrer Monografie
  über Hybride Bibliotheken, S. 73.} Ein modernes, beschleunigtes
Zeitdispositiv scheint nicht nur bei Personen, sondern auch bei
Institutionen eine hastige Suche nach immer neuen, zeitgemäßen
Identitäten ausgelöst zu haben, um Aufmerksamkeitspotentiale zu binden
und attraktiv zu bleiben in rastlosen, technikgetriebenen
Transformationsgesellschaften.

Wie wirkmächtig ist nun der \enquote{digital turn} mit Blick auf die
moderne Informations- oder sogar Wissensgesellschaft? Ist der Gelehrte
alten Typs in den Geisteswissenschaften heute nur noch ein Relikt der
Vergangenheit, weil in den Geisteswissenschaften die Monografie seit
Jahren an Terrain verliert, wie der Präsident der FU Berlin Peter-André
Alt (2014) konstatiert?\footnote{Alt, Peter-André (2014), Artikelflut
  und Forschungsmüll. In: SZ vom 23.06.2014, Nr. 141, S.12.} Realisiert
sich die \enquote{allmähliche Überwindung der}bookishness``, wie Elmar
Mittler (2012) suggeriert?\footnote{Mittler, Elmar (2012),
  Wissenschaftliche Forschung und Publikation im Netz, S. 38.}

Ist das Internet der neue Denkraum für Wissensgenerierung? Michael
Hagner (2015) konstatiert in Anlehnung an Evgeny Morozov
\enquote{intellektuelles Elend, das sich in der digitalen Welt
eingenistet hat} und hat den Eindruck, \enquote{dass intellektuelle
Debatten im Netz allzu schnell in reflexartige
Befindlichkeitsartikulationen und Stereotypen münden.} Das sei
angesichts der Bedeutung, die das Internet als Kommunikationsforum haben
könnte, schlimm. Für die Steigerung der Qualität des Internet als
Reflexionsraum \enquote{wäre eine -- mit Hartmut Rosa gesprochen --
Loslösung von der Diktatur der Schnelligkeit notwendig.}\footnote{Vgl.
  Hagner, Michael (2015) Zur Sache des Buches, S. 44f.}

Valentin Groebner (2012) verweist trotz allgegenwärtiger
Wandlungsdynamik auf die traditionellen Qualitäten des Textbehälters
Buch: \enquote{Ein Buch eröffnet Ihnen die Gelegenheit, Ihre Leser in
einen relativ ruhigen und abgeschlossenen Raum zu entführen.
{[}\ldots{}{]} Er ist ein Versprechen auf Konzentration und gezielte
Aufmerksamkeit.}\footnote{Groebner, Valentin (2012) Wissenschaftssprache
  : eine Gebrauchsanweisung. S. 32.} Er formuliert damit unter den
Konditionen des allgegenwärtigen Netzes neu, was der bereits zitierte
Daniel J. Boorstin schon vor 40 Jahren ohne Empirie der heutigen
sozialen Netzwerken ins Feld führte: \enquote{In our collaborative age
{[}\ldots{}{]} the book {[}\ldots{}{]} remains an island of
individualism, the utopia of the non-collaborator.}\footnote{Boorstin,
  Daniel J. (1974): A design for an anytime, do-it-yourself, energy-free
  communication device. In: Harpers Magazine, Jan. 1, 1974 (248), S.
  83-86} Dazu Harmut Rosa (2005): \enquote{Solche
\enquote{Entschleunigungsoasen} geraten in der Spätmoderne
{[}\ldots{}{]} kulturell {[}\ldots{}{]} verstärkt unter Erosionsdruck
{[}\ldots{}{]}. Wie Helga Nowotny und Hermann Lübbe übereinstimmend
bemerken, gewinnen solche beschleunigungsimmunen Phänomene an gleichsam
\enquote{nostalgischen} Wert oder an Verheißungsqualität, je seltener
sie werden.}\footnote{Rosa, Hartmut (2005) Beschleunigung. S. 143.}

Dem Aufschrecken des Hermeneutikers Hans-Ulrich Gumbrecht und der
Verunsicherung des Informationswissenschaftlers Winfried Gödert ist also
die Besorgnis des Historikers an die Seite zu stellen. Valentin Groebner
(2013) kurz vor einem Konferenzbeitrag: \enquote{Um die Zukunft der
wissenschaftlichen Kommunikation im digitalen Zeitalter sollte es gehen
{[}\ldots{}{]}. Ich war nervös. {[}\ldots{}{]} Aber war ich dazu
überhaupt vernetzt genug und wirklich auf dem Laufenden?}\footnote{Groebner,
  Valentin (2014), Wissenschaftssprache digital, S. 7.} Und
Bibliothekare sind permanent beunruhigt und sehen sich ständig vor der
Notwendigkeit einer Neupositionierung: \enquote{{[}\ldots{}{]} eine
nachhaltige Informationsinfrastruktur war noch nie so nötig wie
jetzt!}\footnote{Mittler, Elmar (2014), Nachhaltige Infrastruktur. In:
  BFP 2014 (38),3: S. 364.}, konstatiert Elmar Mittler (2014), der
selber seit den 70er Jahren des letzten Jahrhunderts an eben dieser
Infrastruktur maßgeblich mit konstruiert hat.

Wenn diese Infrastruktur sich je etabliert hatte, wie hoch war ihre
Halbwerts-, besser Verfallszeit, wenn jetzt erneut \enquote{nachhaltige
Infrastruktur} notwendig ist wie nie? Und was wäre unter
\enquote{Nachhaltigkeit} in einem Kontext ständigen Wandels zu
verstehen? Welche Bestände gälte es zu bewahren? Welche
Migrationsstrategien zu neuen Techniken hin sind die richtigen oder
notwendigen?

Es lassen sich folglich zahlreiche Ankerpunkte für den Bedarf an
Konzeptbildung, für Festschreibungen, für Besinnung auf Bewährtes finden
und benennen, während Fluides,\footnote{Vgl. Eigenbrodt, Olaf (2014),
  Auf dem Weg zur Fluiden Bibliothek. In: Eigenbrodt, Olaf (Hg. et al.,
  2014) Formierungen von Wissensräumen, S. 207-220.}
Konvergenzphänomene, Beschleunigung und Dematerialisierung überall zur
Auflösung von Strukturen zu führen scheinen.

Konzeptuelle Unsicherheit hat bekanntlich sämtliche
Gedächtnisinstitutionen ergriffen: Bibliotheken, Archive, Museen.
Begriffe vom Unikat, der Realie, des Haptischen, der Aura des Originals
beschäftigen alle diese Einrichtungen mit Sammlungsauftrag. Phänomene
der Überführung des Analogen ins Digitale, der Verflüssigung, der
Entmaterialisierung scheinen diesen Einrichtungen die Objekte ihrer
jahrhundertelangen Bemühungen zu entziehen. Digitalgeborenes scheint
Routinen der Integration in Sammlungen zu überfordern. Das Problem der
Perennität kultureller Leistungen, von Information und Wissen stellt
sich in ganz neuen Dimensionen. Der Sammelauftrag, Kernaufgabe aller
Kulturgut bewahrender Institutionen verliert an Kontur und mit dieser an
Überzeugungskraft und damit letztlich seine Finanzierung.

Dem konzeptionellen Tasten und Driften der Gedächtnisinstitutionen
gesellt sich begriffliche Hilflosigkeit zu. Im Bereich der Bibliotheken
aller Sparten und deren anerkannter Grundaktivitäten von Sammeln,
Erschließen und Vermitteln gibt es kaum eine solide, überzeugende
Begrifflichkeit, mit der Bibliotheken ihre Positionierung zwischen der
\enquote{gedruckten Welt} und der \enquote{all-digital-world}\footnote{Vgl.
  Kempf, Klaus (2014), Bibliotheken ohne Bestand? In: BFP 2014, 38(3),
  S. 365-397. Schon das Fragezeichen im Titel darf im Sinne der obigen
  Ausführungen als Zeichen der Verunsicherung gelesen werden.} benennen
könnten. Ein besonderer Akzent liegt dabei oft auf dem Evolutiven, damit
aber auch auf dem bereits zitierten Telos dieser intermediären
Standortbestimmungen. Denn sind Bibliotheken zu 100 Prozent digital (mit
oder ohne eigenen \enquote{Standort}), heißen sie ohne Anführungszeichen
Virtuelle Bibliothek, E-Bibliothek oder Digitale Bibliothek.

Als Benennung von etwas Intermediärem hat sich die Bezeichnung der
Hybridbibliothek gehalten, im deutschen Sprachgebrauch häufig in
Anführungszeichen oder mit dem distanzierenden \enquote{sogenannt}
vorweg.\footnote{So in dem Aufsatz von Frühwald, Wolfgang (2002),
  Gutenbergs Galaxis im 21. Jahrhundert. In: ZfBB 2002 (49) 4, S.
  187-194. Nicht so bei Rösch, Hermann (2004), der den Begriff zu
  Hybrideinrichtungen ausweitet. Siehe: Rösch, Hermann (2004),
  Wissenschaftliche Kommunikation und Bibliotheken im Wandel. In: B.I.T.
  online 2004, Heft 2, S. 104.} Systematisch ausgearbeitet wurde dieser
Begriff zuerst von Stuart A. Sutton, Professor an der School of Library
and Information Science in San Jose, CA, USA, 1996.\footnote{Sutton,
  Stuart A. (1996) Future service models and the convergence of
  functions. The reference librarian as technician, author and
  consultant. In: Low, Kathleen (Hg.) The roles of reference librarians
  today and tomorrow. New York : Haworth, 1996. S. 125-143.} In seinem
\enquote{Library Type Continuum} stellt er graphisch die Entwicklung von
der traditionellen hin zur digitalen Bibliothek als unumgänglich dar:
\enquote{The figure denotes four types of libraries on a continuum
running from the traditional to the digital.}\footnote{Ebd., S. 129.}
Die vier Entwicklungsstufen
\enquote{traditionell},\enquote{automatisiert}, \enquote{hybrid} und
\enquote{digital} werden von ihm jeweils kurz skizziert. \enquote{Type
III The Hybrid Library} ist für Sutton ganz Übergang:
\enquote{{[}\ldots{}{]} the balance of print and digital
meta-information leans increasingly toward the digital.}\footnote{Ebd.,
  S. 136.} Die Endstation \enquote{Digital Library} hat die
beunruhigenden Charakteristika des Virtuellen: \enquote{With Type IV
{[}Digital Library{]}, we arrive at the library as logical entity. It is
the library without walls -- the library that does not collect tangible
information bearing entities but instead provides intermediated,
geographically unconstrained access to distributed, networked digital
information.}\footnote{Ebd., S. 138.}

Charles Oppenheim und Daniel Smithson (1999), beide Department of
Information Science, Loughborough University, UK, sind etwas moderater
in ihrem Fachaufsatz \enquote{What is the hybrid library?}\footnote{Oppenheim,
  Charles; Smithson, Daniel (1999) What is the hybrid library? In:
  Journal of information science 1999 (25) 2, S. 97-112.}:
\enquote{{[}\ldots{}{]} the hybrid library is not a special service, but
an approach to the library which accords paper and digital the same
status}.\footnote{Ebd., S. 108.} Der Begriff wird hier deutlicher im
Sinne eines konstruktiven Nebeneinanders verstanden, wobei sie betonen:
\enquote{There is a clear consensus that the library in a location will
remain.}\footnote{Ebd., S. 97.}

Klaus Kempf (2003) fragt sich: \enquote{Wo und was ist das Neue bei
diesem Konzept? Das Neben- und Miteinander unterschiedlicher Medientypen
in Bibliotheken wird bereits seit geraumer Zeit mehr oder minder
erfolgreich praktiziert.}\footnote{Kempf, Klaus (2003) Erwerbung und
  Beschaffung in der Hybridbibliothek, S. 39.}

Die Diskussion schien auf der Stelle zu treten, denn in den von
Oppenheim und Smithson gesammelten Interviews mit leitenden
Bibliotheksdirektoren aus 1998\footnote{Vgl. Table 1: Individuals
  contacted for discussions on hybrid library issues 1998 in: Oppenheim,
  Charles; Smithson, Daniel (1999) What is the hybrid library?, S. 101.}
wird auch dies bereits vermerkt:

\begin{quote}
\enquote{The overall impression given by the respondents was that hybrid
libraries had existed in all but name before the projects started. The
only difference now is that a phrase has been coined. There are lots of
versions of hybrid libraries in existence, but they are just not called
hybrid libraries.}\footnote{Ebd., S. 104. Diese Interviewäußerungen sind
  Greg Newton-Ingham und Hazel Woodward zugeordnet.}
\end{quote}

Kempf ergänzt mit etwas dunkel verklärtem Unterton: \enquote{Unabhängig
vom Bibliothekstyp und den jeweiligen lokalen Gegebenheiten kann man die
Aussage wagen, die \enquote{hybride Bibliothek} wird konsequent
nutzerorientiert oder gar nicht mehr sein.}\footnote{Kempf, Klaus (2003)
  Erwerbung und Beschaffung in der Hybridbibliothek, S. 66. Diese
  apodiktische Setzung eines Hamlet'schen Sein oder Nicht-Sein ist
  ebenfalls Teil einer sich wiederholenden Rhetorik, die häufig
  Anwendung findet im Zusammenhang mit für unentrinnbar gehaltenen
  Entwicklungen. In unserem Kontext als Beispiel für eine ähnlich
  apodiktische Fehleinschätzung Emmanuel Le Roy Ladurie (1973):
  \enquote{Der Historiker von morgen wird Programmierer sein oder nicht
  mehr sein.} Zitiert nach: Mallinckrodt, Rebekka von (2004)
  \enquote{Discontenting, surely, even for those versed in French
  intellectual pyrotechnics}, S. 228.}

Gerhard Hacker (2005) resümiert dieses Auf-der-Stelle-treten nach 10
Jahren: \enquote{Seit Beginn der Diskussion ist {[}\ldots{}{]} offen,
{[}\ldots{}{]} ob die Idee der Hybridbibliothek dauerhaft
entwicklungsfähig ist und durch ihre kontinuierliche Verbesserung
künftigen Bedürfnissen gewachsen sein wird. Daran hat sich seit 1998
wenig geändert.} \footnote{Hacker, Gerhard (2005) Die Hybridbibliothek
  -- Blackbox oder Ungeheuer?, S. 283.}

In dieser daueroffenen Übergangssituation wurde das Konzept so auch in
den historischen Wissenschaften rezipiert. Klaus Gantert (2011) in
seiner Beschreibung von Informationsressourcen für Historiker:
\enquote{Bibliotheken, die das bewusste Nebeneinander von
konventionellen und digitalen Angeboten betonen möchten, bezeichnen sich
häufig als hybride Bibliotheken.}\footnote{Gantert, Klaus (2011)
  Elektronische Informationsressourcen für Historiker, S. 161.}

Mangels einer besseren Bezeichnung, die sich offensichtlich in den
vergangenen 20 Jahren nicht eingestellt hat, soll daher hier auf Begriff
und Konzept der Hybridbibliothek erneut ausdrücklich hingewiesen werden,
auf ihr Verstetigungspotential, ihre Tragfähigkeit, wenn nicht sogar
Notwendigkeit für geisteswissenschaftliche Bibliotheken.

\section*{\texorpdfstring{Zur Bezeichnung
\enquote{Hybridbibliothek}}{Zur Bezeichnung Hybridbibliothek}}\label{zur-bezeichnung-hybridbibliothek}

Rainer Kuhlen (2002) hielt \enquote{Hybridbibliothek} für eine
einfältige Namensgebung für ein Nebeneinander von gedruckten und
elektronischen Informationsobjekten.\footnote{Kuhlen, Rainer (2002):
  Abendländisches Schisma. Der Reformbedarf der Bibliotheken. In: FAZ
  Nr. 81.2002 vom 08.04.2002, S. 46.}

Es ist in der Tat keine glückliche Benennung, eher eine
\enquote{Sackgasse der Jargonbildung}.\footnote{Bachmann-Medick, Doris
  (2006) Cultural turns, S. 11.} Diese Gefahr durch Beinamen für
Bibliothekstypen erkennt bereits Ulrich Naumann (2004) in seinem Beitrag
\enquote{Über die Zukunft der namenlos gemachten Bibliothek}. Er kommt
zu dem Ergebnis, \enquote{dass eine namenlos gemachte Bibliothek keine
Bibliothek mehr sein wird und damit auch keine Zukunft hat
{[}\ldots{}{]}.} Überhaupt sei die Entwicklung von der traditionellen
\enquote{Festkörper}-Bibliothek zur Hybrid-Bibliothek nur eine
Transformation bibliothekarischer Tätigkeitsfelder und mache damit noch
keine Umbenennung der Sache nötig. Fast ungehalten schließt er seine
Überlegungen mit dem Ausruf: \enquote{Und in Deutschland heißen diese
Einrichtungen nun einmal \enquote{Bibliothek}!}\footnote{Naumann, Ulrich
  (2004). In: Bibliotheksdienst 28.2004 (11), S. 1399-1416. Aufzählung
  der Bibliothekstypen S. 1416, Fußnote 46.}

Es ist in den Fachbeiträge zu einer irritierenden Dauermarotte geworden,
den Traditionsbegriff \enquote{Bibliothek} und das Traditionsmedium
\enquote{Buch} mit einem Fragezeichen zu versehen oder mit
thanatologischen Formeln zu umdräuen.\footnote{Als rezentes Beispiel für
  überflüssige Fragezeichen siehe den Fachbeitrag von Achim Bonte (2015)
  Was ist eine Bibliothek?, worin es aber dem Autor durchaus um
  zeitgemäße Antworten geht, aber nicht ohne Fußnotenverweis zum
  \enquote{dead-end-job} librarian (S. 95) oder der Hefttitel Nr.
  10.2015 von BuB: Die Frankfurter Buch(?)messe. Für Thanatologisches
  siehe Klaus Ceynowa (2014) Der Text ist tot. Für die Kombination von
  Beidem z.B. Rob Bruijnzeels (2015): Die Bibliothek: aussterben,
  überleben oder erneuern? In: Bibliothek -- Forschung und Praxis,
  39(2015)2, S. 225-234.} Diese sind Teil verstetigter Rhetoriken, die
schon vor Jahrzehnten kritisiert wurden:

\begin{quote}
\enquote{Viele wissenschaftliche Untersuchungen, ja ganze Wissenschaften
sehen sich seit {[}\ldots{}{]} Jahren in mannigfachen Zusammenhängen
unserer öffentlichen Kultur durch Relevanzfragen bedrängt.
{[}\ldots{}{]} Relevanzfragen sind in der Wissenschaftspraxis nicht
Fragen einer Normalsituation. Es ist nicht normal, wenn Wissenschaftler
in einem Maße, wie es für die Gegenwart konstatierbar ist, statt mit
ihrer Wissenschaft sich mit der Beantwortung der Frage beschäftigen,
wofür ihre Wissenschaft gut sei. In solcher Selbstbeschäftigung steckt
ein pathologisches Moment; sie ist ein Krisenzeichen. Relevanzfragen
sind Indizien eines Schwunds kultureller
Selbstverständlichkeiten.}\footnote{Lübbe, Hermann (Hg. 1978) Wozu
  Philosophie? Stellungnahmen eines Arbeitskreises. Berlin, 1978. S. V.}
\end{quote}

Aleida Assmann hat diese Formen der Entwertung richtig als Mechanismen
der Herstellung von \enquote{Obsoleszenz}, des
\enquote{Zur-Vergangenheit-Erklärens} und als markante rhetorische
Formen des \enquote{Brechens}, des \enquote{Abwurfes von Balast}
analysiert.\footnote{Assmann, Aleida (2013), Ist die Zeit aus den
  Fugen?, S. 142. Auch S. 203.} Im Hinblick auf die Rezeption solcher
Fachbeiträge durch die je systemische Umgebung der einzelnen
Bibliotheken über Jahre hin erscheint dies als unverantwortliches Spiel
mit dem Feuer.

Es hat den Anschein, dass sich Teile der Fachgemeinde logisch in einer
Rhetorik der Rechtfertigung für den Traditionsbegriff
\enquote{Bibliothek} dauerhaft eingerichtet haben, wie sie Odo Marquard
als \enquote{Aggregatzustand der Tribunalsucht} beschrieben hat:

\begin{quote}
\enquote{Gegenwärtig herrscht weithin die Tendenz, alles und jedermann
zur Legitimation zu verpflichten. Jegliches soll in einen
\enquote{context of justification} eintreten {[}\ldots{}{]} und sich
rechtfertigen, insbesondere dann, wenn es in Legitimationskrisen geraten
ist; und das scheint heute {[}\ldots{}{]} überall der Fall. Und sollte
es irgendwo noch keine Legitimationskrise geben, wird sie notfalls
erfunden: im Interesse der Ubiquisierung des Rechtfertigungsverlangens.
Denn heute bedarf offenbar alles der Rechtfertigung: {[}\ldots{}{]} das
Leben, die Bildung, die Badehose, nur eines bedarf -- warum eigentlich?
-- keiner Rechtfertigung: die Notwendigkeit der Rechtfertigung von allem
und jedem.}\footnote{Marquard, Odo (1984) Entlastungen. Theodizeemotive
  in der neuzeitlichen Philosophie. Berlin : Siedler, 1984. S. 245.}
\end{quote}

Das \enquote{ceterum censeo}, dass man im Übrigen der Meinung sei, die
digitalen und die meisten anderen Medien- und Funktionsvarianten würden
alle dauerhaft und zureichend von dem Begriff \enquote{Bibliothek}
abgedeckt, führt folglich nicht zur Subsummierung der Hybridaufgabe
unter den Oberbegriff Bibliothek und nicht zum Verschwinden von
Behelfsbezeichnungen für die \enquote{zwitterhaften} Gesamtaufgaben
einer Bibliothek.

Ein Beispiel für das allmähliche Verblassen der Urteilsfähigkeit
gegenüber medialen Funktionsunterschieden in hybriden Wissensräumen ist
die Reflexion der Herausgeber des \enquote{Historisches Wörterbuch des
Mediengebrauchs} zu ihrer Wahl des Buchformates für die
Veröffentlichung: " \ldots{} das Format Buch ist nur dort überholt, wo
man es als simplen Container für Wissen versteht. Das Buch kann mehr.
Die Herausgeber haben sich für das Buch entschieden, weil es handlich
{[}sic{]} ist.``\footnote{Christians, Heiko (Hg. et al. 2015)
  Historisches Wörterbuch des Mediengebrauchs. Köln : Böhlau, 2015, S.
  7.} Mehr folgt nicht; die große Schlichtheit der Begründung erstaunt.

Peter Haber (2010) hatte richtig erkannt, dass der analoge Wissensraum
allmählich ins Hintertreffen zu geraten drohe,\footnote{Vgl. Haber,
  Peter (2010), Reise nach Digitalien und zurück. Ein
  historiographischer Betriebsausflug. S. 11.} was Verdrängungsszenarien
Nahrung gäbe.

Die eher unglückliche Suche nach einer neuen Benennung für eine
Traditionseinrichtung, die technikaffin, modern und medienintegrativ
auch das Aufkommen angeblich \enquote{körperloser} Medien begleitet,
zeigt die weiter oben bereits vermutete Unsicherheit im
Selbstverständnis und bei der klaren Positionierung von Bibliotheken als
dauerhafte physische Institutionen, denen Hybridität,
\enquote{Mischverhältnisse}, genuin aneignet.

Wie unglücklich eine Merkmalshervorhebung mittels des Begriffes
\enquote{hybrid} für Bibliotheken in multimedialen Zeiten ist, wird
deutlich in den kommunikationstheoretischen Überlegungen von Christina
Schachtner und Nicole Duller (2014). Die Autorinnen verwenden den
Begriff gerade nicht zur Charakterisierung analogmedialer und
digitalmedialer Begegnungsräume, sondern für das \enquote{hybride}
Kommunikationspotential von digitalen Medien selbst:

\begin{quote}
\enquote{Die Unterscheidung zwischen diskursiver und präsentativer
Symbolik eignet sich dazu, auch den Bedeutungsgehalt Digitaler Medien zu
bestimmen. {[}\ldots{}{]} Digitale Medien präsentieren sich als
\enquote{Bedeutungsmischlinge} {[}\ldots{}{]}. Die diskursive Symbolik
Digitaler Medien zeigt sich in Form von Algorithmen {[}\ldots{}{]}. Auf
eine präsentative Symbolik trifft man in Gestalt von Internetauftritten
und Websites {[}\ldots{}{]}. Wenn sich der Bedeutungsinhalt Digitaler
Medien aus diskursiven und präsentativen Elementen speist, so kann man
sie als hybride Medien bezeichnen. Hybridität ist Bestandteil einer
übergeordneten Symbolik, die sich bei Digitalen Medien zeigt
{[}\ldots{}{]}. Dieses Objektverständnis kontrastiert mit der
verbreiteten Auffassung, dass zwischen Materialität und Immaterialität
strikt zu trennen ist. {[}\ldots{}{]} Mit den Digitalen Medien rückt die
Kombination von Materialität und Immaterialität verstärkt ins
Bewusstsein, denn das eine kann ohne das andere nicht funktionieren. Die
Software ist es, die die Hardware überhaupt erst belebt; aber ohne
Hardware hätte die Software keinen Sinn. Digitale Medien werden zu
solchen erst in der Verschränkung von Hard- und Software, von
Materialität und Immaterialität.}\footnote{Schachtner, Christina;
  Duller, Nicole (2014) Kommunikationsort Internet. Digitale Praktiken
  und Subjektwerdung. In: Carstensen, Tanja (Hg. et al. 2014) Digitale
  Subjekte, S. 81-154. Hier S. 92f.}
\end{quote}

Zu begrüßen ist hier, dass kommunikationstheoretisch der Fokus auf die
Funktionsweisen von Medien in ihrer Materialität (oder ihrem Fehlen) und
ihrem diskursiven Potential vorbereitet wird, wie er im Folgenden hier
ins Visier genommen wird.

\section*{Mediensymbiose --
Medienkonkurrenz}\label{mediensymbiose-medienkonkurrenz}

Für die intermediale Hybridproblematik, wie sie hier in den Blick
genommen wird für Bibliotheken mit geisteswissenschaftlichem
Schwerpunkt, mithin für das Nebeneinander von analogem Buch und
digitaler Ressource zum Zweck geisteswissenschaftlicher Forschung kann
man, je nach medientheoretischem Standpunkt, Ergänzungs- oder
Verdrängungsprozesse konstatieren. Es bieten sich entsprechend die
Begriffe der Mediensymbiose oder Medienkonkurrenz an, kann von einem
dezidierten Nebeneinander bei funktionaler Ausdifferenzierung gesprochen
werden oder einem irreversiblen, evolutionsgetriebenen
Verdrängungsprozess.

Diese Begriffe führen auf das Terrain der Medien- und
Kommunikationswissenschaften. Hier sollen aus der Vielzahl von Theorien
zumindest zwei Ansätze vorgestellt werden, die den
bibliothekskonzeptionellen Hybridansatz dauerhaft stützen können: Das
Komplementaritätsgesetz (1913) von Wolfgang Riepl und die
Mediensystematik (1972) von Harry Pross.

Dabei wird im Hintergrund hier von der Idee einer Medienevolution
ausgegangen, wie sie Rudolf Stöber in seiner Mediengeschichte (2013)
vorstellte.\footnote{Stöber, Rudolf (2013) Neue Medien. Geschichte. Von
  Gutenberg bis Apple und Google. Medieninnovation und Evolution.
  {[}Gründlich revidierte, aktualisierte Neuaufl.{]} -- Bremen : edition
  lumière, 2013.}

Eine medienevolutionäre Annäherung an das Thema erscheint schon deshalb
angemessen, weil seit dem Aufkommen des \enquote{Hybridkonzeptes} Mitte
der 90er Jahre und seiner Verstetigung im Bereich der
Geisteswissenschaften seit 20 Jahren eigentlich kaum mehr von
revolutionärem Umbruch gesprochen werden kann. Das analoge Buch erfüllt
beharrlich und relativ unspektakulär in seinem Wirtschaftssegment und
speziell in den Geisteswissenschaften weiterhin seine traditionellen
angestammten Medienfunktionen. Der Aufmerksamkeitsfokus der
Expertengemeinden hingegen in Wissenschaft, Politik, Verlags- und
Bibliothekswesen scheint sich völlig auf den Digitalen Wandel und das
neue materielle und immaterielle Medienfunktionspotential des Digitalen
zu konzentrieren.

Die breit geführte Diskussion wächst sich immer dann zu einer
Kontroverse aus, wenn die Akzente zu sehr auf Medienkonkurrenz und
Verdrängungsszenarien gelegt werden, die sich aus dem bereits als
problematisch erkannten technischen Fortschrittstelos speisen, der die
Zwangsläufigkeit von Entwicklung mit der unwissenschaftlichen Annahme
unterlegt, das Neue sei das Bessere und das Bessere sei eben der Feind
des Guten oder Alten. Vorsichtiger wäre hier wohl die Annahme, das Neue
sei zunächst das Neue, müsse in den anvisierten medialen
Funktionsnischen sein Gutes erweisen und dort, wo es als das Bessere
gelten könne, sorge es für neue Funktionszuweisungen, eine
Ausdifferenzierung oder in seltenen Fällen für Verdrängung.\footnote{Zur
  Dialektik kultureller Ersetzungsmechanismen von Alt durch Neu vgl.
  Hermann Lübbe (1988) Der verkürzte Aufenthalt in der Gegenwart, dort
  exemplifiziert am Beispiel des \enquote{Buches} vor dem Hintergrund
  seiner \enquote{These vom abnehmenden Verpflichtungscharakter des
  Neuen mit zunehmender Menge seiner Auftritte} (S. 153):
  \enquote{Zunächst nimmt generell mit der Menge des Neuen pro
  Zeiteinheit der Neuigkeitswert des Neuen ab. Wie sich das auswirkt,
  ist uns exemplarisch aus der neuzeitlichen Geschichte des Lesens
  bekannt. {[}\ldots{}{]} Mit der Klage über die {[}\ldots{}{]}
  steigende Flut der Bücher {[}\ldots{}{]} ergab sich als primäre
  Leser-Reaktion diese: Man las öfter, man las mehr und man schaltete um
  vom intensiven aufs extensive Lesen, das heißt man las schneller.
  {[}\ldots{}{]} Das ist der Vorgang, den Ernst Curtius als Vorgang der
  Erschütterung der Autorität des Buches charakterisiert hat.
  {[}\ldots{}{]} Je rascher {[}\ldots{}{]} im Zeitalter temporaler
  Innovationsverdichtung das Neue veraltet, um so tiefer im Kurs sinkt
  sein Neuigkeitswert, und komplementär dazu restabilisiert sich die
  Geltung des Alten. {[}\ldots{}{]} Die Geschwindigkeit, mit der Altes
  noch älter wird, nimmt mit dem historischen Abstand vom gegenwärtigen
  Fortschritt ab. {[}\ldots{}{]} \enquote{Ikonische} Konstanz' hat Hans
  Blumenberg das genannt. {[}\ldots{}{]} Klassik -- das ist
  {[}\ldots{}{]} nichts anderes als erwiesene Selektionsresistenz in den
  Prozessen der {[}\ldots{}{]} Umorganisation {[}\ldots{}{]}.} S. 159ff.
  Lübbe beendet diesen konzentrierten Gedankengang mit dem
  bedenkenswerten Fazit: " {[}\ldots{}{]} wenn die Menge des Guten
  ohnehin schon sehr groß ist und überdies noch sich ständig
  fortschrittsabhängig erweitert, werden die Unkosten der Prüfung in
  Permanenz schließlich größer, als der Vorteil denkbarer Entdeckungen
  von etwas noch Besserem es jemals sein könnte." (S. 163).}

Entfällt die Kompetenz oder der Wille zur Bewertung und verliert sich
die Kraft des Urteils, kommt es zu dem, was Felix Stalder (2016) für
einen \enquote{dramatischen Wechsel} im Verhalten seiner Studierenden
hält: \enquote{Für die Studierenden von heute ist das \enquote{Neue}
nicht mehr neu, sondern selbstverständlich {[}sic!{]}, während sie
vieles, das bis vor Kurzem als normal galt -- etwa dass man ein Buch
physisch in der Bibliothek abholen muss --, inzwischen als unnötig
kompliziert erfahren.}\footnote{Stalder, Felix (2016) Kultur der
  Digitalität, S. 282.}

Neue Medien entstehen, so die Hypothese hier, aus Unzulänglichkeiten
Alter Medien oder aufgrund spezifisch neuer Entwicklungen,
gesellschaftlicher Bedürfnisse und technischer Neuerungen. Dieser
evolutionär zu betrachtende Prozess führt zu technologiegestützten
Übernahmen von Funktionen durch Neue Medien. Dies kann zu einer
verdrängenden Binnendifferenzierung innerhalb der bestehenden
Medienlandschaft führen (Medienkonkurrenz) oder zur Entstehung neuer
Funktionsnischen (Medienkomplementarität). Diese Nischenbildungen und
Funktionsverschiebungen gilt es, mit besonderer Sorgfalt zu analysieren.
Dabei kann es sich durchaus um Prozesse der Ausdifferenzierung und
Bildung von Funktionsnischen handeln, die Alten Medien
Alleinstellungsmerkmale zuweisen ebenso wie sie Neuen Medien
technikgestützte Alleinstellungsmerkmale zuweisen. Verdrängungsszenarien
greifen nur, so wird hier angenommen, wenn Funktionsnischen verlustfrei
von Neuen Medien besser \enquote{überabgedeckt} werden, mithin das Neue
tatsächlich das Bessere ist.

Die Rolle der öffentlichen Aufmerksamkeit für das Neue darf bei solchen
oft als \enquote{dramatisch} empfundenen Veränderungsprozessen durch
mediale Innovation nicht unterschätzt werden. Es ist anzunehmen, dass
gerade sie ein auslösender Faktor für den Glauben an Verdrängungstelos
oder den Eindruck von Änderungsszenarien \enquote{galaktischen} Ausmaßes
oder der Empfindung von \enquote{tektonischen} Verschiebungen ist.

\subsection*{Riepl'sches Komplementaritätsgesetz
(1913)}\label{rieplsches-komplementarituxe4tsgesetz-1913}

Für Hypothesenbildung nützlich und sehr anregend bietet sich immer noch
das Riepl'sche Komplementaritätsgesetz zur Stützung von Hybridansätzen
an. Hermann Rösch, Informationswissenschaftler, erwähnte es
verschiedentlich eher beiläufig: \enquote{Wolfgang Riepl hatte bereits
1913 darauf hingewiesen, dass neue Medien die alten nicht ersetzen,
sondern ergänzen. Die alten Medien positionieren sich neu im
Kommunikationsgefüge, es erwachsen ihnen neu zugeschnittene
Funktionsprofile.}\footnote{Rösch, Hermann (2005) Wissenschaftliche
  Kommunikation und Bibliotheken im Wandel. S. 92.}

Insofern dieser Grundsatz der Komplementarität (bei Wolfgang Riepl heißt
es im Original tatsächlich \enquote{Grundgesetz}\footnote{Riepl,
  Wolfgang (zuerst 1913) Das Nachrichtenwesen des Altertums. Nachdr.
  Hildesheim : Olms, 1972, S. 5.}) belastbar erscheint, können hybride
Mediensituationen folglich als gängiges Phänomen verstanden und
tendenziell für eine Dauererscheinung gehalten werden und von daher
schon als solide mittelfristige konzeptionelle Basis angesehen werden.
Dabei gilt es hingegen in dem hier gewählten Kontext, argumentativ zu
untermauern, warum dem Buch in geisteswissenschaftlichen
Spezialbibliotheken nicht nur eine ökologische Überlebensnische gewährt
werden solle, sondern ob es aufgrund von ausdifferenzierten
Funktionsprofilen seine volle Berechtigung auf Speicherstandort in
geisteswissenschaftlichen Bibliotheken habe.

Wo Wolfgang Riepl in der medientheoretischen Literatur zitiert wird,
mangelt es nicht an Beispielen und Gegenbeispielen für die Brauchbarkeit
oder eben Unbrauchbarkeit dieses Ansatzes.

Etwas ausführlicher untersucht Rudolf Stöber (2013) Riepls
Gesetz\footnote{Stöber, Rudolf (2013) Neue Medien. Geschichte, S.
  438-443.}. Nach etlichen Beispielen und medientheoretischen
Überlegungen resümiert er: \enquote{Da soziale Kommunikation den
archimedischen Punkt der Kommunikationswissenschaft ausmacht, ist
Funktionswandel der springende Punkt für die Erörterung der Frage, ob
Medien sterben können oder nicht.}\footnote{Ebd., S. 441.} Diesen
zentral wichtigen Hinweis auf die Funktionen eines Mediums und deren
Wandel gilt es im Folgenden zu beachten.

Urs Meier (2013) bilanziert nach 100 Jahren Riepl'schem Gesetz:

\begin{quote}
\enquote{Mit diesem einen Satz hat Riepl eine Hypothese hinterlassen,
deren heuristisches Potenzial erst Jahrzehnte später erkannt wurde. Sie
ist auch nach hundert Jahren noch nicht erledigt, sondern stimuliert
stets von neuem Forschung und Publizistik zu Fragen der
Medienentwicklung. {[}\ldots{}{]} Stark ist auch der Antrieb, aus
Riepl's Hypothese plausible Szenarien für die Weiterentwicklung der
Medien gewinnen zu wollen. Der Denkansatz, wonach bewährte Medien durch
technisch-ökonomisch-gesellschaftliche Entwicklungen nicht verdrängt,
sondern lediglich in ihren Funktionen verändert werden, hat sich immer
wieder als fruchtbar und richtig erwiesen. Der Medienwissenschaft gibt
er Anlass, genau solche Funktionsverschiebungen zu untersuchen
{[}\ldots{}{]}. Welche Aufgaben haben Bibliotheken zu erfüllen, wenn
Bücher zunehmend online lesbar sind?}\footnote{Meier, Urs (2014) 100
  Jahre Riepl'sches Gesetz. In: Kappes, Christoph (Hg. et al. 2014)
  Medienwandel kompakt 2011 -- 2013, S. 12f.}
\end{quote}

Genau diese Frage trifft den Gegenstandsbereich des Hybridkonzeptes. Es
sind die Funktionsverschiebungen von Medien, die nach Riepls Ansatz zu
analysieren sind und konzeptionell zu begleiten, die
kommunikationsrelevanten Impulse, die von ihrer Materialität und
Immaterialität ausgehen. Es ist hingegen gerade \emph{nicht} die
Institution der Bibliothek und deren funktionale Struktur, die
Gegenstand dieser Frage nach Medienevolution ist.\footnote{\enquote{Man
  darf {[}\ldots{}{]} nicht den Niedergang gedruckter Medien im
  Wissenschaftsbetrieb mit der Kultur der Bibliotheken verbinden und
  befürchten, dass diesen nun auch der Untergang drohe.} Ulrich Johannes
  Schneider, Direktor der UB Leipzig 2016 auf dem Bibliothekskongress.
  In: Kongressnews, Nr. 1 vom Montag, den 14. März 2016, S. 6.}

Die Kultureinrichtung ist \emph{sekundär} in Bezug auf das Lesemedium
wie auch Verlage oder Buchhandel. Möchte man Verschiebungen im medialen
Gefüge und neue Funktionszuschreibungen von Medien verstehen, ist
direkter Transfer von Medienfunktionsverschiebungen auf Institutionen zu
vermeiden. Beim Aufkommen von Mikrofiches und Mikrofilm ohne allzu
großes Aufmerksamkeitspotential hat keine Bibliothek mit dem Gedanken
gespielt, sich Mikrobibliothek zu nennen. Gerät nun der Mikrofiche in
Vergessenheit, gerät auch die Bibliothek in Vergessenheit? Verschwindet
die Videokassette, verschwindet auch die Bibliothek undsofort?
Digitalisieren sich Formen des \enquote{Lesens} und werden mobil und
ortlos, digitalisiert sich auch die Bibliothek und wird virtuell und
fluide?

Im Folgenden wird dafür plädiert, Funktionsverschiebung nur auf der
medialen Ebene und im Vergleich der Medien untereinander zu betrachten.
\emph{Hier liegt tatsächlich der Kern der Hybridproblematik. Welche
Funktionen können oder müssen innerhalb der geisteswissenschaftlichen
Forschung dem Digitalmedium bzw. dem Analogmedium zugeordnet bleiben, wo
sind sie austauschbar, wo kann man Alleinstellungsmerkmale ausmachen?}

Die aus dem Riepl'schen Ansatz abzuleitende Frage ist folglich:
\enquote{Welche Aufgabe haben analoge Bücher zu erfüllen, wenn sie
zunehmend online ortlos lesbar werden?} Auch hier hilft die Präzisierung
der Fragerichtung, um den medialen Kern freizulegen. Es sind ja nicht
\enquote{Bücher}, die \enquote{zunehmend online lesbar} sind. \emph{Das
analoge Buch darf in seiner Medialität und Materialität nicht mit dem
digitalen Parallelmedium einer binären Datei verwechselt werden}.
Wolfgang Frühwald: \enquote{In den großen Internetprojekten
{[}\ldots{}{]} werden \emph{Texte}, nicht \emph{Bücher}
digitalisiert.}\footnote{Frühwald, Wolfgang (2011) Gutenbergs Galaxis
  oder Von der Wandlungsfähigkeit des Buches. In: Zintzen, Clemens (Hg.
  2011) Die Zukunft des Buches. S. 9-21. Hier S. 13.} Das
medientechnische Funktionieren, ihr jeweiliges \enquote{An- und
Ausschalten}, ihre je eigenen Mensch-Objekt-Schnittstellen, ihre
diskursiven und präsentativen Kommunikationspotenziale, ihr
epistemisches Funktionieren, ihre \enquote{Vernetztheit} sind
verschieden, weil sie unterschiedlichen Medienklassen zugehören. Das
wird weiter unten noch klarer ausgeführt.

Es mag erstaunen, wenn hier auch 25 Jahre nach der ersten Einführung
eines E-Book-Readers von Sony 1990 noch darauf insistiert wird: Es ist
objektiv nicht das Buch, was online lesbar wird, es ist eine mediale,
über elektronische Endgeräte vermittelte binäre Computerdatei:
\enquote{Mediale Substitutionslogiken greifen zu kurz; vielmehr geht es
um ein Bewusstsein je medienspezifischer Möglichkeiten und
Grenzen.}\footnote{Krameritsch, Jakob (2007) Geschichte(n) im Netzwerk,
  S. 23.}

Es ist in diesem Punkt nicht uninteressant zu beobachten, wie von Hard-
und Softwareseite aus, mimetische Anstrengungen unternommen werden, um
das analoge Buch zu suggerieren. Man denke an die Versuche zur
Reduzierung des \enquote{Selbstleuchtens}\footnote{Zur zentralen
  Bedeutung von Sendelicht, Zeigelicht und Beleuchtungslicht für den
  Lesevorgang, der spirituellen Optik des \enquote{lumen oculorum}, des
  Augenlichtes als Eigenlicht seit der Frühscholastik siehe Illich, Ivan
  (1991) Im Weinberg des Textes, hier S. 25-27.} der Digitaltexte, die
visuelle und akustische Imitation der \enquote{Blätterfunktion}, die
Digitalwerkzeuge für Textmarkierung oder Marginalien und s
weiter.\footnote{Zum Stand der Entwicklung von E-Books und ihrer
  gerätetechnischen Basis, den E-Readern siehe: Jungbluth, Anja (2015)
  Vor Kindle. Die Anfänge des E-Books. In: Perspektive Bibliothek,
  4(2015)2, S. 87-106. Hier besonders Kapitel 2.4: Vor- und Nachteile
  gegenüber Print.}

Zwecks Verdeutlichung der technisch unterschiedlichen Herkunftswelten
kann hier das unterschiedliche Funktionieren eines vergleichbaren
Elements beider Medienvarianten herangezogen werden: Die
\enquote{Fußnote} und der \enquote{Link} als Formen analoger
beziehungsweise digitaler \enquote{Vernetztheit} in wissenschaftlichen
Texten.

Relativ früh hat Peter Sloterdijk (1993) die Kernproblematik der zu
erwartenden technikgetriebenen Funktionserweiterung von Text durch
\enquote{Hypertext} erkannt:

\begin{quote}
\enquote{Das Prinzip Text gründete {[}\ldots{}{]} in der Begrenzbarkeit
der Fäden und Gewebe; das Prinzip Buch hatte seine regulative Idee in
der Vorstellung, daß in irgendeiner Tiefe {[}\ldots{}{]} eine Äquivalenz
von Buchform und Weltform in Kraft sei {[}\ldots{}{]}. Heute steht es um
den Kinderglauben an die weltbeschwörende Macht des aus einfachen Zeilen
gewobenen Buches schlecht; {[}\ldots{}{]} nun bricht mit der Entdeckung
des Hypertextes die Katastrophe der Buchförmigkeit über uns herein --
Linearität erweist sich als ein zu schwaches Organisationsprinzip, um
der neuen Weltform des verzweigten und verknäuelten Wissens gewachsen zu
sein, {[}\ldots{}{]}. Das Prinzip Zeile insgesamt wird abgelöst vom
Prinzip Knoten oder Schnittpunkt, jedes Wort könnte Ausgangspunkt sein
zu einem Sprung in ein anderes Archiv, jeder Satz könnte gleichsam in
mehreren Richtungen weitergehen, der Text wird vom zweidimensionalen
Gewebe zum dreidimensionalen Verweisungsknäuel, {[}\ldots{}{]}.
{[}\ldots{}{]} das arme alte Buch seufzt unter den Spannungen einer
Polyvalenz, zu deren Beherbergung es anfangs nicht geschaffen war.
{[}\ldots{}{]} was früher die Fußnote war, wird jetzt zum selbständig
nutzbaren Fahrzeug in einer Nebenwelt der mitwißbaren
Parallelinformation. Aus dem Buch wird der Knotenpunkt im
bibliographischen Archipel, aus der Zeile das multidimensionale
Informationsknäuel, aus der Fußnote die Fernreise
{[}\ldots{}{]}.}\footnote{Sloterdijk, Peter (1993) Zum Empfang des
  Ernst-Robert-Curtius-Preises, S. 52ff.}
\end{quote}

Das war zwar präzise und beziehungsreich analysiert, die
\enquote{Katastrophe der Buchförmigkeit} steht aber noch immer
aus.\footnote{Vgl. ebenso die Skepsis bei Zimmer, Dieter E. (2000) Die
  Bibliothek der Zukunft, Kapitel \enquote{Hypertext oder Absage ans
  Lineare}, S. 52-60.} Nebenbei bemerkt scheint also die
\enquote{Äquivalenz von Buchform und Weltform} noch immer in Kraft zu
sein, wobei man die \enquote{Buchförmigkeit} als
\enquote{Buchfrömmigkeit} lesen könnte: \enquote{Vielleicht kann das
Christentum als Buchreligion gar nicht anders, als zu den materiellen
Verkörperungen der Heiligen Schrift ein besonders inniges ambivalentes
Verhältnis zu entwickeln.}\footnote{Groebner, Valentin (2014),
  Wissenschaftssprache digital, S. 56. Ebenso Kopp, Vanina (2016) Der
  König und die Bücher: \enquote{Das Buch ist der traditionelle Träger
  des abendländischen Wissens, es ist die Basis des Christentums als so
  genannter}Religion des Buches" {[}\ldots{}{]}``, S. 33.}

Es bleibt die Zusammenstellung eines Handapparates in einer
Präsenzbibliothek nach wie vor praktisch die analoge Materialisierung
der Fußnotenvernetztheit eines Kerntextes, der auf ein Thema hinführt.
In gut sortierten analogen Forschungssammlungen sollte sich ein hoher
Anteil dieses analogen Netzwerkes \enquote{materialisieren} lassen.
Fehlgehende \enquote{Analog-Links} lassen sich durch Fernleihe oder
Umstieg in die andere, die digitale Medienklasse realisieren.

Das Zusammenrufen digitaler Links aus einer elektronischen Netzressource
heraus per Mausklick sollte ebenfalls fast lückenlos gelingen durch
Download auf eigenen Speicherplatz oder durch netzunterstützte
Fernkonsultation auf fremden Servern. Der
\enquote{Nicht-Materialisierung} eines \enquote{Analoglinks} entspricht
dann annähernd der http-Statuscode 404 = \enquote{Die angeforderte
Ressource wurde nicht gefunden} oder http-Statuscode 406 = \enquote{Die
angeforderte Ressource steht nicht in der gewünschten Form zur
Verfügung} und so weiter. Solche fehlgehenden Digitallinks können dann
durch ergänzende Online-Recherchen oder Lizensierung von Digitalcontent
realisiert werden oder durch den Umstieg auf die andere, die analoge
Medienklasse.

Es dürfte deutlich geworden sein, dass zu jedem Moment die Herstellung,
die Aktivierung der \enquote{Vernetztheit} und ihre Komplettierung einer
je eigenen medialen Logik und Logistik folgt. Ein umfassendes
Hybridverständnis nimmt genau dies deutlich in den Blick. Welches sind
vor dem Hintergrund der Riepl'schen Komplementaritätsannahme die
Funktionsverschiebungen und neuen Funktionszuteilungen zwischen den
beiden Medienklassen analog und digital? Welche konzeptionellen
Strategien ermöglichen \enquote{Lesen} und \enquote{Schreiben},
\enquote{Wissensarbeit} in allen ihren Formen in der Institution
Bibliothek, die sich selbstredend als hybrider Wissensraum darstellt?
Wie wäre Ausgewogenheit im Angebot der medientypischen
\emph{Funktionsweisen} in Bibliotheken zu realisieren, besonders in
geisteswissenschaftlichen Bibliotheken? Peter Strohschneider fand für
diese Funktionsbündelung den etwas sperrigen Begriff \enquote{hybride
Ko-Operationsfelder, in denen Materielles (Sammlungsgut bzw. technische
Apparaturen), Epistemisches ({[}\ldots{}{]} Wissen und seine Ordnungen)
und Soziales ({[}\ldots{}{]} Forschergruppen {[}\ldots{}{]}) aufeinander
bezogen sind.}\footnote{Strohschneider, Peter (2012) Faszinationskraft
  der Dinge, S. 23.}

Solange die Anstrengungen von Bibliotheken im Bereich der
Informationskompetenzvermittlung und der Teaching Library im Telos der
digitaltechnisch getriebenen Medienverdrängung ihren Ursprung haben,
solange sie ihren didaktischen Fokus nur im Digitalen platzieren, in
ihren Magerstufen reduziert auf die Vermittlung von Hard- und
Softwarebedienung, wird Kompetenzvermittlung für umfassende und
effiziente Wissensgenerierungsprozesse unter den Bedingungen
multimedialer Diversität nicht gelingen. Ausgleichend müsste die
Reauratisierung analoger Herkunftswelten und -arbeitstechniken dazu
gehören, wie sie zum Beispiel vom Forschungsfeld der Materialität von
Wissensarbeit zunehmend wieder geleistet wird.

Erst hier kommt dann auch wieder richtig die Institution Bibliothek ins
Spiel mit ihren \enquote{Strategiebausteinen} Lernort, Makerspace,
Wissenschaftssalon oder alles das, was in den letzten Jahren, teils als
Resultate konzeptionellen Driftens, auch immer in Vorschlag gebracht
wurde. Selbst die Idee der barocken Kunst- und Wunderkammer wurde jüngst
von Achim Bonte (2015, SLUB Dresden) wiederbelebt.\footnote{Bonte, Achim
  (2015) Was ist eine Bibliothek? In: ABI Technik 35(2015)2, S. 103.}

\subsection*{Mediensystematik (1972) von Harry
Pross}\label{mediensystematik-1972-von-harry-pross}

Ein zweites mediales Analysekonzept kann diese Überlegungen noch
verstärken. Aus der Vielfalt medientheoretischer Ansätze soll dazu das
Modell von Harry Pross (1972) wiederaufgegriffen werden. Hier findet
sich ein medientechnischer Aspekt, der den Riepl'schen Ansatz und die
Hybrididee der \emph{medialen Funktionsweisen}, wie jetzt genauer
formuliert werden kann, vom Geräteaufwand und der Apparatur her
plausibel macht.\footnote{Wiederaufgegriffen hat diesen Ansatz auch
  Thordolf Lipp (2011) in Bezug auf Intangible Cultural Heritage und ihn
  um eine Vierte Medienklasse, die Quartärmedien fortgeschrieben. Mit
  dieser Medienklasse meint er ausschließlich das Internet und
  \enquote{vermischt} dann wieder, was Harry Pross so überzeugend
  unterschieden hatte: \enquote{Das Quartärmedium Internet vereint
  Elemente aller vorhergehenden Medientechnologien {[}\ldots{}{]}.} Vgl.
  Lipp, Thordolf: Arbeit am medialen Gedächtnis. In: Robertson-von
  Trotha, Caroline Y. (Hg. et al. 2011) Neues Erbe, S. 39-67. Hier S.
  49.}

Harry Pross unterschied primäre, sekundäre und tertiäre Medien nach
ihrer technikfreien oder technikgebundenen Vermittlung und legt damit
den Akzent auf einen zentralen Aspekt der anthropozentrischen
Kommunikation.

Nach ihm sind primäre Medien alle Kommunikationsformen des menschlichen
Elementarkontaktes mit dem Außen und der Welt wie Lachen, Weinen,
Sprache, Gesten, Zeremoniell, Taxis und so weiter ohne Geräte. Der
Medienbegriff definiert sich hier durch Absenz von Apparatur und
Technik. Werner Faulstich nennt sie \enquote{Menschmedien}.\footnote{Siehe
  Werner Faulstichs fünfbändige \enquote{Geschichte der Medien} (1996 --
  2004). \enquote{Die Publizistikwissenschaften nennt sie Primärmedien,
  aber das ist eher verschleiernd. Ich nenne sie Menschmedien
  {[}\ldots{}{]}.} In: Podiumsdiskussion \enquote{Begann die Neuzeit mit
  dem Buchdruck?} 2005, S. 21 (siehe Bibliographie).} Ihre
\enquote{Lesbarkeit} ist prinzipiell
anthropologisch-universell.\footnote{Ähnlich die kommunikativen
  Universalien bei Aby Warburg und seine Theorie der \enquote{Engramme
  leidenschaftlicher Erfahrung} als Gedächtnisschatz in: Raulff, Ulrich
  (2003) Wilde Energien. Vier Versuche zu Aby Warburg, S. 36f:
  \enquote{Im \enquote{Urprägewerk} frühgriechischer und
  kleinasiatischer Kulte waren dauerhafte Formeln körperlichen Ausdrucks
  der Leidenschaft geprägt worden, die sich unweigerlich jedem
  Nachgeborenen aufdrängten, der vom \enquote{Ausdruckszwang} ergriffen
  wurde.}}

Wichtig ist Christa Karpenstein-Eßbachs (2004) Aufgreifen dieses
Medienbegriffs aus kultur\-wissen\-schaftlich-anthropologischer Sicht und
ihr Hinweis auf die \emph{Einheit aller Sinnestätigkeiten}, die durch
das Einbringen von Artefakten verschoben würde: \enquote{{[}\ldots{}{]}
wie die \enquote{natürliche} apparatfreie Wahrnehmung die Einheit der
Sinne kennt, streben {[}\ldots{}{]} Artefakte und Medien nach der
Ermöglichung synästhetischer Sinneserfahrung. Wir haben es hier mit
verschiedenen Technisierungen der Sinne zu tun. Diese {[}\ldots{}{]}
erzeugen eine eigene Welt der artifizierten Sinnestätigkeit.
{[}\ldots{}{]} zeitenthoben und raumentbunden {[}\ldots{}{]}.}\footnote{Karpenstein-Eßbach,
  Christa (2004) Einführung in die Kulturwissenschaft der Medien, S. 13.}
Rudolf Stöber (2013) spricht unter dem Blickwinkel der Medienevolution
von den \enquote{primären Proto-Medien (Sprache, Gestik,
Mimik)}.\footnote{Stöber, Rudolf (2013) Neue Medien. Geschichte, S. 404.}

Sekundärmedien nach Harry Pross sind solche Kommunikationsmittel, die
eine Botschaft zu einem potentiellen Empfänger transportieren, ohne dass
dieser ein technisches Empfangsgerät benötigt außer seinen natürlichen
Sinnesorganen, hier also Bild, Schrift, Druck, Graphik, Fotographie,
auch Brief, Flugschrift, Buch, Zeitschrift, Zeitung, also Presse im
weitesten Sinne. Hier besteht stets Gerätegebrauch auf der Senderseite,
die Wahrnehmung der Kommunikation auf Empfängerseite ist prinzipiell
apparatefrei. Bei Stöber sind es auf der zweiten Medienevolutionsstufe
die sekundären Basis-Medien (Schrift und Bild).\footnote{Ebd., S. 404.}

Tertiäre Medien umfassen Telegrafie, Nachrichtenagenturen, Schallplatte,
Tonband, Film, Radio und Fernsehen. Hier benötigen nach Harry Pross
sowohl Sender als auch Empfänger Geräte, womit die elektronischen
Kommunikationsmittel jeglicher Art zu diesen tertiären Medien
zählen.\footnote{Vgl. ebd., S. 15-17.}

Diese durch Apparatur auf beiden Seiten des Kommunikationskanals
definierte Medienklasse erweist sich als ein sehr weittragendes und
belastbares Analyseinstrument. Stöber charakterisiert diese dritte
Medienevolutionsstufe als \enquote{Verbreitungs-Medien}.\footnote{Ebd.,
  S. 404} Für ihn ist Harry Pross' mediale Dreiteilung überzeugend und
einfach, weil sie in ihrem Kern auf das Kriterium der Technik
abhebt.\footnote{Vgl. ebd., S. 17.} Das Analogmedium Buch gehört nach
diese Einteilung zu den Sekundärmedien. Jedes Digitalmedium gehört zu
den Tertiärmedien. \emph{Damit ist das Hybridkonzept im Kern der Versuch
einer Symbiose zweier Medienklassen mit ganz unterschiedlichen medialen
Charakteristika.} Im Folgenden soll durchgespielt werden, wie weit die
Mediensystematik von Harry Pross zu einer Klärung zentraler Begriffe der
Hybridproblematik wie Symbiose, Konvergenz, Verdrängung oder
Komplementarität beitragen könnte.

Das Schreiben, egal aus welchem Jahrhundert und mit welcher Technik,
ermöglicht das Aussenden eines Kommunikationsangebotes. Das Auslesen
desselben ist technikfreie Rezeption beim Empfänger. Welche zeitliche
Dehnung und Verzögerung für die Rezeption eines ausgesendeten
Kommunikationsangebotes in Schriftform diese Freistellung des Empfängers
von Technik und Gerät ermöglicht, veranschaulicht der Rosettastein (196
v. Chr.) oder jedes jahrhundertealte Druckwerk. Das \enquote{Anschalten}
und Auslesen des \enquote{Device} (im Sinne von Daniel J. Boorstin,
siehe Fußnote 31) erfolgt durch \enquote{In-Augenschein-nehmen}.

Das E-Book beziehungsweise jedes digitale Medium und sein
Kommunikationsangebot ist technikvermittelt auf Sender- und
Empfängerseite. Damit gehört alles Digitale in die Klasse der tertiären,
doppelseitig technisch vermittelten Medien.

Die Theorie von Harry Pross liefert mit dem Aufzeigen dieser doppelten
medientechnischen Vermitteltheit eine überzeugende Grundlage für die
Begründung von dauerhaftem hybridem Nebeneinander in Mediensammlungen:
Analogtexte und Digitaltexte gehören nicht in die gleiche
Medienkategorie und repräsentieren unterschiedliche Grade der
Technisierung von Sinneserfahrungen und damit letztlich auch von
geisteswissenschaftlichen Wissensgenerierungsprozessen, wie sie hier
relevant sind für die Frage nach ihrer Ermöglichung in hybriden
Mediensammlungen.

Analogtexte sind senderseitig technikvermittelt. Wenn sich hier
\enquote{Code} manifestiert, bleibt er in Abhängigkeit von der
kulturellen Nähe oder Ferne von Sender und Empfänger prinzipiell von
Menschen immer und stets auslesbar und damit interpretierbar. Dies ist
das zentrale Kriterium für textorientierte Forschung in den
Geschichtswissenschaften, welches das Modell von Harry Pross mit seiner
schlichten Einteilung in technische Medienklassen deutlich sichtbar
macht.

In dem oben genannten Beispiel war es der
ägyptisch-demotisch-altgriechische schrifttechnisch vermittelte
Rosettastein 196 v. Chr. und dessen Rezeption durch den Franzosen
Jean-François Champollion 1822.\footnote{Zu den epistemologischen
  Umständen der Entzifferung siehe die Spezialuntersuchung von Markus
  Messling (2012) Champollions Hieroglyphen. Philologie und
  Weltaneignung. Berlin : Kulturverlag Kadmos, 2012.} Die zeitliche
Dehnung der Kommunikation kann natürlich noch länger sein. So ist es
dieses technikfreie In-Augenschein-nehmen, das Uwe Jochum als
Basisprinzip für das Buch \enquote{an der Wand} vorsichtig fragend sogar
bis zu den Botschaften der Höhlenmalerei zurückverfolgt.\footnote{Jochum,
  Uwe (2015) Bücher. Vom Papyrus zum E-Book. S. 9ff. Ebenso Werner
  Faulstich: \enquote{Schreibmedien gab es von Anfang an -- die Wand,
  später die Tafel {[}\ldots{}{]}. Der sogenannten Höhlen'malerei'
  beispielsweise, einer von Tierbildern und Inzisionen geprägten Nutzung
  des Kommunikationsmediums Wand, wird dabei kulturkonstituierende
  Bedeutung zugesprochen.} In: Faulstich, Werner (1998) Medien zwischen
  Herrschaft und Revolte, S. 8.}

Wird der kulturelle Kontext eines gemeinsamen analogen Codes verlassen,
kann er trotzdem mit Hilfe von Gedächtnisinstitutionen, Memorialkultur
und perzeptiver Intelligenz auf Empfängerseite wieder erschlossen
werden, ohne dass ein rein technisches Problem auf Empfängerseite jede
Rezeption abschließend unmöglich macht. Sein Sinngehalt bleibt in ihm
eingelagert, solange er physisch existiert.

Digitaltexte sind sender- und empfängerseitig technikvermittelt. Wenn
sich hier Code manifestiert, ist er in Abhängigkeit von der gerade
gültigen Maschinenkompatibilität prinzipiell übertragbar, aber von
Menschen an beiden Kommunikationsenden prinzipiell nicht lesbar, da es
sich primär um intermaschinellen Code handelt. Dieser Code ist auf
Sender- und Empfängerseite jeweils nur über
Mensch/Maschine-Schnittstellen produzierbar und rezipierbar.

Die zeitliche Dehnfähigkeit dieser Kommunikation wird bestimmt durch die
kulturtechnische Gültigkeit und Kompatibilität der Codeversion und
Hardwaretechnik. \enquote{Code ist {[}\ldots{}{]} nur innerhalb der
sozio-technischen Umgebungen interpretierbar, in denen er eingesetzt
wird.}\footnote{Meßner, Daniel (2015) Coding History. In: Schmale,
  Wolfgang (Hg. 2015) Digital Humanities, S. 159.} Wird hier der
kulturtechnische Kontext verlassen, ist der Code prinzipiell hard- und
softwareseitig an beiden Enden der Kommunikation nicht mehr zugänglich.
Hier können nur Techniken der digitalen Langzeitarchivierung\footnote{Vgl.
  dazu die beiden Bände der Reihe \enquote{Kulturelle Überlieferung --
  digital} des Karlsruher Instituts für Technologie (KIT): Bd. 1: Neues
  Erbe (2011) und Bd. 2: Digitales Kulturerbe (2015), beide hg. von
  Caroline Y. Robertson-von Trotha; genaue Angaben siehe Bibliographie.}
wie Code-Migration oder Emulation früherer kulturtechnischer Hardware-
und Digitalumgebungen die zeitliche Dehnung überwinden und den Code
wieder erschließen. Er muss im schlimmsten Fall technikbedingt als
verloren gelten. Hermeneutik hilft hier nicht mehr weiter, da sein
diskursives Potential sich nur im reibungslosen Zusammenspiel von Hard-
und Software entfaltet. Christiane Heibach (2011) spricht vom volatilen
Charakter von Soft- und Hardware, der zu hinterfragen sei.\footnote{Heibach,
  Christiane (2011) (De)Leth(h)e. In: Das Ende der Bibliothek?, S. 62.}

Die explizit technikzentrierte Medieneinteilung von Harry Pross ist
deshalb für die Analyse von Mediensymbiose oder Medienkonkurrenz oder
gar -verdrängung so fruchtbar. Sie nimmt primär die Technikabhängigkeit
von medialer Kommunikation in den Blick. Da diese bei digitalen Texten
ungleich höher ist beziehungsweise mit Blick auf die
Sender/Empfänger-Situation gedoppelt, mit Blick auf die
Hardware-/Softwarevoraussetzungen auf Sender- und Empfängerseite
vervierfacht, lassen sich aus diesem Blick auf die technischen
Medienkonditionen Argumente für einen dezidierten Hybridansatz in den
textorientierten Geisteswissenschaften ableiten.

Besonders die unterschiedliche Dehnfähigkeit für potenziell
erfolgreiches Statthaben von Kommunikation bei sekundären oder tertiären
Medien muss die Geschichtswissenschaften und Gedächtnisinstitutionen in
besonderem Maße interessieren. Die Gefahr des gänzlichen Verschwindens
manifester Kommunikationsfähigkeit ist bei tertiären Medien ungleich
höher.\footnote{Das Verschwinden von Wissensbeständen für die Frühe
  Neuzeit thematisiert Martin Mulsow (2012) Prekäres Wissen. Eine andere
  Ideengeschichte der Frühen Neuzeit. Berlin : Suhrkamp. Für digitale
  Wissensbestände gibt es wohl nur Einzelhinweise, aber noch keine
  systematische, breit angelegte Untersuchung.}

\subsection*{Latenz}\label{latenz}

Konkret gerät damit insbesondere von digitaltechnischer Seite her auch
in Gefahr, was Peter Strohschneider, Präsident der Deutschen
Forschungsgemeinschaft, für wissenschaftliche Sammlungen, auch für
Bibliotheken, fordert: Zukunftsoffenheit und institutionalisierte
Latenz. \enquote{Die künftige Gebrauchsform {[}Lektüre{]} der Bücher ist
{[}\ldots{}{]} jederzeit schon antizipiert, allerdings nicht auch ihr
künftiger Sinn. {[}\ldots{}{]} Die Bibliothek {[}akkumuliert{]} im
Medium der Bücher dynamische Sinnsysteme {[}\ldots{}{]}. Darin speichert
die Bibliothek, was {[}\ldots{}{]} erst mit dem Modus der Sammlung
institutionalisiert wird: Latenz.}\footnote{Strohschneider, Peter (2012)
  Faszinationskraft der Dinge, S. 17, Fußnote 24.} Strohschneider
weiter: \enquote{Dies {[}ein schon gegebenes und doch im epistemischen
Prozess noch nicht antizipierbares Potenzial{]} bewahren die Dinge als
Latenz: als Möglichkeit einer späteren Befassung mit anderen
Erkenntnisinteressen, anderem Aufmerksamkeitsfokus, anderen Methoden, in
anderen Theorierahmen.}\footnote{Ebd., S. 18.} Das lässt sich mit
Sicherheit nur erhoffen, wenn eine spätere Rezeption der möglichen
Deutungsvarianten, wie das nur bei sekundären Medien grundsätzlich
anzunehmen wäre, nicht in Frage steht.

Und diese Zukunftsoffenheit sieht Strohschneider geradezu als
anthropologisches Konstituens: \enquote{Es gibt neben dem
antizipierbaren Zukünftigen auch die offene Zukunft: das unbekannte und
das unerwartbare Neue. Und Menschen wissen, dass es dies gibt: Menschen
sind zukunftsoffene Wesen. Deswegen brauchen sie neben der
Vorratshaltung auch die Sammlung.}\footnote{Ebd., S. 17.}

Hier kann ein interessanter Nebengedanke nur angedeutet werden. Die
Latenzidee geht von einem Variantenreichtum späterer Perzeption aus, wie
er in einem \enquote{geschlossenen} Wissenscontainer wie dem Buch oder
einem Kunstwerk transportiert wird. Offensichtlich haben die Starre des
Gedruckten beziehungsweise andere Formen analoger Fixierung nur physisch
den Anschein der mangelnden Offenheit und Flexibilität gegenüber der
Idee \enquote{fluider} digitaler Dokumente. Die intellektuelle Offenheit
paaren analog fixierte Objekte nachweislich mit soliden
konservatorischen Qualitäten und ihrer Rezeptionsoffenheit als sekundäre
Medien. \enquote{Erkenntnisse, die aus Archiven kommen, können
jahrhundertelang schlummern.}\footnote{Wyss, Beat (2010) Bilder von der
  Globalisierung, S. 9.} Oder mit den Worten Wolfgang Frühwalds:
\enquote{Auch die obskurste Monographie besitzt ihr
Auferstehungspotential.}\footnote{Frühwald, Wolfang (2002) Gutenbergs
  Galaxis im 21. Jahrhundert. In: ZfBB 49(2002)4, S. 190.}

Nun wird aber gerade die vernetzte Offenheit, die Entgrenztheit
digitaler verlinkter Wissensobjekte häufig als ihre Qualität
herausgestellt gegenüber sequenzieller, linearer Enge gutenbergscher
Wissenscontainer. Rafael Ball (2013): \enquote{Die Nachteile des
gedruckten Buchs als Medium sind evident: Es lässt nur eine einzige
Antwort auf gestellte Fragen zu und -- einmal gedruckt -- muss es nahezu
unwidersprochen bleiben und beansprucht scheinbar ein für allemal
Gültigkeit. Dynamische Dokumente hingegen konterkarieren diese Gefahr
mit dem Angebot der digitalen Beliebigkeit und permanenten
Veränderbarkeit von Information und Inhalten.}\footnote{Ball, Rafael
  (2013) Was von Bibliotheken wirklich bleibt. S. 49f.}

Wenn also Digitalobjekte -- zumal im Rahmen kollaborativer
Arbeitsprozesse in den Geisteswissenschaften -- permanentem Palimpsest
zugänglich werden, haben sie allerdings alle Eigenschaften eines
\enquote{fluiden} Dokumentes. In welchen \enquote{Petrischalen},
metaphorisch gesprochen, solche fluiden entgrenzten Wissensobjekte dann
in digitalen wissenschaftlichen Sammlungen aufbewahrt werden für
späteres Verifizieren, Falsifizieren oder Ausloten ihrer Latenz(en), ist
noch immer nicht klar.\footnote{Christiane Heibach (2011) (De)Let(h)e,
  S. 61: \enquote{{[}\ldots{}{]} \emph{Hypercard} von Apple, das erste
  Programm, mit dem Mitte der 1980er Jahre Hypertexte erstellt werden
  konnten, sind so schnell wieder verschwunden, daß die Texte --
  darunter die ersten literarischen Hyperfictions -- heute nicht mehr
  lesbar sind. Die Flüchtigkeit und beständige Veränderung der Software
  steht dem Ziel der nachhaltigen Speicherung diametral entgegen.}}
Erstaunlich auch, dass Rafael Ball die von Peter Strohschneider
angedeutete \enquote{intellektuelle Offenheit} in Büchern nicht
wahrzunehmen scheint. Ähnlich Michael Hagner: \enquote{Als ob der
Buchdruck für Starrheit und Monumentalität stünde, besteht das
Faszinosum des Buches {[}\ldots{}{]} in seiner Flexibilität, seiner
historischen Wandelbarkeit. Mit diesen Eigenschaften ist es für Denken,
Phantasie und Gedächtnis des Menschen konstitutiv.}\footnote{Hagner,
  Michael (2015) Zur Sache des Buches, S. 132.}

Selbst wenn man Rafael Ball zugute hält, dass es ihm bei seiner Aussage
um naturwissenschaftliche gedruckte Sachaussagen geht, die in ihrer
analogen Erstarrung dann unwidersprochen im Buchspeicher für
\enquote{Ewigkeiten} konservatorisch aufbewahrt werden, kann man doch
für solche Inhalte nicht im Gegenzug \enquote{digitale Beliebigkeit}
einfordern.

Für geisteswissenschaftliche Wissensgenerierungsprozesse gilt vielmehr,
dass auch epistemisch prekäres Wissen zu bewahren wäre: \enquote{Es
liegt auf der Hand, daß historische Wissensbestände in vielen, wenn
nicht gar in der Mehrheit aller Fälle überholt sind und daher dem
Wahrheitskriterium nicht bzw. nicht mehr entsprechen. Das nicht bzw.
nicht mehr wahre Wissen auszuklammern, kann für eine an der Genese und
der Zirkulation von Wissen interessierte Kulturwissenschaft nicht in
Frage kommen.}\footnote{Grunert, Frank (Hg. et al. 2015) Wissensspeicher
  der Frühen Neuzeit, Einleitung S. VIII.}

Der Verweis auf die Starrheit der Physis des Buches, des
\enquote{Schwarz auf Weiß}, welches man getrost nach Hause tragen könne,
zielt auf die haptische Realität des Buches, die es im Gegenteil gerade
als entscheidende Qualität und noch immer gültiges
Alleinstellungsmerkmal eines sekundären Mediums zu erkennen gilt. Der
Verweis ignoriert die semantische Offenheit und die tatsächlich in
Lektürevorgängen vorhandene \enquote{Fluidität}, nämlich die
Sinnkonstruktionen von Text auf der Seite des Lesers.

Die Neurowissenschaftlerin Maryanne Wolf (2009, Tufts-Universität
Boston, Leseforschung) formulierte diesen Sachverhalt wie folgt:
\enquote{Lesen ist neuronal und intellektuell ein Akt der verschlungenen
Wege, der durch die unvorhersagbaren Abstecher in Gestalt der
Schlussfolgerungen und Gedanken der Leser genauso bereichert wird wie
durch die unmittelbare Botschaft, die der Text an das Auge
sendet.}\footnote{Wolf, Maryanne (2009) Das lesende Gehirn, S. 18. Hier
  zitiert nach Hagner, Michael (2015) Zur Sache des Buches, S. 243.}

\subsection*{Deep Reading}\label{deep-reading}

\begin{quote}
\enquote{Was man in Frage stellen muß, ist {[}\ldots{}{]} die
Gleichstellung von Lektüre mit Passivität. {[}\ldots{}{]} Analysen
zeigen, das \enquote{jede Lektüre ihren Gegenstand verändert}
{[}\ldots{}{]} und daß schließlich ein verbales und ikonisches
Zeichensystem ein Reservoir von Formen ist, die darauf warten, vom Leser
ihre Bedeutung zu bekommen. Wenn somit das \enquote{Buch ein Resultat
(eine Konstruktion) des Lesers ist}, muß man die Vorgehensweise dieses
letzteren als eine Art von lectio betrachten, als eine dem
\enquote{Leser} eigene Produktion. {[}\ldots{}{]} Er erfindet in den
Texten etwas anderes als das, was ihre \enquote{Intention} war. Er löst
sie von ihrem {[}\ldots{}{]} Ursprung. Er kombiniert ihre Fragmente und
schafft in dem Raum, der durch ihr Vermögen, eine unendliche Vielzahl
von Bedeutungen zu ermöglichen, gebildet wird, Un-Gewußtes.}\footnote{Certeau,
  Michel de (1988) Kunst des Handelns. Berlin : Merve, 1988. S. 300.}
\end{quote}

Womit sich zumindest in einem Punkt fluide Digitalobjekte und
analog-sequenzielle Sinncontainer auf interessante Weise wieder begegnen
könnten: In der Rolle des Textes als bloßem Katalysator: Er löst
epistemische Prozesse aus, seine Aussagen liegen im Analogen
anschließend aber wieder unverändert vor für eine prüfende Folgelektüre,
im Digitalen wäre das nicht mehr unbedingt gesichert. Frank Grunert
(2015): \enquote{{[}\ldots{}{]} damit Aussagen als Wissen kommunizierbar
sind, muß Wissen als identisches über eine gewisse Dauer wahrnehmbar
sein. Dieser letzte Punkt ist für den vorliegenden Zusammenhang von
besonderem Interesse, unterstreicht er doch das besondere Gewicht, das
der Wissensspeicherung {[}\ldots{}{]} zukommt.}\footnote{Grunert, Frank
  (Hg. et al. 2015) Wissensspeicher der Frühen Neuzeit, Einleitung S.
  IX.}

Wenn auf Leserseite von \enquote{fluider} Rezeption auszugehen ist und
den Textseiten im Digitalen Fluidität inhärent ist zur Ermöglichung von
Hypertext und Kollaboration, dann verflüchtigt sich, was in
Geisteswissenschaften stets prekär war: Nachvollziehbarkeit,
Überprüfbarkeit. Nochmals: \enquote{Deep Reading ist insofern
schöpferisch, als dass der Leser, der über den Text hinausgeht, einen
individuellen Zugang zum Text erschafft {[}\ldots{}{]} Die Basis für ein
solches Lesen ist das Buch, und zwar vor allem die Beschränkung
{[}\ldots{}{]}.}\footnote{Weidenbach, Lukas (2015) Buchkultur und
  digitaler Text, S. 37.}

Man kann also mit einiger Wahrscheinlichkeit davon ausgehen, dass es bei
einem Buch stets der Leser ist, der gedanklich \enquote{abschweift} im
Sinne kreativer, intellektueller Hirnleistung, im Digitalen tut dies der
Text über Links: Hyper Reading ist die durch das Internet induzierte
Leseform, die den Lesefluss durch Anklicken von Verweisen auf andere
Texte, Bilder, Filme, akustische Dokumente usw. unterbricht.\footnote{Vgl.
  Hagner, Michael (2015) Zur Sache des Buches, S. 223.} Es wird die
Sofortverfügbarkeit elektronischer Ressourcen zum epistemischen Problem
durch instante Umlenkung von Aufmerksamkeitspotenzial. Im Kern stellt
sich bei Hypertext die Frage der Herstellung inhaltlicher Kohärenz, wenn
die Rezeptionspfade verästelt, verschlungen und je individuell sind.

Peter Strohschneider spricht darüber hinaus von dem
\enquote{anthropologisch fundamentalen Verfügen über Dinge}, die
\enquote{Kristallisationspunkte des Wissens}\footnote{Vgl.
  Strohschneider, Peter (2012), Faszinationskraft der Dinge, S. 16.}
seien und bedient sich damit einer völlig entgegengesetzten Metaphorik
zum Fluiden. Elmar Mittler (2012) sprach in konservatorischer Absicht
vom \enquote{Einfrieren der liquid documents},\footnote{Mittler, Elmar
  (2012) Wissenschaftliche Forschung und Publikation im Netz. In:
  Füssel, Stephan (Hg. 2012) Medienkonvergenz -- transdisziplinär, S.
  80.} was deren Offenheit wohl in Frage stellen dürfte. Klaus Ceynowa
(2016): \enquote{Seien wir ehrlich: Es weiß heute niemand, wie man
derart fluide Wissensbestände verlässlich sammelt, referenziert und über
lange Zeiträume stabil bewahrt.}\footnote{Ceynowa, Klaus (2016) Anker im
  Fluss des Wissens. Begehrte \enquote{Ruinen}. Die Bibliothek der
  Zukunft muss dynamischen Objekten Dauer verleihen. In: FAZ Nr. 74.2016
  vom 30.03.2016, S. N4.}

Destabilisierung der Textformate, ihre mediale Klonierung bzw.
Multiplizierung, so Michel Hagner (2015), zwinge zu wechselndem
Rezeptionsverhalten am selben Text und fragmentiere, destabilisiere
damit auch die Lektüre. Die geistige Aneignung eines Textes hänge von
dessen materieller Form ab. Die Kohärenz des Lesens wäre in Frage
gestellt.\footnote{Vgl. Hagner, Michael (2015) Zur Sache des Buches, S.
  231.} Bernard Stiegler (2014) fragt: \enquote{Kann eine tiefe
Untersuchung von Wörtern, Gedanken, Wirklichkeit und Tugend in einer Art
von Lernen gelingen, die durch eine ständig geteilte Aufmerksamkeit und
Multitasking charakterisiert wird?}\footnote{Stiegler, Bernard (2014):
  Licht und Schatten im digitalen Zeitalter. In: Reichert, Ramón (2014)
  Big Data, Hier S. 42.}

Der interessanten Frage, wie sich Latenz bei tertiären Medien
institutionalisieren ließe, kann hier nicht weiter nachgegangen werden.
Die Frage verweist zum einen auf das technische Feld der
Langzeitarchivierung digitaler Medien und auf die ungelösten
konservatorischen Herausforderungen gegenüber Latenz, die in
\enquote{fluiden} oder entgrenzten und verlinkten multimedialen
Digitalobjekten, Sloterdijks \enquote{multidimensionalen
Informationsknäulen}, transportiert wird und für spätere
geschichtswissenschaftliche Rezeption interpretierbar bleiben
sollte.\footnote{Zum Forschungsstand von Geschichtswissenschaft und
  Hypertext Krameritsch, Jakob (2007) Geschichte(n) im Netzwerk.
  Hypertext und dessen Potenziale für die Produktion, Repräsentation und
  Rezeption der historischen Erzählung. Münster : Waxmann, 2007.}

Zum Zweiten verweist sie auf die Massendatenhaltung (Big Data als
tertiäre Medien). Diese Massendaten, so wird immer deutlicher, enthalten
ein noch nicht ausgelotetes Potential an Latenz. Allein auf Grund ihrer
Quantität erlangen sie sozialstatistisch eine nie vorher gekannte
Relevanz, die quasi zwangsläufig zu qualitativen Aussagen tendiert.
Hinzu kommt ihr schier unbegrenztes Korrelationspotenzial mit Hilfe von
Algorithmen.

Sie befinden sich jedoch nur zu einem geringen Teil in
nichtkommerzieller Hand (in Hochschulrechenzentren oder auf
öffentlich-finanzierten Repositorien). \enquote{Cloud-computing} oder
\enquote{Social media} beruhen auf der Inanspruchnahme von Diensten
multinationaler Wirtschaftsunternehmen und sind damit
öffentlich-institutionellem Zugriff im Bemühen um Archivierung oder
Auswertung grundsätzlich entzogen. Man denke zum Beispiel an eine für
öffentlich finanzierte Forschung wünschenswerte, aber unzugängliche
Komplettkopie der aktuellen Facebook- oder Google-Datenbanken.

\begin{quote}
\enquote{Dem Sammeln großer Datenmengen ist {[}\ldots{}{]} eine
Machtgeschichte der möglichen Herstellung sozialprognostischen Wissens
inhärent. {[}\ldots{}{]} Wenn man es so betrachtet, ist das Social Web
zur wichtigsten Datenquelle zur Herstellung von Regierungs- und
Kontrollwissen geworden. {[}\ldots{}{]} In diesem Sinne kann man sowohl
von \emph{datenbasierten} als auch \emph{datengesteuerten}
Wissenschaften sprechen, da die Wissensproduktion von der Verfügbarkeit
computertechnologischer Infrastrukturen und der Ausbildung von digitalen
Anwendungen und Methoden abhängig geworden ist.} \footnote{Reichert,
  Ramón (Hg. 2014) Big Data, Einführung, S. 9f.}
\end{quote}

Die Frage, wie sich für diesen Bereich Latenz zu Zwecken späterer
Entdeckung, Auswertung und Deutung für geisteswissenschaftliche
Forschung konservieren ließe, muss gegenwärtig noch offen bleiben.

Auf noch etwas höherer Abstraktionsebene geht es um die
\enquote{Ontologie des Digitalen}. Der Computer als tertiäres Medium
verweise auf \enquote{das digitale Sein als ein ortloses und
rechnerisches Sein}, das den physischen sinnlichen Körper nicht mehr
erreiche.\footnote{Vgl. Kossek, Brigitte (Hg. 2012), Einleitung: digital
  turn?, S. 17.}

Die Mediensystematik von Harry Pross von 1972 erweist sich folglich,
trotz ungebremster Innovation im Bereich der tertiären Medien, als ein
überraschend weitreichendes Analyseinstrument.

\section*{Blended Library}\label{blended-library}

Während in dem Wort \enquote{hybrid} die Verschiedenheit, das
Zwitterhafte von zweierlei Herkunft stets richtig mitschwingt, hat das
englische Wort \enquote{blend} eher den Charakter der harmonisierten
Mischung zweier oder mehrerer Bestandteile. Man denke an \enquote{a
blend of tea}, eine harmonische Teemischung. Die Blended Library
thematisiert wie die Hybridbibliothek das Miteinander analoger und
digitaler Medien und Verfahren in Bibliotheken.

Harald Reiterer, Leiter der Arbeitsgruppe Mensch-Computer Interaktion,
Fachbereich Informatik und Informationswissenschaften der Universität
Konstanz (2011 et al.):

\begin{quote}
\enquote{An die {[}so im Original{]} Stelle der von Medienbrüchen und
Einstiegshürden bei der Bedienung gekennzeichneten Koexistenz von
\enquote{digital} und \enquote{analog} entsteht {[}so im Original{]}
eine gegenseitige Ergänzung und Kooperation, die entscheidende Mehrwerte
für die Benutzer bei der Recherche und dem Wissenserwerb verspricht.
Somit führt die \enquote{Blended Library} reale und virtuelle Angebote
homogen zusammen und schaffte eine Umgebung, in der Realität und
Virtualität nicht konkurrieren {[}sic{]}, sondern benutzergerecht
verschmelzen.}\footnote{Heilig, Mathias; Rädle, Roman; Reiterer, Harald
  (2011) Die Blended Library, S. 239.}
\end{quote}

Reiterer und seine Koautoren gehen in ihrem Ansatz zunächst von
Medienkonkurrenz und dem Verdrängungspotenzial der Digitalisierung aus
und schlagen dagegen einen erneuten Paradigmenwechsel hin zur
Mediensymbiose vor: " {[}\ldots{}{]} weg von der Entwicklung rein
virtueller Welten, hin zur Einbettung von Informationstechnologien in
die soziale und physische Welt einer Bibliothek.``\footnote{Ebd., S.
  217.}

\begin{quote}
\enquote{Die Vielfalt, Flexibilität, Natürlichkeit und (Be-)greifbarkeit
realer Arbeitsumgebungen soll bewusst gegenüber der körperlosen und
beliebigen \enquote{everytime and everywhere} Nutzung virtueller Objekte
und Dienste bewahrt und genutzt werden. An die Stelle der Koexistenz von
realen und digitalen Bibliotheken soll eine Vermischung beider Welten
treten.}\footnote{Ebd., S. 219.}
\end{quote}

Diese Vermischung jedoch soll erreicht werden durch verschiedenste
innovative Benutzerschnittstellen zum Digitalen. In Reiterers (et al.)
Konzept sind es unter anderem die sechs \enquote{Blends} ZOIL (eine
zoombare objektorientierte digitale Visualisierung), \enquote{Suche}
(ein großes hochauflösendes Display), \enquote{Virtuelles Fenster}
(Tablet-PC mit einer Applikation \enquote{Augmented Reality}),
\enquote{Notizen} (Anoto.com-Digitalschreibstift und
echtzeitübertragungsfähiger Digitalbeschreibstoff),
\enquote{Such-Tokens} (Multitouch-Computertischfläche mit einer Art Puck
als Tangible User Interface) und \enquote{hybrides Medium} (physisches
Objekt Buch oder DVD und Objekterkennung sowie Digitalanreicherung durch
ZOIL).\footnote{Vgl. Heilig, Mathias; Rädle, Roman; Reiterer, Harald
  (2011) Die Blended Library, S. 223-236 sowie Gebhardt, Christoph;
  Rädle, Roman; Reiterer, Harald (2014) Employing blended interaction to
  blend the qualities of digital and physical books. In: M\&C Best
  Paper, 3.2014, S. 36-42 (Online-Ressource siehe Bibliographie).}

Die Grundannahmen dieses Konzeptes, dass nämlich kognitive Prozesse
maßgeblich durch körperliche und soziale Interaktion mit Objekten und
Lebewesen der Umwelt und umfassendes Ansprechen aller Sinnesorgane, des
Körperbewusstseins und der Orientierung in Realräumen beeinflusst
wird,\footnote{Vgl. Reiterer, Harald (2014) Blended Interaction. In:
  Informatik-Spektrum, 37(2014)5, S. 459f. \enquote{Es kommt dabei zu
  einer Vermischung von realer und digitaler Welt, daher sprechen wir
  auch von \enquote{Blended Interaction} {[}\ldots{}{]}.} S. 460.}
basieren auf aktuellen Erkenntnissen der Kognitionsforschung. Sie werden
aber ausschließlich auf die -- in der Tat -- sehr notwendige Optimierung
von Mensch-Maschine-Schnittstellen und digitale Usability angewendet,
heilen jedoch nicht die unvermeidbaren Medienbrüche eines hybriden
Informationsraumes, wie ihn die Realbibliotheken heute darstellen.

Harry Pross' schlichte Medieneinteilung könnte auch hier im Kontext der
Blended Library die Einsicht in die \emph{Unüberbrückbarkeit der
medialen Funktionsweisen} nahelegen. Der \enquote{Wissensarbeiter} in
dem ausführlich bebilderten Aufsatz von Harald Reiterer (et al.) nutzt
zu keinem Zeitpunkt und bei keinem \enquote{Blend}, die allesamt
tertiäre Medien sind, ein aufgeschlagenes Sekundärmedium
\enquote{Buch}.\footnote{Heilig, Mathias; Rädle, Roman; Reiterer, Harald
  (2011) Die Blended Library, Abb. 10 und Text S. 238.} Von
\enquote{Verschmelzung} kann folglich nicht die Rede sein und es bleibt
im Kern in diesem Konzept ungeklärt, wie Analoges und Digitales sich
\enquote{vermischen}.

Der Frage, die gänzlich auf das Feld der Medienwissenschaft führt, soll
nicht weiter nachgegangen werden, wie überhaupt sich Rivalitäten oder
Synergien, Medienkonkurrenz oder Mediensymbiose zwischen sekundären und
tertiären Medien etablieren, ob sie gesellschaftlich konstruiert sind
oder ob es sich im Kern um anthropologische Grundphänomene der
Mensch-Objekt-Schnittstellen handelt, die sich zwangsläufig einstellen,
wenn die apparatefreie, primäre, synästhetische Elementarkommunikation
(zum Beispiel die antike Rede nach Platon;\footnote{Uwe Jochum (2015)
  Bücher, S. 134: \enquote{Der Verdacht, Schrift und Buch seien zur
  Speicherung des menschlichen Denkens -- der Fülle des Gedachten und
  der Denkprozesse -- nicht geeignet, reicht bis zu Platon
  {[}\ldots{}{]} zurück, der darauf hingewiesen hat, dass Schriftzeichen
  an sich bedeutungslos seien, solange sie nicht durch miteinander
  sprechende und denkende Menschen mit Bedeutung aufgeladen werden.}}
der Rhetor als \enquote{Menschmedium} nach Werner Faulstich) verlassen
wird und in die sekundäre oder tertiäre Medienklasse gewechselt wird.

\begin{quote}
\enquote{Die Anthropologie der Medien fragt nach der Differenz von
leiblich gebundener und medial-technischer Sinnestätigkeit. Mit der
Technisierung der Sinne entsteht ein neuer Modus des Sinns und eine
Modifikation der Sinnestätigkeit. Technische Medien sind keine einfachen
Analogien zu oder Ausweitungen von Sinnesorganen des Leibes.}\footnote{Karpenstein-Eßbach,
  Christa (2004) Einführung in die Kulturwissenschaft der Medien, S. 68.}
\end{quote}

\section*{Fluide Bibliothek}\label{fluide-bibliothek}

Als Steigerungsstufe von \enquote{hybrid} versteht Olaf Eigenbrodt
(2014, Staats- und Universitätsbibliothek Hamburg) sein Konzept der
\enquote{fluiden} Bibliothek: \enquote{Neben dem Begriff der Digitalen
Bibliothek wurde vor allem der der Hybriden Bibliothek populär und hat
sich bis heute zur Beschreibung von Bibliotheken gehalten, die sowohl
digitale als auch analoge Medien vorhalten. Beide Welten blieben aber
mehr oder weniger voneinander getrennt.}\footnote{Eigenbrodt, Olaf
  (2014) Auf dem Weg zur Fluiden Bibliothek. In: Formierung von
  Wissensräumen. S. 207-220. Hier S. 209.} Ihn beschäftigt nun diese
\enquote{Abschottung} des analogen Bestandes von dem seiner Vorstellung
nach dominanten \enquote{digitalen Raum} hybrider Bibliotheken.

Als Lösung oder zumindest Abmilderung dieses Problems schwebt ihm ein
\enquote{futuristischer Versuch}\footnote{Ebd., S. 209.} einer neuen Art
von Konvergenz vor. Dazu müsse sich der physikalische Ort, die
Architektur verflüssigen. Raumausstattung, auch die
\enquote{eineindeutigen} realen Medienstandorte müssten \enquote{fluide}
werden. Dies gelänge durch das Vernetzen alles Festen durch RFID.
\enquote{Einrichtungsgegenstände wie Tische, Präsentationsmöbel und
Regale müssen mit RFID-Antennen ausgerüstet werden
{[}\ldots{}{]}.}\footnote{Ebd., S. 214.} Nutzer und Medien und Möbel
fänden dann technikvermittelt zueinander per
RFID-Kommunikation.\footnote{\enquote{Hier wird technomediale Vernetzung
  zum zentralen Existenzmodus.} Sabine Maasen; Barbara Sutter (2016)
  Dezentraler Panoptismus. Subjektivierung unter techno-sozialen
  Bedingungen im Web 2.0, hier S. 176.} Nur mit einem entsprechenden
\enquote{mobilen Endgerät}, das man auszuleihen oder mitzuführen habe,
fände der Nutzer Zusammenhängendes in Regalen (die in Grenzen auch
\enquote{Fließen} lernen könnten nach diesem Konzept, obwohl Eigenbrodt
-- eine analoge Reminiszenz -- klar erkennt, dass \enquote{insbesondere
mit Büchern bestückte Regale sehr schwer und damit auch potentiell
kippgefährdet}\footnote{Eigenbrodt, Olaf (2014) Auf dem Weg zur Fluiden
  Bibliothek. In: Formierung von Wissensräumen, S. 210.} seien!).

Auf diesen Regalen befänden sich die vom elektronischen Suchgerät
angezeigten Medien in \enquote{chaotischer Lagerhaltung}, dafür aber
unter Umständen mit einer neuen Dimension von \enquote{Heureka} oder
\enquote{Serendipity}.\footnote{Vgl. dazu im Kontext von Hypertext:
  Krameritsch, Jakob (2007) Geschichte(n) im Netzwerk, S. 185-193.} Das
\enquote{Gesetz der guten Nachbarschaft}\footnote{Nach Aby Warburg. Vgl.
  Saxl, Fritz: Die Geschichte der Bibliothek Warburgs (1886-1944) in:
  Gombrich, Ernst H. (zuerst 1970, dt. Ausgabe 1992) Aby Warburg. Eine
  intellektuelle Biographie, S. 436.} als topographischen Resultats
intellektueller Zuordnung von Buchinhalten durch systematische
Aufstellung wird hier aufgegeben zu Gunsten des Prinzips einer
\enquote{beliebigen} Nachbarschaft unter Aufgabe kohärenter
Wissenstopographien.

Für die neuen Regalfunktionalitäten kann sich Eigenbrodt sogar
Frontalpräsentation {[}sic{]} der Bücher vorstellen als Alternative zur
Präsentation von Buchrücken-\footnote{Eigenbrodt, Olaf (2014) Auf dem
  Weg zur Fluiden Bibliothek. In: Formierung von Wissensräumen, S. 212.}
\enquote{Die steuernden Elemente bei der Nutzung von physischen
Beständen sind nicht mehr Ordnung und Suche, sondern Zufälligkeit und
Entdeckung, mithin genau die Elemente, die die Informationssuche und das
Lernverhalten in digitalen Umgebungen mitbestimmen.}\footnote{Ebd., S.
  212f.}

Eine erste Definition der Fluiden Bibliothek könne lauten: \enquote{Die
Fluide Bibliothek ist eine hybride Informationseinrichtung, in der
digitale und physische Räume zu einer konsistenten Informationsumgebung
integriert sind.}\footnote{Ebd., S. 213.} Eigenbrodt spricht hier
vorsichtig von \enquote{Integration}, nicht Konvergenz, da
\enquote{absolute Konvergenz im Sinne einer Verschmelzung digitaler und
physischer Realitäten {[}\ldots{}{]} eine technologische Utopie
{[}ist{]}.}\footnote{Ebd., S. 211.} Damit urteilt Eigenbrodt indirekt
über das hier von Reiterer (et al.) weiter oben vorgestellte Konzept der
\enquote{Blended Library}. Was Eigenbrodt für technologische Utopie
hält, wurde in den vorliegenden Ausführungen als Nicht-Überbrückbarkeit
medialer Unterschiede von sekundären und tertiären Medien nach Harry
Pross erklärt.

Der Begriff des \enquote{Fluiden} könnte sich, wovor bereits gewarnt
wurde, auch als eine \enquote{Sackgasse der Jargonbildung} (vergleiche
Fußnote 88) erweisen. Die Metapher des Fluiden ist wesentlich an
Eigenschaften digitaler Medien gebunden.\footnote{Köstlbauer, Josef:
  \enquote{Verflüssigung, Entgrenzung, Variabilität, Beschleunigung sind
  oft genannte, zentrale Eigenschaften digitaler Medien.} Siehe:
  Köstlbauer (2015): Spiel und Geschichte im Zeichen der Digitalität, S.
  95. Dort auch Hinweis auf die Herkunft der Metapher des Flüssigen bei
  Manuel Castells (1996) The Rise of the Network Society und Zygmunt
  Bauman (2000): Liquid Modernity.} Ob sie auch architektonisch
Festgefügtes treffend charakterisiert, ist fraglich. Veraltete Begriffe
wie \enquote{Mehrzweckhalle} oder \enquote{Multifunktionsbau} wären
sicher adäquater, aber genügen nicht der Diskursmode des
\enquote{Fluiden}.\footnote{In diesem Punkt auf \enquote{veraltetem},
  aber solidem Diskursstand Inge Kloepfer (2014) Der irre Boom der
  Bibliotheken. FAZ Nr. 11 vom 16.03.2014, S. 24: \enquote{Bibliotheken
  erfüllen zunehmend die Rolle multifunktionaler Stadthallen, die für
  verschiedene Zwecke genutzt wurden und werden.}} Wie sehr
Erosionserscheinungen klarer Konturen mit tiefenpsychologischen
Schichten der Gesellschaft, mithin also des Individuums zusammenhängen
könnten, hat nicht nur Hartmut Rosa, sondern früher bereits Siegfried
Kracauer (1929) in Worte gefasst:

\begin{quote}
\enquote{Jeder typische Raum wird durch typische gesellschaftliche
Verhältnisse zustande gebracht, die sich ohne die störende
Dazwischenkunft des Bewußtseins in ihm ausdrücken. Alles vom Bewußtsein
Verleugnete, alles, was sonst geflissentlich übersehen wird, ist an
seinem Aufbau beteiligt. Die Raumbilder sind die Träume der
Gesellschaft. Wo immer die Hieroglyphe irgendeines Raumbildes entziffert
ist, dort bietet sich der Grund der sozialen Wirklichkeit
dar.}\footnote{Krakauer, Siegfried (zuerst 1929) Über Arbeitsnachweise.
  Konstruktion eines Raumes. In: Ders., Werke, Bd. 5,3: Essays,
  Feuilletons, Rezensionen 1928 -- 1931. Hrsg. von Inka Mülder-Bach.
  Berlin : Suhrkamp, 2011, S. 250. Im Typoskript des Krakauer Nachlasses
  lautet der letzte Teilsatz: " {[}\ldots{}{]}, dort sind die Ideologien
  durchschaut, und der Grund der sozialen Wirklichkeit bietet sich dar."}
\end{quote}

Auf die Bibliothek als traditionell \enquote{feststehenden} Ort bezogen
scheint Hartmut Rosas Analyse zunehmend wahrscheinlich zu werden:

\begin{quote}
\enquote{Die Vorstellung, dass die kulturell und strukturell bedeutsamen
Raumqualitäten heute nicht mehr durch {[}\ldots{}{]} lokal fixierte,
immobile Institutionen, durch feststehende Orte und Plätze, sondern
gleichsam hin- und herfließende, immer wieder ihre Richtung und Gestalt
ändernde \emph{Ströme} oder \emph{Flüsse} {[}\ldots{}{]} bestimmt
werden, ist gegenwärtig dabei, kulturelle Hegemonie zu
erlangen.}\footnote{Rosa, Hartmut (2005) Beschleunigung. S. 342f.}
\end{quote}

Diese Annahme wird in der Dankesrede des Preisträgers der
Karl-Preusker-Medaille 2015, von den Ausführungen Konrad Umlaufs,
bestätigt:

\begin{quote}
\enquote{Künftige Bibliotheken werden kaum noch als Bibliotheken zu
erkennen sein. Sie werden in fluiden Gebäuden untergebracht sein
{[}\ldots{}{]} Wo im Gebäude {[}\ldots{}{]} Bibliothek anfängt, wird man
nicht erkennen können. Vielleicht findet Bibliothek auf den
Galerieflächen {[}\ldots{}{]} statt. Öffnungszeiten wird es nicht mehr
geben, weil die fluiden Gebäude jederzeit zugänglich sind; eine Bindung
{[}\ldots{}{]} an die Anwesenheit bibliothekarischen Personals wird es
nicht geben.}\footnote{Umlauf, Konrad (2015) Dankesrede des Preisträgers
  zur Verleihung der Karl-Preusker-Medaille am 30. Oktober 2015.
  Online-Ressource: siehe Bibliographie.}
\end{quote}

Es wäre an dieser Stelle sicher eine vertiefende Analyse der
\enquote{Hieroglyphe} im Sinne Kracauers bzw. des Jargonbegriffs
\enquote{Fluid} geboten, die hier aber nicht geleistet werden
kann.\footnote{Zur Einordnung dieses \enquote{symbolischen},
  analogischen Konzeptes, die das Innen und Außen als zwei
  korrespondierende Aspekte der Persönlichkeit ansieht, in diesem Fall
  das Außen als symbolische Repräsentation des inneren Menschen siehe
  Wolfgang Müller-Funk (1995) Erfahrung und Experiment: \enquote{Was der
  Geist ist, tut sich an den Artefakten und Maschinerien kund, die er
  geschaffen hat {[}\ldots{}{]}. Das Da-Sein der Maschine ist nicht
  Ausdruck von Seinsvergessenheit (Heidegger), sondern von
  Seinsvergegenwärtigung im Sinne der beinahe verzweifelten Entäußerung
  eines unsichtbaren Innen, das sich sichtbar ins Licht rücken möchte.},
  ebd., S. 147. \enquote{Wissenschaftlich ist das analogistische
  Verfahren höchst umstritten {[}\ldots{}{]}. Gleichwohl ist nicht zu
  übersehen, daß Wissenschaft sich nicht der \enquote{magischen}
  Praktiken des Analogisierens und anderer vergleichbarer Operationen
  entledigen kann {[}\ldots{}{]}.}, ebd., S. 148. Konkret in Bezug auf
  Subjektkonstitution und das Internet vlg. den Sammelband von
  Carstensen, Tanja (Hg. et al. 2014) Digitale Subjekte, besonders S.
  147: \enquote{Das uneindeutige Subjekt {[}\ldots{}{]} sucht sich Orte,
  an denen es nicht ständig zur Eindeutigkeit aufgerufen wird
  {[}\ldots{}{]}. Das Internet zählt heutzutage zu den Orten, die diese
  Möglichkeit signalisieren.}} Nach Hartmut Rosa in aller Kürze
bezeichne die Metapher des \enquote{flows} oder \enquote{fluids} die
Wahrnehmung einer eher unbestimmten Situation mit hohen,
unvorhersehbaren Veränderungsraten. Sie sei unter anderem eine
Erosionserscheinung personaler Identität als Folge des technik- und
modernisierungsimplizierten Zeitregimes steigender
Beschleunigung\footnote{Vgl. Rosa, Hartmut (2005) Beschleunigung. Hier
  S. 467f.} und struktur- und objektbezogen eine \enquote{richtungslose
Dynamisierung}.\footnote{Ebd., S. 437.}

\subsection*{Enabling Spaces}\label{enabling-spaces}

Es dürfte klar geworden sein, dass entscheidende theoretische
Vorarbeiten für die Proportionierung medialer Anteile von Analogem und
Digitalem in epistemologischen Räumen bei den Kognitionswissenschaften
gesucht werden müssen. Die physische, festgefügte Bibliothek hat, davon
wird hier ausgegangen, unbestritten ihren Platz in Prozessen der
Wissensgenerierung. Sie ist in ihrer Partialeigenschaft als öffentlicher
Lernort Teil einer strukturierbaren Topographie epistemischer Praktiken,
die sich von der Couch oder dem Ledersessel im Privaten (primären Orten)
über den Ausbildungs- und Arbeitsplatz (sekundäre Orte) bis zur
Bibliothek als \enquote{Drittem Ort} (Ray Oldenburg, Soziologe, zuerst
1989\footnote{Der \enquote{Dritte Ort} ist von dem US-amerikanischen
  Soziologen Ray Oldenburg in Bezug auf Problemfelder der
  US-amerikanischen Gesellschaft definiert worden. Die von ihm zur
  Lösung positiv herausgearbeiteten Raumqualitäten wurden in der Folge
  auch im Bibliothekswesen rezipiert. Vgl. zuletzt Robert Barth (2015)
  Die Bibliothek als Dritter Ort. In: BuB 67(2015)7, S. 426-429.})
erstreckt.

Tatsächlich liefern die Kognitionswissenschaften unter dem Einfluss
eines \enquote{socio-epistemological creative turn}\footnote{Peschl,
  Markus E.; Fundneider, Thomas (2012) Vom \enquote{digital turn} zum
  \enquote{socio-epistemological creative turn}. In: Brigitte Kossek (et
  al 2012) Digital Turn?, S. 47-62.} wichtige theoretische Hinweise für
Konzepte von Mensch-Raum-Technik-Schnitt\-stellen und die
Möglichkeitsbedingungen für kreative kognitive Prozesse. Die Autoren
Peschl und Fundneider (2012) stellen die Frage, \enquote{wie Räume
aussehen müssten, in denen -- aus einer epistemologischen Perspektive
betrachtet -- Prozesse der \emph{Wissensgenerierung} und Innovation an
erster Stelle stehen.}\footnote{Ebd., S. 48.} Sie liefern den Begriff
der \enquote{enabling spaces} und den theoretischen Rahmen, aber
bedauerlicher Weise nicht auch ganz konkrete Vorschläge für die
Raumgestaltung oder ein konkretes Design entgegen ihrer ausgreifenden
Ankündigung: \enquote{Es geht um eine Integration unterschiedlicher
Dimensionen, die zu einem umfassenden Raumverständnis zusammengefügt
werden, das soziale, kognitive, emotionale, organisationale,
technologische Aspekte ebenso umfasst wie architektonische.}\footnote{Ebd.,
  S. 48.}

Aus dem Umfeld der \enquote{Digital Humanities} sind zu dem
kulturtechnischen Verständnis von digitalem Code und hybridem
Nebeneinander ebenfalls zahlreiche Anregungen zu entnehmen.

\section*{Digital Humantities}\label{digital-humantities}

Im Hinblick auf die Hybridproblematik und die Bibliothek als Werkstatt
der Geisteswissenschaften darf eine Sondierung der Digital Humanities
als Experimentierfeld neuer digitaler Forschungsmethoden in den
Geisteswissenschaften nicht fehlen. Beim Blick auf den Diskurs innerhalb
dieser \enquote{Spezialarena}\footnote{Vgl. Keller, Reiner (2013)
  Diskursanalyse, S. 426.} geht es um die Frage, wieweit von den Digital
Humanities ein Wandlungsdruck im Sinne des \enquote{Library type
continuum}\footnote{Sutton, Stuart A. (1996) Future service models and
  the convergence of functions, S. 136.} von Stuart A. Sutton (1996) auf
Hybridbibliotheken ausgeht:

\begin{quote}
\enquote{Reicht das Zusammenkommen von Geisteswissenschaften und
Digitalität aus, damit die Geisteswissenschaften \enquote{anders}
werden? In welchem Sinn würde die Digitalität etwas \enquote{Neues} in
die Geisteswissenschaften einbringen, und welche Folgen hätte dies für
die Art und Weise, wie die Geisteswissenschaften {[}\ldots{}{]}
betrieben werden?}\footnote{Frabetti, Federcia (2014): Eine neue
  Betrachtung der Digital Humanities im Kontext originärer Technizität.
  In: Reichert, Ramón (2014) Big Data, Hier S. 85.}
\end{quote}

Bei eher zögerlicher Annahme der digitalen Kommunikations- und
Medientechnologien durch Historiker\footnote{Mareike König (2015)
  konstatiert z.B. für Wissenschaftsblogs mangelnde Akzeptanz und
  niedrige Nutzung. Vlg. König, Mareike (2015) Herausforderung für
  unsere Wissenschaftskultur, S. 60. Ebenso Donk, André (2015) Die
  Wirklichkeit der Wissenschaft im digitalen Zeitalter, S. 165-184. Hier
  S. 171 und Fußnote 27 mit Nachweisen. Die 2006 von Jakob Nielsen
  veröffentlichte 90-9-1-Relation (auch Ein-Prozent-Regel: 90\% aller
  Beiträge in Foren werden von nur 1\% der User geschrieben, 10\% der
  Beiträge stammen von 9\% der User und 90\% der User bleiben passive
  Betrachter. Online-Ressource siehe Bibliographie) scheint sich nicht
  wesentlich verschoben zu haben.} ist es dennoch zu einer
Institutionalisierung und Strukturbildung dieser Teildisziplin innerhalb
der Kulturwissenschaften gekommen. Es wäre zu sondieren, ob von dieser
Seite und in welcher Stärke ein Druck ausgeht auf die mediale
Ausrichtung der dort tätigen Spezialbibliotheken. Ob eine Einreihung in
das \enquote{Werkzeug des Historikers}\footnote{Klassiker der Einführung
  in die historischen Hilfswissenschaften von Ahasver von Brandt, zuerst
  ersch. 1958, zuletzt in 18. Aufl. 2012.} zur Folge haben würde, daß
\enquote{{[}\ldots{}{]} the digital {[}\ldots{}{]} will
\enquote{disappear} once it becomes accepted and integrated into the
infrastructure of humanities research overall, which will have expanded
{[}\ldots{}{]}},\footnote{Evans, Leighton; Rees, Sian (2012): An
  interpretation of digital humanities. In: Berry, David M. (Hg. 2012)
  Understanding digital humanities, S. 26f. Die beiden Autoren liefern
  einen Überblick der Positionen auf der \enquote{Computational
  Conference} 2010 (vgl. Fußnote 53).} ist eine interessante, aber
spekulative Frage.

Der Einstieg in die Notwendigkeit digitaler Forschungsmethoden erfolgt
häufig mit dem Gemeinplatz von der Überfülle computervermittelter
Wissensobjekte -- in Reichweite der Fingerspitzen. David M. Berry (2012)
in einem grundlegenden Werk zum Verständnis der Digital Humanities (DH):
\enquote{This new infinite archive {[}\ldots{}{]} at the researcher's
fingertips. {[}\ldots{}{]} It is now quite clear that historians will
have to grapple with abundance, not with scarcity.}\footnote{Berry,
  David M. (2012), Unterstanding digital humanities. Hier: Introduction,
  S. 2.}

Dies ist eine der stereotypen Einleitungen in die Digitalproblematik
unter dem Blickwinkel der Informationsflut und der Massendaten, die im
Grunde nichts Neues annonciert, da das Zuviel schon in der
Gutenberg-Galaxis galt und weiter zurück bis zu den ersten Sammlungen
von Schriftzeichen auf Trägerstoffen an zentralen Orten.\footnote{Vgl.
  ausführlich diachronisch vom 16.-21. Jahrhundert dazu Waquet,
  Françoise (2015) L'ordre matériel du savoir, besonders Kapitel 6~: La
  surabondance et l'urgence, S. 249-273.}

Die Klagen der Gelehrten, Wissenschaftler und Informationssuchenden über
ein Zuviel sind Legion. Dem kann nur mit Gelassenheit und mit der
Einsicht in die anthropologischen Grenzen menschlicher Aufnahmefähigkeit
begegnet werden. Darauf verweist Valentin Groebner lakonisch zu Recht
mit Blick auf die \enquote{analoge} Überfülle der Buchwelt:
\enquote{Jeder Mensch kann in seinem Leben zwischen 3000 und 5000
Büchern lesen, hat Arno Schmidt ausgerechnet.}\footnote{Groebner,
  Valentin (2012) Wissenschaftssprache : eine Gebrauchsanweisung. S. 29,
  entnommen bei: Arno Schmidt: Julianische Tage, in: Ders.: Trommler
  beim Zaren, 1966, S. 183-191.}

Die Bemerkung verweist aber vor allem auf die begrenzenden Ressourcen
Zeit und Aufmerksamkeit für Prozesse der Wissensgenerierung. Valentin
Groebner mit Blick auf die digitale Informationswelt: \enquote{Die
endlose vernetzte Fülle des Internets schluckt {[}\ldots{}{]} die
Ressource, die für ihre Nutzung genau so unentbehrlich ist wie
Batteriestrom und Übertragungsgeschwindigkeit: Zeit.}\footnote{Ebd., S.
  31.}

Im Sinne einer Kompensationstheorie, die Einsatz von Technik wesentlich
als Kompensierung menschlicher Mängelerfahrungen interpretiert, hilft
hier nur \enquote{maschinelles}, also computergetriebenes Lesen.

\begin{quote}
\enquote{Das menschliche Leben ist begrenzt, somit auch die Menge an
Büchern, die in einem Menschenleben gelesen werden können. Mit der
Digitalisierung wird die Menge an Büchern, die der Mensch lesen kann,
zwar nicht größer, aber durch die maschinellen Lesemöglichkeiten des
Computers ergänzt.\footnote{Baillot, Anne ; Schnöpf, Markus (2015) Von
  wissenschaftlichen Editionen als interoperable Objekte. In: Schmale,
  Wolfgang (Hg. 2015) Digital Humanties, S. 150.} {[}\ldots{}{]} Die
Bücher {[}\ldots{}{]} werden dabei nicht mehr linear gelesen, sondern
anhand computerlinguistischer Methoden massenausgewertet. Die Digital
Humanities gehen mit der Einführung quantifizierender Methoden in den
Geisteswissenschaften einher.\footnote{Ebd., S. 151.} {[}\ldots{}{]} die
Hermeneutik verschiebt sich in die Überprüfung der automatisch
generierten Ergebnisse.}\footnote{Ebd., S. 156.}
\end{quote}

In diese Richtung zielt auch die Reflexion von Rigoberto Carvajal (2015,
Informatiker): \enquote{Naheliegend wäre es, Big Data über die Größe der
Daten zu definieren. Doch wo setzen wir die Grenze -- bei 100 Gigabyte,
10 Terabyte, 1 Petabyte? {[}\ldots{}{]} Da sich das, was wir unter groß
verstehen, mit der Zeit ändert, müssen wir Big Data anders definieren,
und zwar in Bezug auf unsere Fähigkeit, die Datenmengen zu
verarbeiten.}\footnote{Carvajal, Rigoberto (2015) Wie groß ist Big Data?
  In: Kulturaustausch 65(2015)4, S. 52.} Also gälte es, das Sortier-,
Ordnungs- und Strukturierungspotential von Computern, genauer von
Algorithmen für Relevanzentscheidungen einzusetzen.

Zu den selektiven Mechanismen begrenzter Aufmerksamkeit hat Aleida
Assmann (2001) Grundlegendes gesagt: \enquote{Die
Aufmerksamkeitsökonomie gilt als die neue Ökonomie des
Informationszeitalters, denn wo die Produktion von Information ins
Gigantische wächst, wird das, was Information einen Wert zuweist, immer
knapper und wichtiger, nämlich: Aufmerksamkeit.}\footnote{Assmann,
  Aleida (Hg. 2001) Aufmerksamkeiten. Einleitung, S. 11. Zeitgleich
  Dieter E. Zimmer, Redakteur der ZEIT (2000) Die Bibliothek der
  Zukunft, S. 33: " {[}\ldots{}{]} je mehr Informationsvehikel auf immer
  mehr Kanälen auf Wanderschaft geschickt werden, desto geringer wird
  die Wahrscheinlichkeit, dass sie die Adressaten erreichen. Es liegt
  nicht an der Enge der Kanäle, {[}\ldots{}{]} es liegt daran, dass die
  Aufnahmekapazität der Adressaten stagniert. Angesichts des
  Informationsregens, der auf sie niederprasselt, fühlen sie sich
  zunehmend verwirrt, überfordert und sogar desinformiert. Am Ende muss
  all die Fülle ja durch einen Flaschenhals: das individuelle
  menschliche Gehirn {[}\ldots{}{]}." S. 39: \enquote{In der
  Informationsgesellschaft tobt ein Kampf um eine der wertvollsten
  nichterneuerbaren Ressourcen, unsere Aufmerksamkeit. {[}\ldots{}{]}
  Aufmerksamkeitsheischer üben sich in den neuen Künsten Pushing und
  Spamming.}} Die geradezu existenziell-schicksalhafte Verkettung von
Digitalem und den Grenzen menschlicher Aufmerksamkeit verknüpft Matthew
Crawford (2015, Physiker und Politikphilosoph) völlig zu Recht mit einem
Kernproblem unserer überinformierten Gegenwartsgesellschaft, der
Kohärenzbildung: \enquote{Wir sind Zeugen einer Krise der
Aufmerksamkeit. Es geht heute um nicht weniger als die Frage, ob wir ein
kohärentes Ich aufrechterhalten können.}\footnote{Crawford, Matthew B.
  (2015) Erfahrungen aus zweiter Hand. In: Kulturaustausch 65(2015)4, S.
  36.} Gerade in diesem Bereich liegen ganz zentrale Forschungsfelder
noch relativ brach was die Subjektkonstruktion in hybriden und zunehmend
digitalen Kulturen betrifft.\footnote{Einführend siehe Carstensen, Tanja
  (Hg. et al. 2014) Digitale Subjekte. Praktiken der Subjektivierung im
  Medienumbruch der Gegenwart. Bielefeld : transcript, 2014. Einleitung
  S. 19: \enquote{Das uneindeutige Subjekt, so lautet unsere
  abschließende These, sucht sich Orte, an denen es nicht ständig zur
  Eindeutigkeit aufgerufen wird; einer dieser Orte ist heutzutage der
  Cyberspace.}}

Im Kontext einer solchen Überfülle, gar einer
\enquote{infobesity},\footnote{Begriff, der ca. 2003 in der
  amerikanischen Presse auftaucht (Morris, J.H.: Tales of technology.
  Consider a cure for pernicious infobesity. The Pittsburgh
  Post-Gazette, 2003) zur Kennzeichnung der pathologischen Seiten von
  Überinformation. Hinweis auf den Begriff bei Waquet, Françoise (2015)
  L'ordre matériel du savoir, S. 258, Fußnote 29.} des \enquote{Viel zu
Viel} an digitalen und analogen Wissensobjekten wird zwangsläufig auch
der Sammlungsbegriff problematisch und es muss über Archivierung,
Speicherung, dauerhafte Verfügbarkeit und auch verantwortungsbewusstes
Vergessen neu nachgedacht werden.

Das denkbare Handlungsspektrum im Angesicht des Viel-zu-viel geht von
Ohnmacht bis \enquote{Sammelwut}. Das Umkippen von qualifizierten
Sammlungen in bloße Vorratshaltung von \enquote{Informationen} erweist
sich als ein ernstzunehmendes Problem.\footnote{Zur Typologie von
  Akkumulationsformen (Sammelsurium, Sammlung und Vorratshaltung) vgl.
  Strohschneider, Peter (2012), Faszinationskraft der Dinge.} Es kann zu
ausufernden Formen von analoger Lagerhaltung (Stichwort:
Speicherbibliothek) oder digitaler Massendatenspeicherung auf Vorrat
ohne akute Fragestellungen führen. Eine rein auf das Quantitative
zielende Daten-\enquote{Sammelwut} verursacht zu einem bedeutenden Teil
das mit, was als Informationsüberflutung wahrgenommen wird und was kaum
noch als Information auswertbar ist noch als Wissen integrierbar.

Peter Haber (2005): \enquote{Wie kann eine Gesellschaft mit dieser Menge
von Wissen -- wenn es sich denn wirklich um Wissen handelt -- umgehen?
Zwei Aspekte werden die Diskussion zu diesem Thema in Zukunft prägen:
Zum einen werden bei der Informationsbeschaffung und -au\-then\-ti\-fi\-zierung
neue Kompetenzen nötig sein. Zum anderen wird die Wissensgesellschaft
neu lernen müssen, zu \emph{vergessen}.}\footnote{Haber, Peter (2005)
  Archive des Wissens, S. 81.}

Von kaum zu unterschätzender Wichtigkeit beim Nachdenken über die
Ursachen der digitalen Informationsflut ist, dass jedes Wissensobjekt --
wenn es nicht schon originär digitalen Ursprungs ist -- immerhin
willentlich eine Konversion ins Digitale erfahren muss, um einer
digitalen Sammlung hinzugefügt zu werden. Nicht selten mit erheblichem
Aufwand und unter hohen Kosten.

Hierhin gehören alle Formen der Retrokonversion. Man könnte John Locke
und G.W. Leibniz paraphrasieren und sagen: Nichts ist im Digitalen, was
nicht vorher durch die Konversion gegangen ist -- außer dem Digitalen
selbst. Unter Letzerem wäre die immense Menge von \enquote{digital
born}-Objekten zu verstehen sowie die Ableitungen von sekundären
digitalen Wissensbeständen aus primären digitalen Grunddaten, also das,
was unter der Bezeichnung \enquote{Big data} und ihrer Massenauswertung
mit digitalen Werkzeugen verhandelt wird, verkürzt gesagt, die Arbeit
der Algorithmen als einem Hauptmerkmal tertiärer elektronischer Medien
im Sinne von Harry Pross' Medieneinteilung (1972) in drei Klassen.

Ein notwendiger Ansatz ist sicher, dass es sich um eine gesellschaftlich
produzierte digitale Informationsflut handelt, der bisher die ebenso
gesellschaftlich zu definierenden Filter fehlen.

\begin{quote}
\enquote{{[}\ldots{}{]} eine Frage, die {[}\ldots{}{]} eng mit der
Entwicklung von letztlich normativen Kriterien des Speicherwürdigen
verbunden ist. {[}\ldots{}{]} Die Frage nach der Bewahrung digitaler
Bestände ist somit nicht nur eine technische der Formate und der
Haltbarkeit der Datenträger, sondern auch eine strukturelle der
digitalen Wissensgenerierung und damit eine, die die Offenlegung und
gleichzeitige Hinterfragung der Normen und Kriterien erfordert, nach
denen bisher über Speicherwürdiges entschieden wurde
{[}\ldots{}{]}.}\footnote{Heibach, Christiane (2011) (De)Leth(h)e. In:
  Das Ende der Bibliothek?, S. 65.}
\end{quote}

David M. Berry (2012): \enquote{To mediate a cultural object, a digital
or computational device requires that this object be translated into the
digital code that it can understand. {[}\ldots{}{]} a computer requires
that everything is transformed from the continuous flow of everyday life
into a grid of numbers that can be stored as a representation which can
then be manipulated using algorithms.}\footnote{Berry, David M. (2012),
  Unterstanding digital humanities. Hier: Introduction, S. 2.}

\enquote{Everything {[}\ldots{}{]} from the continuous flow of everyday
life}. Der Sammlungsgedanke wird hier -- das Wort darf gewagt werden --
total. Beachtenswert ist hier auch, dass das wollende Subjekt des oben
genannten Satzes der \enquote{computational device} ist, welcher zu
verstehen verlangt. Daher mahnt Peter Haber (2000) zu Recht an:
\enquote{Letztlich wird es nach dem \emph{digital turn} um die
gesellschaftliche Definition neuer Organisationsmechanismen des
Erinnerns und des Vergessens gehen. Die schier unendlichen
Speicherkapazitäten des Internet haben die Vision des totalen Archivs
wieder aufleben lassen. {[}\ldots{}{]} Als AuthentifikatorInnen des
Vergangenen werden HistorikerInnen mehr denn je an der Schnittstelle von
Kultur und Technik die Regeln von Aufbewahren, Sammlung und Erschließung
mitdefinieren müssen.}\footnote{Haber, Peter (2005) Archive des Wissens,
  S. 85.}

In seiner Habilitationsvorlesung an der Universität Basel zehn Jahre
später spricht Peter Haber dann von dem
\enquote{Volldigitalisierungsphantasma},\footnote{Vgl. Haber, Peter
  (2010), Reise nach Digitalien und zurück. Ein historiographischer
  Betriebsausflug. S. 11. Thomas Thiel, FAZ-Journalist sprach vom
  \enquote{Machtantritt des Nerds in den Geisteswissenschaften}. Vgl.
  FAZ Nr. 171 vom 25.07.2012, S. N5: Eine Wende für die
  Geisteswissenschaften? Standardisierung und Digitalisierung. Der
  Wissenschaftsrat wertet Forschungsinfrastrukturen auf.} das dazu
führe, dass der analoge Informationsraum ins Hintertreffen zu geraten
drohe und dass nur noch in digitale Entwicklung investiert werde.

Dieses Hintertreffen konzeptionell auszubalancieren, ist nicht zuletzt
einer der Aspekte der Verstetigung der Hybridansätze
geisteswissenschaftlicher Spezialbibliotheken, wie er hier argumentativ
unterstützt wird.

Praktiker der Digital Humanities beginnen zu erkennen, dass nicht jede
Fragestellung mit digitalen Werkzeugen angegangen werden muss, auch wenn
sich die Gesellschaft als Ganzes einem permanenten digitalen
Wandlungsdruck ausgesetzt sieht. In einem einschlägigen Sammelband zu
praktischen Fallbeispielen spricht die Herausgeberin Claire Warwick
(2012) das Problem der Verwertbarkeit und sinnvollen
Zielgruppendefinition für Digitalprojekte an:

\begin{quote}
\enquote{There is little point in creating digital resources in either
sector if they are not used, however. We know that people use digital
resources if they fit their needs, yet many digital humanities resources
are still designed without reference to user requirements. This often
means that expensive digital resources remain unused or unappreciated by
their intended audience.}\footnote{Warwick, Claire (Hg., 2012), Digital
  Humanities in practice. Hier: Introduction, S. XV.}
\end{quote}

Diesem Pragmatismus Claire Warwicks diametral gegenüber steht die
Ansicht von David M. Bell: \enquote{That is, computational technology
has become the very condition of possibility required in order to think
about many of the questions raised in the humanities today.}\footnote{Berry,
  David M. (2012), Unterstanding digital humanities. Hier: Introduction,
  S. 3.}

Herausgeber Bell fährt in seiner Einleitung zu den Digital Humanities
fort mit einer kurzen Skizze der ersten, zweiten und dritten Welle und
schlägt vor, nach dem \enquote{digital turn} nun den
\enquote{computational turn} in den Blick zu nehmen: \enquote{This means
that we can ask the question: what is culture after it has been
\enquote{softwarised}?}\footnote{Berry, David M. (2012), Unterstanding
  digital humanities. Hier: Introduction, S. 6.}

Die Antwort auf dieses von Peter Haber zu Recht als
\enquote{Volldigitalisierungsphantasma} benannte Streben ist relativ
nüchtern: Code und Algorithmen, die nur maschinenvermittelt über
\enquote{electronic devices} wieder erreichbar und dechiffrierbar wären.
Dort liegt im Tertiären (nach Harry Pross' Medientheorie), was von Peter
Haber (2010) als das weite Wunderland \enquote{Digitalien}\footnote{Haber,
  Peter (2010), Reise nach Digitalien und zurück. Ein
  historiographischer Betriebsausflug. Zum Stand der
  \enquote{Algorithmizität} hochtechnisierter Gesellschaften siehe
  Stalder, Felix (2016) Kultur der Digitalität, besonders S. 164-202. Zu
  Virtueller Realität (VR) aus philosophisch-ethischer Perspektive vlg.
  Thomas Metzinger (2016) Wer, ich? Spiegel-Gespräch. In: Der Spiegel
  2016, Nr. 19 vom 7.5.2016, S. 68-71.} bezeichnet wurde.

Innerhalb der Kulturwissenschaften wird Berrys \enquote{computational
turn} klar als dezidierte methodische Hinwendung zur Informatik erkannt
und verortet. Die Digital Humanities wären als eine Art
\enquote{technischer Dienstleister} für Geistes- und
Kulturwissenschaften gestartet. Nun erfolge die disziplinäre Verbindung
mit der Informatik\footnote{Vgl. Klawitter, Jana (et al., 2011)
  Kulturwissenschaftliche Forschung. In: Dies. (Hg. et al., 2012)
  Kulturwissenschaften digital, S. 11, hier Fußnote 2.}.

Jana Klawitter (2011) und ihre Ko-Autoren positionieren sich dezidiert
\enquote{hybrid} in dem hier vertretenen Sinne:

\begin{quote}
\enquote{Im Hinblick auf Digitalisierungs- und Interneteinflüsse in den
Kulturwissenschaften wird in diesem Band anstelle einer Ablösung der
klassischen, nicht-digitalen Forschung von einer Erweiterung dieser um
\enquote{das Digitale} und somit von einem verzahnten Komplex zwischen
digitalen und nichtdigitalen Phänomenen ausgegangen -- dies sowohl im
intra- als auch im interdisziplinären Zusammenspiel. {[}\ldots{}{]}
Berücksichtigung muss somit finden, dass nicht allein auf
computerbasierte Repräsentationen digital und im Internet referenziert
werden kann, sondern auch auf jedes Lebewesen oder physische Objekt --
also auf das \enquote{genuin Analoge}.}\footnote{Ebd., S. 12.}
\end{quote}

In den Kulturwissenschaften ist folglich eine Tendenz erkennbar, die
sich gegen Verdrängungsphantasmen und im Sinne von häufig beobachtbaren
gesellschaftlichen Pendelbewegungen für ein \enquote{come back} des
Analogen ausspricht.

Den eigentlichen Kern der Steigerungsstufe \enquote{computational}
präzisiert Daniela Pscheida (2013) wie folgt: \enquote{Hier geht es
längst nicht mehr nur um den Aufbau digitaler Datenbanken {[}\ldots{}{]}
und {[}\ldots{}{]} Zugriff auf verteilte Ressourcen (\emph{digital
humanities}), sondern {[}\ldots{}{]} um die Entwicklung von Algorithmen
und Systemen zur Analyse digitaler Daten (computational
humanities).}\footnote{Pscheida, Daniela (2013), Wissen und Wissenschaft
  unter digitalen Vorzeichen, S. 21.}

Diese dritte Welle der DH, so fährt Bell fort, nähme Wissen und Macht in
den Blick, die Institutionen und die Idee der Universität selbst, die
sich im 21. Jahrhundert überlebt habe:

\begin{quote}
\enquote{What I would like to suggest is that today we are beginning to
see instead the cultural importance of the digital as the unifying idea
of the university. {[}\ldots{}{]} However, I want to propose that,
rather than learning a practice for the digital, {[}\ldots{}{]} we
should be thinking about what reading and writing actually should mean
in a computational age {[}\ldots{}{]}.}\footnote{Berry, David M. (2012),
  Unterstanding digital humanities. Hier: Introduction, S. 8.}
\end{quote}

Ähnlich Daniela Pscheida (2013): \enquote{Das ist weit mehr als die
bloße Veränderung von Arbeitsprozessen und Handlungspraxen. Stattdessen
deutet sich hier die Notwendigkeit an, den wissenschaftlichen
Erkenntnisprozess grundlegend neu zu denken.}\footnote{Pscheida, Daniela
  (2013), Wissen und Wissenschaft unter digitalen Vorzeichen, S. 22.}

In der Definition von \enquote{Digitalität} durch Wolfgang Schmale
(2015) finden sich diese Gedanken durchaus wieder, wie allerdings auch
die bereits mehrfach hinterfragten \enquote{Jargonbegriffe}:

\begin{quote}
\enquote{Rationalisierung, Verflüssigung, Entgrenzung,
Dekontextualisierung, Personalisierung und gegebenes
Veränderungspotenzial sind Kernelemente der \enquote{digitalen
Vernunft}. Digitalität ermöglicht einen eigenen wissenschaftlichen
Diskurs, der sich vom gängigen textuellen Diskurs unterscheidet
{[}\ldots{}{]}.}\footnote{Schmale, Wolfgang (2015), Einleitung Digital
  Humanities, S. 10.}
\end{quote}

Vor dem Hintergrund dieser Ankündigungen wird das eingangs erwähnte
Aufschrecken des Hermeneutikers Hans Ulrich Gumbrecht verständlicher.

\section*{Schlussüberlegungen}\label{schlussuxfcberlegungen}

\begin{quote}
\enquote{Was ist eigentlich der Kern des gegenwärtigen Medienwandels?
Geht es überhaupt um einen Wandel der Medien, oder {[}\ldots{}{]} um
eine transmedialisierende Synthese der antithetischen Medien (hier:
Buchdruck, dort: digitale Medien)?}\footnote{Sahle, Patrick (2013)
  Digitale Editionsformen, Bd. 1: Das typografische Erbe, S. 8. Auch
  hier tritt die durchaus richtige Kernfrage auf der Stelle. Kay
  Kirchhoff schon 1998: Verdichtung, Weltverlust und Zeitdruck, S. 399f:
  \enquote{Wie neu sind die sogenannten}Neuen Medien" hinsichtlich ihrer
  strukturellen und funktionalen Dimensionen denn nun wirklich?``.}
\end{quote}

Auf den vorangegangen Seiten war die Institution \enquote{Bibliothek} in
ihrer konzeptionellen Dauerverunsicherung im Angesicht des umfassenden
digitalen Wandels Ausgangspunkt für die Frage nach dem
Verstetigungspotenzial des Hybridkonzeptes als ihrer medialen Basis --
dies unter Begrenzung des Fokus auf Wissensgenerierung in den
Geisteswissenschaften durch Arbeit an Texten.

Zentrale Begriffspaare waren dabei digitaler Wandel und analoge
Herkunftswelten, Beschleunigung und Zeitregime, Informationsflut und
Aufmerksamkeitsökonomie, fluide Entgrenzung und analog-sequenzielle
Starre sowie gegen teleologisch-zielverhaftete Prozesse die Betonung
evolutiver, grundsätzlich offener Prozesse.

Die durch die engere zeitnahe Diskursanalyse 2014-2016 gesichteten
Zentralbegriffe wurden anschließend in Verbindung gebracht mit der
medialen Hybridproblematik, wie sie die \enquote{Spezialarena}\footnote{Keller,
  Reiner (2013) Diskursanalyse, S. 426.} der Forschungsbibliotheken in
der Geisteswissenschaft als \enquote{Werkstätten des Wissens}\footnote{Titel
  der Aufsatzsammlung von Helmut Zedelmaier (2015) Werkstätte des
  Wissens zwischen Renaissance und Aufklärung. Tübingen : Mohr Siebeck,
  2015.} betreffen.

Die Argumentation führte dann auf das Terrain der Medienwissenschaft.
Für eine mediensymbiotische bzw. medienkomplementäre Fundierung des
Hybridkonzepts wurden zwei medientheoretische Konzepte näher betrachtet,
der Komplementäransatz von Wolfgang Riepl (1913) und die apparate- und
technikbezogene Mediensystematik von Harald Pross (1972).

Die mit diesen beiden Theorien verbundenen medientheoretischen und
medientechnischen As\-pekt\-e erlaubten in Ansätzen einen klärenden Blick
auf unerlässliche epistemische Prozesse geisteswissenschaftlicher
Wissensgenerierung. Beispielhaft wurde das \enquote{Ent-Decken} und
Erfassen von Latenz, den semantischen Tiefenstrukturen von
Wissenskontexten und das Close- oder Deep-Reading als in hohem Maße
konzentriertes und sinnerschließendes Lesen betrachtet. Verbindungen zur
Kognitionswissenschaft wurden angedeutet, die für Leseprozesse die
besondere Bedeutung des Haptischen unterstreicht:

\begin{quote}
\enquote{'There is physicality in reading', says cognitive scientist
Maryanne Wolf {[}\ldots{}{]}. The human brain may perceive a text in its
entirety as a kind of physical landscape. When we read, we construct a
mental representation of the text that is likely similar to the mental
maps we create of terrain and indoor spaces.}\footnote{Zitiert nach:
  Jabr, Ferris (2013) Why the brain prefers paper. In: Scientific
  American, 309(2013)5, S. 48-53. Hier S. 50f.}
\end{quote}

Bei der Sichtung von Fundstellen zu allen hier kombinierten
Problemfeldern wurde die Verstetigung von Rhetoriken (Markus
Buschhaus)\footnote{Siehe dazu auch: Heßler, Martina (2016): Zur
  Persistenz der Argumente im Automatisierungsdiskurs. In: APuZ
  66(2016)18-19, S. 17-24. S. 18: \enquote{Seit mehr als einem halben
  Jahrhundert sind es ähnliche Argumentationsfiguren, Versprechungen,
  behauptete Notwendigkeiten und Befürchtungen, die mit der
  Automatisierung der Arbeitswelt einhergehen und nur leicht variieren.}}
sichtbar, was meint, dass über Jahre beziehungsweise Jahrzehnte
Argumente des Für und Wider im Kern gleich bleiben und trotz einer
technikgetriebenen Wandlungsgeschwindigkeit der Oberflächenphänomene und
trotz modischer Anpassungen des Diskursjargons die Landschaft der
Problemlagen -- häufig ungelöster -- die selbe bleibt.\footnote{In
  modernisierungskritischer Haltung dafür Stichwort gebend Paul Virilio:
  \enquote{Rasender Stillstand} (zuerst 1990, dt. Übers. 1992), den
  Hartmut Rosa als Anreger für zentrale Thesen seiner
  Beschleunigungsanalyse der Spätmoderne zitiert.}

Exemplarisch wurde von einem als zentral zu bewertenden Zitat des
Direktors der Library of Congress Daniel J. Boostin von 1974
ausgegangen, weil hier bereits ein Diskursjargon in Gebrauch war, der
auch 40 Jahre später als zeitgenössisch in 2016 gelesen werden kann.
Peter Sloterdijks Reflexion über die \enquote{Katastrophe der
Buchförmigkeit} von 1993 (zeitgleich zu Norbert Bolz \enquote{Ende der
Gutenberg-Galaxis}) und seine Annahme, dass neue \enquote{Fahrzeug in
die mitwißbaren Nebenwelten} sei der Hypertext, hat sich so nicht
bestätigt. Die Konstruktion des Prototyps eines der qualifizierten
wissenschaftlichen Analog-Edition vergleichbaren
\enquote{multidimensionalen Informationsknäuels} (Peter Sloterdijk 1993)
steht auch in 2016 noch aus, möchte man nicht pauschal das Internet als
ebendieses \enquote{Knäuel} ansehen.

Im Besonderen die Verabschiedungsrhetoriken sowohl zur Institution
\enquote{Bibliothek} auf systemischer Ebene als auch auf medialer Ebene
zum Buch als Printmedium wurden in Zusammenhang gebracht mit
kulturwissenschaftlichen Hinweisen auf Mechanismen von
Aufmerksamkeitsentzug, von Antiquation und Obsolezenz als kultureller
technikgetriebener Ersetzungsmechaniken (nach Aleida Assmann 2001,
2013)\footnote{Siehe auch stellvertretend für weitere Fundstellen Gyburg
  Radke-Uhlmann (2010, dort am Beispiel von Platon-Reinterpretationen):
  \enquote{}Es gehört {[}\ldots{}{]} zur Logik des neuzeitlichen Wende-
  und Neuheitsbewusstseins, dass das, was in der Auseinandersetzung mit
  den unmittelbaren Vorgängern kritisiert wird, als Merkmal und Fehler
  der ganzen vorangehenden Tradition behauptet wird. Das Überwundene
  gilt als Ganzes, als einheitliche Masse, als Fehlentwicklung in der
  abendländischen Tradition. Dabei werden gravierende Unterschiede und
  sogar unterschiedliche Schulzugehörigkeiten und Kontinuitäten sowie
  Diskontinuitäten innerhalb dieser überwundenen Vergangenheit
  nivelliert. Alles steht unter dem Generalverdacht und ist deshalb auch
  jeweils für sich gar nicht mehr interessant. Die Nivellierung von
  Unterschieden innerhalb der Tradition wird gerechtfertigt mit der
  Unerheblichkeit der möglicherweise zu erschließenden Unterschiede."
  In: Hempfer, Klaus W. (2010 et al.) Der Dialog im Diskursfeld seiner
  Zeit, S. 29.} in Gesellschaften, die zunehmend als in permanenter
Transformation befindlich sich begreifen.

Verstetigung wurde auch sichtbar in der Form einer teleologisch
unterlegten, letztlich unwissenschaftlichen Rhetorik zu permanentem
technischen Wandel, welcher auf einen Zustand der Volldigitalisierung
von Kultur und Wissenschaft hin tendiere. Es wurde der Bogen geschlagen
von der esoterischen Noosphäre Marshall McLuhans (1962) als
\enquote{kosmischer Membran} elektrischer Sinnerweiterung\footnote{Vgl.
  stellvertretend für andere Fundstellen auch Franz Josef Rademacher,
  Professor für Datenbanken und Künstliche Intelligenz Universität Ulm
  (1994): \enquote{Wenn man die Menschheit als Lebewesen sieht, die
  Struktur der Menschheit als Körper, dann sind wir im Moment Zeuge
  eines gewaltigen Evolutionsschrittes der Menschheit {[}\ldots{}{]},
  indem dieser Körper gerade sein Nervensystem ausbildet.} Zitiert nach
  Michael Giesecke (2002) Von den Mythen der Buchkultur zu den Visionen
  der Informationsgesellschaft, S. 363.} bis hin zum
\enquote{computational turn} in den Kulturwissenschaften, vertreten von
David M. Berry (2012, 2014), der sich als idealen Forschungskontext eine
volldigitalisierte Gesamtkultur vorstellt: \enquote{This means that we
can ask the question: what is culture after it has been
\enquote{softwarised}?}\footnote{Berry, David M. (2012) Unterstanding
  digital humanities. Hier: Introduction, S. 6.}

Es wurde mit Odo Marquard und Hans Ulrich Gumbrecht darauf hingewiesen,
dass solche Entwicklungen von den Geisteswissenschaften kritisch zu
begleiten wären.\footnote{Die Max-Weber-Stiftung positioniert sich in
  diesem Bereich zusammen mit der Union der Deutschen Akademien der
  Wissenschaften mit ihrer Reihe \enquote{Geisteswissenschaften im
  Dialog} (GiD): \enquote{Anhand von exemplarischen Fragestellungen und
  im Dialog mit anderen Disziplinen macht GiD einer breiten
  Öffentlichkeit die Orientierungs- und Sinnstiftungskompetenz der
  Geisteswissenschaften erfahrbar.} Pressemitteilung vom 19. April 2016.}
Dafür steht Odo Marquards Wortschöpfung der
\enquote{Inkompetenzkompensationskompetenz} (1973, wiederaufgegriffen
insbesondere von Aleida Assmann 2013) und Gumbrechts
\enquote{Epistemologie der elektronischen Zeit} (2014).

Es besteht die Gefahr, grundsätzliche Brüche und Funktionsunterschiede
zu übersehen, die dem analogen und digitalen Medium je inhärent sind,
nicht zuletzt wegen ihrer Zugehörigkeit zu klar unterscheidbaren
Medienklassen nach Harry Pross (1972).

Eine der unhintergehbaren Erkenntnisse ist sicher, dass doppelt
apparatevermittelte Tertiärmedien klar analysierbar anders funktionieren
als die vorgängigen Medienklassen und dass das realweltliche
\enquote{anthropologisch fundamentale Verfügen über Dinge} (Peter
Strohschneider) von der gedoppelten, technikgetriebenen
Apparatevermitteltheit des Digitalen dominiert und gefiltert ist.

Hier liegt ein noch wenig bearbeitetes Forschungsfeld der
Hybridproblematik, wenn man Hartmut Rosas beiläufig gestellte Frage
aufgreift, inwieweit menschliche unmittelbar physische Weltbeziehungen
durch Bildschirme und ihre Symbolströme moduliert werden:\footnote{Vgl.
  Rosa, Hartmut (2016) Resonanz. Eine Soziologie der Weltbeziehung.
  Berlin, 2016, S. 155ff.}

\begin{quote}
\enquote{Dies wirft {[}\ldots{}{]} die kulturhistorisch ebenso wie
leibphänomenologisch oder sinnesphysiologisch interessante Frage auf,
wie sich die Natur des menschlichen und seines biographischen
Weltverhältnisses insgesamt ändert, wenn \emph{Bildschirme} zum
Leitmedium nahezu aller Weltbeziehungen werden.}\footnote{Ebd., S. 155.}

\enquote{Wir sind auf dem Weg in eine Gesellschaft, in der der größte
Teil unserer Weltbeziehungen bildschirmvermittelt und in der unser
Weltverhältnis als ganzes bildschirm-symbolvermittelt geprägt ist. Dies
hat {[}\ldots{}{]} offensichtlich zwei Konsequenzen. Zum Ersten wird der
Bildschirm zu einer Art Nadelöhr, durch das sich unsere Welterfahrung
und Weltaneignung vollzieht, was eine tendenzielle Uniformierung oder
Mono-Modularisierung des Weltbezugs zur Folge hat. {[}\ldots{}{]} Und
zum Zweiten wird die physische Welterfahrung dadurch {[}\ldots{}{]}
trotz aller technischen Neuerungen extrem reduziert
{[}\ldots{}{]}.}\footnote{Ebd., S. 157.}
\end{quote}

So kann hier mit Harry Pross nur ein Fenster aufgestoßen werden, um
einen befremdlichen Blick auf Klaus Ceynowas \enquote{haptische
Manipulation von 3D-Digitalisaten im virtuellen Raum}\footnote{Vgl.
  Ceynowa, Klaus (2015) Vom Wert des Sammelns und vom Mehrwert des
  Digitalen. In: Bibliothek -- Forschung und Praxis, 39(2015)3, hier S.
  274.} zu werfen oder sein \enquote{reales 3D-Drucker-Simulacrum des
Originals}.\footnote{Ebd., S. 274. Ceynowa lässt hier offen, ob er hier
  willentlich an den Simulakrum-Begriff von Jean Baudrillard anknüpft,
  den dieser in eher modernisierungskritischer Haltung in die Diskussion
  Ende der 70-iger Jahre einbringt; vgl. Baudrillard, Jean (1978) La
  précession du simulacre. In: Traverses, 10.1978 sowie (1985)
  Simulacres et simulation.}

Trotzdem meint eben dieser Autor einige Monate später (März 2016):
\enquote{Zunächst ist festzuhalten: Das Buchzeitalter will nicht
vergehen. {[}\ldots{}{]} Die oft bedrohlich als \enquote{Disruption}
beschriebene digitale Transformation ist weitgehend vollzogen, ohne dass
dies den Tod des Gedruckten bedeutet hätte}.\footnote{Ceynowa, Klaus
  (2016) Anker im Fluss des Wissens. Begehrte \enquote{Ruinen}. Die
  Bibliothek der Zukunft muss dynamischen Objekten Dauer verleihen. In:
  FAZ Nr. 74.2016 vom 30.03.2016, S. N4. Siehe auch den Beitrag von
  Steffen Heizereder: Das Buch lebt \ldots{} und lebt \ldots{} und lebt
  \ldots{} In: Bub 68(2016)4, S. 152-153.}

Aus dieser Gemengelage 2014-2016 heraus wurde hier die unpopuläre Frage
gestellt, wie es mit der Buchzentriertheit, der Textorientierung in den
Geisteswissenschaften sich verhalte, und als direkte Konsequenz daraus,
für den medialen Hybridansatz ihrer Forschungsbibliotheken? Es wurde
versucht, den Marquard'schen, stets spürbaren \enquote{context of
justification} zu verlassen (\enquote{Die Bibliothek: aussterben,
überleben oder erneuern?}\footnote{Bruijnzeels, Rob (2015): Die
  Bibliothek: aussterben, überleben oder erneuern? In: Bibliothek --
  Forschung und Praxis, 39(2015)2, S. 225-234.}) und gute Gründe zu
sichten für einen dezidierten medial begründeten Hybridansatz. Im
Kernbereich geisteswissenschaftlicher Verstehensbemühungen wurden
\enquote{Latenz} und \enquote{Deep Reading} unter medialen Aspekten
erkundet.

\begin{quote}
\enquote{Die Kulturtechnik des Lesens war lange Zeit darauf beschränkt,
von Menschen ausgeübt zu werden. {[}\ldots{}{]} heute kombinieren wir
ganz selbstverständlich das dem Menschen eigene intellektuelle,
semantische Lesen mit dem oberflächlichen, syntaktischen, aber
außerordentlich schnellen Lesen des Computers. Eine Suchmaschine ist
eigentlich nichts anderes als eine Lesemaschine, mit deren Hilfe wir
unsere zeitliche Begrenztheit im Leseprozess zu überwinden suchen. Lesen
ist damit zu einer hybriden Kulturtechnik geworden.}\footnote{Klawitter,
  Jana (et al., 2011) Kulturwissenschaftliche Forschung, S. 10.}
\end{quote}

Das Zeitregime der Moderne, wesentlich durch das Paradoxon von
technikgetriebener Beschleunigung bei gleichzeitiger gesellschaftlicher
Zeitknappheit (Hartmut Rosa, 2005) charakterisiert, verkürzt gesagt
durch ein individuell nicht mehr ausschöpfbares Wachstum an
Handlungsoptionen (in unserem Kontext \enquote{Informationsflut} durch
Totaldigitalisierung von Kultur), wurde als eine Systemvariable
herausgestellt, die als solche auch auf Leseprozesse entscheidenden
Einfluss hat. \enquote{Lesen unter den Bedingungen der Moderne steht
unter dem Diktat der Zeit.}\footnote{Zedelmaier, Helmut (2015)
  Werkstätte des Wissens zwischen Renaissance und Aufklärung, S. 16.
  Vgl. ausführlich diachronisch vom 16.-21. Jahrhundert dazu Waquet,
  Françoise (2015) L'ordre matériel du savoir, besonders S. 249-273.}

Zum systemischen Umfeld der hier in den Blick genommenen
Spezialbibliotheken gehören die Geisteswissenschaften selbst. Daher
wurde ein Blick auf die Trends in den Digital Humanities geworfen.
Werden sie zur Leitdisziplin in den Kulturwissenschaften? Wird xml oder
eine andere Auszeichnungssprache als gleichwertig anerkannt werden wie
Sprachkenntnisse im Französischen oder Latein? Wird man die Syntax und
Semantik von Code studieren, Algorithmen auf die in ihnen enthaltenen
Herrschaftsstrukturen hin untersuchen, Computerrituale des Alltags
analysieren? Ist Software anerkannte Quelle und Hardware ihre Realie?
Verdrängt das \enquote{computational subject} (David M. Berry 2012) den
Wissensarbeiter \enquote{alten Stils} (Peter-André Alt 2014)?

Oder werden die Digital Humanities schlicht, was jüngst feststellbare
Tendenz ist, dem Werkzeugkasten des Historikers hinzugefügt? Es gibt
deutliche Anzeichen dafür. Aber auch hier ist diese spannungsgeladene
Stillstandsituation bei konzeptueller Verunsicherung zu spüren.
Christian Rohr (2015) in eben dieser verstetigten Rhetorik der
Verunsicherung in seiner Einleitung zu \enquote{Historische
Hilfswissenschaften}:

\begin{quote}
\enquote{Wozu brauchen wir Historische Hilfswissenschaften? Diese Frage
haben sich in den letzten beiden Jahrzehnten {[}sic{]} vermutlich viele
Fakultäten bzw. Institute gestellt {[}\ldots{}{]}. Sind Historische
Hilfswissenschaften nicht mehr zeitgemäß? Repräsentieren sie gleichsam
eine vergangene Geschichtssicht {[}\ldots{}{]}. Haben sie in Zeiten, in
denen für neuere Teilfächer {[}\ldots{}{]} neue Lehrstühle geschaffen
werden, keine Daseinsberechtigung mehr? {[}\ldots{}{]} was haben die
Vertreter dieser Disziplin falsch gemacht, dass sie so in die Defensive
geraten sind? Warum verschwinden die Hilfswissenschaften {[}\ldots{}{]}?
Warum wirkt auf viele die Arbeit mit Archivquellen als altmodisch und
verstaubt? Es scheint, dass die Historischen Hilfswissenschaften ein
Problem damit haben, ihre Relevanz {[}\ldots{}{]} deutlich genug
aufzuzeigen. {[}\ldots{}{]} Man spricht lieber von}Digital Humanities"
{[}\ldots{}{]}.``\footnote{Rohr, Christian (2015) Historische
  Hilfswissenschaften. Eine Einführung. Wien {[}u.a.{]} : Böhlau, 2015.
  S. 11.}
\end{quote}

Zunächst also wie für das Bibliothekswesen verstetigte
Untergangsszenarien, \enquote{dead-end-job}-Ängste,
Verdrängungsphantasmen und allgemeiner \enquote{context of
justification} und dann doch ein ganz unspektakuläres, pragmatisches
\enquote{Trotzdem}: Auf den folgenden 280 Seiten erklärt der Autor
engagiert, was die traditionellen Hilfswissenschaften aktuell leisten
und wozu sie dienen.

Dies illustriert die Tragweite von Aleida Assmanns Feststellung:
\enquote{Ohne die stetige Erneuerung von Wertschätzung gibt es kein
Überleben im kulturellen Gedächtnis.}\footnote{Assmann, Aleida (2001)
  Aufmerksamkeiten. Einleitung, S. 13.}

\begin{quote}
\enquote{Wenn Software und Code die Möglichkeitsbedingung für die
Vereinheitlichung der heute an der Universität produzierten
Wissensformen werden, dann könnte die Fähigkeit zum selbstständigen
Denken {[}\ldots{}{]} weniger wichtig werden. {[}\ldots{}{]} Das Denken
könnte sich stattdessen einer {[}\ldots{}{]} Denkmethode zuwenden,
{[}\ldots{}{]} wie man die Technik nutzen muss, um ein brauchbares
Ergebnis zu erzielen -- ein umwälzender Prozess des reflexiven Denkens
und des gemeinschaftlichen Überdenkens. {[}\ldots{}{]} Das
computergestützte Subjekt ist entscheidend für ein datenzentriertes
Zeitalter {[}\ldots{}{]}. Kurzum, Bildung ist noch immer eine
Schlüsselidee an der digitalen Universität, aber {[}\ldots{}{]} für ein
Subjekt, das {[}\ldots{}{]} offen für neue pädagogische Methoden ist,
die dies ermöglichen. {[}\ldots{}{]} Dies ist ein Subjekt, das bevorzugt
per Computer kommuniziert {[}\ldots{}{]}.}\footnote{Berry, David M.
  (2014) Die Computerwende, S. 60.}
\end{quote}

Diese Positionen der \enquote{Third-wave-digital-humanities} von David
M. Berry (2014) kommen im Grunde der Aufkündigung einer Wertschätzung
der etablierten Wissenspraktiken in den Kulturwissenschaften gleich. So
bleibt Wolfgang Giesecke (2002) trotz diverser Versuche der
\enquote{Verschmelzung} von Medienklassen immer noch unwiderlegt:
\enquote{Was die Kombination der verbalen, nonverbalen, natürlichen und
technisierten Medien angeht, tappt unsere Kultur im Dunkeln.}\footnote{Giesecke,
  Wolfgang (2002) Von den Mythen der Buchkultur zu den Visionen der
  Informationsgesellschaft, S. 425. Zeitnah dazu Zimmer, Dieter E.
  (2000), S. 57: \enquote{Das epochale Wunder, das manche zu erwarten
  scheinen, sobald sich die Narration von der leidigen Linearität
  befreit, dürfte jedoch auf sich warten lassen.} Ähnlich resümiert
  Jakob Krameritsch (2007) Geschichte(n) im Netzwerk, S. 21f:
  \enquote{Nach dem -- vor allem auf der Diskursebene inszenierten Hype
  rund um den Hypertext, begann sich angesichts der spärlichen
  überzeugenden Hypertext-Produkte bzw. --Prozesse Anfang des 21.
  Jahrhunderts zunehmend Ernüchterung breit zu machen.}}

Zeitgleich ist daher auch eine Tendenz zur Entdeckung oder
Wiederentdeckung der Kulturgeschichte intellektueller Wissenspraktiken
erkennbar und ein Interesse am Aufarbeiten der Materialität von
Wissensgenerierung\footnote{Vgl. Waquet, Françoise (2015) L'ordre
  matériel du savoir. Comment les savants travaillent. XVIe -- XXIe
  siècles, S. 14: \enquote{Cet inventaire laisse apercevoir la nature
  hybride de bien des techniques intellectuelles.}} und facheigener
Arbeitstechniken durch historische Wissensforschung. Helmut Zedelmaier
(2015):

\begin{quote}
\enquote{Wie wir Informationen suchen, wie wir lesen und das Gelesene
verarbeiten, all das hat sich radikal verändert. {[}\ldots{}{]} Die
Erfahrung von Veränderung fördert Differenzierung und schärft die
historische Aufmerksamkeit. Daraus erklärt sich das Interesse für die
Vergangenheit des Wissens und die damit verbundenen Praktiken.
{[}\ldots{}{]} Erfahrungskontexte neuer Kommunikationstechnologien haben
die historische Aufmerksamkeit durchdrungen. \enquote{Wissensgeschichte}
hat Konjunktur. {[}\ldots{}{]} Auch darum geht es in diesem Buch: um die
historische Relativierung der viel beschworenen \enquote{digitalen
Revolution}, die einiges von ihrem revolutionären Charakter verliert,
sobald man sich genauer auf die buchgestützte Welt und ihre Werkstätten
einlässt.}\footnote{Zedelmaier, Helmut (2015) Werkstätte des Wissens
  zwischen Renaissance und Aufklärung. Tübingen : Mohr Siebeck, 2015.
  Einleitung S. 1-3.}
\end{quote}

Bei dem in dieser Arbeit konstatierten Phänomen einer konzeptionellen
Verstetigung durch Verharren im Hybridmodell in den
geschichtswissenschaftlichen \enquote{Werkstätten} könnte es sich vom
Ergebnis her um eine \enquote{Reauratisierung der
Leseerfahrung}\footnote{Thiel, Thomas (2009) Rezension Maryanne Wolf:
  Das lesende Gehirn. In: FAZ, 28.09.2009, S. 8.} handeln, um eine
Wiederentdeckung der \enquote{Eigenzeiten}\footnote{Rosa, Hartmut (2005)
  Beschleunigung. S. 439.} von wissensgenerierenden Leseprozessen
gegenüber digital induzierter systemischer Beschleunigung, um einen
\enquote{Gegenfluss von Aufmerksamkeit}\footnote{Vgl. Assmann, Aleida
  (2001) Aufmerksamkeiten. Einleitung, S. 12.} für analoge
Herkunftswelten.

Diese Wiederentdeckung analoger Herkunftswelten verweist in letzter
Konsequenz eigentlich auf nichts anderes als die Bewahrung
lebensweltlicher Realitäten\footnote{Dazu aus soziologischer Perspektive
  auf aktuellem Forschungsstand Rosa, Hartmut (2016) Resonanz. Eine
  Soziologie der Weltbeziehung. Berlin, 2016.} des Menschen an sich. Die
gegenstands- und bedeutungslos gewordenen
\enquote{Informationswissenschaften}, wie Winfried Gödert alarmiert
konstatierte,\footnote{Vgl. Fußnote 13.} müssten vermutlich deutlicher
den Fokus \enquote{Mensch} \footnote{Zur Diskussion um eine
  transdisziplinäre Rezentrierung auf den Fokus \enquote{Mensch} siehe
  den Sammelband der Heidelberger Jahrbücher 54, hrsg. von Markus
  Hilgert et al.: Menschen-Bilder. Darstellungen des Humanen in der
  Wissenschaft. Heidelberg, 2012.} in spätmodernen Zeiten
wiedergewinnen,\footnote{Wersig, Gernot (2000) Zur Zukunft der Metapher
  \enquote{Informationswissenschaft}. In: Schröder, Thomas A. (Hg. 2000)
  Auf dem Weg zur Informationskultur wa(h)re Information?, S. 275f.:
  \enquote{Vielleicht ist es das, wovor Informationswissenschaftler
  immer Angst gehabt haben, daß sich ganz unten in der Tiefenstruktur
  von Informationswissenschaften das Bedürfnis nach einer Wissenschaft
  verborgen hat, die sich einfach mit Menschen befaßt, die für ihr
  ständiges Handeln verschiedene Formen von Handlungsgrundlagen brauchen
  -- Wissen, Vernunft, Werte, Urteile usw. -- und diese sich auf
  unterschiedliche Art und Weise aneignen und für unterschiedliche
  Handlungszusammenhänge benutzen. {[}\ldots{}{]} Und vielleicht hatten
  sie alle gemeinsam, daß ihnen vor einer solchen Wissenschaft
  schwindelte, die zumindest Elias einmal als}Menschenwissenschaft"
  angesprochen hatte."} damit sie in der digitalen Cloud kondensieren
und realweltlich wieder auf die Erde hienieden zurückfallen könnten.

\section*{Literaturverzeichnis}\label{literaturverzeichnis}

\textbf{Alt}, Peter-André (2014), Artikelflut und Forschungsmüll. In:
Süddeutsche Zeitung vom 23.06.2014, Nr. 141, S.12.

\textbf{Assmann, Aleida} (Hg. et al. 2001) Aufmerksamkeiten. Archäologie
der literarischen Kommunikation VII. München : Fink, 2001.

\textbf{Assmann, Aleida} (2013) Ist die Zeit aus den Fugen? Aufstieg und
Fall des Zeitregimes der Moderne. München : Hanser, 2013.

\textbf{Bachmann-Medick}, Doris (2006) Cultural turns. Neuorientierungen
in den Kulturwissenschaften. Reinbek bei Hamburg : Rowohlt, 2006.

\textbf{Baillot}, Anne ; \textbf{Schnöpf}, Markus (2015) Von
wissenschaftlichen Editionen als interoperable Objekte. In: Schmale,
Wolfgang (Hg. 2015) Digital Humanties, S. 139-156.

\textbf{Ball}, Rafael (2013) Was von Bibliotheken wirklich bleibt. Das
Ende eines Monopols. Ein Lesebuch. Wiesbaden : Dinges \& Fricke, 2013.

\textbf{Bellingradt}, Daniel: {[}Rezension zu:{]} Stöber, Rudolf: Neue
Medien. Von Gutenberg bis Apple und Google. In: Jahrbuch für
Kommunikationsgeschichte, 15.2013, S. 167-169.

\textbf{Berry}, David M. (Hg., 2012), Unterstanding digital humanities.
Houndmills : Palgrave Macmillan, 2012.

\textbf{Berry,} David M. (2014), Die Computerwende. Gedanken zu den
Digital Humanities. In: Reichert, Ramón (Hg.) 2014, Big Data, S. 47-64.

\textbf{Boes}, Andreas (2016 et al.): Digitalisierung und
\enquote{Wissensarbeit}. Der Informationsraum als Fundament der
Arbeitswelt der Zukunft. In: Aus Politik und Zeitgeschichte,
66(2016)18-19, S. 32-39.

\textbf{Bolz}, Norbert (1993) Am Ende der Gutenberg-Galaxis. Die neuen
Kommunikationsverhältnisse. München : Fink, 1993.

\textbf{Bonte}, Achim (2015) Was ist eine Bibliothek? Physische
Bibliotheken im digitalen Zeitalter. In: ABI Technik, 35(2015)2, S.
95-104.

\textbf{Boorstin}, Daniel J. (1974): A design for an anytime,
do-it-yourself, energy-free communication device. In: Harpers Magazine,
Jan. 1, 1974 (248), S. 83-86.

\textbf{Borscheid}, Peter (2004): Das Tempo-Virus. Eine Kulturgeschichte
der Beschleunigung. Frankfurt, Main : Campus, 2004.

\textbf{Burke}, Peter (zuerst 2012, dt. Übers. 2014) Die Explosion des
Wissens. Von der Encyclopédie bis Wikipedia. Berlin : Wagenbach, 2014.

\textbf{Buschhaus}, Markus (2008) Am einen \& am anderen Ende der
Gutenberg-Galaxis. In: Grampp, Sven (Hg. et al., 2008) Revolutionsmedien
-- Medienrevolutionen, S. 205-228.

\textbf{Carstensen}, Tanja (Hg. et al. 2014) Digitale Subjekte.
Praktiken der Subjektivierung im Medienumbruch der Gegenwart. Bielefeld
: transcript, 2014.

\textbf{Carvajal}, Rigoberto (2015) Wie groß ist Big Data? In:
Kulturaustausch 65(2015)4, S. 52-53.

\textbf{Ceynowa}, Klaus (2014) Der Text ist tot. Es lebe das Wissen!
Kultur ohne Text. In: Hohe Luft, 1.2014, S. 52-57.

\textbf{Ceynowa}, Klaus (2015) Vom Wert des Sammelns und vom Mehrwert
des Digitalen. Verstreute Bemerkungen zur gegenwärtigen Lage der
Bibliothek. In: Bibliothek -- Forschung und Praxis, 39(2015)3, S.
268-276.

\textbf{Ceynowa}, Klaus (2016) Anker im Fluss des Wissens. Begehrte
\enquote{Ruinen}. Die Bibliothek der Zukunft muss dynamischen Objekten
Dauer verleihen. In: FAZ Nr. 74.2016 vom 30.03.2016, S. N4.

\textbf{Christians}, Heiko (Hg. et al 2015), Historisches Wörterbuch des
Mediengebrauchs. Köln : Böhlau, 2015.

\textbf{Cole}, Tim (2015) Kein Grund zur Panik. Warum wir der digitalen
Zukunft optimistisch entgegentreten sollten. In: Kulturaustausch
65(2015)4, S. 18-19.

\textbf{Crawford}, Matthew B. (2015) Erfahrungen aus zweiter Hand. In:
Kulturaustausch 65(2015)4, S. 36-37.

\textbf{Deutsches Historisches Institut Paris} : Jahresbericht 1.
September 2011 -- 31. August 2012. Paris : Deutsches Historisches
Institut Paris, 2012.

\textbf{Diekmann}, Stefanie (Hg. et al. 2007) Latenz. 40 Annäherungen an
einen Begriff. Berlin : Kadmos, 2007.

\textbf{Eigenbrodt}, Olaf (2014), Auf dem Weg zur Fluiden Bibliothek.
Formierung und Konvergenz in integrierten Wissensräumen. In: Ders. (Hg.
et al. 2014): Formierungen von Wissensräumen. Optionen des Zugangs zu
Information und Bildung. Berlin : de Gruyter Saur, 2014. S. 207-220.

\textbf{Eigenbrodt}, Olaf (Hg. et al., 2014): Formierungen von
Wissensräumen. Optionen des Zugangs zu Information und Bildung. Berlin :
de Gruyter Saur, 2014.

\textbf{Ellrich}, Lutz (Hg. et al. 2009) Die Unsichtbarkeit des
Politischen. Theorie und Geschichte medialer Latenz. Bielefeld :
Transcript, 2009.

\textbf{Evans}, Leighton; \textbf{Rees}, Sian (2012): An interpretation
of digital humanities. In: Berry, David M. (Hg. 2012) Unterstanding
digital humanities. Houndmills, 2012, S. 21-41.

\textbf{Fabian}, Bernhard (1983) Buch, Bibliothek und
geisteswissenschaftliche Forschung. Göttingen : Vandenhoeck \& Ruprecht,
1983.

\textbf{Faulstich}, Werner (2002) Einführung in die Medienwissenschaft.
Probleme, Methoden, Domänen. München, 2002.

\textbf{Faulstich}, Werner (1998) Medien zwischen Herrschaft und
Revolte. Die Medienkultur der frühen Neuzeit (1400 -- 1700). Göttingen :
Vandenhoeck \& Ruprecht, 1998.

\textbf{Faulstich}, Werner (2005) Begann die Neuzeit mit dem Buchdruck?
Ist die Ära der Typographie im Zeitalter der digitalen Medien endgültig
vorbei? Podiumsdiskussion unter der Leitung von Winfried Schulze. In:
Historische Zeitschrift. Beihefte (Neue Folge), 41.2005, S. 11-38.

\textbf{Frabetti}, Federcia: Eine neue Betrachtung der Digital
Humanities im Kontext originärer Technizität. In: Reichert, Ramón (2014)
Big Data, S. 85-102.

\textbf{Frankenberger}, Rudolf (2004 et al.)Die moderne Bibliothek. Ein
Kompendium der Bibliotheksverwaltung. München : Saur, 2004.

\textbf{Frühwald}, Wolfgang (2002) Gutenbergs Galaxis im 21.
Jahrhundert. Die wissenschaftliche Bibliothek im Spannungsfeld von
Kulturauftrag und Informations-Management. Plenarvortrag beim 92.
Deutschen Bibliothekartag am 9. April 2002 in Augsburg. In: ZfBB
49(2002)4, S. 187-194.

\textbf{Frühwald}, Wolfgang (2011) Gutenbergs Galaxis oder Von der
Wandlungsfähigkeit des Buches. In: Zintzen, Clemens (Hg., 2011) Die
Zukunft des Buches. S. 9-21.

\begin{description}
\tightlist
\item[\textbf{Füssel}, Stephan (Hg.) Medienkonvergenz --
transdisziplinär. Berlin]
de Gruyter, 2012.
\end{description}

\textbf{Fuhrmans}, Marc (2016) Change Management -- Mainstream oder
unverzichtbarer Werkzeugkasten? In: Perspektive Bibliothek, 5.1(2016),
S. 3-24.

\textbf{Ghanbari}, Nacim (2013 et al.) Was sind Medien kollektiver
Intelligenz? Eine Diskussion. In: Zeitschrift für Medienwissenschaft,
2013, H. 8, S. 145-155

\textbf{Gantert}, Klaus (2011) Elektronische Informationsressourcen für
Historiker.

\textbf{Gersmann}, Gudrun (2005) Begann die Neuzeit mit dem Buchdruck?
Ist die Ära der Typographie im Zeitalter der digitalen Medien endgültig
vorbei? Podiumsdiskussion unter der Leitung von Winfried Schulze. In:
Historische Zeitschrift. Beihefte (Neue Folge), Bd. 41.2005, S. 11-38.

\textbf{Gfrereis}, Heike (Hg. et al. 2013) Zettelkästen. Maschinen der
Phantasie. Marbach a.N. : Deutsche Schillergesellschaft, 2013.

\textbf{Goldsmith}, Kenneth (2015) Der digitale Todestrieb. In:
Kulturaustausch 65(2015)2, S. 55.

\textbf{Gombrich}, Ernst H. (zuerst 1970, dt. Ausgabe 1992) Aby Warburg.
Eine intellektuelle Biographie. Hamburg : Europäische Verlagsanstalt,
1992.

\textbf{Grampp}, Sven (Hg. et al. 2008) Revolutionsmedien --
Medienrevolutionen. Konstanz : UVK, 2008.

\textbf{Groebner}, Valentin (2012) Wissenschaftssprache -- eine
Gebrauchsanweisung. Konstanz : Konstanz university press, 2012.

\textbf{Groebner}, Valentin (2014) Wissenschaftssprache digital.
Konstanz : Konstanz university press, 2014.

\textbf{Grunert}, Frank (Hg. et al. 2015) Wissensspeicher der Frühen
Neuzeit. Formen und Funktionen. Berlin {[}u.a.{]} : de Gruyter, 2015.

\textbf{Gumbrecht}, Hans Ulrich (Hg. et al. 2011) Latenz. Blinde
Passagiere in den Geisteswissenschaften. Göttingen : Vandenhoek \&
Ruprecht, 2011.

\textbf{Gumbrecht}, Hans Ulrich (2014) Das Denken muss nun auch den
Daten folgen. In: Frankfurter Allgemeine Zeitung Nr. 59 vom 11.03.2014,
S. 14.

\textbf{Haber}, Peter (2005) Archive des Wissens. Neue Herausforderungen
für ein altes Problem. In: Historisches Forum 7(2005)1, S. 73-85.

\textbf{Habermas}, Rebekka (Hg. et al. 2004) Interkultureller Transfer
und nationaler Eigensinn. Europäische und anglo-amerikanische Positionen
der Kulturwissenschaftren. Göttingen : Wallstein, 2004.

\textbf{Hacker}, Gerhard (2005) Die Hybridbibliothek -- Blackbox oder
Ungeheuer? In: Ders. (Hg. et al. 2005) Bibliothek leben. Das deutsche
Bibliothekswesen als Aufgabe für Wissenschaft und Politik. Wiesbaden,
2005, S. 278-295.

\textbf{Hagner}, Michael ; \textbf{Hirschi}, Caspar (2013) Editorial.
In: Nach Feierabend. Züricher Jahrbuch für Wissensgeschichte, 9.2013, S.
7-11.

\textbf{Hagner}, Michael (2011) Ein Buch ist ein Buch ist ein Buch;
keine Datei. In: Zintzen, Clemens (Hg. 2011) Die Zukunft des Buches.
Stuttgart : Steiner {[}u.a.{]}, 2011. S. 49-51.

\textbf{Hagner}, Michael (2015) Zur Sache des Buches. 2., überarb. Aufl.
Göttingen : Wallstein, 2015.

\textbf{Harnischfeger}, Horst (2005) Zwischen Gutenberg und Google --
Medien. In: Maaß, Kurt-Jürgen (Hg., 2005) Kultur und Außenpolitik, S.
161-171.

\textbf{Heibach}, Christiane (2011), (De)Leth(h)e. Über das Problem des
Vergessens im Digitalen Zeitalter. In: Jochum, Uwe (Hg. et al. 2011) Das
Ende der Bibliothek?, S. 53-70.

\textbf{Heizereder}, Steffen (2016): Das Buch lebt \ldots{} und lebt
\ldots{} und lebt \ldots{} In: Bub 68(2016)4, S. 152-153.

\textbf{Heßler}, Martina (2016): Zur Persistenz der Argumente im
Automatisierungsdiskurs. In: Aus Politik undZeitgeschichte,
66(2016)18-19, S. 17-24.

\textbf{Hilgert}, Markus (Hg. et al. 2012) Menschen-Bilder.
Darstellungen des Humanen in der Wissenschaft. Heidelberg : Springer,
2012.

\textbf{Hölscher}, Lucian (2009) Semantik der Leere. Grenzfragen der
Geschichtswissenschaft. Göttingen : Wallstein, 2009.

\textbf{Hörisch}, Jochen (zuerst 2001) Eine Geschichte der Medien. Vom
Urknall zum Internet. 4. Aufl., Nachdr. Frankfurt, Main : Suhrkamp,
2010.

\textbf{Holtorf}, Christian (2013), Der erste Draht zur Neuen Welt.
Göttingen, 2013.

\textbf{Horstmann}, Wolfram: Die Bibliothek als Werkstatt der
Wissenschaft. Rede zur Amtseinführung, Göttingen am 24. Juli 2014. In:
Bibliothek, Forschung und Praxis 38(2014)3, S. 503-505.

\textbf{Hotea}, Meda Diana (2015) ETHorama. Ein unkomplizierter Zugang
zu digitalen Bibliotheksinhalten. In: ABI Technik. 35(2015)1, S. 11-22.

\textbf{Hug}, Theo (2012) Kritische Erwägungen zur Medialisierung des
Wissens im digitalen Zeitalter. In: Brigitte Kossek (Hg. et al, 2012)
Digital Turn?, S. 23-46.

\textbf{Illich}, Ivan (zuerst 1990) Im Weinberg des Textes. Als das
Schriftbild der Moderne entstand. (Dt. Übers.) Frankfurt, Main :
Luchterhand, 1991.

\textbf{Jabr}, Ferris (2013) Why the brain prefers paper. In: Scientific
American, 309(2013)5, S. 48-53.

\textbf{Jochum}, Uwe (2007) Kleine Bibliotheksgeschichte. 3. Aufl.,
Stuttgart, 2007.

\textbf{Jochum}, Uwe (2011) Die Selbstabschaffung der Bibliotheken. In:
Ders. (Hg. et al, 2011) Das Ende der Bibliothek? Vom Wert des Analogen,
S. 11-25.

\textbf{Jochum}, Uwe ; \textbf{Schlechter}, Armin (Hg. 2011) Das Ende
der Bibliothek? Vom Wert des Analogen. Frankfurt, Main : Klostermann,
2011.

\textbf{Jochum}, Uwe (2015): Bücher. Vom Papyrus zum E-Book. Darmstadt :
Zabern, 2015.

\textbf{Jungbluth}, Anja: Vor Kindle. Die Anfänge des E-Books. In:
Perspektive Bibliothek, 4(2015)2, S. 87-106.

\textbf{Karpenstein-Eßbach}, Christa (2004) Einführung in die
Kulturwissenschaft der Medien. Paderborn : Fink, 2004.

\textbf{Keller}, Reiner (2013) Diskursanalyse. In: Umlauf, Konrad (Hg.
et al. 2013) Handbuch Methoden der Bibliotheks- und
Informationswissenschaft. S. 425-443.

\textbf{Kempf}, Klaus (2003) Erwerbung und Beschaffung in der
Hybridbibliothek. In: Littger, Klaus Walter (Hg. et al. 2003)
Entwicklungen und Bestände. Bayerische Bibliotheken im Übergang zum 21.
Jahrhundert. S. 35-67.

\textbf{Kempf}, Klaus (2014) Bibliotheken ohne Bestand? In: Bibliothek :
Forschung und Praxis 2014 (38), 3. S. 365-397.

\textbf{Kirchmann}, Kay (1998) Verdichtung, Weltverlust und Zeitdruck.
Grundzüge einer Theorie der Interdependenzen von Medien, Zeit und
Geschwindigkeit im neuzeitlichen Zivilisationsprozess. Wiesbaden :
Springer Fachmedien, 1998.

\textbf{Klawitter}, Jana (Hg. et al. 2011) Kulturwissenschaften digital.
Neue Forschungsfragen und Methoden. Frankfurt, Main {[}u.a.{]} : Campus,
2011.

\textbf{Klawitter}, Jana (2011 et al.) Kulturwissenschaftliche
Forschung. Einflüsse von Digitalisierung und Internet. In: Dies. (Hg. et
al. 2011) Kulturwissenschaften digital, S. 9-29.

\textbf{Köstlbauer}, Josef (2015): Spiel und Geschichte im Zeichen der
Digitalität. In: Schmale, Wolfgang (Hg. 2015) Digital Humanities, S.
95-124.

\textbf{Kopp}, Vanina (2016) Der König und die Bücher. Sammlung, Nutzung
und Funktion der königlichen Bibliothek am spätmittelalterlichen Hof in
Frankreich. Ostfildern : Thorbecke, 2016.

\textbf{Kossek}, Brigitte (Hg. et al. 2012), Digital turn? Zum Einfluss
digitaler Medien auf Wissensgenerierungsprozesse von Studierenden und
Hochschullehrenden. Göttingen : V \& R unipress, 2012.

\textbf{König}, Mareike (2015) Blogs als Wissensorte der Forschung.
Konferenzbeitrag. In: Die Zukunft der Wissensspeicher. Forschen, Sammeln
und Vermitteln im 21. Jahrhundert. Düsseldorf, 5.-6. März 2015.

\textbf{König}, Mareike (2015) Herausforderung für unsere
Wissenschaftskultur. Weblogs in den Geisteswissenschaften. In: Schmale,
Wolfgang (Hg. 2015) Digital Humanities, S. 57-74.

\textbf{Kotter}, John P. (2014) Accelerate. Building strategic agility
for a faster-moving world. Boston : Harvard, 2014. {[}Dt. Übers. München
: Vahlen, 2015{]}.

\textbf{Krakauer}, Siegfried (zuerst 1929) Über Arbeitsnachweise.
Konstruktion eines Raumes. In: Ders., Werke, Bd. 5,3: Essays,
Feuilletons, Rezensionen 1928 -- 1931. Hrsg. von Mülder-Bach, Inka.
Berlin : Suhrkamp, 2011. S. 249-257.

\textbf{Krameritsch}, Jakob (2007) Geschichte(n) im Netzwerk. Hypertext
und dessen Potenziale für die Produktion, Repräsentation und Rezeption
der historischen Erzählung. Münster : Waxmann, 2007.

\textbf{Krebs}, Stefan (2015) Zur Sinnlichkeit der Technik(geschichte).
Ist es Zeit für einen \enquote{sensorial turn}. In: Technikgeschichte
82(2015)1, S. 3-9.

\textbf{Kuhlen}, Rainer (2002): Abendländisches Schisma. Der
Reformbedarf der Bibliotheken. In: FAZ Nr. 81.2002 vom 08.04.2002, S.
46.

\textbf{Lepper}, Marcel (Hg. et al. 2016) Handbuch Archiv. Geschichte,
Aufgaben, Perspektiven. Stuttgart : Metzler, 2016.

\textbf{Lipp}, Thordolf: Arbeit am medialen Gedächtnis. Zur
Digitalisierung von Intangible Cultural Heritage. In: Robertson-von
Trotha, Caroline Y. (Hg. et al. 2011) Neues Erbe, S. 39-67.

\textbf{Littger}, Klaus Walter (Hg. et al. 2003): Entwicklungen und
Bestände. Bayerische Bibliotheken im Übergang zum 21. Jahrhundert.
Wiesbaden : Harrassowitz, 2003.

\textbf{Lübbe}, Hermann (Hg. 1978) Wozu Philosophie? Stellungnahmen
eines Arbeitskreises. Berlin {[}u.a.{]} : de Gruyter, 1978.

\textbf{Lübbe}, Hermann (1988) Der verkürzte Aufenthalt in der
Gegenwart. Wandlungen des Geschichtsverständnisses. In: Kemper, Peter
(Hg.) \enquote{Postmoderne} oder Der Kampf um die Zukunft. Frankfurt,
Main : Fischer, 1988, S. 145-164.

\textbf{Mallinckrodt}, Rebekka von (2004) \enquote{Discontenting,
surely, even for those versed in French intellectual pyrotechnics}.
Michel de Certeau in Frankreich, Deutschland und den USA. In: Habermas,
Rebekka (Hg. et al. 2004) Interkultureller Transfer und nationaler
Eigensinn. Göttingen : Wallstein, 2004. S. 221-241.

\textbf{Marquard}, Odo (1984) Entlastungen. Theodizeemotive in der
neuzeitlichen Philosophie. In: Jahrbuch Wissenschaftskolleg 1982/83.
Berlin : Siedler, 1984. S. 245-258.

\textbf{Marquard}, Odo (1985) Über die Unvermeidlichkeit der
Geisteswissenschaften. In: Marquard, Odo: Apologie des Zufälligen.
Stuttgart : Reclam, 1986. S. 98-116.

\textbf{Marquard}, Odo (1986) Apologie des Zufälligen. Stuttgart :
Reclam, 1986.

\textbf{Maasen}, Sabine; \textbf{Sutter}, Barbara (2016) Dezentraler
Panoptismus. Subjektivierung unter techno-sozialen Bedingungen im Web
2.0. In: Geschichte und Gesellschaft, 42(2016)1, S. 173-194.

\textbf{McLuhan}, Herbert Marshall (amerikan. Erstausg. 1962) Die
Gutenberg-Galaxis. Die Entstehung des typographischen Menschen. Dt.
Übers. Hamburg : Ginko Press, 2011.

\textbf{Meckel}, Miriam (2011), Next. Erinnerungen an eine Zukunft ohne
uns. Erinnerungen eines ersten humanoiden Algorithmus. Reinbek bei
Hamburg, 2011.

\textbf{Meier}, Urs (2014) 100 Jahre Riepl'sches Gesetz. In: Kappes,
Christoph (Hg. et al. 2014) Medienwandel kompakt 2011 -- 2013.
Netzveröffentlichungen zu Medienökonomie, Medienpolitik \& Journalismus.
Wiesbaden : Springer VS, 2014, S. 11-17.

\textbf{Meßner}, Daniel (2015) Coding History. Software als
kulturwissenschaftliches Forschungsobjekt. In: Schmale, Wolfgang (Hg.
2015) Digital Humanities, S. 157-174.

\textbf{Metzinger}, Thomas (2016) Wer, ich? Spiegel-Gespräch. In: Der
Spiegel 2016, Nr. 19 vom 7.5.2016, S. 68-71.

\textbf{Mittler}, Elmar (2012) Wissenschaftliche Forschung und
Publikationen im Netz. In: Füssel, Stephan (Hg. 2012) Medienkonvergenz
-- transdisziplinär, S. 31-80.

\textbf{Müller-Funk}, Wolfgang (1995) Erfahrung und Experiment. Studien
zur Theorie und Geschichte des Essayismus. Berlin : Akademie-Verlag,
1995.

\textbf{Mulsow}, Martin (2012) Prekäres Wissen. Eine andere
Ideengeschichte der Frühen Neuzeit. Berlin : Suhrkamp, 2012.

\textbf{Nekroponte}, Nicholas (1995): Being digital. Dt.: Total digital
(1995). München : Goldmann, überarb. Ausg. 1997.

\textbf{Oexle}, Otto Gerhard (Hg. 1998) Naturwissenschaft,
Geisteswissenschaft, Kulturwissenschaft. Einheit -- Gegensatz --
Komplementarität? Göttingen : Wallstein, 1998.

\textbf{Olbertz}, Jan-Hendrik (2014), Kein Grund für Kulturpessimismus,
aber \ldots{} In: Die Politische Meinung 59.2014, Nr. 526, S. 47-51.

\textbf{Olms}, Dietrich: Bücher ohne Verlage -- Verlage ohne Bücher.
Stand und Zukunft der Informationsvermittlung in den
Geisteswissenschaften. In: Zintzen, Clemens (Hg., 2011) Die Zukunft des
Buches. Mainz : Akademie der Wissenschaften und der Literatur
{[}u.a.{]}, 2011. S. 65-74

\textbf{Oppenheim}, Charles; \textbf{Smithson}, Daniel (1999), What is
the hybrid library? In: Journal of information science 25(1999)2, S.
97-112.

\textbf{Ott}, Karl-Heinz (2014), Gewichtige Werke oder digitales
Gewurstel. In: Die Politische Meinung 59(2014)526, S. 84-89.

\textbf{Paravicini}, Werner (Hg. 1994) Das Deutsche Historische Institut
Paris. Festgabe. Sigmaringen : Thorbecke, 1994.

\textbf{Paravicini}, Werner (2010) Die Wahrheit der Historiker. München
: Oldenbourg, 2010.

\textbf{Peschl}, Markus E.; \textbf{Fundneider}, Thomas (2012) Vom
\enquote{digital turn} zum \enquote{socio-epistemological creative
turn}. In: Kossek, Brigitte (Hg. et al. 2012) Digital Turn? Zum Einfluss
digitaler Medien auf Wissensgenerierungsprozesse von Studierenden und
Hochschlulehrenden. Göttingen : V\&R unipress, 2012. S. 47-62.

\textbf{Pfeifer}, Wolfgang (Hg. 1993) Etymologisches Wörterbuch des
Deutschen, Bd. 1 (1993), 2. Aufl.

\textbf{Plassmann}, Engelbert ; \textbf{Syré}, Ludger (2004) Die
Bibliothek und ihre Aufgabe. In: Frankenberger, Rudolf (Hg. et al 2004)
Die moderne Bibliothek. Ein Kompendium der Bibliotheksverwaltung.
München : Saur, 2004. S. 11-41.

\textbf{Pscheida}, Daniela (2013), Wissen und Wissenschaft unter
digitalen Vorzeichen. In: Aus Politik und Zeitgeschichte 63(2013)18-20,
S. 16-22.

\textbf{Radke-Uhlmann}, Gyburg (2010) Zur Dialogizität des platonischen
Parmenides. In: Hempfer, Klaus W. (2010 et al.) Der Dialog im
Diskursfeld seiner Zeit. Von der Antike bis zur Aufklärung. Stuttgart :
Steiner, 2010. S. 27-45.

\textbf{Raulff}, Ulrich (2003) Wilde Energien. Vier Versuche zu Aby
Warburg. Göttingen : Wallstein, 2003.

\textbf{Rautenberg}, Ursula (Hg. 2003) Reclams Sachlexikon des Buches.
2., verb. Aufl., Stuttgart : Reclam, 2003.

\textbf{Rautenberg}, Ursula (Hg. 2015) Reclams Sachlexikon des Buches.
3., vollst. überarb. u. aktual. Aufl., Stuttgart : Reclam, 2015.

\textbf{Reichert}, Ramón (Hg. 2014) Big Data. Analysen zum digitalen
Wandel von Wissen, Macht und Ökonomie. Bielefeld : transcript, 2014.

\textbf{Reiterer}, Harald (2014) Blended Interaction. Ein neues
Interaktionsparadigma. In: Informatik-Spektrum, 37(2014)5, S. 459-463.

\textbf{Riepl}, Wolfgang (zuerst 1913) Das Nachrichtenwesen des
Altertums (Nachdr.) Hildesheim : Olms, 1972.

\textbf{Robertson-von Trotha}, Caroline Y (Hg. et al. 2011) Neues Erbe.
Aspekte, Perspektiven und Konsequen-zen der digitalen Überlieferung.
Karlsruhe : Karlsruher Institut für Technologie (KIT), 2011.

\textbf{Robertson-von Trotha}, Caroline Y (Hg. et al. 2015) Digitales
Kulturerbe. Bewahrung und Zugänglichkeit in der wissenschaftlichen
Praxis. Karlsruhe : Karlsruher Institut für Technologie (KIT), 2015.

\textbf{Rödder}, Andreas (2015) 21.0. Eine kurze Geschichte der
Gegenwart. München : Beck, 2015.

\textbf{Rösch}, Hermann (2004) Wissenschaftliche Kommunikation und
Bibliotheken im Wandel. In: B.I.T. online 7(2004) 2, S. 113-124.

\textbf{Rösch}, Hermann (2005) Wissenschaftliche Kommunikation und
Bibliotheken im Wandel. Sammeln und Ordnen, Bereitstellen und Vermitteln
in diversen medialen Kontexten und Kulturen. In: Historisches Forum
7(2005) 1, S. 87-113.

\textbf{Rohr}, Christian (2015) Historische Hilfswissenschaften. Eine
Einführung. Wien {[}u.a.{]} : Böhlau, 2015.

\textbf{Rosa}, Hartmut (2005) Beschleunigung. Frankfurt, Main :
Suhrkamp, 2005.

\textbf{Rosa}, Hartmut (2016) Resonanz. Eine Soziologie der
Weltbeziehung. Berlin : Suhrkamp, 2016.

\textbf{Sahle}, Patrick (2013) Digitale Editionsformen. Zum Umgang mit
der Überlieferung unter den Bedingungen des Medienwandels. Dissertation
in 3 Bände. Norderstedt : Books on Demand, 2013.

\textbf{Schachtner}, Christina; Duller, Nicole (2014) Kommunikationsort
Internet. Digitale Praktiken und Subjektwerdung. In: Carstensen, Tanja
(Hg. et al. 2014) Digitale Subjekte, S. 81-154.

\textbf{Schaefer-Rolffs}, Aike (2013) Hybride Bibliotheken. Navigatoren
in der modernen Informationslandschaft. Stategien und Empfehlungen für
Bibliotheken/Informationsexperten. Berlin : Simon-Verlag für
Bibliothekswissen, 2013.

\textbf{Schmale}, Wolfgang (Hg. 2015) Digital Humanities. Praktiken der
Digitalisierung, der Dissemination und der Selbstreflexivität. Stuttgart
: Steiner, 2015.

\textbf{Schmale}, Wolfgang (2015), Einleitung Digital Humanities. In:
Schmale, Wolfang (Hg., 2015) Digital Humanities. Praktiken der
Digitalisierung, der Dissemination und der Selbstreflexivität. Stuttgart
: Steiner, 2015. S. 9-13.

\textbf{Schneider}, Ulrich Johannes (2015) Wozu Lesesäle? In: FAZ Nr.
186 vom 13. August 2015, S. 12.

\textbf{Schneider}, Ulrich Johannes (2016) Die Aufgabe einer
wissenschaftlichen Bibliothek ist offen. Interview. In: Kongressnews
b.i.t. online Nr. 1 vom Montag, den 14. März 2016, S. 4-6.

\textbf{Sloterdijk}, Peter (1993) Zum Empfang des
Ernst-Robert-Curtius-Preises für Essayistik. Bonn : Bouvier, 1993. S.
43-56.

\textbf{Stalder}, Felix (2016) Kultur der Digitalität. Berlin :
Suhrkamp, 2016.

\textbf{Stegbauer}, Christian (2014) Beziehungsnetzwerke im Internet.
In: Weyer, Johannes (Hg., 2014) Soziale Netzwerke. München : de Gruyter
Oldenbourg, 2014. S. 239-263.

\textbf{Stiegler}, Bernard (2014): Licht und Schatten im digitalen
Zeitalter. Programmatische Vorlesung auf dem Digital Inquiry Symposium
am Berkeley Center for New Media. In: Reichert, Ramón (2014) Big Data,
S. 35-46.

\textbf{Stöber}, Rudolf (2013) Neue Medien. Geschichte. Von Gutenberg
bis Apple und Google. Medieninnovation und Evolution. {[}Gründlich
revidierte, aktualisierte Neuaufl.{]} -- Bremen : edition lumière, 2013.

\textbf{Strauss}, Simon (2015) Und wo sind hier die Bücher. Bibliothek
der Zukunft. In: FAZ Nr. 229 vom 2. Okt. 2015, S. 20.

\textbf{Strohschneider}, Peter (2012), Faszinationskraft der Dinge. Über
Sammlung, Forschung und Universität. In: Denkströme. Journal der
Sächsischen Akademie der Wissenschaften, 8(2012), S. 9-26.

\textbf{Sühl-Strohmenger}; Wilfried (2016) Das hochkomplexe Feld des
Publizierens. {[}Rezension zu:{]} Michael Hagner: Zur Sache des Buches.
In: Bub 68(2016)4, S. 208-209.

\textbf{Sutton}, Stuart A. (1996) Future service models and the
convergence of functions. The reference librarian as technician, author
and consultant. In: Low, Kathleen (Hg.) The roles of reference
librarians today and tomorrow. New York : Haworth, 1996. S. 125-143.

\textbf{Thiel}, Thomas (2009) {[}Rezension zu:{]} Maryanne Wolf: Das
lesende Gehirn. In: FAZ, 28.09.2009, S. 8.

\textbf{Thiel}, Thomas (2012) Eine Wende für die Geisteswissenschaften?
Standardisierung und Digitalisierung. Der Wissenschaftsrat wertet
Forschungsinfrastrukturen auf. In: FAZ Nr. 171 vom 25. Juli 2012, S. N5.

\textbf{Ulucan}, Sibel: Hybride Bibliotheken -- eine
Begriffsneubestimmung. In: Libreas, 21(2012)2, S. 87-92.

\textbf{Umlauf}, Konrad (Hg. et al. 2012) Handbuch Bibliothek.
Geschichte, Aufgaben, Perspektiven. Stuttgart : Metzler, 2012.

\textbf{Umlauf}, Konrad (Hg. et al. 2013) Handbuch Methoden der
Bibliotheks- und Informationswissenschaft. Bibliotheks-,
Benutzerforschung, Informationsanalyse. Berlin ; Boston : de Gruyter
Saur, 2013.

\textbf{Waquet}, Françoise (2015) L'ordre matériel du savoir. Comment
les savants travaillent. XVIe -- XXIe siècles. Paris : CNRS Éditions,
2015.

\begin{description}
\tightlist
\item[\textbf{Warwick}, Claire (Hg. 2012), Digital Humanities in
practice. London]
Facet, 2012.
\end{description}

\textbf{Weidenbach}, Lukas (2015) Buchkultur und digitaler Text. Zum
Diskurs der Mediennutzung und Medienökonomie. Hamburg :
Diplomica-Verlag, 2015.

\textbf{Weltweit vor Ort}. Das Magazin der Max Weber Stiftung. Bonn :
Max Weber Stiftung, 2011ff.

\textbf{Wersig}, Gernot (2000) Zur Zukunft der Metapher
\enquote{Informationswissenschaft}. In: Schröder, Thomas A. (Hg.) Auf
dem Weg zur Informationskultur wa(h)re Information? Festschrift für
Norbert Henrichs zum 65. Geburtstag. Düsseldorf : Univ.- u.
Landesbibliothek Düsseldorf, 2000. S. 267-276.

\textbf{Weyer}, Johannes (2014) Netzwerke in der mobilen
Echtzeit-Gesellschaft. In: Weyer, Johannes (Hg.) Soziale Netzwerke. 3.,
überarb. Aufl., München : de Gruyter Oldenbourg, 2014. S. 3-37.

\textbf{Wirth}, Sabine (2014): {[}Artikel{]} Computer/Internet. In:
Wodianka, Stephanie (Hg. et al.) Metzler Lexikon moderner Mythen (2014)
S. 84.

\textbf{Wodianka,} Stephanie (Hg. et al.) Metzler Lexikon moderner
Mythen. Stuttgart : Metzler, 2014.

\textbf{Wyss}, Beat (2010) Bilder von der Globalisierung. Die
Weltausstellung von Paris 1889. Berlin : Insel Verlag, 2010.

\textbf{Zedelmaier}, Helmut (2015) Werkstätte des Wissens zwischen
Renaissance und Aufklärung. Tübingen : Mohr Siebeck, 2015.

\textbf{Zimmer}, Dieter E. (2000) Die Bibliothek der Zukunft. Text und
Schrift in den Zeiten des Internet. Hamburg : Hoffmann und Campe, 2000.

\textbf{Zintzen}, Clemens (Hg. 2011) Die Zukunft des Buches. Vorträge
des Symposions der Geistes- und Sozialwissenschaftlichen Klasse und der
Klasse der Literatur in der Akademie der Wissenschaften und der
Literatur, Mainz, am 20. Mai 2010. Mainz : Akademie der Wissenschaften
und der Literatur {[}u.a.{]}, 2011.

\section*{Online-Ressourcen}\label{online-ressourcen}

\emph{Letzte Prüfung der Verfügbarkeit: 30. Juni 2016.}

\textbf{Gebhardt}, Christoph; Rädle, Roman; Reiterer, Harald (2014)
Employing blended interaction to blend the qualities of digital and
physical books. In: M \& C Best Paper, 3.2014, S. 36-42.
\url{http://hci.uni-konstanz.de/downloads/icom.2014.0028.pdf}

\textbf{Gödert}, Winfried (2015), Hashtag Erschließung : eine vermutlich
Konsequenzen lose Besinnung bei dräuendem Cloud-Gewitter.
\url{http://eprints.rclis.org/24643/}

\textbf{Haber}, Peter (2010), Reise nach Digitalien und zurück. Ein
historiographischer Betriebsausflug. Habilitationsvorlesung Basel, 2.
November 2010. \url{http://www.histnet.ch/repository/hnwps/hnwps-01.pdf}

\textbf{König}, Mareike (2016) Was sind Digital Humanities?
Definitionsfragen und Praxisbeispiele aus der Geschichtswissenschaft.
\url{https://dhdhi.hypotheses.org/2642}

\textbf{Meier}, Urs (2013) 100 Jahre Riepl'sches Gesetz. In:
Journal21.ch vom 23. Januar 2013.
\url{https://www.journal21.ch/100-jahre-rieplsches-gesetz}

\textbf{Nielsen}, Jakob (2006) The 90-9-1 rule for participation
inequality in social media and online communities:
\url{https://www.nngroup.com/articles/participation-inequality/}

\textbf{Paperage}: \url{https://www.youtube.com/watch?v=OujTD3TAFYY}

\textbf{Pinfield}, Stephen (et al., 1998) Realizing the Hybrid Library.
In: D-Lib Magazine, october 1998.
\url{http://www.dlib.org/dlib/october98/10pinfield.html}

\textbf{Rusbridge}, Chris (1998), Towards the hybrid library. In: D-Lib
Magazine, july/august 1998.
\url{http://www.dlib.org/dlib/july98/rusbridge/07rusbridge.html}

\textbf{Schaefer-Rolffs}, Aike: Kurzvita auf
\url{http://www.simon-bw.de/schaefer-rolffs/index.php}

\textbf{Umlauf}, Konrad (2015) Dankesrede des Preisträgers zur
Verleihung der Karl-Preusker-Medaille am 30. Okt. 2015.
\url{https://bideutschland.de/download/file/KPM_2015_Dankesrede.pdf}

%autor
\begin{center}\rule{0.5\linewidth}{\linethickness}\end{center}

\textbf{Andreas Hartsch} (MA L.I.S.) ist Bibliothekar in der
Spezialbibliothek zur Geschichtswissenschaft des Deutschen Historischen
Instituts Paris. Interessenschwerpunkte in Kulturwissenschaften,
Medientheorie, Informationswissenschaft. Studium in Hannover, Stuttgart
und Köln. Als Bibliothekar zuvor tätig an der Universitätsbibliothek
Heidelberg.

\end{document}
