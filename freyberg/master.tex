\documentclass[a4paper,
fontsize=11pt,
%headings=small,
oneside,
numbers=noperiodatend,
parskip=half-,
bibliography=totoc,
final
]{scrartcl}

\usepackage{synttree}
\usepackage{graphicx}
\setkeys{Gin}{width=.4\textwidth} %default pics size

\graphicspath{{./plots/}}
\usepackage[ngerman]{babel}
\usepackage[T1]{fontenc}
%\usepackage{amsmath}
\usepackage[utf8x]{inputenc}
\usepackage [hyphens]{url}
\usepackage{booktabs} 
\usepackage[left=2.4cm,right=2.4cm,top=2.3cm,bottom=2cm,includeheadfoot]{geometry}
\usepackage{eurosym}
\usepackage{multirow}
\usepackage[ngerman]{varioref}
\setcapindent{1em}
\renewcommand{\labelitemi}{--}
\usepackage{paralist}
\usepackage{pdfpages}
\usepackage{lscape}
\usepackage{float}
\usepackage{acronym}
\usepackage{eurosym}
\usepackage[babel]{csquotes}
\usepackage{longtable,lscape}
\usepackage{mathpazo}
\usepackage[normalem]{ulem} %emphasize weiterhin kursiv
\usepackage[flushmargin,ragged]{footmisc} % left align footnote

\usepackage{listings}

\urlstyle{same}  % don't use monospace font for urls

\usepackage[fleqn]{amsmath}

%adjust fontsize for part

\usepackage{sectsty}
\partfont{\large}

%Das BibTeX-Zeichen mit \BibTeX setzen:
\def\symbol#1{\char #1\relax}
\def\bsl{{\tt\symbol{'134}}}
\def\BibTeX{{\rm B\kern-.05em{\sc i\kern-.025em b}\kern-.08em
    T\kern-.1667em\lower.7ex\hbox{E}\kern-.125emX}}

\usepackage{fancyhdr}
\fancyhf{}
\pagestyle{fancyplain}
\fancyhead[R]{\thepage}

%meta
%meta

\fancyhead[L]{L. Freyberg \\ %author
LIBREAS. Library Ideas, 30 (2016). % journal, issue, volume.
\href{http://nbn-resolving.de/urn:nbn:de:kobv:11-100244249}{urn:nbn:de:kobv:11-100244249}} % urn
\fancyhead[R]{\thepage} %page number
\fancyfoot[L] {\textit{Creative Commons BY 3.0}} %licence
\fancyfoot[R] {\textit{ISSN: 1860-7950}}

\title{\LARGE{Die Lesbarkeit der Welt} \\ Rezension zu 'The Concept of Information in Library and Information Science. A Field in Search of Its Boundaries: 8 Short Comments Concerning Information'. In: Cybernetics and Human Knowing. Vol. 22 (2015), 1, 57-80. Kurzartikel von Luciano Floridi, Søren Brier, Torkild Thellefsen, Martin Thellefsen, Bent Sørensen, Birger Hjørland, Brenda Dervin, Ken Herold, Per Hasle und Michael Buckland.} %title %title
\author{Linda Freyberg} %author

\setcounter{page}{170}

\usepackage[colorlinks, linkcolor=black,citecolor=black, urlcolor=blue,
breaklinks= true]{hyperref}

\date{}
\begin{document}

\maketitle
\thispagestyle{fancyplain} 

%abstracts

%body
Es ist wieder an der Zeit den Begriff \enquote{Information} zu
aktualisieren beziehungsweise einen Bericht zum Status Quo zu liefern.
Information ist der zentrale Gegenstand der Informationswissenschaft und
stellt einen der wichtigsten Forschungsgegenstände der Bibliotheks- und
Informationswissenschaft dar. Erstaunlicherweise findet jedoch ein
stetiger Diskurs, der mit der kritischen Auseinandersetzung und der
damit verbundenen Aktualisierung von Konzepten in den
Geisteswissensschaften vergleichbar ist, zumindest im deutschsprachigen
Raum\footnote{Siehe dazu
  \url{http://libreas.eu/ausgabe20/texte/09kaden_kindling_pampel.htm}.}
nicht konstant statt. Im Sinne einer theoretischen Grundlagenforschung
und zur Erarbeitung einer gemeinsamen begrifflichen Matrix wäre dies
aber sicherlich wünschenswert.

Bereits im letzten Jahr erschienen in dem von Søren Brier (Siehe
\enquote{The foundation of LIS in information science and
semiotics}\footnote{In: LIBREAS, 4
  (2006):\url{http://libreas.eu/ausgabe4/001bri.htm}.} sowie
\enquote{Semiotics in Information Science. An Interview with Søren Brier
on the application of semiotic theories and the epistemological problem
of a transdisciplinary Information Science}\footnote{LIBREAS interview,
  19 (2011):\url{http://libreas.eu/ausgabe19/texte/07treude.htm}.})
herausgegebenen Journal \enquote{Cybernetics and Human Knowing} acht
lesenswerte Stellungnahmen von namhaften Philosophen beziehungsweise
Bibliotheks- und Informationswissenschaftlern zum Begriff der
Information. Unglücklicherweise ist das Journal \enquote{Cybernetics \&
Human Knowing} in Deutschland schwer zugänglich, da es sich nicht um ein
Open-Access-Journal handelt und lediglich von acht deutschen
Bibliotheken abonniert wird.\footnote{Nämlich von der Staatsbibliothek
  zu Berlin sowie den Universitätsbibliotheken in Bayreuth, Bielefeld,
  Friedrichshafen, Hamburg, Köln, Lüneburg und München, siehe
  Zeitschriftendatenbank:
  \url{http://dispatch.opac.d-nb.de/DB=1.1/SET=1/TTL=1/SHW?FRST=1\&PRS=HOL}.}
Aufgrund der schlechten Verfügbarkeit scheint es sinnvoll hier eine
ausführliche Besprechung dieser acht Kurzartikel anzubieten.

Das Journal, das sich laut Zusatz zum Hauptsachtitel thematisch mit
\enquote{second order cybernetics, autopoiesis and cyber-semiotics}
beschäftigt, existiert seit 1992/93 als Druckausgabe. Seit 1998
(Jahrgang 5, Heft 1) wird es parallel kostenpflichtig elektronisch im
Paket über den Verlag Imprint Academic in Exeter angeboten. Das Konzept
Information wird dort aufgrund der Ausrichtung, die man als
theoretischen Beitrag zu den Digital Humanities (avant la lettre)
ansehen könnte, regelmäßig behandelt. Insbesondere die phänomenologisch
und mathematisch fundierte Semiotik von Charles Sanders Peirce taucht in
diesem Zusammenhang immer wieder auf. Dabei spielt stets die Verbindung
zur Praxis, vor allem im Bereich Library- and Information Science (LIS),
eine große Rolle, die man auch bei Brier selbst, der in seinem Hauptwerk
\enquote{Cybersemiotics} die Peirceschen Zeichenkategorien unter anderem
auf die bibliothekarische Tätigkeit des Indexierens anwendet,\footnote{Siehe
  Brier, Søren: Cybersemiotics (2008), 424 ff.} beobachten kann.

Die Ausgabe 1/ 2015 der Zeitschrift fragt nun \enquote{What
underli\sout{n}es Information?} und beinhaltet unter anderem Artikel zum
Entwurf einer Philosophie der Information des Chinesen Wu Kun sowie zu
Peirce und Spencer Brown. Die acht Kurzartikel zum Informationsbegriff
in der Bibliotheks- und Informationswissenschaft wurden von den
Thellefsen-Brüdern (Torkild und Martin) sowie Bent Sørensen, die auch
selbst gemeinsam einen der Kommentare verfasst haben.

\section*{Acht
Informationsbegriffe}\label{acht-informationsbegriffe}

Den Anfang macht Luciano Floridi\footnote{Sein Ansatz ist freilich nicht
  unumstritten, vgl. u.a.
  \url{http://libreas.tumblr.com/post/101354211366/john-searle-luciano-floridi}.}
mit seinem Kommentar \enquote{Information -- Reductionist,
Anti-reductionist and Non-reductionist Approaches}. Floridi ist einer
der präsentesten und (Social Media) aktivsten zeitgenössischen
Philiosophen und stellt die Frage \enquote{Was ist Information?} auf
dieselbe Stufe wie fundamentale Fragen nach dem Sein und der Wahrheit.
Auf die nach der Information existiere keine triviale Antwort und so,
seine Hauptthese, können die Antworten in drei Ansätze unterteilt
werden: \enquote{reductionist, anti-reductionist and non-reductionist}.
Reduktionistische Zugriffe auf den Informationsbegriff verfolgen die
Hauptintention, eine einheitliche, für alle Gebiete gleichermaßen valide
Definition zu erschaffen, eine Unified Theory of Information (UTI).
Floridi sieht dieses Vorhaben als gescheitert an, was sich seines
Erachtens in der Tatsache manifestiert, dass diese UTI trotz jahrelangen
Bestrebungen bis dato nicht existiert. Im deutschsprachigen Raum haben
sich vor allem Capurro/Fleissner/ Hofkirchner mit dieser Problematik
auseinandergesetzt und kamen, teilweise aus anderen Gründen, auf
dasselbe Ergebnis.\footnote{Siehe unter anderem Capurro, Rafael;
  Fleissner, Peter; Hofkirchner, Wolfgang: Is A Unified Theory of
  Information Feasible? (1999).} Anti-reduktionistische Ansätze haben
zusammenfassend hingegen das Problem sich hauptsächlich mit der Negation
reduktionistischer Ansichten zu beschäftigen. Aus der Ablehnung der
Shannonschen mathematischen Theorie der Kommunikation als Urquelle des
Informationsbegriffs sowie der Betonung der Vieldeutigkeit des
Informationsbegriffs in den jeweiligen Kontexten, resultiert ebenso
keine Klärung des Informationsbegriffs. Der nicht-reduktionistische
Ansatz definiert den Informationsbegriff als \enquote{network of
connected concepts, linked by mutual and dynamic influences that are not
necessarily genetic or genealogical}, welche \enquote{can be centralized
in various ways or completely decentralized and perhaps
multi-centered},\footnote{Floridi (2015), 59.} wobei vor allem die de-
oder multizentralisierten Ansätze, sich gegen eine UTI beziehungsweise
die Möglichkeit einer UTI positionieren. Je nach Kontext oder
Fragestellung kann Information als \enquote{interpretation, power,
narrative, message or medium, conversation, construction, a
commodity}\footnote{Ebenda.} definiert werden. Information ist in diesem
Verständnis nicht so sehr auf das Objekt seiner Repräsentation bezogen,
sondern vielmehr als Medium oder prozesshaftes Element anzusehen. Die
zentralisierten Ansätze muten noch ätherischer an. Bei ihnen liegt der
Bezug zur Metaphysik nahe.

Søren Briers Beitrag \enquote{How to Define an Information Concept for a
Universal Theory of Information?} beginnt mit dem Bezug auf objektive
Informationsbegriffe wie die Claude Shannons und Norbert Wieners. Diesen
mangele es an einem intersubjektiven Bezug auf Kommunikation. Sie
klammern also soziale und kognitive Kontexte aus. In gewohnter Manier
dekliniert Brier alle von ihm vielzierten Bezugstheoretiker aus dem
(bio)semiotischen und systemtheoretischen Umfeld durch. Briers Idee ist
es, \enquote{to develop a theory, which can encompass the living,
experiencing body and its consciousness's integration with
communicational networks such as natural and artificial languages in
humans}.\footnote{Thellefsens/Sørensen (2015), 61.} Information solle
zunächst, angelehnt an Floridi, \enquote{meaningfulness and
truthfulness} besitzen, einen Neuigkeitswert (Gregory Bateson)
aufweisen, der sich prozessual aus dem jeweiligen Kontext ergibt und
sich nicht nur auf menschliche Kommunikation beziehen, sondern auf alle
lebenden Systeme. Bezugnehmend auf die Peirce'sche Semiotik sieht Brier
das Konzept Information als zeichenbasiert an und erweitert es um einen
biosemiotischen Ansatz, welcher besagt, \enquote{that signs are real
relational processes manifesting as tokens connecting all living beings
with each other and with the environment}. Peirces universelles
Zeichenverständnis, seine Universalkategorien sowie Konzepte wie das des
Kontinuums machen seinen Ansatz auch in diesem Kontext anwendbar.
Information ist in den Zeichenprozess (semiosis) zu integrieren
\enquote{as well as matter/energy if we want a universal concept of
information}, so Brier in seinem Schlussstatement.

Der Beitrag \enquote{A Semeiotic Account of Information}schließt
unmittelbar an Brier beziehungsweise an Peirce an. Torkild und Martin
Thellefsen und Bent Sørensen isolieren für ihre Argumentation zwei
Dimensionen des Informationsbegriffs: Die ontologische und
epistemologische. Diese interagieren in der Semiose natürlich
miteinander und sind somit gleichermaßen vorhanden. Die ontologische
Dimension bezieht sich auf das Zeichen selbst, wobei die
epistemologische Sichtweise sich auf die Interpretation bezieht. Das
Ziel ist es, beide Dimensionen gleichermaßen zu berücksichtigen, vor
allem, weil, so die These, im Bereich der LIS die ontologische
Auffassung des Informationsbegriffs vernachlässigt wird. Dies gelingt,
wie bei Brier, mit der Bezugnahme auf das Peircesche Œuvre, speziell auf
seine Sichtweise des \enquote{universe as a whole {[}\ldots{}{]}
{[}as{]} an argument (or type of sign) (CP 1.119)},\footnote{Ebd., 63.}
sein Kontinuitätskonzept und seine sehr positive Annahme, dass die Welt
verstehbar sei. Die Schlüsselelemente im Prozess des Verstehens seien
\enquote{information, emotion and knowledge}.\footnote{Ebd., 64.}
Information wird zunächst emotional erfasst und anschließend durch einen
kognitiven Prozess mit vorhandenem Wissen kontextualisiert, wobei sie
potentiell zu neuem Wissen führen kann. Bedeutung wird also erzeugt,
indem ein Objekt zunächst emotional als Wahrnehmen eines unspezifischen
Punktes am Horizont erfasst wird, so deren Beispiel. Im nächsten Schritt
werden spezifische Eigenschaften des Punktes erkennbar und dieser
beispielsweise als Hund identifiziert. Schlussendlich überwiegt das
\enquote{knowledge dominant level},\footnote{Ebd., 66.} bei dem der
kognitive Prozess des Erkennens und der Informationsverarbeitung
abgeschlossen ist und der Hund als Nachbarhund Jake erkannt wird. Der
hier dargestellte Prozess, der Kognition, Emotionen einbezieht, könne
nun auch für den Bereich LIS fruchtbar sein, da er sowohl die
epistemologische als auch die ontologische Ebene des
Informationsbegriffes berücksichtigt.

Birger Hjørland setzt am Dokumentbegriff an, da die LIS nach seiner
Lesart aus der Dokumentation entstand. Der Titel \enquote{The Concept of
Information -- Again and Again} impliziert eine gewisse Redundanz dieses
Themas und Hjørland beantwortet in seinem kurzen Beitrag auch die
Sinnfrage des ewigen Fragens. Einmal sei die LIS laut Jonathan Furner
(2004) \enquote{to be able to do so very well without the concept of
information}.\footnote{Hjørland (2015), 68.} Eine Konzeptdefinition
sollte daher immer anwendungsbezogen und zielgreichtet sein. Hjørland
positioniert sich dann aber doch noch als Verfechter eines
(inter)subjektiven Informationsbegriffs, der das Prozesshafte, das
Sich-Informieren und den sozialen Kontext mitdenkt. Aus dieser
Sichtweise , die unter anderem Fritz Machlup, Gregory Bateson und Per
Hasle vertreten, resultieren 6 Implikationen\footnote{Ebd., 68f.}:

\begin{enumerate}
\def\labelenumi{\arabic{enumi}.}
\item
  \enquote{Anything can be informative/information}, woraus sich das
  amüsante Paradox ableiten lässt, dass Informationsspezialisten daher
  Spezialisten für Alles seien, jedoch ein Spezialist für alles
  letztendlich kein Spezialist mehr ist und somit
  Informationsspezialisten Spezialisten für Nichts seien.
\item
  \enquote{{[}I{]}nformation is a relational concept}, welches den Zweck
  und die Perspektive für den Rezipienten berücksichten müsse.
\item
  \enquote{The specification of information {[}\ldots{}{]} must be
  relative to the context and purpose} des Informationssystems.
\item
  \enquote{{[}P{]}otential possibilities in documents (and relevance of
  documents)} im Kontext der jeweiligen Fragestellung innerhalb einer
  Wissenschaft sind relevanter als der reine Dokumentinhalt.
\item
  \enquote{Information science is a meta-field}, welches andere Diskurse
  verfolgen und berücksichtigen solle.
\item
  Aus 5 folgt die Fähigkeit zu \enquote{information criticism and
  knowledge criticism}, die Informationswissenschaftler gerade bei
  innerhalb verschiedener Disziplinen umstrittenen Fragestellungen und
  in Ermangelung einer neutralen Instanz besitzen und pflegen sollen.
\end{enumerate}

Es folgt mit Brenda Dervin die einzige Frau dieser Runde, die in
\enquote{Information as Verb: An Information Concept} ein
interpretatives Paradox postuliert. Bei der Definition von Konzepten
träte das unlösbare Problem auf, dass man sich Sprache und Symbolen
bediene, die stets von dem eigentlichen Objekt entfernt sind, was zu
einem \enquote{ever-present interpretive paradox}\footnote{Dervin
  (2015), 70.} führt. Diese interpretativen Differenzen können sich
neben der Sprache-Objekt-Differenz auf eine \enquote{time-space-context
discrepancy}\footnote{Ebenda.} oder \enquote{phenomenological,
psychological, cultural and/or experiential differences}\footnote{Ebd.,
  71.} beziehen. Dervin Lösung ist die Ververbung des
Informationsbegriffes, also die Ververbung des Informationsbegriffes in
ein \enquote{informationing},\footnote{Ebenda.} welches den kulturellen,
sozialen und kognitiven Kontext mit einbezieht und durch seinen Verweis
auf die Tätigkeit seine Relativität betont. Der Enstehungskontext, die
Autorinnen-Person sowie der Interpretationskontext schieben sich vor
Objektivierungs- und Vereinheitlichungsbestrebungen. Den Begriff
Information ersetzt sie von nun an mit der Plural-Verb-Konstruktion
\enquote{informationings}, die auch auf andere Konzepte wie Kultur,
Geschichte anwendbar sind (\enquote{culturings, historyings}\footnote{Ebenda.}).
Diese Aktivierung des Informationsbegriffs hätte, so ihre These, vor
allem eine Disziplinierung der Kommunikation zur Folge.

Ken Herold bezieht sich in seiner Begriffsdefinition auf das Verhältnis
von Wahrnehmung und Intuition. In seinem Kommentar \enquote{Intuiting
Information} geht er von Elijah Chudnoffs Thesen zum Verhältnis dieser
Konzepte aus. Alan Turings Schriften zur Nützlichkeit von Intuition im
Bereich Computertechnologie bilden den Ausgangspunkt, da diese auch auf
den Informationsbegriff anzuwenden seien. In einem sehr
naturwissenschaftlichen Stil reiht er nummerierte Kurzthesen zu
verschiedenen, mit Buchstaben gekennzeichneten Themenblöcken aneinander.
Auf der dritten Seite seines Kommentar folgen schließlich Thesen zum
Informationsbegriff, die er aus den oben genannten Konzepten ableitet
und mit Descartes Zeitbegriff sowie mit Floridis Informationsbegriff
kontextualisiert. Demnach besäße Information einen repräsentativen
Charakter (\enquote{(I2) Information experiences possess (replacement)
presentational phenomenology}\footnote{Herold (2015), 73.}) und,
angelehnt an Floridi, eine semantische Ebene.

Per Hasle klärt den Informationsbegriff in seinem Beitrag
\enquote{Information as Representation or as Rhetoric} weitestgehend mit
Wittgenstein. Die Auffassung, dass Sprache die Realität repräsentiert,
findet sich im Tractatus und spiegelt sich in der Praxis des Indexierens
wieder, bei der ebenso angenommen wird, \enquote{that information is
some kind of object and that the representation system is neutral, at
least ideally}\footnote{Hasle (2015), 75.}, was bei der Organisation von
Wissen, vor allem der Auffindbarkeit und Verfügbarmachung von Wissen
eine sehr zielführende Sichtweise ist. Der späte Wittgenstein wendet nun
ein, dass Sprache jedoch weit davon entfernt ist, neutral zu sein. Hasle
weist diese Nichtneutralität der Sprache überzeugend am Beispiel der
Rhetorik nach, in der Sprache immer etwas will, also von Motivation des
Sprechenden geleitet ist. Für den Bibliothekskontext bedeutet das nun,
dass die Verbreitung von Information und Wissen \enquote{not a
transmission, but a dialogue}\footnote{Hasle (2015), 76.} ist.

Michael Buckland plädiert in seinem Beitrag \enquote{Information
Suspect} für ein Mißtrauen oder, milder ausgedrückt, eine gesteigerte
Aufmerksamkeit gegenüber der Verwendung des Informationsbegriffs, da
einerseits die Beziehung von Objekt und Sprache, aber auch sprachliche
Unklarheiten generell zu berücksichtigen sind. Die Bezeichnungen eines
Phänomens sollen sich, so Buckland, nur auf ihren jeweiligen Kontext
beziehen. Diese Vorgehensweise, die Eigenschaften eines Phänomens zu
betrachten und zu beschreiben und daraufhin den passenden Term zu
wählen, gelten gleichermaßen für den Begriff Information. In einigen
Kontexten sei es angemessener \enquote{{[}to{]} use some other more
precise word or phrase (such as data, document, or knowledge
imparted).}\footnote{Buckland (2015), 77.} Buckland lehnt zudem die
Sichtweise, Information habe wahr zu sein, ab und sieht Ansätze, die die
gesamte physische Welt als Information ansehen als nicht zielführend an.

\section*{Zusammenfassung}\label{zusammenfassung}

In Bezug auf die Definition des Informationsbegriffs changieren die
Positionen der Autoren und der Autorin zwischen

\begin{itemize}
\item
  dem Willen nach einem (einheitlichen) Konzept,
\item
  der Notwendigkeit eines anwendbaren Konzeptes
\item
  der Infragestellung der Notwendigkeit, der Anerkennung der
  Unmöglichkeit oder der Ablehnung eines (einheitlichen)
  Informationsbegriffs
\end{itemize}

Die Herausgeber dieser Sammlung unterteilen die Kommentare in
\enquote{systems-oriented}, \enquote{use-oriented} und
\enquote{domain-oriented perspective}, die alle parallel sowie ebenso
als Mischformen im Bereich der LIS koexistieren. Die Überschrift
\enquote{A Field in Search of Its Boundaries} ist sehr treffend, da in
jedem Beitrag die Grenzen des Informationsbegriffes ausgelotet und
abgesteckt werden und dieser auf diese Art greifbarer und somit
anwendbarer wird.

Der Informationsbegriff wird bei fast allen Autoren auf das Phänomen der
Sprache beziehungsweise die Zeichenebene bezogen und zu seiner Anwendung
in Beziehung gesetzt. Dieser Bezug zum (geschriebenen) Wort liegt im
Bereich LIS nahe. Die Objekte sind oftmals Bücher oder Textdokumente und
die LIS beschäftigt sich mit der meistens sprach- beziehungsweise
schriftbasierten Verfügbarmachung von Wissen. Als Unterscheidungsmerkmal
zwischen den einzelnen Kommentaren kann der Grad an Pragmatismus
fungieren, also als wie kontextabhängig das Konzept
\enquote{Information} angesehen wird und ob ein Objektivitätsanspruch
des jeweilige Informationsbegriffes vorliegt. Die ontologische
Informationsauffassung, bei der Informationen objektiv die Realität
repräsentieren, wird jeweils um eine soziale, kognitive und
kommunikative Dimension erweitert.

Die Repräsentation funktioniert nur durch Relation, also durch
Kontextualisierung und diese ist prozessual, also ein aktiver Vorgang.
Die Prozesshaftigkeit und Relationalität kann durch Peircesche
Endlos-Semiose oder durch die Ververblichung der Information
(\enquote{informationings}) verdeutlicht werden. Die
Kommunikationssituation kann in diesem Sinne über menschliche
Kommunikation hinaus gedacht werden. So wird die ganze Welt als Argument
lesbar.

\section*{Referenzen}\label{referenzen}

Brier, Søren (2008): Cybersemiotics. Why Information Is Not Enough!
Toronto Studies in Semiotics and Communication. Toronto : Univ. of
Toronto Press.

Brier; Søren (2006): The foundation of LIS in information science and
semiotics. In: LIBREAS. Library Ideas, 4:
\url{http://libreas.eu/ausgabe4/001bri.htm}.

Capurro, Rafael; Fleissner, Peter; Hofkirchner, Wolfgang (1999): Is A
Unified Theory of Information Feasible? In: Hofkirchner, Wolfgang
(Hrsg.): The Quest for a Unified Theory of Information. Proceedings of
the Second Conference on the Foundations of Information Science.
Amsterdam etc. : Gordon\&Breach, 9-30.

Kaden, Ben; Kindling, Maxi; Pampel, Heinz (2012) : Stand der
Informationswissenschaft 2011. In: LIBREAS. Library Ideas, 20:
\url{http://libreas.eu/ausgabe20/texte/09kaden_kindling_pampel.htm}.

LIBREAS (2014): Luciano Floridis \enquote{4th Revolution} hat ein
Bewusstseinsproblem. Meint John Searle.
\url{http://libreas.tumblr.com/post/101354211366/john-searle-luciano-floridi}
Zu: John R. Searle (2014) What Your Computer Can't Know. In: New York
Review of Books. October 9-22. 2014 Vol. LXI, Number 15.
\url{http://www.nybooks.com/articles/2014/10/09/what-your-computer-cant-know/}

Peirce, Charles S. (1903): Lowell Lectures on Logic, CP 1.119.

Treude, Linda (2011): LIBREAS interview: Semiotics in Information
Science. An Interview with Søren Brier on the application of semiotic
theories and the epistemological problem of a transdisciplinary
Information Science``. \emph{LIBREAS. Library Ideas}, 19:
\url{http://libreas.eu/ausgabe19/texte/07treude.htm}.

%autor
\begin{center}\rule{0.5\linewidth}{\linethickness}\end{center}

\textbf{Linda Freyberg} (geb. Treude) studierte Bibliothekswissenschaft
und Kunstgeschichte an der Humboldt-Universität zu Berlin. Sie ist
Dozentin am Fachbereich Informationswissenschaften der Fachhochschule
Potsdam und Stipendiatin ebendort im Rahmen des Professorinnenprogrammes
am Institut für Urbane Zukunft. Sie promoviert zur Zeit zum Thema
``Iconicity in Information'' an der Leuphana Universität Lüneburg am
Promotionskolleg Wissenskulturen / Digitale Medien und ist Redakteurin
der LIBREAS.Library Ideas.

\end{document}
