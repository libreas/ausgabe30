\documentclass[a4paper,
fontsize=11pt,
%headings=small,
oneside,
numbers=noperiodatend,
parskip=half-,
bibliography=totoc,
final
]{scrartcl}

\usepackage{synttree}
\usepackage{graphicx}
\setkeys{Gin}{width=.4\textwidth} %default pics size

\graphicspath{{./plots/}}
\usepackage[ngerman]{babel}
\usepackage[T1]{fontenc}
%\usepackage{amsmath}
\usepackage[utf8x]{inputenc}
\usepackage [hyphens]{url}
\usepackage{booktabs} 
\usepackage[left=2.4cm,right=2.4cm,top=2.3cm,bottom=2cm,includeheadfoot]{geometry}
\usepackage{eurosym}
\usepackage{multirow}
\usepackage[ngerman]{varioref}
\setcapindent{1em}
\renewcommand{\labelitemi}{--}
\usepackage{paralist}
\usepackage{pdfpages}
\usepackage{lscape}
\usepackage{float}
\usepackage{acronym}
\usepackage{eurosym}
\usepackage[babel]{csquotes}
\usepackage{longtable,lscape}
\usepackage{mathpazo}
\usepackage[normalem]{ulem} %emphasize weiterhin kursiv
\usepackage[flushmargin,ragged]{footmisc} % left align footnote

\usepackage{listings}

\urlstyle{same}  % don't use monospace font for urls

\usepackage[fleqn]{amsmath}

%adjust fontsize for part

\usepackage{sectsty}
\partfont{\large}

%Das BibTeX-Zeichen mit \BibTeX setzen:
\def\symbol#1{\char #1\relax}
\def\bsl{{\tt\symbol{'134}}}
\def\BibTeX{{\rm B\kern-.05em{\sc i\kern-.025em b}\kern-.08em
    T\kern-.1667em\lower.7ex\hbox{E}\kern-.125emX}}

\usepackage{fancyhdr}
\fancyhf{}
\pagestyle{fancyplain}
\fancyhead[R]{\thepage}

%meta
%meta

\fancyhead[L]{F. Tullio \\ %author
LIBREAS. Library Ideas, 30 (2016). % journal, issue, volume.
\href{http://nbn-resolving.de/urn:nbn:de:kobv:11-100244213}{urn:nbn:de:kobv:11-100244213}} % urn
\fancyhead[R]{\thepage} %page number
\fancyfoot[L] {\textit{Creative Commons BY 3.0}} %licence
\fancyfoot[R] {\textit{ISSN: 1860-7950}}

\title{\LARGE{Zur Legitimation Öffentlicher Bibliotheken}} %title %title
\author{Fabio Tullio} %author

\setcounter{page}{147}

\usepackage[colorlinks, linkcolor=black,citecolor=black, urlcolor=blue,
breaklinks= true]{hyperref}

\date{}
\begin{document}

\maketitle
\thispagestyle{fancyplain} 

%abstracts

%body
\section*{Einleitung}\label{einleitung}

Der vorliegende Artikel spiegelt Ergebnisse meiner im Oktober 2014 als
Abschlussarbeit an der Universität Duisburg-Essen eingereichten
Exploration der Legitimation Öffentlicher Bibliotheken aus einer
neoinstitutionalistischen Perspektive der Organisationssoziologie.

Bibliotheken stehen eher selten im Mittelpunkt soziologischer
Betrachtungen und sind noch seltener in den soziologischen
Nachschlagewerken des 21. Jahrhunderts zu finden. Der soziologischen
Forschung liegt an dieser Stelle eine Erkenntnislücke vor, auf die mit
der Arbeit in gegebenem Rahmen eingegangen wurde. Mit der Frage nach der
Legitimation ist in dieser Arbeit der Wandel der Erklärung und
Rechtfertigung Öffentlicher Bibliotheken in Deutschland bearbeitet
worden. Unter Rekurs auf Berger und Luckmann (2009) lässt sich die
Legitimation als eine sekundäre Objektivation von Sinn, die einer
Institution zugeschrieben wird, verstehen. Dies geschieht durch
Legitimatoren, die einer Institution Sinnhaftigkeit im Kontext einer
symbolischen Sinnwelt zuschreiben.\footnote{Vgl. hierzu: Berger und
  Luckmann, 2009: Die gesellschaftliche Konstruktion der Wirklichkeit.}
In Bezug auf die als Öffentliche Bibliotheken verstandenen Einrichtungen
wurden hierfür die Erklärungen und Rechtfertigungen der
Bibliothekswissenschaft in Betracht gezogen. Im vorliegenden Artikel
wird der Steuerung sowie der Definition der Bibliothek besondere
Aufmerksamkeit geschenkt. Quellen hierfür sind Lehrbücher, Monographien
und Nachschlagewerke der Bibliothekswissenschaft.

Das Buch \emph{Studien zur Soziologie der Bibliotheken} von Peter
Karstedt aus dem Jahre 1965, welches ich auf meiner Suche nach einer
Bibliothekssoziologie fand, wurde zum intellektuellen Ausgangspunkt
meiner Arbeit. Karstedt merkte dort eine Fremdheit zwischen Soziologie
und Bibliothekswissenschaft an, die auch heute noch weiterbesteht. Er
schrieb hierzu:

\begin{quote}
Dass inzwischen in zweiter Auflage erschienene Handbuch der
Bibliothekswissenschaft kennt keine Bibliothekssoziologie. Die neuen
Nachschlagewerke der Soziologie kennen nicht das Schlagwort Bibliothek.
Um so erstaunlicher, daß ein Buch über Bibliothekssoziologie in neuer
Auflage herauskommt, nachdem es seit mehr als zwei Jahren vergriffen war
(Karstedt, 1964, Vorwort zur zweiten Auflage).
\end{quote}

Die Soziologie hat bekannterweise viele Bindestrichsoziologien
hervorgebracht, doch lässt sich eine Bibliothekssoziologie weder in den
Standardlexika, den Handbüchern noch in den Nachschlagewerken zu
speziellen Soziologien auffinden. Wer eine spezielle Soziologie der
Bibliotheken sucht, der wird heute am ehesten in den Nachschlagewerken
der Bibliothekswissenschaft fündig. Diese rekurriert zwar nicht selten
auf eine Bibliothekssoziologie, die sogar von diversen Autoren definiert
wird, doch existiert eine Bibliothekssoziologie in dieser Form im
Kontext soziologischen Denkens bis heute nicht. Es lässt sich
feststellen, dass trotz diverser nennenswerter Beiträge wie denen von
Karstedt (1965) und Heidtmann (1974) kaum Literatur existiert, die sich
explizit mit der Bibliothek als Organisation respektive Institution aus
Soziologischer Perspektive befasst. Dass Bibliotheken seit 2008 nicht
mehr im Datenreport des Bundesamt für Statistik aufgeführt werden, sei
hier nur angemerkt. Der vorliegende Artikel kann letztlich nur
deskriptiven Charakters sein und kann in dem Sinne keine Analyse der
Institution Bibliothek oder des Organisationsfeldes der Öffentlichen
Bibliotheken leisten.

\section*{Legitimation}\label{legitimation}

Als sinnstiftende Gebilde haben Institutionen zwar eine Neigung zur
Dauerhaftigkeit, doch ist der Zustand der Institutionalisierung kein
unwiderruflicher Prozess. Berger und Luckmann schrieben hierzu:

\begin{quote}
Institutionen sind dauerhaft, solange sie nicht problematisch
werden.\footnote{Berger und Luckmann, 2009, 126.}
\end{quote}

Wenn Institutionen problematisch werden, spielt die Legitimation eine
entscheidende Rolle für das Überleben der Institution:

\begin{quote}
Legitimation sagt dem Einzelnen nicht nur, warum er eine Handlung
ausführen \emph{soll} und die andere nicht ausführen darf. Sie sagt ihm
auch, warum die Dinge sind, \emph{was} sie sind.\footnote{Berger und
  Luckmann, 2009, 100.}
\end{quote}

Solange eine Institution ein Faktum ist, welches keiner weiteren
subjektiven oder biographischen Unterstützung bedarf, ist sie für alle
Betroffenen Gewissheit. Ihre Anwesenheit bedarf in diesem Falle keiner
Erklärung oder Rechtfertigung respektive Legitimation. Berger und
Luckmann schrieben hierzu:

\begin{quote}
Das Problem der Legitimation entsteht unweigerlich erst dann, wenn die
Vergegenständlichung einer (nun bereits historischen) institutionalen
Ordnung einer neuer Generation vermittelt werden muß.\footnote{Berger
  und Luckmann, 2009, 99f.}
\end{quote}

Die Legitimierung stellt im Sinne Berger und Luckmanns die letzte Stufe
und insbesondere die Vollendung der Institutionalisierung dar. Sie
bezeichnet einen Vorgang, der die sekundäre Objektivation von Sinn
beschreibt. In diesem wird einer Institution, die bereits zu einem
historischen Faktum geworden ist, durch institutionalisierte
Legitimatoren eine neue Sinnhaftigkeit im Kontext institutionaler
Ordnungen zugeschrieben. Diese dient dazu, Bedeutungen, die
ungleichartigen Institutionen schon anhaften, zu Sinnhaftigkeit zu
integrieren.\footnote{Berger und Luckmann, 2009, 99.} Die Integration
dieser oder jener Art einer Institution ist letztlich in Relation zu den
Motiven der Legitimatoren zu betrachten und zu verstehen, während es die
Funktion der Legitimierung ist:

\begin{quote}
primäre Objektivationen, die bereits institutionalisiert sind, objektiv
zugänglich und subjektiv ersichtlich zu machen.\footnote{Berger und
  Luckmann, 2009, 99.}
\end{quote}

Der Grund für die Legitimation, so Berger und Luckmann, ist gelegt,
sobald ein System sprachlicher Objektivationen menschlicher Erfahrung
weitergegeben wird. Die Weitergabe eines Vokabulars legitimiert in
diesem Verständnis aus sich heraus:

\begin{quote}
Die fundamentalen, legitimierenden Erklärungen sind sozusagen in das
Vokabular eingebaut.\footnote{Berger und Luckmann, 2009, 100f.}
\end{quote}

\section*{Bibliothek und Stadt}\label{bibliothek-und-stadt}

Karstedt zeichnete den Weg der uneingeschränkt Öffentlichen Bibliotheken
in die (europäische) Öffentlichkeit unter Bezug auf die
gesellschaftlichen Verhältnisse, Wirtschaftsformen und die sozialen
Strukturen ihrer jeweiligen Umwelten nach, die sich höchst
unterschiedlich ausgedrückt haben.\footnote{Vgl.: Karstedt, 1965: Zur
  historischen Soziologie der Bibliotheken.} Den genauen Punkt des
Ursprunges zu benennen, lehnt dieser aufgrund der Unmöglichkeit einer
Bestimmung zwar ab, doch beschreibt er einen soziologischen Ort des
vermutlichen Ursprungs:

\begin{quote}
Er konnte nur dort liegen, wo politische Macht zuerst als nicht mehr an
eine Person gebunden in Erscheinung trat.\footnote{Vgl.: Karstedt, 1965,
  14ff.}
\end{quote}

Das war nach Karstedt in den Städten mit Ratsverfassungen im 12.
Jahrhundert der Fall.\footnote{Vgl. hierzu auch den Beitrag von Hans
  Dörries, in Schöller, 1969.} Karstedt schreibt hierzu:

\begin{quote}
Erst jetzt, da das Sozialgebilde Stadt, in seiner verfassungsmäßigen
Struktur ausgewachsen und erstarrt als Stadtrepublik, das volle soziale
Selbstbewußtsein erreicht hat, ist auch die öffentliche Hand vorhanden,
welche die öffentliche Bibliothek der ganzen Stadt, die Stadtbibliothek
entstehen lassen kann.\footnote{Karstedt, 1965, 18.}
\end{quote}

Karstedt wird hierzu auch noch etwas konkreter, wenn er zum Ende seiner
historischen Betrachtung des Weges der Öffentlichen Bibliothek in die
Öffentlichkeit zusammenfassend anmerkt:

\begin{quote}
Hier, also in erster Linie in den alten freien Reichsstädten, ist die
Demokratie, ist der moderne Staatsgedanke und mit ihm die öffentliche
Bibliothek zuerst Wirklichkeit geworden. Und hier hat er sich zäh
gehalten auch in der Zeit, als es im Staate nur eine Fürstensouveränität
und private Fürstenbibliotheken gab und die politische Bedeutung der
Städte völlig geschwunden war. Sobald den Städten die Möglichkeit, ihre
eigenen Angelegenheiten in demokratischer Selbstverwaltung zu besorgen,
erneut gegeben war, kommt es auch wieder zu neuen Gründungen
öffentlicher Bibliotheken.\footnote{Karstedt, 1965, 39.}
\end{quote}

Regional zeitversetzt ist im Europa des 19. Jahrhunderts die Forderung
auf den Mitgebrauch der Bildungsmittel verschiedentlich eingelöst
worden, was sich insbesondere in den Geschichten der
Nationalbibliotheken oder der Anwesenheit eines Bibliotheksgesetzes
widerspiegelt.

\begin{quote}
Das Volk erhob nicht nur seine Ansprüche auf die Mitwirkung bei den
Staatsgeschäften, sondern auch auf Mitgebrauch der in staatlichem Besitz
befindlichen Bildungsmittel.\footnote{Karstedt, 1965, 23.}
\end{quote}

Einige der Nationalbibliotheken, die sich in Europa aus Hofbibliotheken
heraus entwickelten, wurden im 19. Jahrhundert als Staatsinstitute
gegründet und standen in dieser Form erstmals einem öffentlichen
Publikum zum Gebrauch zur Verfügung. Die Frage, wem die Hofbibliotheken
und damit die in ihnen befindlichen Bildungsmittel gehörten, führte 1874
in Wien so weit, dass diese Frage in der konkreten Form, ob die
Hofbibliothek in Wien ein Hof- oder ein Staatsinstitut sei, in aller
Öffentlichkeit folgendermaßen beantwortet wurde:

\begin{quote}
die Bibliothek sei nicht Privateigentum des Hofes, ihre Benutzung durch
das Publikum sei ein Recht und keine Gnade.\footnote{Karstedt, 1965, 23.}
\end{quote}

Der 24. Oktober 1828 gilt in Deutschland als der Tag der Bibliothek. An
diesem Tag richtete Karl Benjamin Preusker in Großhain eine
Schulbibliothek für Lehrer und Schüler ein, die 1832 zur ersten
unentgeltlichen Bürgerbibliothek erweitert wurde. Bis zur
preußisch-deutschen Reform des Bibliothekswesens lässt sich beobachten,
dass die großen uneingeschränkt Öffentlichen Bibliotheken überwiegend im
Nebenamt von Gelehrten geleitet und/oder betrieben wurden und ihre
Bezeichnung als Institut respektive Staatsinstitut weitestgehend
verbreitet war. Mit der Institutionalisierung des Berufsbibliothekars in
den 1890er Jahren wurde die Bibliothek und mit ihr die Öffentliche
Bibliothek von einem Institut jedoch vorerst zu einem funktionalen Teil
der Verwaltung und somit zu einer Behörde, für die allgemein bis ins 20
Jahrhundert hinein auch der Begriff der Volksbibliothek verwendet
wurde.\footnote{Vgl.: Plassmann et al., 2011, 72.} Mit dem
Berufsbibliothekar, der durch die preußisch-deutschen Reformen die Form
eines mechanisch tätigen Verwaltungsbeamten annahm, entwickelten sich
nach 1900 innerhalb der bibliothekarischen Fachwelt jedoch erhebliche
Spannungen, die auch als Richtungsstreit bekannt wurden und bis weit in
das 20. Jahrhundert strahlten.\footnote{Vgl.: Leyh, 1968 sowie Jochum,
  1999, XI Die öffentlichen Bibliotheken, insbesondere 154ff. Dieser
  Richtungsstreit, dessen Anfängen die Bücherhallenbewegung wichtige
  Impulse gab, kann unter Rekurs auf Jochum auch als die Weiterführung
  einer politischen Debatte verstanden werden.} Die Frage, was
Bibliotheken im Kern sind, ob Behörden oder Institute, beschäftigte
weite Teile der bibliothekarischen Fachwelt bis weit ins 20. Jahrhundert
hinein, so schrieb Rolf Kluth 1970:

\begin{quote}
Es ist nicht einfach, Klarheit in den organisatorischen Rahmen des
Bibliotheksbetriebes zu bringen. Das liegt z.T. an der allgemeinen
Uneinheitlichkeit, z.T. aber auch an der Eigenart der Bibliothek, die
ein Institut und keine Verwaltung ist.\footnote{Kluth, 1970, 125.}
\end{quote}

Eine Vielzahl uneingeschränkt Öffentlicher Bibliotheken hat sich aus
Stiftungen, Lesegesellschaften, Gründungsbewegungen und Vereinen heraus
entwickelt, die im Laufe der Zeit der Öffentlichen Hand übergeben
wurden. Dass Öffentliche Bibliotheken in Deutschland weitestgehend als
kommunalisiert bezeichnet werden können, lässt sich etwa ab der Zeit der
Weimarer Republik feststellen.\footnote{Vgl. hierzu die von Jörg Fligge
  und Alois Klotzbücher, 1997, herausgegebene Sammlung von Fallstudien
  und Überblicksreferaten: Stadt und Bibliothek. Literaturversorgung als
  kommunale Aufgabe im Kaiserreich und in der Weimarer Republik.} Die
Definition bestimmter persönlicher und/oder sachlicher Voraussetzungen,
die zur Benutzung einer Öffentlichen Bibliothek berechtigen, steht in
einem direkten Zusammenhang mit der Kultur- und Bildungspolitik bzw.
Staatsform und/oder Verfassung einer Gesellschaft.\footnote{Vgl. hierzu:
  Fritz Milkau: Die Zulassung zur Bibliotheksbenutzung, S. 362 ff, in:
  Georg Leyh: Handbuch der Bibliothekswissenschaft, Band 2., 1961.} Die
rechtliche Grundlage zur Nutzung des Informations-, Medien- und
Dienstleistungsangebotes in öffentlich zugänglichen bibliothekarischen
Einrichtungen der Bundesrepublik Deutschland leitet sich heute in erster
Linie aus den in Artikel 5. GG gegebenen Grundrechten ab.\footnote{Vgl.
  hierzu bspw.: Cobabus, 2004.} Das Verständnis der Öffentlichkeit einer
Öffentlichen Bibliothek\footnote{Historisch ist im Bereich der
  Öffentlichen Bibliotheken bzw. der in dieser Arbeit beschriebenen
  Einrichtungen zwischen der Öffentlichen und der Wissenschaftlichen
  Stadtbibliothek zu unterscheiden. Vgl. hierzu: Vodosek und Arnold,
  2008.} ist mit der Gründung der Bundesrepublik Deutschland im
weitesten Sinn an das der Public Library aus dem anglo-amerikanischen
Raum angelehnt worden, welches sich insbesondere durch einen
Freihandbestand,\footnote{Der Freihandbestand steht im Gegensatz zur
  Thekenbibliothek.} den Informationsdienst und weitestgehende Befreiung
von pädagogischer Bevormundung und geplanter Einwirkung auf die
Nutzerinnen und Nutzer auszeichnet.\footnote{Vgl. zur Geschichte der
  Bibliothek: Uwe Jochum, Kleine Bibliotheksgeschichte 1999; zum Weg der
  Bibliothek in die Öffentlichkeit: Peter Karstedt, Studien zur
  Soziologie der Bibliotheken, Kap. 1., 1965; zur Bibliothek als
  Non-Profit-Organisation: Konrad Umlauf, Leistungsmessung und
  Leistungsindikatoren für Bibliotheken im Kontext der Ziele von
  Nonprofit Organisationen, 2003; zur Frage des sozialen Auftrags der
  Bibliothek: Kaiser und Schuldt, Hat die Bibliothek einen sozialen
  Auftrag und wenn ja, welchen? - Ein Dialog; zur Bibliothek als
  Demokratische Institution: Pelaya und Sanllorenti, Der schwierige
  Auftrag der Bibliotheken, in Heinrich-Böll-Stiftung, 2010; zum
  Bibliothekswesen der Gegenwart: Plassmann et al., Bibliotheken und
  Informationsgesellschaft in Deutschland, 2011; zu Bibliotheken in der
  Antike: Fritz, Antike öffentliche Bibliotheken und ihre
  bildungspolitische sowie kulturelle Bedeutung, 2007.}

Öffentliche Bibliotheken, so kann an dieser Stelle festgehalten werden,
erfüllen heute in der Regel als Anstalten des öffentlichen Rechts
Aufgaben im Interesse der Öffentlichkeit\footnote{Umstätter, 2011,
  163ff.} wie zum Beispiel die Literaturversorgung der Bevölkerung oder
spezieller Gebrauchsöffentlichkeiten. Diese Aufgabenerfüllung wird durch
öffentliche Mittel finanziert, die wiederum von den Unterhaltsträgern,
also Städten und Gemeinden, aufzubringen sind.

\section*{Bibliotheksverwaltung/Bibliotheksmanagement}\label{bibliotheksverwaltungbibliotheksmanagement}

Bis in die 70er Jahre hinein wurden bibliothekarische Einrichtungen in
der Bundesrepublik Deutschland weitestgehend als Behörden oder auch
Institute\footnote{Vgl. hierzu beispielsweise den Beitrag von Hansjörg
  Süberkrüb in: Rakowski, 1968.} verstanden und beschrieben. In diesen
wurde das Handeln, sofern sie in öffentlicher Trägerschaft geführt
wurden, durch Grundsätze des Verwaltungshandelns wie Haushaltsrecht,
öffentliches Dienstrecht und Verwaltungsvorschriften
organisiert.\footnote{Vgl.: Plassmann et al., 2011, 263.} Auch als Kern
der Bibliothekswissenschaft bezeichnet, stand die Lehre der
Bibliotheksverwaltung bis weit in die 90er Jahre hinein, allgemein von
der Fachwelt anerkannt, im Mittelpunkt bibliothekarischer Ausbildung.
Gekennzeichnet wurde die Bibliotheksverwaltung in Anlehnung an die
klassische Definition von Anselm Graesel durch die Lehre vom Gebäude und
dem Bibliothekspersonal, der Erwerbung, der Katalogisierung und der
Benutzung.\footnote{Vgl.: Kunze, 1958; Leyh, 1961; Fuchs, 1968;
  Umstätter und Ewert, 1997.} Diese Themen wurden in den Hand- und
Lehrbüchern zur Bibliotheksverwaltungslehre in der Regel von
Bibliothekaren und ihren Mitarbeitern definiert und beständig durch die
Weiterführung diverser Hand- und Lehrbücher erweitert. Die
Bibliotheksverwaltung hat durch die Integration betriebswirtschaftlicher
Denkansätze seit den 70er Jahren jedoch einen Wandlungsprozess erfahren,
dem in Folgenden nachgegangen wird.

In den 60er Jahren erscheinen die ersten Artikel zu
betriebswirtschaftlichen Problembereichen in der Bibliotheksverwaltung.
In den 70er Jahren erscheinen neben weiteren Artikeln die ersten
Abhandlungen zu ganzheitlichen betriebswirtschaftlichen Ansätzen des
Bibliotheksbetriebs. Beispielhaft hierfür ist eine Abhandlung von Robert
Funk (1975). Mit seinem Beitrag wollte Funk sich sowohl an Bibliothekare
als auch Verwaltungspraktiker, die ein Interesse an
Wirtschaftlichkeitsproblemen und insbesondere an Kostenproblemen haben,
wenden.\footnote{Funk, 1975, Vorwort.} Funk hielt Kostenrechnungen in
der gesamten Universität beziehungsweise in einem abgestimmten System
für wünschenswert, um größere Genauigkeit zu erzielen. Hierzu schrieb
Funk 1975:

\begin{quote}
Außer den praktischen Schwierigkeiten bei der Anwendung der technischen
Verfahren der Kostenerfassung, -verteilung etc., die im Zeitablauf zu
bewältigen sind, um später zu einer exakten Bibliothekskostenabrechnung
zu führen, sind der Kostenrechnung dennoch gewissen Grenzen durch
Rechtsverordnungen, insbesondere auch verwaltungsrechtliche Gesetze und
Vorschriften gegeben. Aber das größte Hemmnis bei der Einführung einer
Kostenrechnung ist im psychologischen Verhalten des Personals gegenüber
den betriebswirtschaftlichen Methoden zu suchen.\footnote{Funk, 1975,
  149.}
\end{quote}

In der Zeit vor 1975 war die Auffassung dessen, was unter
Bibliotheksverwaltung zu verstehen ist in beiden Teilen Deutschlands
weitestgehend identisch, so schrieb Horst Kunze in seinem in Leipzig
erschienen Lehrbuch 1958:

\begin{quote}
Die Bibliotheksverwaltungslehre umfaßt einerseits die Grundlagen
allgemeiner Verwaltungskunde, andererseits die spezifisch
bibliothekarischen Kenntnisse von der Einrichtung und Organisation der
Bibliotheken. {[}\ldots{}{]}. In Übereinstimmung mit A. Graesel hat sich
die bisher allgemein gebräuchliche Kurzdefinition der
Bibliotheksverwaltungslehre ergeben, die geradezu als die klassische
anzusehen ist: \emph{Die Lehre von der Erwerbung, der Katalogisierung
und der Benutzung der Bücher.} {[}\ldots{}{]}. Der eigentliche
Schwerpunkt der bibliothekarischen Arbeit ist {[}..{]} heute in den
zweckmäßigsten Methoden und Formen der Literaturauswertung
(Bestandserschließung) zu sehen.\footnote{Kunze, 1958, 61.}
\end{quote}

Hermann Fuchs schrieb in seinem in Wiesbaden erschienen Lehrbuch 1968
hierzu:

\begin{quote}
Die \emph{Bibliotheksverwaltungslehre} stellt den eigentlichen Kern der
Bibliothekswissenschaft dar. Sie umfaßt die spezifisch
bibliothekarischen Kenntnisse von der Einrichtung und Organisation einer
Bibliothek. Hierzu rechnen neben der Lehre vom Gebäude und dem
Bibliothekspersonal nach der Definition von Arnim Graesel in seinem
\emph{Handbuch der Bibliothekslehre} (1902) die \emph{Lehre von der
Erwerbung, der Katalogisierung und der Benutzung der Bücher}. Die dafür
auch gebrauchten Bezeichnungen Bibliothekslehre, Bibliothekskunde,
Bibliothekonomie haben sich nicht durchzusetzen vermocht. Neuerdings
pflegt man auch die allgemeine Verwaltungs- oder Bürokunde mehr oder
weniger ausführlich dabei abzuhandeln.
\end{quote}

Im Hinblick auf die Erklärung der Bibliotheksverwaltung durch die
Bibliothekswissenschaft, fiel mir das Lehrbuch von Wilhelm Krabbe und
dessen Entwicklung auf. Als Kurzgefasstes Lehrbuch der
Bibliotheksverwaltung im Jahre 1937 erschienen, ist es erweitert und mit
Wilhelm Martin Luther 1953 in dritter Auflage herausgegeben worden. 1997
ist es in Weiterführung von Gisela Ewert und Walther Umstätter völlig
neu bearbeitet und noch als Lehrbuch der Bibliotheksverwaltung
herausgegeben worden. Seit 2011 liegt es als Lehrbuch des
Bibliotheksmanagements von Walther Umstätter vor.

Noch Ende der 90er Jahre, so Ewert und Umstätter, wollten weite Teile
der Bibliothekswissenschaft, der bibliothekarischen Tradition
verpflichtet, dem englischsprachigem Management einen akzeptablen
deutschsprachigen Begriff entgegensetzen.\footnote{Ewert und Umstätter,
  1997, 18.} Hiermit wurde versucht, nachzuweisen, dass die Verwaltung
einer Bibliothek über das Management eines öffentlichen Unternehmens,
auch unter betriebswirtschaftlichen Erfordernissen, hinausgeht und ein
eigenständiges Spezialgebiet ist. Ewert und Umstätter schrieben 1997 zur
Bibliotheksverwaltung:

\begin{quote}
Bibliotheksverwaltung umfaßt alle Handlungen (Operationen), die zur
Leitung und Organisation der bibliotheksspezifischen Arbeits-,
Informations- und Kommunikationsprozesse zielgruppenorientiert
erforderlich sind, einschließlich der Fragen des Gebäudes, seiner
Einrichtung und Ausstattung sowie der Fragen zur
Personalführung.\footnote{Vgl.: Ewert und Umstätter, 1997, 17.}
\end{quote}

Die Bezeichnung Bibliotheksverwaltung fand zwar noch Ende der 90er Jahre
Verwendung, doch spiegelt sich in dem Neologismus des
Bibliotheksmanagements unter anderem der Wandel des Verständnisses der
Steuerung des Bibliotheksbetriebs wider. Der Begriff des
Bibliotheksmanagements hebt in seiner Interpretation nach Umstätter auf
die Dimension der laufenden Anpassung an sich verändernde Anforderungen
der Umwelt ab und wird von ihm wie folgt definiert:

\begin{quote}
Bibliotheksmanagement umfasst alle Handlungen (Operationen), die zur
Leitung und Organisation der bibliotheksspezifischen Arbeits-,
Informations- und Kommunikationsprozesse zielgruppenorientiert
erforderlich sind, einschließlich der Frage des Gebäudes, seiner
Einrichtung und Ausstattung sowie der Fragen zur Personalführung und der
Bibliothekszusammenarbeit.\footnote{Umstätter, 2011, 21.}
\end{quote}

Die Bibliotheksverwaltung im erweiterten Sinne wurde im Zuge der
Neudefinition ebenfalls durch Bibliotheksmanagement ersetzt.

Bibliotheksmanagement im erweiterten Sinne umfasst alle Handlungen
(Operationen), die zur Leitung und Organisation von
bibliotheksspezifischen Arbeits-, Informations- und
Kommunikationsprozessen im weltweit strukturierten Bibliothekssystem
erforderlich sind, einschließlich der Fragen nach den Gebäuden,
Einrichtungen und ihrer Vernetzung untereinander so wie der nach den
Qualifikationen des Personals.\footnote{Umstätter, 2011, 206.}

Die Definition zum Begriff der Bibliotheksbetriebslehre im Lexikon der
Bibliotheks- und Informationswissenschaften von 2011 zeigt die
Entwicklung des Verständnisses der Steuerung durch die Verwendung des
Managementbegriffs und die in Klammern stehende Bezeichnung der
Bibliothek als Betrieb noch etwas deutlicher an:

Als Einrichtungen (=Betriebe) in überwiegend öffentlich-rechtlicher
Trägerschaft unterliegen Bibliotheken dem öffentlichen Dienst- und
Haushaltsrecht, das auch für den gesamten Bereich der öffentlichen
Verwaltung maßgeblich ist. Insofern bezieht sich die
Bibliotheksbetriebslehre primär auf das Management und die Organisation
der bibliotheksspezifischen Dienstleistungen.\footnote{Umlauf und
  Gradmann, 2011, 96.}

Seit den 1990er Jahren geht die Bibliotheksbetriebslehre, welche in den
1970er Jahren begann die Bibliotheksverwaltungslehre abzulösen, in
uneinheitlichen Managementlehren auf,\footnote{Vgl. hierzu: den Eintrag
  zur Bibliotheksverwaltungslehre von Umlauf und Gradmann, 2011.}
wodurch sich in Anbetracht der dezentralen Organisation der einzelnen
Bibliothekssysteme, welche durch die Diversität der einzelnen
Bundesländer und sozial-räumliche Disparitäten in diesen gekennzeichnet
sind, auch eine Vielfalt an Ausgestaltungen von Managementlehren in den
einzelnen Einrichtungen auffinden lässt.

Ende der 90er Jahre schrieben Gisela Ewert und Walther Umstätter zur
Situation des Bibliothekswesens:

\begin{quote}
Das Bibliothekswesen befindet sich aufgrund seines rasanten Wachstums
zweifelsohne seit längerem in einer Identitätskrise. Bedingt durch die
quantitativ und qualitativ veränderten Anforderungen der modernen
Wissensgesellschaft stößt es mit den klassischen Methoden auf räumliche
und insbesondere auf finanzielle Grenzen.\footnote{Ewert und Umstätter,
  1999, 1.}
\end{quote}

Die von Ewert und Umstätter benannte Identitätskrise des
Bibliothekswesens löste in den vergangenen Dekaden eine breit angelegte
Suche nach einem neuen Verständnis von Bibliotheken und ihrer Aufgabe
aus. Im Fokus der Diskussionen standen nach Ewert und Umstätter:
technische Lösungsmöglichkeiten, die Suche nach einem neuen Berufsbild
mit veränderten Ausbildungs- und Besoldungsstrukturen sowie veränderte
und/oder neue Finanzierungsmodelle und Kriterien, um auch im
Bibliothekswesen mit Erfolg nach Prinzipien der Wirtschaftlichkeit und
des Marketings arbeiten zu können.\footnote{Vgl.: Ewert und Umstätter,
  1999, 1.}

\section*{Zur
Bibliothekswissenschaft}\label{zur-bibliothekswissenschaft}

Unter Rekurs auf die Arbeiten von Uwe Jochum lässt sich die Geschichte
der Bibliothekswissenschaft seit der Neuzeit auch als Suche nach einer
Bestimmung dessen, was Bibliotheken sind und Bibliothekare zu leisten
haben, bezeichnen. Diese Suche bewegt sich in einem von Jochum
angedeutetem Kontinuum zwischen Bildung und Markt betreffend ihrer
Funktionen und Aufgaben. 1970 merkte Rolf Kluth zur Entwicklung der
Bibliothekswissenschaft an:

\begin{quote}
Es ist erstaunlich, daß eine wirkliche, \emph{wissenschaftliche
Standortbestimmung} der Bibliothek überhaupt erst seit kurzer Zeit
möglich ist. Alle Versuche, eine Bibliothekswissenschaft zu entwickeln,
mußten scheitern, da der wissenschaftliche Gehalt der Bibliothek nicht
zu ermitteln war. Erst die Entstehung des Begriffes der
Kommunikationswissenschaft und des Systems der
Kommunikationswissenschaften hat hier den Weg frei gemacht. Innerhalb
der \emph{Kommunikationswissenschaft} hat die Bibliothek einen
spezifischen Stellenwert, im Rahmen der Kommunikationswissenschaften ist
\emph{Bibliothekswissenschaft} möglich.\footnote{Kluth, 1970, 5.}
\end{quote}

Der Begriff Bibliothekswissenschaft tauchte zu Beginn des 19.
Jahrhunderts in Martin Schrettingers Versuch eines vollständigen
Lehrbuchs der Bibliothekswissenschaft (1808) erstmalig auf, seine
Definition bot dem Diskurs eine grundlegende Orientierung. Schrettinger
definierte Bibliothekswissenschaft wie folgt:

\begin{quote}
der auf festen Grundsätze systematisch gebaute und auf einen obersten
Grundsatz zurückgeführte Inbegriff aller zur zweckmäßigen Einrichtung
einer Bibliothek erforderlichen Lehrsätze.\footnote{Schrettinger, 1829,
  16, zitiert nach: Umstätter, 2011, 19.}
\end{quote}

Die Versuche, eine Bibliothekswissenschaft als universitäres Fach zu
etablieren, scheiterten jedoch zunächst an der Kritik der
Wissenschaftlichkeit und der eines fehlenden Kanons der
Bibliothekswissenschaft. Die heterogenen Versuche, der
Bibliothekswissenschaft einen Kern zu geben, schieden sich an
grundsätzlichen Fragen wie zum Beispiel der eigenen wissenschaftlichen
Betätigung, der Ausbildung oder fundamentalen Fragen der Orientierung
der Arbeit am Universellen oder Fachspezifischen. Die
Institutionalisierung des Berufsbibliothekars in den 1890er Jahren
während der preußisch-deutschen Reform des Bibliothekswesens machte die
Bibliothekare vorerst zu einem funktionalen Teil der Verwaltung. Die
Bestimmung der Bibliothekswissenschaft schwankte seitdem zwischen einem
Ideal von gelehrter Tätigkeit und einer Wirklichkeit, die aus
Verwaltungstätigkeit bestand. Seit den 1960er Jahren ist die Bibliothek
unter dem Einfluss der Theoreme und den Modellen der Informations- und
Kommunikationswissenschaften im weitesten Sinne zu einem Gedächtnis
(Speicher der Speicher)\footnote{Kluth, 1970, 7.} der
Informationsgesellschaft im Informationszeitalter geworden. Die
Bibliothekswissenschaft hat sich seitdem zu einer der
Informationswissenschaft nahen Betätigung entwickelt, die heute als
Bibliotheks- und Informationswissenschaft verstanden und im Rahmen der
Umstellung auf Bachelor und Master Studienstrukturen auch so gelehrt
wird. Plassmann et al. sprechen in diesem Kontext von einem Scheitern
des deutschen Sonderwegs, der Bibliotheks- und
Informationswissenschaften voneinander getrennt behandelt.

\section*{Zur Definition der
Bibliothek}\label{zur-definition-der-bibliothek}

Eine Definition der Bibliothek liegt für die Soziologie nicht vor. Eine
solche lässt sich weder in den Handbüchern, den Lexika, den
Wörterbüchern noch anderen einschlägigen soziologischen
Nachschlagewerken auffinden. Lediglich das Handwörterbuch zur
Gesellschaft Deutschlands behandelt das Thema Bibliothek, jedoch nur
unter dem Überbegriff der Kulturinstitution. Historisch ist die
Bibliothek somit als Stützpfeiler einer demokratischen Öffnung für
(Volks-) Bildung festgehalten,\footnote{Vgl.: Schäfers und Zapf, 2001,
  408.} doch in den modernen Nachschlagewerken der Soziologie nicht
aufzufinden.

In den sogenannten Alltagslexika, die im Wesentlichen das Wissen um die
Alltagswelt in universellem Zugang repräsentieren, findet sich in der
Definition der Bibliothek in der Regel eine Fokussierung auf die
Bibliothek als Büchersammlung und das Gebäude der Bibliothek, die über
die vergangenen Dekaden hinweg relativ stabil blieb. Beispielhaft lässt
sich hierzu die Definition im Lexikon der Deutschen Buch-Gemeinschaft
von 1963 zitieren:

\begin{quote}
\textbf{Bibliothek} (griech.), Büchersammlung (auch das sie
beherbergende Gebäude). Nach der Zweckbestimmung unterscheidet man
allgemeine u. Fach-B.en, Behörden- u. Anstalts-B.en, öffentliche u.
Privat-B.en usw., nach der Benutzungsart Ausleih- u. Stand-
(Präsenz-)B.en. Gesammelt werden in B.en außer Büchern, Zeitungen und
Zeitschriften auch Karten, Musikalien, Schallplatten, Tonbänder,
Handschriften usw. Die öffentlichen B.en vergrößern ihren Bestand durch
staatl. bzw. kommunale Zuschüsse, Austausch und vor allem durch die
Pflichtexemplare.\footnote{Deutsche-Buch-Gemeinschaft, 1963, Band 1,
  245.}
\end{quote}

An der Definition der Bibliothek in der Brockhaus Enzyklopädie von 1996
lässt sich diese Fokussierung noch feststellen, doch lässt sich hier
auch eine Erweiterung vorfinden, die sich auf die Aufgabe der Bibliothek
bezieht:

\begin{quote}
\textbf{Bibliothk} {[}griech., eigtl. Büchergestell{]} die, -/-en,
\textbf{Bücherei}, öffentliche oder private, planmäßig angelegte
Büchersammlung, auch das Gebäude, in dem sie untergebracht ist. Aufgabe
einer B. im modernen Sinn ist es, jede Art von Literatur, Medien und
Information zu vermitteln.\footnote{Brockhaus, 1996, Band 3, 292.}
\end{quote}

Die Funktionsorientierung der Definition der Bibliothek entwickelte
sich, nach Gisela Ewert und Walther Umstätter, durch die Aufnahme der
besonderen Zweckbestimmung der Bibliothek als zur Benutzung aufgestellte
Büchersammlung im Sinn der Definition Anselm Graesels, durch Wilhelm
Krabbes und Wilhelm Martin Luthers Definition der Bibliothek. Die im
vorhergegangenen unter Rekurs auf Uwe Jochum angemerkte Suche der
Bibliothekswissenschaft nach einer Bestimmung dessen, was Bibliotheken
sind und Bibliothekare zu leisten haben, hat eine Vielzahl heterogener
Erklärungsversuche hervorgebracht, auf die an dieser Stelle nicht
eingegangen werden kann. Eine soziologische Orientierung lässt sich
beispielsweise im Ansatz von Plassmann et al. finden, welcher die
Bibliothek im Rahmen systemtheoretischer Ansätze luhmannscher Prägung
als Informationsdienstleister in einer Informationsgesellschaft
verortet.

1999 erschien mit einem Beitrag von Ewert und Umstätter im
Bibliotheksdienst eine Definition der Bibliothek, die im
Bibliothekswesen bis heute weite Verbreitung gefunden hat und innerhalb
der bibliothekarischen Fachwelt anhaltend diskutiert wird. Diese
Definition ist nach Ewert und Umstätter als auf alle Typen und Formen
von Bibliotheken anwendbar zu verstehen. Sie ist Ausdruck des
vorhandenen Wissens um die Bibliothek und als Beitrag zur Suche nach
einem notwendigerweise neuen Verständnis der Bibliothek formuliert
worden.

\begin{quote}
Die Bibliothek ist eine Einrichtung, die unter archivarischen,
ökonomischen und synoptischen Gesichtspunkten publizierte Informationen
für die Benutzer sammelt, ordnet und verfügbar macht.\footnote{Ewert und
  Umstätter, 1999, 10.}
\end{quote}

Diese Definition erarbeiteten Ewert und Umstätter in Auseinandersetzung
mit den herkömmlichen Definitionen der Bibliothek, die nach ihrer
Auffassung relativ einheitlich an dem lokationsorientierten Aspekt der
Sammlung beziehungsweise des Aufbewahrungsortes festhielt. Im
wesentlichen bezogen Ewert und Umstätter sich bei der Definition der
Bibliothek auf drei Aspekte, die im Wissen um die Bibliothek und in den
herkömmlichen Definitionen zutage treten:

\begin{itemize}
\tightlist
\item
  den der Lokation,
\item
  den der Sammelobjekte, und
\item
  den der Ziele von Bibliotheken.
\end{itemize}

Ewert und Umstätter verwenden in ihrer Definition den Begriff der
Einrichtung, da mit diesem sowohl ortsgebundene Sachverhalte
gekennzeichnet werden (Bibliothek als Raum, Gebäude oder Gebäudekomplex)
als auch distribuierte Einheiten und sogar jene, die sich auf virtuelle
Räume beziehen. Der Terminus der publizierten Information trägt
informationstheoretischen Erkenntnissen und der zunehmenden
Digitalisierung Rechnung. Der Zusatz publizierte Information wird zur
definitorischen Präzisierung verwendet, um den Gegensatz zum Archiv, das
nach Ewert und Umstätter insbesondere nichtveröffentlichte Informationen
sammelt, hervorzuheben. In Bezug auf die Aufgaben der Bibliothek führen
Ewert und Umstätter drei Aspekte ein, die sie als zielorientierte
Qualitätskriterien verstehen:

\begin{itemize}
\tightlist
\item
  den archivarischen,
\item
  den ökonomischen, und
\item
  den synoptischen Aspekt der Bibliothek.
\end{itemize}

Der archivarische Aspekt wird von Ewert und Umstätter insbesondere vor
dem Hintergrund der Informationsflut eingeführt. Ewert und Umstätter
schrieben hierzu:

\begin{quote}
Er kann ohne jede Übertreibung, hinsichtlich des bereits existierenden
und des noch zu erwartenden Publikationsaufkommens, als eine der größten
bibliothekswissenschaftlichen Herausforderungen unserer Zeit angesehen
werden.\footnote{Ewert und Umstätter, 1999, 8f.}
\end{quote}

Mit dem ökonomischen Aspekt berücksichtigen Ewert und Umstätter:

\begin{quote}
daß Bibliotheken gemäß ihres Auftrages wirtschaftlich agieren und als
Dienstleistungseinrichtungen nach dem Prinzip umweltbezogener
Wirtschaftlichkeit handeln müssen.\footnote{Ewert und Umstätter, 1999,
  9.}
\end{quote}

Was nach Ewert und Umstätter nicht nur bedeutet, Haushalts-, Finanz- und
Wirtschaftspläne effizient aufzustellen, sondern:

\begin{quote}
es bedeutet in diesem Zusammenhang vielmehr, daß sie das wirtschaftliche
Interesse des Unterhaltsträgers zu respektieren haben. Mit anderen
Worten: Die Informationsversorgung der bibliothekarischen Zielgruppen
(die der Nutzer), muß ökonomisch optimiert werden. In dieser
Zielrichtung unterscheiden sich Bibliotheken von
Buchhandlungen.\footnote{Ewert und Umstätter, 1999, 9.}
\end{quote}

Der synoptische Aspekt bezieht sich auf die Informationsvermittlung, die
im Gegensatz zu passiven Dokumentationsangeboten steht. Durch aktive
Informationsversorgung und passive Dokumentation, so Ewert und
Umstätter, bietet die Bibliothek mithilfe der digital verfügbaren
Angebote über die eigenen Bestände hinaus eine Synopsis des weltweiten
Informationsangebotes.\footnote{Ewert und Umstätter, 1999, 9.}

Die Definition der Bibliothek hat in diesem Lehrbuch, wie den
vorhergegangen Ausführungen entnommen werden kann, anhaltend
Veränderungen und Erweiterungen erfahren. Olaf Eigenbrodt verweist in
diesem Zusammenhang zwar auf den bedeutenden Beitrag der Definition von
Ewert und Umstätter, die Definition der Bibliothek vom Medium Buch und
dem konkreten Aufbewahrungsort dieses Mediums emanzipiert zu haben, doch
bewertet Eigenbrodt den Versuch der Definition, welcher seiner
Auffassung nach in seiner Entstehungszeit schon anachronistisch war, als
einen letzten Versuch, um den Gegenstand der Bibliothekswissenschaft
dingfest zu machen\footnote{Eigenbrodt, 2013, 110.}, der sich seiner
Auffassung nach diesem Versuch jedoch entzieht. Nach Eigenbrodt reicht
die Definition Ewerts und Umstätters heute nicht mehr aus, um die
Funktionen, Aufgaben und Arbeitsfelder der Bibliotheken, die hierunter
verstanden werden sollen, zu beschreiben. Eigenbrodt kritisiert
insbesondere das Festhalten an funktionalistischen und positivistischen
Definitionen, die seiner Auffassung nach wissenschaftliche Bibliotheken
schon immer nur teilweise beschrieben und Öffentliche Bibliotheken
eigentlich ignoriert haben.\footnote{Eigenbrodt, 2013, 110.} Mit seinem
Beitrag versucht Eigenbrodt eine weitere Debatte zu eröffnen, deren
Gegenstand die Frage ist:

\begin{quote}
ob sich auf der Grundlage der vorliegenden Erkenntnisse eine neue,
vielleicht erweiterte Definition finden lässt, oder, ob eine solche
eventuell gar nicht mehr notwendig ist.\footnote{Eigenbrodt, 2013, 113.}
\end{quote}

\section*{Schlussbetrachtung}\label{schlussbetrachtung}

Der Artikel versucht aufzuzeigen, dass eine soziologisch verankerte
Betrachtung der Bibliothek nützlich doch innerhalb der Soziologie leider
weitestgehend abwesend ist. Versucht wurde in exploratorischer Art, der
Entwicklung der Erklärung und Rechtfertigung der Steuerung und
Definition einer Öffentlichen Bibliothek nachzugehen. Hierfür wurde eine
bewusste Auswahl an Quellen betrieben.

Die Frage was Bibliotheken im Kern sind, beschäftigt die
bibliothekarische Fachwelt anhaltend. Die Definition der Bibliothek
liegt seit ihrer Formulierung durch Ewert und Umstätter 1999 in einer
Form vor, die über den lokationsorientierten Aspekt der Sammlung
beziehungsweise des Aufbewahrungsortes von Büchern hinausgeht und
innerhalb des Bibliothekswesens weite Verbreitung erfahren hat. Von
Bedeutung ist an dieser Stelle, dass in der Definition ein ökonomischer
Aspekt zum Tragen kommt, welcher dass wirtschaftliche Interesse des
Unterhaltsträgers fokussiert. Wie sich die Definition weiterentwickeln
wird ist offen.

Die Einführung und Entwicklung betriebswirtschaftlicher Methoden der
Steuerung, Planung und Organisation zur Auftragserfüllung und
Zielerreichung in Bibliotheken spielt heute eine besondere Rolle für die
Legitimation der Einrichtungen. Der Neologismus Bibliotheksmanagement
zeigt dies deutlich an. Die einst auch als Institute bezeichneten
Einrichtungen scheinen sich derzeit von einer verwalteten Behörde zu
einem betriebswirtschaftlich organisierten Dienstleister zu entwickeln,
für den die betriebswirtschaftliche Denkweise und Organisation von
Aufbau- und Ablauforganisation, durch ihre Abhängigkeit als
unselbstständiger Teil der Verwaltung, notwendigerweise zunehmend an
Bedeutung gewinnt.

\section*{Quellenverzeichnis}\label{quellenverzeichnis}

AG-Soziologie. 2004. \emph{Denkweisen Und Grundbegriffe Der Soziologie.
Eine Einführung. 15. Auflage}. Frankfurt/New York: Campus Verlag.

Berger, Peter L. und Thomas Luckmann. 2009. \emph{Die Gesellschaftliche
Konstruktion Der Wirklichkeit}. Frankfurt a. M.: Fischer Taschenbuch
Verlag, 22. Auflage.

Beyersdorff, Günter. 1982. \enquote{Düstere Zukunft Für Bibliotheken?
Haushaltskrise - Entwicklung Neuer Technologien - Kommerzialisierung Der
Information.} \emph{BuB} 1: 43--54.

Brockhaus, ed. 1996. \emph{Brockhaus Enzyklopädie in 24 Bänden, 20.
Aufl., Band 3. (Bed-Brom)}. Leipzig-Mannheim: F.A. Brockhaus GmbH.

Cobabus, Norbert, ed. 2004. \emph{Bürgerrechte Und Bibliotheken. Die
Aushöhlung Des Freien Zugangs Zu Information Und Bildung Durch Die
ökonomisierung Der Gesellschaft}. Nümbrecht: Kirsch Verlag.

Deutsche-Buch-Gemeinschaft, ed. 1963. \emph{Das Dbg-Lexikon.} Frankfurt
a. M. - Berlin: Verlag Ullstein GmbH.

Eigenbrodt, Olaf. 2013. \enquote{Ist Eine Klare Definition von
Bibliotheken Noch Möglich?} \emph{BuB - Forum Bibliothek Und
Information; 66, 2013 Heft 2}.

Ewert, Giesela und Walther Umstätter. 1997. \emph{Lehrbuch Der
Bibliotheksverwaltung.} Stuttgart: Anton Hiersemann.

Ewert, Gisela und Walther Umstätter. 1999 Heft 6. \enquote{Die
Definition Der Bibliothek: Der Mangel an Wissen über Das Unzulängliche
Wissen Ist Bekanntlich Auch Ein Nichtwissen.} \emph{Bibliotheksdienst}
33: 957--71.

Fligge, Jörg und Alois Klotzbücher, ed. 1997. \emph{Stadt Und
Bibliothek. Literaturversorgung Als Kommunale Aufgabe Im Kaiserreich Und
in Der Weimarer Republik.} Wiesbaden: Harrassowitz Verlag, Wiesbaden.

Fritz, Manuela. 2007. \emph{Antike öffentliche Bibliotheken Und Ihre
Bildungspolitische Sowie Kulturelle Bedeutung.} Innsbruck: Innsbruck
University Press.

Fuchs, Hermann. 1968. \emph{Bibliotheksverwaltung, 2. Auflage}.
Wiesbaden: Otto Harrassowitz.

Funk, Robert. 1975. \emph{Kostenanalyse in Wissenschaftlichen
Bibliotheken. Eine Modelluntersuchung an Der Universitätsbibliothek Der
Technischen Universität Berlin.} Bibliothekspraxis, Band 17. Pullach bei
München: Verlag Dokumentation.

Hauke, Petra und Konrad Umlauf, ed. 2006. \emph{Vom Wandel Der
Wissensorganisation Im Informationszeitalter. Festschrift Für Walther
Umstätter Zum 65. Geburtstag.} Bad Honnef: Bock+Herchen Verlag.

Heidtmann, Frank. 1973. \emph{Zur Soziologie von Bibliothek Und
Bibliothekar. Betriebs Und Organisationssoziologische Aspekte.} Berlin:
DBV.

Jochum, Uwe. 1999. \emph{Kleine Bibliotheksgeschichte}. Stuttgart:
Philipp Reclam jun. GmbH \& Co., 2., durchgesehene und bibliographisch
ergänzte Auflage.

Kaiser, Wolfgang, and Karsten Schuldt. 2011. \enquote{Hat Die
öffentliche Bibliothek Einen Sozialen Auftrag Und Wenn Ja, Welchen? -
Ein Dialog.} \emph{LIBREAS. Library Ideas} 19: 45--69.

Karstedt, Peter. 1965. \emph{Studien Zur Soziologie Der Bibliotheken}.
Beiträge Zum Buch- Und Bibliothekswesen, Band 1. Wiesbaden: Otto
Harrassowitz.

Kluth, Rolf. 1970. \emph{Grundriß Der Bibliothekslehre.} Wiesbaden: Otto
Harrassowitz Verlag.

Kunze, Horst. 1958. \emph{Bibliotheksverwaltungslehre.} Leipzig:
Harrassowitz Verlag.

Leyh, Georg. 1968. \emph{Die Bildung Des Bibliothekars.} Darmstadt:
Wissenschaftliche Buchgesellschaft.

---------, ed. 1961. \emph{Handbuch Der Bibliothekswissenschaft.}
Wiesbaden: Otto Harrassowitz Verlag.

Pelaya, Lucia, and Ana Sanllorenti. n.d. \enquote{Der Schwierige Auftrag
Der Bibliotheken.} \emph{Argentina Copyleft}.

Plassmann, Engelbert; Herrmann Rösch; Jürgen Seefeld; Konrad Umlauf.
2011. \emph{Bibliotheken Und Informationsgesellschaft in Deutschland.
Eine Einführung}. Wiesbaden: Otto Harrassowitz Verlag, 2., gründlich
überarbeitete und erweiterte Auflage.

Powell, Walter W. und Paul J. DiMaggio, ed. 1991. \emph{The New
Institutionalism in Organisational Analysis}. London; Chicago: The
University of Chicago Press.

Rakowski, Frank, ed. 1968. \emph{Die öffentliche Bibliothek. Auftrag Und
Verwirklichung.} Berlin: Deutscher Büchereiverband.

Schäfers, Bernd und Wolfgang Zapf, ed. 2001. \emph{Handwörterbuch Zur
Gesellschaft Deutschlands}. Opladen: Leske+Budrich.

Schöller, Peter, ed. 1969. \emph{Allgemeine Stadtgeographie.} Darmstadt:
Wissenschaftliche Buchgesellschaft.

Scott, Richard W. 1995. \emph{Institutions and Organisations.} Thousand
Oaks: Sage Publications,

Senge, Konstanze. 2011. \emph{Das Neue Am Neo-Institutionalismus. Der
Neo-Insitutionalismus Im Kontext Der Organisationswissenschaft}.
Wiesbaden: VS Verlag für Sozialwissenschaften.

Senge, Konstanze und Kai-Uwe Hellmann, ed. 2006. \emph{Einführung in Den
Neo-Institutionalismus}. Wiesbaden: VS Verlag für Sozialwissenschaften.

Umlauf, Konrad. 2003. \emph{Leistungsmessung Und Leistungsindikatoren
Für Bibliotheken Im Kontext Der Ziele von Nonprofit Organisationen}.
Vols. 116, Berliner Handreichungen zur Bibliothekswissenschaft. Berlin:
Institut für Bibliothekswissenschaft der Humboldt-Universität.

Umlauf, Konrad und Stefan Gradmann, ed. 2011. \emph{Lexikon Der
Bibliotheks- Und Informationswissenschaft.} Stuttgart: Anton Hiersemann.

Umstätter, Walther. 2011. \emph{Lehrbuch Des Bibliotheksmanagements}.
Stuttgart: Anton Hierseman KG.

Vodosek, Peter und Werner Arnold, ed. 2008. \emph{Auf Dem Weg in Die
Informationsgesellschaft: Bibliotheken in Den 70er Und 80er Jahren Des
20. Jahrhunderts.} Wiesbaden: Harrassowitz Verlag, Wiesbaden.


%autor
\begin{center}\rule{0.5\linewidth}{\linethickness}\end{center}

\textbf{Fabio Tullio} hat Soziologie und Wirtschaftsgeographie an der
Universität Duisburg-Essen studiert. Während seines Studiums hat er
sechs Jahre als studentischer Mitarbeiter in der Universitätsbibliothek
am Campus Duisburg gearbeitet. Derzeit studiert er Bibliotheks- und
Informationswissenschaft sowie Informatik an der Humboldt Universität zu
Berlin. 

Kontakt: fabio.tullio@student.hu-berlin.de.

\end{document}
